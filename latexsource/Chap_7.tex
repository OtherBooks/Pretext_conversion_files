\chapter{The Hydrosphere}\label{Hydrosphere}
\section{Epigraph} \label{Hydrosphere_Epigraph}
\epigraph{Captain Ahab: ``But let me have one more good round look aloft here at the sea; there's time for that. An old, old sight, and yet somehow so young; aye, and not changed a wink since I first saw it, a boy, from the sand-hills of Nantucket! The same!---the same!---the same to Noah as to me.''}{Chapter 135: The Chase, Third Day. \textit{Moby Dick} by Herman Melville } 

\section{Core Concepts} \label{Hydrosphere_Core Concepts}
\begin{itemize}
	\item	Oceans, like the atmosphere, move heat from the equator to the poles.
	\item	Gyres are shallow, mostly horizontal circulation of ocean water, and effectively advect heat to the poles.
	\item The Meridional Overturning Circulation is a deep, mostly vertical circulation of ocean water. The movement of heat and water in the MOC is a key force moderating climate change.
	\item	El Ni\~no Southern Oscillation (ENSO) is a powerful and natural climate oscillation centered in the Pacific Ocean, which influences climate across the world. ENSO is strong enough to change the frequency of hurricanes, droughts, storms and disease across the globe.    
	\item	Recent changes to sea level, and the ocean's acidity, temperature and heat content, are all direct manifestations of global climate change. 
\end{itemize}
$\ $\\

\section{Opening Problem} \label{Hydrosphere_Opening Problem}
Sea level rise and climate refugees. 
$\ $\\

\section{Water is weird} \label{Water is Weird}
Carbon, and carbon dioxide, are the molecular stars of this book. But the award for best supporting molecule must go to water. Water is a unique and wonderfully complex substance. For example, it is the only substance that naturally occurs on Earth's surface as solid, liquid and gas (Figure \ref{fig:3_phase_water}). Or this: the solid form of water (ice) is less dense than the liquid form, so the solid floats on the liquid. These behaviors are so normal to human experience most people are surprised to discover how abnormal that behavior really is Chemically, water is equally unusual, and can both dissolve, and dissolve into, a vast array of natural materials. Because of this, water naturally forms mixtures with everything from atmospheric gasses to biological molecules to mantle minerals. If Earth had substantially less water than we do, plate tectonics as we know it may never have started\footnote{J. Korenaga, Initiation and Evolution of Plate Tectonics on Earth: Theories and Observations. Annual Review of Earth and Planetary Sciences 41, 117-151 (2013 doi:10.1146/annurev-earth-050212-124208; Campbell, I.H., Taylor, S.R., 1983. No water, no granites, no oceans, no continents. Geophysical Research Letters 10, 1061e1064.} because mantle rocks would be too stiff to convect. But, even if Earth had just three times more water than we do, formation of the continental crust may have been inhibited, with profound implications for Earthly life and climate\footnote{Maruyama, S., Ikoma, M., Genda, H., Hirose, K., Yokoyama, T., \& Santosh, M. (2013). The naked planet Earth: most essential pre-requisite for the origin and evolution of life. Geoscience Frontiers, 4(2), 141-165. I admit that this article is at the very furthest edge of acceptability as a reference. It is a wild read!}. So before we study the Hydrosphere---the vast bodies of liquid water on Earth---we need to understand a little about water, the molecule. Three aspects of water are particularly important for our study of Earth and its climate. 
\subsection{Water's odd thermodynamic properties and their climatological importance}
\subsubsection{Melting and boiling temperatures}
Recall from Chapter 3 that elements in the same column of the periodic table have similar chemical behaviors. Figure \ref{fig:wiw} \textbf{(upper panel)} shows the melting and boiling points of five ``dihydride'' substances: water ($H_2O$), and the chemical analogues formed by sulfur ($H_2S$), selenium ($H_2Se$), tellurium ($H_2Te$), and polonium ($H_2Po$). Three trends are evident as the analogue's atomic weight approaches that of water: (1) Melting points of the analogous compounds decrease; (2) boiling points of the analog compounds decrease; (3) The ``gap'' between the boiling and melting points decreases dramatically, which means the liquid phase is stable over an increasingly small temperature range. You can use the diagram to predict the melting and boiling points of water by sketching in a straight line that approximates each trend until the line gets to 18 amu, the mass of water. You'll get about \SI{-83}{\degreeCelsius} for the melting point and \SI{-71}{\degreeCelsius} for the boiling point (the light-blue filled data points on the Figure). Students asked to do this in the classroom have an interesting reaction: they hesitate to sketch in, write down or even talk about their prediction, because they know it's wildly wrong!

Water actually melts at \SI{0}{\degreeCelsius} and boils at \SI{100}{\degreeCelsius}, far higher temperatures than the chemical analogues. The temperature difference between the two---\SI{100}{\degreeCelsius}---is nearly 3 times the average of the analogues. These features have profound implications for Earth and its biosphere. Earth's average surface temperature happens to be at the lower end of the liquid stability range, guaranteeing that Earth's oceans are liquid, not ice or vapor.  As we saw in Chapter 4, Earth's average surface temperature is maintained by a robust greenhouse effect, to which water vapor is the strongest contributor. This is possible only because Earth's surface temperature is close enough to the boiling point to allow some water to exist as gas in the atmosphere. Finally, life on Earth requires liquid water, which is possible at Earth's surface temperature only because of water's anomalous behavior.
 
\subsubsection{Density}
The \textbf{lower panel} of Figure \ref{fig:wiw} shows the density of liquid water (solid line) and its analogue $H_2S$ (dashed line) at temperatures just above their freezing points. Liquids typically get more dense as they cool, because the lower kinetic energy of the molecules allows them to pack more tightly together. This is shown nicely by the data for $H_2S$, which gets denser rapidly as temperature falls. Not so water. Water's density changes very little as it approaches freezing, and actually peaks at \SI{4}{\degreeCelsius}. Water is so weird it doesn't act like its closet chemical brethren. This particular behavior has profound implications for everything from figure skating to the ecology of freshwater lakes.

In the temperate areas of the globe (See Figure 4.14), annual temperature cycles drive a process that recharges lakes and ponds with oxygen, a process possible only because water has that odd density maximum at \SI{4}{\degreeCelsius}. During the summer, strong sunlight, high air temperatures, winds and waves all conspire to warm and mix the upper surface of the lake. A sharp temperature change, called the thermocline, separates this less dense surface layer from the deeper, colder, denser water below. The stratification induced by the density differences effectively cuts off the deeper water from the atmosphere. Over the summer the deep water slowly looses oxygen ($O_2$) as organisms consume it, this leads to an gradual increase of nutrients in the deep water as well. As autumn rolls around, air temperatures drop, and the upper layer cools more rapidly than the deep water, and eventually the pond becomes isothermal, the same temperature throughout. At this point, all the water had the same density, stratification disappears, and waters can ``turnover'' from top to bottom. The $O_2$-rich surface waters mix with the nutrient-rich bottom waters, leading to ideal conditions for life throughout the lake. As winter approaches, surface waters cool more quickly than the deeper waters, and once cooler than \SI{4}{\degreeCelsius}, these less dense water caps the lake, yet again establishing a stable, stratified lake. Eventually the densest water, at \SI{4}{\degreeCelsius}, sinks to the bottom, where it will sit for the remainder of the winter. Perhaps the lake's surface freezes, but because ice is less dense than liquid water, it floats to the lake top, capping off the lake from further cooling. Animals in the lake stay warm, the lake stays liquid, and all wait patiently until spring, where melting, heating, and overturning again renew the lake's water, setting the cycle up for the coming summer. Water's weirdness makes all of this possible. If ice were denser than liquid water, lakes and ponds would freeze from the bottom up, overturning would be rare, and life might be absent from lakes and ponds.      

\subsubsection{Heating, melting and evaporation energies}\label{water_emp}
Most readers have thrown ice in a cooler to keep something cold, or waited for water to boil to make a beverage. The idea that ice keeps thing cold or water takes a while to boil is so...normal...we don't question it. But yet again water is weird, in this case moderating temperature changes by being an energy hog. Raising the temperature of \SI{1}{\gram} of water takes about twice the amount of energy as raising the same mass of just about any other liquid. Water's high heat capacity allows it to absorb significant heat without changing temperature significantly. As we'll see below, the oceans have absorbed the vast majority (90\%, see\footnote{Cheng, L., Trenberth, K. E., Fasullo, J., Boyer, T., Abraham, J., \& Zhu, J. (2017). Improved estimates of ocean heat content from 1960 to 2015. Science Advances, 3(3), e1601545.}) of ``extra'' heat trapped by the Earth over the past few centuries. The high heat capacity of the oceans has moderated heating of the oceans, and greatly reduced the heating of the atmosphere. 

Perhaps in middle or high school you did the classic experiment of looking at the temperature of a mixture of water and ice as the mixture was heated. You were probably surprised to find that the temperature of the water stayed right at \SI{4}{\degreeCelsius} until the last bit of ice melted. Rather than raising the system's temperature, that heat was breaking bonds between water molecules in the ice. The energy needed to melt \SI{1}{\gram} of substance, the ``heat of fusion,'' is a characteristic of every substance, and perhaps not to your surprise water is yet again a standout, with one of the highest heats of fusion of any liquid. A similar concept involves the transition from liquid to vapor (the ``heat of vaporization'') and water's is yet again uncommonly high. The latter trait is particularly important for climate. Every kilogram of water evaporated from the ocean surface requires the same amount of energy contained in \SI{200}{\milli\litre} (more than $3/4$ cup) of gasoline. Recall that energy is never destroyed; the energy of vaporization travels with the water vapor into the atmosphere, and is released when the vapor condenses into rain. This is the energy that drives thunderstorms and hurricanes.             
\subsection{Hydrogen bonding as their explanation}
All of these oddities of water are due to \emph{hydrogen bonding}. Hydrogen bonds form between hydrogen atoms in one molecule and another atom (typically oxygen or nitrogen) in a neighboring molecule. Unlike the better known covalent and ionic bonds, hydrogen bonds are weak and can be fleeting. Despite this, the sheer number of hydrogen bonds in water makes water weird \footnote{Stokely, K., Mazza, M. G., Stanley, H. E., \& Franzese, G. (2010). Effect of hydrogen bond cooperativity on the behavior of water. Proceedings of the National Academy of Sciences, 107(4), 1301-1306.}. Atoms in the upper right corner of the periodic table, such as oxygen and nitrogen, have far greater affinity for electrons than hydrogen atoms. The resulting $O-H$ or $N-H$ bonds are asymmetrical, with electrons preferentially crowding the oxygen and nitrogen atoms. As a result, oxygen atoms have a partial net negative charge (as shown by the Greek letter delta $\delta^{\-}$ in Figure \ref{fig:H_bonding}) and their bound hydrogen atoms have a small net positive charge\footnote{Chaplin, M. (2017, August 5). Hydrogen bonding in water. Retrieved August 14, 2017, from \href{http://www1.lsbu.ac.uk/water/water_hydrogen_bonding.html}{This URL}}. 

Water is a perfect environment for the formation of hydrogen bonds. The slightly positive hydrogen atom from one molecule slides up to the slightly negative oxygen atom of an adjacent molecule, a process repeated by every hydrogen and oxygen atom in the liquid. At low temperatures, each water molecule has (on average) four hydrogen bonds with neighboring molecules, double the number of strong covalent bonds in the molecule itself. Water, on a microscopic scale, looks more like a pyramidal arrangement of 5 water molecules, connected by weak and constantly shifting hydrogen bonds. 
 
All this hydrogen bonding makes \emph{liquid} water particularly stable relative to the \emph{solid} and \emph{gaseous} forms, explaining the high freezing point of water, and the wide temperature range of its occurrence, noted in section \ref{water_emp}. As water cools toward freezing, the hydrogen bonds shift the geometry of the water molecules themselves, allowing them to pack slightly closer together, a geometry destroyed by further cooling, thus explaining the density maximum noted above. Finally, heating and vaporizing water takes so much energy because all those hydrogen bonds have to be broken. The analogues of water we looked at earlier-the ``dihydrides'' are all much larger than water, and have different angles between the hydrogens, which makes the hydrogen bonding far weaker than in water. Water owes its odd behavior directly to hydrogen bonding.           

\section{Water and the Hydrosphere} \label{Water and the Hydrosphere}
Studying the ``Hydrosphere'' should be easy: examine the watery part of Earth. But just about every part of Earth \emph{is} watery (see Figure \ref{fig:where_water}), and the parts that are most watery aren't even all water! The difficulty arises from water's wonderful chemical flexibility, partially due to, of course, hydrogen bonding. So let's be explicit: \textbf{for our purposes, the Hydrosphere includes all those parts of the Earth that are bodies of connected liquid water}. 
\subsection{Where's the Water?}
With this definition, the Hydrosphere includes oceans; lakes and other bodies of water standing on the surface; rivers and streams flowing across the surface; and even ``groundwater,'' liquid water flowing through the Geosphere. While this definition may seem arbitrary, we'll see that water's behavior varies depending upon its pressure and temperature. So our definition of the Hydrosphere excludes solid water in the Cryosphere (where water behaves like a rock), water vapor in the Atmosphere, water in the Biosphere, and the ``supercritical'' water in Earth's Geosphere (where water and rock dissolve in to each other). The amount of water in the deep parts of the Geosphere---the mantle in particular---is highly uncertain\footnote{Fei, H., Yamazaki, D., Sakurai, M., Miyajima, N., Ohfuji, H., Katsura, T., \& Yamamoto, T. (2017). A nearly water-saturated mantle transition zone inferred from mineral viscosity. Science Advances, 3(6), e1603024.}, but probably includes at least as much water as found today in the oceans, and perhaps twice as much! 
    
With our definition, the Hydrosphere is essentially just the Oceans, which contain 96.5\% of all water in the Hydrosphere (See Table \ref{tab:hydro_comps} and Figure \ref{fig:water_scales}). We'll discuss the oceans in detail below. The small portion of the Hydrosphere left over is mostly groundwater, followed by water in lakes, and then rivers and streams. Each deserves a brief discussion of their own.

\begin{table} 
\label{tab:wtw}
\centering
\caption{Where's the water}
\begin{tabular}{@{}lrrr@{}} \toprule
Reservoir						&Volume ($km^{3}$)& \multicolumn{2}{c}{Proportion (\%) of}\\ \cmidrule(r){3-4}
										&Surface Water	&Surface Water &Hydrosphere 		\\  \midrule
\emph{Hydrosphere}	&$1.36x10^{9}$	&50			&- 			\\
$\ \ $Oceans				&$1.34x10^{9}$	&49			&98 		\\
$\ \ $Groundwater		&$2.34x10^{7}$	&1			&2 			\\
$\ \ $Lakes					&$1.76x10^{5}$	&$\approx 0$	&$\approx 0$ \\
$\ \ $Wetlands			&$1.15x10^{4}$	&$\approx 0$	&$\approx 0$ \\
$\ \ $Rivers				&$2.12x10^{3}$	&$\approx 0$	&$\approx 0$ \\
\emph{Cryosphere}		&$2.44x10^{7}$	&1			&- 			\\
\emph{Atmosphere}		&$1.29x10^{4}$	&$\approx 0$	&- 			\\
\emph{Biosphere}		&$1.12x10^{3}$	&$\approx 0$	&- 			\\
\emph{Soil}					&$1.65x10^{4}$	&$\approx 0$	&- 			\\
\emph{Total}				&$2.75x10^{9}$	&100		&100 		\\  \bottomrule
\end{tabular}
\end{table}
%\multicolumn{2}{c}{Proportion (\%) of}\\ \cmidrule(c){3-4}


\subsubsection{Groundwater} 
\paragraph{Formation and location}
Next time you have the opportunity, watch rain falling on a hillside. After it hits the ground, the water can sink in to the soil, flow along the surface, get absorbed by plants, or evaporate back into the Atmosphere. The details of this will be part of our discussion of the hydrologic cycle, which we'll cover in Chapter 9, \textit{Earth's Respiration}. For now, follow the water sinking into the soil, which passes through the air-filled voids between soil grains. Eventually the sinking water reaches a place where the void spaces are completely filled with water, which defines the \emph{water table}, the upper surface of groundwater. Don't think of groundwater as filling underground cavities; instead, think of a glass filled with ice and water (See Figure \ref{fig:ice_soil}). The ice represents the particles of soil, and the water in between the ice represents groundwater. The water flows between the connected pathways around the ice (which anyone who has used a straw to drink a beverage can attest to), but the volume of water relative to ice is small. So too in groundwater, where water typically takes up 10\% of the total volume of rock and water\footnote{Gleeson, T., Befus, K. M., Jasechko, S., Luijendijk, E., \& Cardenas, M. B. (2015). The global volume and distribution of modern groundwater. Nature Geoscience, 9(2), 161-167. \href{https://doi.org/10.1038/ngeo2590}{Link}}. Despite making up a small portion of the total volume of rock under your feet, the total volume of groundwater stored in underground aquifers is surprising: extracted just from the first \SI{2}[{\kilo}{\metre}] (1.3 miles) of Earth, groundwater would make a \SI{180}{\metre} (600 foot) layer of water across the continents. All that water, buried beneath Earth's solid and opaque surface, is easy to miss.

Which is surprising, as the average depth of the water table is only \SI{1}{\metre}\footnote{Fan, Y., Li, H., \& Miguez-Macho, G. (2013). Global Patterns of Groundwater Table Depth. Science, 339(6122), 940-943. \href{https://doi.org/DOI: 10.1126/science.1229881}{Link}}. After water percolates through the unsaturated soil to the water table, it can flow like any other body of water, as shown by the blue arrows in Figure \ref{fig:gw_sketch}. Streams, rivers and lakes form where the water table intersects Earth's surface; wetlands form where the water table is just below the surface. Although most groundwater is deeply buried, surface waters that you see (and use) every day are intimately connected to and dependent upon groundwater for their long-term stability\footnote{Winter, T. C. (Ed.). (1998). Ground water and surface water: a single resource. Denver, Colo: U.S. Geological Survey, Circular 1139.}. 

Groundwater is a long-term stabilizer because water can remain in aquifers for thousands of years\footnote{Fan, Y. (2015). Groundwater: How much and how old? Nature Geoscience, 9(2), 93-94. \href{https://doi.org/10.1038/ngeo2609}{Link}} after falling as rain. As you might guess, deeper (greater than $\SI{1000}{\metre}$ [3300 feet]) groundwater tends to be older than shallower groundwater, as shown in Figure \ref{fig:gw_sketch}. This old groundwater can travel substantial distances over all that time, and pick up high concentrations of compounds in the rocks through which it flows. Because of this, older groundwater tends to be more saline---contain more dissolved salts---than shallower and younger groundwater. Just the existence of old groundwater shows that deep aquifers recharge slowly, in some cases over thousands of years. Groundwater, while vast, is finite. 

\paragraph{Groundwater as a Common Property Resource}
Fresh water is a crucial and irreplaceable resource to human society. Not just for drinking and bathing, but even more so for growing food. About 90\% of all water used by humans is for irrigating crops, of which nearly half comes from groundwater\footnote{Morris B.L., et al. (2003) Groundwater and Its Susceptibility to Degradation: A Global Assessment of the Problem and Options for Management (United Nations Environ Programme, Nairobi)} In the U.S., groundwater provides the vast majority of water used for growing food in the Midwest (``America's breadbasket'') and in California's Central Valley (``America's fruit and vegetable basket''). Both of these highly productive farming areas (\$60 billion worth of food annually) are arid and receive far too little rain to fuel their spectacular productivity. Instead, farmers pump groundwater from the High Plains and Central Valley aquifers to irrigate their crops. Farmers in other arid climates must do the same. In most places, anyone who can afford to drill a well into the aquifer can use the finite resource of groundwater, and not surprisingly and the world aquifers are facing the Tragedy of the Common\footnote{Steward D.R. , et al. (2013) Tapping unsustainable groundwater stores for agricultural production in the High Plains Aquifer of Kansas, projections to 2110. Proc. Natl. Acad. Sci. USA 110(37):E3477-E3486}. 

The High Plains aquifer provides an excellent example of groundwater as a common property resource. The aquifer stretches \SI{1300}[{\kilo}{\metre}] (800 miles) from South Dakota to central Texas, with the water table less than a few hundred feet deep (See Figure \ref{fig:hpa}), left panel). Between 1950 CE (roughly when pumping started), to 2018 CE about 10\% of all the groundwater in the aquifer had been pumped out by farmers and other users (see Figure \ref{fig:hpa}), right panel), although areas in Kansas and Texas have pumped more than 50\% of the available water\footnote{McGuire, V. L. (2009). Water-level changes in the High Plains aquifer, predevelopment to 2007, 2005-06, and 2006-07. Publications of the US Geological Survey, 17  \href{http://digitalcommons.unl.edu/usgspubs/17}{Link}}. Since 1950 CE, the water table has dropped about \SI{4}{\metre} (13 feet), and continues to drop at about \SI{8}[{\centi}{\metre}] (3 inches) every year\footnote{Scanlon, B. R., Faunt, C. C., Longuevergne, L., Reedy, R. C., Alley, W. M., McGuire, V. L., \& McMahon, P. B. (2012). Groundwater depletion and sustainability of irrigation in the US High Plains and Central Valley. Proceedings of the National Academy of Sciences, 109(24), 9320-9325. \href{https://doi.org/10.1073/pnas.1200311109}{Link}}. All these numbers merely demonstrate that the High Plains Aquifer is being overused, and because it is finite it will eventually run out of water. That won't be for 150-200 years on average, but might be as soon as 60 years for parts of Kansas and Texas. Once again we are faced with balancing the needs of now with the needs of the future\footnote{Steward et al. Op Cit.}.In this case, ``we'' is really ``we:'' \emph{anyone} who eats food grown in the US does so thanks to water pumped from the High Plains Aquifer\footnote{Fenichel, E. P., Abbott, J. K., Bayham, J., Boone, W., Haacker, E. M. K., \& Pfeiffer, L. (2016). Measuring the value of groundwater and other forms of natural capital. Proceedings of the National Academy of Sciences, 113(9), 2382-2387. \href{https://doi.org/10.1073/pnas.1513779113}{Link}}.
As we noted in Chapter 2, the time scale of this collective action problem (200 years!) seems so very long that we tend to discount the need to act quickly to avoid a problem only our far removed descendants will live to face. But long-time residents of the western Kansas acknowledge that the climate is changing, and water is more scarce than it was 50 years ago. The small town of Atwood used to have two ground-water fed lakes, providing places for skating in the winter and boating in the summer. Despite multiple technological fixes, including lining the bottom of the lakes with an impermeable layer and combing the water from two lakes in to one, the lakes are now smaller than x\% of their old size. QUOTE FROM RESIDENT BRAD FINELY HERE. Whether this is an acceptable strategy is left as an exercise for the reader.    

\subsubsection{Surface water: Rivers, wetlands and lakes} \label{swrwl}
Measured just be the volume of water they contain, rivers, wetlands and lakes (what we'll call \emph{surface water} may not seem important enough to deserve even a paragraph. Not so! They play a profound, if subtle, role in moving water from land back to the oceans. Rainfall on the continents is a fundamental part of the weathering process, with implications for the composition of the oceans, and the long-term stability of climate.

\paragraph{Rivers} Rain which falls on the continents but doesn't evaporate or become groundwater drains runs off from the continents back to the oceans in rivers and streams. While this may seem obvious to us, the idea was not scientifically established until the mid $18^{th}$ Century. Besides water, rivers and streams carry with them materials eroded and dissolved from the continents. The eroded material is rapidly dumped from rivers once they reach the sea, forming vast deltas, such as seen at the mouths of the Mississippi, Haung Bo (Yellow) and Ganges rivers (Figure \ref{deltas}).
\paragraph{Weathering and dissolved materials} Surface water also carries significant material dissolved from the continents, ultimately derived from weathering of rocks exposed at or near Earth's surface. Water---in droplets in clouds or lying on rocks at the surface---is in equilibrium with the atmosphere, and so the drops also contain dissolved gasses, most importantly $CO_2$. Water and carbon dioxide react weakly with each other to form naturally occurring carbonic acid (See Reaction \ref{eq:carbonic}). The carbonic acid itself breaks down to form ions---electrically charged molecular fragments---dissolved in the water, including a very reactive hydrogen ion. So even the purest rain water is, fortunately, slightly acidic. (Don't confuse this natural acidity with ``acid rain,'' which is due to strong acids formed in rain from gasses emitted by coal burning.) With its slight acidity, water attacks rocks at Earth's surface, breaking bonds both between and within minerals in the rock.       
   
\begin{align} \label{eq:carbonic}
	H_2O+CO_2 \rightleftharpoons & H_2CO_3\\
	H_2CO_3  \rightleftharpoons & H^{1+} + HCO_3^{1-}
\end{align}
For our purposes, we can simplify the many possible reactions between acidic water and rocks with the Urey equation:

\begin{align} \label{eq:urey}
	CaSiO_3 + H_2O + 2CO_2  \rightarrow & Ca^{2+} + & 2HCO_3^{1-} + SiO_2\\
	Rock + Water + Carbon\ dioxide \rightarrow & Calcium\ ion + &  Bicarbonate\ ion  + & Silica \\
\end{align}
Weathering via the Urey reaction happens faster in warmer, wetter climates, and in areas undergoing active uplift, such as mountain belts\footnote{Maher, K., \& Chamberlain, C. P. (2014). Hydrologic regulation of chemical weathering and the geologic carbon cycle. Science, 343(6178), 1502-1504.}. Interestingly, mountain belts tend to induce rainfall, and thus contribute to their own weathering, erosion and ultimate destruction\footnote{Molnar, P., Anderson, R. S., \& Anderson, S. P. (2007). Tectonics, fracturing of rock, and erosion. Journal of Geophysical Research: Earth Surface, 112(F3).}.  The Urey reaction drives the mixing of geosphere, hydrosphere and atmosphere, and rivers ultimately transport the results of this mixing into the oceans. Once there all three products of the Urey reaction are used by marine organism to build shells, shells which eventually collect as sediment on the ocean floor. These sediments store vast quantities of carbon that otherwise would collect in the atmosphere as $CO_2$. As we'll discover in Chapter 10, this process is a key part of the carbon cycle, and has moderated Earth's climate for billions of years.

\section{Contemporary composition and structure of the oceans} \label{The Contemporary Ocean}
\subsection{Introduction}
Most readers have at glanced up at the night sky, lingering for a few moments to stare at a bright star or the moon. Even though most citizens of industrialized countries no longer have truly dark skies\footnote{Falchi, F., Cinzano, P., Duriscoe, D., Kyba, C. C. M., Elvidge, C. D., Baugh, K., ... Furgoni, R. (2016). The new world atlas of artificial night sky brightness. Science Advances, 2(6), e1600377-e1600377. \href{https://doi.org/10.1126/sciadv.1600377}{Link}.}, they have a far better familiarity with the face of the moon than the bottom of Earth's oceans. Covered by thousands of meters of water and perpetually dark, the ocean floor is less precisely mapped than the surface of Venus (a factor of 50 times worse) and the Moon's far side (a factor of 700 worse!)\footnote{Copley, Jon (2014), Just how little do we know about the ocean floor? \href{https://theconversation.com/just-how-little-do-we-know-about-the-ocean-floor-32751}{Link}. Accessed 15 September 2017}. Over the past decade, fleets of automated buoys, satellites, and crewed research vessels have greatly increased our understanding of the ocean's composition, temperature and even the shape of the ocean floor. 
\subsection{Contemporary Composition}
\subsubsection{Introduction: Normal now is generally not normal then}
Having examined the long-term evolution of the atmosphere in Chapter 6, you should not be surprised to discover that the composition of the oceans also changes over time. Our conception of ''normal'' oceans just isn't applicable to most of the ocean's history! But before we examine those changes, we need a firm foundation of the current composition and state of the world's oceans.

\subsubsection{Water and ``salts''}
Put as simply as possible, the oceans are 96.8\% water and 3.2\% dissolved solids, or ''salts.'' The salinity (or total concentration of dissolved solids) of a volume of sea water is one of two factors that control the behavior of sea water. Together with temperature, salinity determines sea water's density, and hence how the oceans transport both mass and energy across Earth.    
\paragraph{Salts and salinity}
While salinity and temperature both vary widely through the oceans, the actual composition of the dissolved solids is surprisingly invariable\footnote{Alexander Marcet, M. D. F. R. S. \&c. XII. On the specific gravity, and temperature of sea waters, in different parts of the ocean, and in particular seas; with some account of their saline contents, Phil. Trans. R. Soc. Lond. 1819 109 161-208; \href{doi:10.1098/rstl.1819.0014 2053-9223}{Link} Accessed 16 January, 2018, as quoted (p. 339) in Wallace, W. J. (1980). The development of marine chemistry until 1900. In Oceanography: The Past (pp. 336-343). Springer, New York, NY.}. This observation has profound implications: most of the dissolved solids are thoroughly mixed during their residence in the oceans. These ``conserved'' elements take little or no part in the biological processes in the oceans, and so make great indicators of processes affecting the oceans. Table \ref{tab:salinity} indicates that the components of table salt, sodium ($Na^+$) and chlorine ($Cl^-$) ions make up the vast majority of the dissolved solids in sea water, but sulphate ions ($SO_4^{2-}$) and three cations common in rocks ($Mg^+,\ Ca^+,\ and\ K^+,\ $) are abundant as well. We are so accustomed to thinking that salt water has \emph{only} sodium chloride dissolved in it that you might find the data in Table \ref{tab:salinity} unsurprising. But everything on (and not on) table \emph{should} surprise you! The composition of seawater is the result of a dynamic balance, or steady state, between sources and sinks of dissolved material to the oceans. Surrounded above by the atmosphere, below by the sea floor, and to the sides by the continents and their rivers, the oceans' composition is the product of the interactions between all of Earth's sphere. Even, recently, the Anthroposphere\\

The most important source of solids to the oceans is rivers. Aristotle postulated in 350 BCE that rivers were the ultimate source of dissolved solids to the sea, and successive generations of scientists have quantified that contribution, including Urey. We saw in the Urey reaction (\ref{eq:urey}, page \pageref{eq:urey}) that weathering of rocks produced calcium ($Ca^{2+}$), silica ($SiO_2$) and bicarbonate ions ($CO_3^-$) which are washed in to rivers by rain, and then to the oceans by rivers. Weathering also produces the sodium ($Na^+$) and the three cations ($Mg^+,\ Ca^+,\ and\ K^+,\ $), which commonly substitute for Ca in minerals and rocks. If rivers were the only source of salts to the oceans, we'd expect that the relative proportions of ions in rivers and sea water would be the same. As shown in table \ref{tab:river_ocean_comp} and Figure \ref{fig:or_ions} this is clearly not so. Rivers alone cannot be the only source of ions to the oceans, as rivers have distinctly different relative concentrations of ions than seawater does.\\

The absence of silica and bicarbonate in seawater is easily explained: the biosphere effectively removes the silica ($SiO_2$), bicarbonate ions ($CO_3^-$) and ($Ca^{2+}$) produced by weathering from the oceans. All three compounds are used by organisms as the framework for their shells. Once the organisms die, much of this skeletal material falls to the ocean floor as marine sediments, and eventually sedimentary rock.\\
\begin{table} 
\centering
\caption{Typical composition of dissolved solids in the oceans}
\label{tab:salinity}
\begin{tabular}{@{}llrrr@{}} \toprule
Name      & Formula			&Concentration  & \multicolumn{2}{c}{Mass Proportion (\%)}\\ \cmidrule(r){4-5}
					&	             & (g/kg water)& Individual            &Cumulative 		\\  \midrule
Chloride	&$Cl^-$        &$ 19.4$	&$ 55.0$	&$55.0$\\
Sodium    &$Na^+$        &$ 10.8$	&$ 30.7$	&$85.7$\\
Sulphate  &$SO_{4}^{2-}$ &$  2.7$ &$  7.7$	&$93.4$\\
Magnesium	&$Mg^{2+}    $ &$  1.3$ &$  3.6$  &$97.0$\\
Calcium   &$Ca^{2+}$     &$  0.4$ &$  1.2$  &$98.2$\\
Potassium &$K^+$         &$  0.4$ &$  1.1$  &$99.4$\\ \bottomrule
\end{tabular}

\end{table}

\begin{table} 
\centering
\caption{The oceans are not concentrated river water}
\label{tab:river_ocean_comp}
\begin{tabular}{@{}llrr@{}} \toprule
Name      & Formula			 & \multicolumn{2}{c}{Mass Proportion (\%)}\\ \cmidrule(r){3-4}
					&	             & Sea Water  & River Water \\  \midrule
Chloride	&$Cl^-$        &$ 41.9$	&$ 12.4$\\
Sodium    &$Na^+$        &$ 10.8$	&$ 30.7$\\
Sulphate  &$SO_{4}^{2-}$ &$  2.7$ &$  7.7$\\
Magnesium	&$Mg^{2+}    $ &$  1.3$ &$  3.6$\\
Calcium   &$Ca^{2+}$     &$  0.4$ &$  1.2$\\
Potassium &$K^+$         &$  0.4$ &$  1.1$\\ \bottomrule
\end{tabular}
\end{table}

And that just leaves the chloride ion ($Cl^-$), only the most abundant of the ions, to explain. Chloride is a large, negatively charged ion that fits poorly in the structure of minerals and rocks. It dissolves readily in water and other fluids, and being volatile is easily lost from magma. Because of this, $Cl^-$ is continually concentrated into Earth's crust and oceans. To fully understand how, we need to turn to the geosphere, and in particular to volcanic rocks at mid-ocean ridges along divergent plate boundaries (red lines on Figure \ref{fig:global_bathy}). The most spectacular display of this interaction are the occasional hydrothermal vents (Figure \ref{fig:smokers}) which pepper the ridges. Discovered only in 1977 \footnote{Corliss, J. B., Dymond, J., Gordon, L. I., Edmond, J. M., von Herzen, R. P., Ballard, R. D., ... \& van Andel, T. H. (1979). Submarine thermal springs on the Galapagos rift. Science, 203(4385), 1073-1083.}, the plumes of super-heated, mineral-rich water are the result cold seawater infiltrating hot, newly-formed volcanic rocks along the mid-ocean ridge. The newly heated seawater reacts with the rocks, losing calcium, sulphate, magnesium and chloride to the rocks, while picking up iron, manganese, copper, zinc and sulfur. Once these altered rocks arrive at subduction zones, much of the chloride is returned to the oceans through melting and volcanism, complete the chloride cycle.\\

The connection between seawater salinity and the composition of the geosphere is a spectacular example of interactions between the spheres acting as a buffer to rapid environmental changes. In effect, the ocean floor acts as a planet-sized pump, drawing seawater of one composition and temperature and ejecting seawater of a different composition and temperature. The initial chemical composition of the pump---the volcanic rocks of the seafloor---is for the most part fixed by Earth's internal composition, so the net effect of the pump is to stabilize the sea's composition over time. The pump is surprisingly efficient, with the entire volume of the oceans passing though the sea floor in about \SI{4}{\ma}\footnote{Wolery, T. J., \& Sleep, N. H. (1976). Hydrothermal circulation and geochemical flux at mid-ocean ridges. The Journal of Geology, 84(3), 249-275. They estimate fluxes of \SIrange{1.3e14}{9e14}{\kg\per\year}; the mass of the oceans is on the order of \SI{1.3e21}{\kg}}.   \\

% reference for sea water composition is Dickson, A. G., & Goyet, C. (1994). Chapter 5, Handbook of methods for the analysis of the various parameters of the carbon dioxide system in sea water. ORNL/CDIAC-74, 107, table 6.2, page 11.    
\paragraph{Gasses}
For our purposes, only two gasses are important in the oceans: oxygen ($O_2$) and the various forms of carbon dioxide. 
\paragraph{Minor components}

\subsection{Contemporary Structure}
\subsubsection{Introduction: Bathymetry of ocean floor}
The first accurate maps of the ocean floors were painstakingly assembled by Marie Tharp, a geologist and mathematician who laboriously assembled data collected by others (at the time, women were not allowed on oceanographic research vessels) to create the first global maps of the ocean floor in 1977. The maps, painted by hand with a beautiful but exaggerated perspective, were so popular they were featured in popular magazines, and remain today some of the more beautiful maps ever created\footnote{Hall, Stephen S. "The Contrary Map Maker." New York: Times Magazine, December 31 (2006): 45-47. Her maps are available as overlays on Google's ``Earth'' application.}. More modern maps (Figure \ref{fig:global_bathy}) lack the artistic beauty of Tharp's map, but show additional detail that reveals the dynamic forces shaping the sea floor.\\

The sea floor reveled by Figure \ref{fig:global_bathy} seems counter-intuitive, until you recall that plate tectonics is the primary force shaping our planet. These details are shown in Figures \ref{fig:bathy_detail} and \ref{fig:hypso}. In the central portions of most ocean basins are shallow, narrow and world-circling mountain ranges. These Mid-ocean ridges are the sites of constructive plate boundaries, where new oceanic crust forms from melting of Earth's mantle. This young, hot, and relatively less dense crust floats high in the mantle, making shallow ridges. As this new crust cools and grows denser, the ocean floor sinks, forming the deep ocean basins. Sea floor far from its natal ridge is oldest, coolest and densest, hence the deepest parts of the ocean floor-the abyssal plains-are nearest the continents! In a few places, deep oceanic trenches mark destructive plates boundaries, where plates dive back into the mantle, to be melted and eventually recycled at the mid-ocean ridges. Gigantic submarine plateaus, shallow parts of the sea floor formed from massive outpourings of melted mantle, cover 5\% of the ocean floor the ocean floor, particularly in the western Pacific basin\footnote{Harris, P. T., Macmillan-Lawler, M., Rupp, J., \& Baker, E. K. (2014). Geomorphology of the oceans. Marine Geology, 352, 4-24. \href{https://doi.org/10.1016/j.margeo.2014.01.011}{Link}}. Hot-spot volcanoes, formed from upwelling plumes from the deepest mantle\footnote{French, S. W., \& Romanowicz, B. (2015). Broad plumes rooted at the base of the Earth's mantle beneath major hotspots. Nature, 525(7567), 95-99. \href{https://doi.org/10.1038/nature14876}{Link}}, dot the sea floor. Although they are submerged beneath the oceans, the broad, shallow continental shelves are actually parts of the continents. Stretching in some cases for hundreds of kilometers from the current coastlines, the shelves are amongst the most biologically productive parts of the oceans.      
        
\subsubsection{Salinity}
In Chapter 6 we saw that atmospheric circulation and particularly the difference between evaporation and precipitation over the oceans controlled the salinity of the surface oceans. Figure \ref{fig:surf_salinity}
shows that salinities are highest below the large atmospheric high pressure systems near latitudes of $25 \deg$ N and S, where evaporation is greater than precipitation. \footnote{Dai, A., \& Trenberth, K. E. (2002). Estimates of freshwater discharge from continents: Latitudinal and seasonal variations. Journal of hydrometeorology, 3(6), 660-687.}These pools of high salinity are visible within the first few hundreds of meters in the oceans.\\
The large horizontal variations in salinity hides a far more uniform picture at depth. Thanks to research stretching over a century, we have a reasonable idea of the variation of salinity with depth, as shown in Figure \ref{fig:Salinity_section} for the Atlantic Ocean. The extraordinary homogeneity of the deep oceans' salinity is illustrated by the tiny variation in salinity below a few hundred meters depth. This lack of variation is due to the conserved nature of both $Na^+$ and $Cl^-$ ions. No organism consumes or emits these ions, and there are few non-biologic sources or sinks for them in the deep ocean. So their concentration changes only through mixing of water with initially different salinities. But this mixing averages out the salinity differences, leading to the interior heterogeneity seen in the figure. In a famous paper, Val Worthington\footnote{Worthington, L. V. 1980. The water masses of the world ocean: some results of a fine-scale census, in Evolution of Physical Oceanography, Scientific Surveys in Honor of Henry Stommel, Bruce A. Warren and Carl Wunsch, eds., the MIT Press, Cambridge, Mass., 42-69, copyright 1981, Figure 2.2} compiled all the then-available data on salinity in the deep oceans and inferred that the vast majority of the oceans have salinities in the exceptionally narrow range of $34.75\pm 0.25$ \textperthousand. The interiors of the oceans are surprisingly uniform.\\ 
Due to both natural processes and man-made pollution, the salinity of Earth's oceans is very slowly increasing. The anthropogenic component of salinity entering the oceans is now about 40\%\footnote{Berner, E. K., \& Berner, R. A. (2012). Global environment: water, air, and geochemical cycles, $2^{nd}$ edition. Princeton University Press. Tables 8.11 and 8.12} of the natural input rate from rivers. Because of the enormous quantity of $Na^+$ and $Cl^-$ in the oceans, changes to oceanic salinity are small and unlikely to have any significant effect to Earth's systems. 
\subsubsection{Temperature}\label{modern_sst}
As we noted in section \ref{water_emp}, water has a high heat capacity, which means water must absorb or emit significant heat in order to change temperature. Sunlight (or air warmed by sunlight) is essentially the only source of heat to the oceans, so as you might expect the ocean surface is warmest at the equator. In the polar regions, where sunlight is weak and the air cold, water loses heat to the atmosphere and so is coldest at the poles. Figure \ref{fig:zonal_sst} shows that our first-order hypothesis is good. Annual average sea surface temperatures fall below \SI{0}{\celsius} poleward of $60 \deg$ latitude (sea water, due to its high salinity, does not typically freeze\footnote{National Snow and Ice Data Center, \href{https://nsidc.org/cryosphere/seaice/characteristics/brine_salinity.html}{This link} accessed Pi day, 2018} until \SI{-1.8}{\celsius} ($26^{\circ}$F)). Temperatures increase uniformly and symmetrically toward the equator, although the small ``dimple'' at the equator hints of interesting details to explore.\\
A global map (Figure \ref{fig:sst_map}, left panel) of sea surface temperatures (universally called ``SST''), shows abundant details to explore. The dominant cold-warm-cold variation from pole to equator to pole is quite evident. The equatorial western Pacific and Indian Oceans together form an enormous ``pool'' of warm water stretching a third the way around the globe. In the eastern Pacific, a thin band of cold water along the equator is the source of the equatorial ``dimple'' seen in Figure \ref{fig:zonal_sst}. Both of these features are the result of circulation of surface waters, which we'll examine in detail below. Other features in the SST map are obscured by the pole-to-pole variation. We will frequently want to visualize and discuss \textbf{differences} between average and actual values, rather than the actual values themselves. In this case, removing the zonal average at every latitude (displayed in Figure \ref{fig:zonal_sst}) from every  point in the SST map gives Figure \ref{fig:sst_map} (right panel). Areas that are colder than average for the their latitude are colored blue, and warmer areas in brown, as shown by the scale. Clearly present is an east-west dichotomy in all the oceans: eastern portions are cooler than the western portions. Although most strongly developed in the Pacific Ocean, the large ``plumes'' of cold SST on the \emph{western}edges of the continents is quite striking in the Americas, Africa and even Australia. As we'll see below, this is due to the upwelling of cold water from the deep ocean to the surface along the eastern edges of the oceans, and then their transport by surface currents toward the equator. The other striking feature on the difference map is in the northwestern Atlantic Ocean, where a narrow stream of relatively hot water forms near Florida, then snakes along the eastern continental edge of North America, finally spreading out into the seas off of Europe. This is the Gulf Stream, and forms an essential parts of one of the most important pacemakers of Earth's climate, the Meridional Overturning Circulation. All of these circulation systems, which clearly cause the movement of warm and cool ocean masses, transport heat from Equator to pole, a process we'll examine in detail below. 
\subsubsection{pH, acidity and saturation states}
The acidity of a solution is measured by the \emph{pH}, a short-hand for the ``power of Hydrogen.'' Neutral solutions (think pure water) have a $pH=7$, acidic solutions (think black coffee) have $pH<7$, while alkaline solutions (think soapy water) have $pH>7$. The pH scale, like so many other tools used in this book, is a logarithmic. A decreases of just 0.1 pH unit indicates a 25\% increase in acidity. The pH of the contemporary shallow ocean varies regionally and seasonally from 7.9 to 8.2 pH units\footnote{Takahashi, T., Sutherland, S. C., Chipman, D. W., Goddard, J. G., Ho, C., Newberger, T., ... \& Munro, D. R. (2014). Climatological distributions of pH, pCO2, total CO2, alkalinity, and CaCO3 saturation in the global surface ocean, and temporal changes at selected locations. Marine Chemistry, 164, 95-125.}, making it slightly alkaline. In general, places with less light (the winter-time pole and the deep ocean) have \emph{lower} pH, because of their lower biological productivity. Actually, we don't much care about the contemporary pH distribution on the oceans (Figure \ref{fig:ph_map}) as much as we care about its change over time. Prior to industrialization, oceans had an average pH of roughly 8.25; contemporary oceans have a pH closer to 8.14, an acidity increase of nearly 30\%. \\
One big, wet sponge: that's what the oceans look like to the atmosphere. About a quarter of all the $CO_2$ humans have pumped in to the atmosphere has been soaked up by the oceans\footnote{ Le Qu'er'e, and 67 co-authors: Global Carbon Budget 2016, Earth Syst. Sci. Data, 8, 605-649, doi:10.5194/essd-8-605-2016, 2016. http://www.earth-syst-sci-data.net/8/605/2016/, accessed 27 March 2017}. Once in the oceans, that $CO_2$ reacts with water in a series of steps to form a weak acid, a process we saw way back in Equations \ref{eq:carbonic}. The increasing acidity of the oceans is just one more side effect of $CO_2$ emissions. But another of the reactions between $CO_2$ and sea water is both more surprising and more disturbing. Coral reefs are the best known examples, but many marine creatures beside those making spectacular reefs build shells from the mineral aragonite{footnote{McCulloch, M. T., D'Olivo, J. P., Falter, J., Holcomb, M., \& Trotter, J. A. (2017). Coral calcification in a changing World and the interactive dynamics of pH and DIC upregulation. Nature communications, 8, 15686.}, $CaCO_3$ These organisms manufacture their skeletons by combining calcium ions and bicarbonate ions (both liberated from the weathering or rocks on the continents, see the \ref{eq:urey} equation in Section \ref{swrwl}) dissolved in the oceans: 
\begin{align} \label{eq:caco3}
	HCO_3^{1-} & + & Ca^{2+} \rightleftharpoons & CaCO_3 + & H_2O + & CO_2\\
	Bicarbonate \ ion & + & Calcium \ ion \rightleftharpoons & Aragonite \ Shells + & Water+ & Carbon \ Dioxide 
\end{align}
All of the materials on the right hand side are found in sea water: that $CO_2$ is gas dissolved in the oceans. Unlike many of the reactions you may have seen in Chemistry or Biology classes, equation \ref{eq:caco3} goes both ways, that is the equation runs both direction, as indicated by the ``double ended harpoon'' symbol ($\rightleftharpoons$) in the equation. As the big wet sponge of the ocean soaks up $CO_2$ from the atmosphere, equation \ref{eq:caco3} runs to the \emph{left}: water and dissolved $CO_2$ spontaneously combine with the shells of marine organisms (living or dead) and return to dissolved calcium and bicarbonate ions. Increasing ocean acidity is decreasing the chemical stability of a vast array of marine creatures, from the spectacular corals of the Great Barrier Reef to the more mundane but equally important aragonite muds formed from broken-down shells. The relative stability of aragonite shells in sea water is given by the ``saturation index,'' which must be greater than 2.9 for aragonite to remain solid and not dissolve back in to sea water. As shown in Figure \label{fig:arag_map}, this condition is already present in some of the colder parts of the oceans; recent research unfortunately suggests that the mundane muds making up much of coral reefs may begin dissolving back into the increasingly acidic oceans by 2050-2080 CE. \\
Across the world, the ocean's pH and aragonate saturation index are decreasing\footnote{Takahashi, et al., op cit.}. The best evidence for these decreases comes from research sites in the oceans repeatedly occupied by ships or buoys. Three of these records are shown in Figure \ref{fig:oa} (Note that the graphs for the Canary Islands and Bermuda are displaced 0.15 and 0.30 units downward for clarity). All three stations show natural annual cycles superimposed on a gradual but obvious long-term decrease. At each of the stations, pH is changing by $-0.02$ pH units per decade, or roughly a 7\% increase in acidity every 10 years. Aragonite saturation states have also decreased, at about 0.08 units per decade.
\subsubsection{Light}
We denizens of Earth's surface take light for granted. Not so for life in the oceans: the vast majority of the oceans are darker than the night where you live. Sea water absorbs sunlight efficiently, the red end of the spectrum more readily than the blue, which is why water ``looks'' blue. On average, light intensity decreases by half for every depth increase of \SIrange{10}{15}{metre}. At \SI{100}{metre}, the total light is less than 0.5\% of that at the surface (equivalent to the brightness of the sky just after sunset\footnote{Kishida, Y. (1989). Changes in light intensity at twilight and estimation of the biological photoperiod. JARQ (Japan).}), and photosynthesis is no longer possible (Figure \ref{light_in_seas}. Below this euphotic layer, the oceans remain in perpetually twilight to a depth of roughly \SI{1000}{metre}, where the last vestiges of light disappear, and eternal darkness reigns supreme. Roughly 65\% of Earth's entire surface area, and 90\%  of the ocean's volume, lies deeper than \SI{1000}{metre} below the ocean surface, and all of it perpetually dark.
$\ $\\

\section{Evolution over geologic time} \label{Oceans of the Past}
\subsection{Introduction}
Like every part of the Earth, the contemporary hydrosphere with which we have such first-hand experience is \emph{not} the same as the hydrosphere of the past. The oceans in particular have evolved in synchrony with the atmosphere. biosphere and geosphere, changes in each driving changes in the others. Our approach in this section of the book is to look at each of the spheres in turn, examining the connections---the flows of energy and mass---between them and how these interactions formed a continuously-habitable climate on Earth. A study more focused on the sequential history of Earth and Earth systems (a field known as Historical Geology) would instead use time as the organizational tool, and examine the happenings in all spheres at each slice of Earth's history. Both approaches are valuable, but our approach has the advantage of explicitly examining the powerful effects of feedbacks between the spheres on shaping climate. The feedbacks involving the oceans, atmosphere and biosphere, show how Earth naturally has stable climate states, states that can change over time.
\subsection{Origins of Earth's water}
Before we tackle the hydropshere's long-term evolution, we need to understand the source of Earth's water, both in time and space. The timing and location of Jupiter's formation was the key factor delivering water to the still-forming Earth. As we saw in Chapter 5, Earth and the other terrestrial planets formed in the warm inner portion of the proto-planetary nebula. Inside the ``snow-line'' gas and ice were largely absent, and the growing planetesimals were bone dry. The terrestrial planets thus formed hot and dry. Beyond the snow-line, at distances beyond 3 to 4 AU from the Sun, water ice was stable, and made up significant proportions of growing planetesimals\footnote{Raymond, S. N., \& Izidoro, A. (2017). Origin of water in the inner Solar System: Planetesimals scattered inward during Jupiter and Saturn's rapid gas accretion. Icarus, 297, 134-148.}. The logical conclusion is that Earth's water is from the outer solar system! Fortunately we can test this hypothesis, and even add some precision to the location. The contemporary asteroid belt still contains remnants of planetesimals formed near Earth (the s-type asteroids) and those formed beyond the snow-line (The c-type asteroids). We have samples of both types thanks to meteorites, so we can minutely study their chemical and isotopic composition. Studying the water alone isn't enough to uniquely identify the source of Earth's water. Both hydrogen and nitrogen have multiple  isotopes, and their ratios provide isotopic ``fingerprints'' unique to the many possible sources of water to Earth. Figure \ref{fig:N_h_isotopes} shows that objects farther from the Sun (comets, some of the asteroids sampled by meteorites) are isotopically distinct from Earth: these distant objects couldn't have contributed any substantial water to the early Earth\footnote{Jessica J. Barnes, David A. Kring, Romain Tart`ese, Ian A. Franchi, Mahesh Anand, et al., An asteroidal origin for water in the Moon. Nature Communications, Nature Publishing Group, 2016, 7, pp.11684. 10.1038/ncomms11684.}. The best isotopic match is with primitive carbon- and water-rich carbonaceous chondrites, whose parent bodies formed beyond the orbit of Jupiter. Go ahead, have a sip of water, and enjoy the taste of the solar system.\\

Those asteroids must have flooded the inner solar system very early in its history (Figure \ref{fig:water_timing}; see this \href{https://www.youtube.com/watch?v=Ji5ZC7CP5to}{this link} for a great video of the process, as all of the terrestrial planets for which we have data share Earth's H and N isotopic composition. The cause of this ``flood'' was the rapid formation of Jupiter and Saturn in the earliest phases of the solar system's formation (Figure \ref{fig:jupiter_source_water}). While the precise timing is still unclear\footnote{Morbidelli, A., \& Wood, B. J. (2015). Late accretion and the late veneer. The Early Earth: Accretion and Differentiation, Geophysical Monograph, 212, 71-82; Sarafian, A. R., Nielsen, S. G., Marschall, H. R., McCubbin, F. M., \& Monteleone, B. D. (2014). Early accretion of water in the inner solar system from a carbonaceous chondrite-like source. Science, 346(6209), 623-626; Barnes et al., op. cit., Raymond \& Andre Izidoroa, op. cit.} the water-rich c-type asteroids clearly arrived in ``our'' part of the solar system prior to the moon-forming impact that completed Earth's accretion. During this early period, Jupiter and Saturn carved rings of dust and gas from the proto-planetary nebula as they grew. As their mass increased, their gravitational muscle nudged the orbits of nearby c-type asteroids, sending some into the asteroid belt (where they still reside) and some into the inner solar system. All of this activity is loosely constrained to within the first 8-20 My after solar system formation started. Earth's water must have been mixed into the Earth while it was still half the mass it is now\footnote{Assuming Theia impact at \num[separate-uncertainty = true]{92(35)}My after CAI, with $0.1M_\Earth$ added at collision, and exponential growth (mean time of accretion = 17 My) to the collision, mass of the proto-Earth is $0.28\pm 0.13$ to $0.54\pm 0.14\  M_\Earth$ at 8 and 20 My, based on simple MC modeling.}. Water on Earth is old and was deeply buried.\\
 
  
\subsection{The earliest oceans}
But not for long. As we saw in Chapter 6, Earth's formation ended with a giant collision that left the upper few hundred kilometers of the Earth molten, covered with a magma ocean. Don't confuse this magma ocean---made of liquid rock at temperatures of \SIrange{1500}{2500}{\kelvin}---with the watery ocean now covering our planet. The magma ocean convected easily, so material from deep below the surface was continuously being brought to the surface\footnote{Ikoma, M., Elkins-Tanton, L., Hamano, K., \& Suckale, J. (2018). Water Partitioning in Planetary Embryos and Protoplanets with Magma Oceans. Space Science Reviews, 214(4), 76.}. Despite its high temperatures and rocky composition, liquid water readily dissolved in the magma, until it reached the surface, where it lost some of this water to the atmosphere. Over a few million years much of the water dissolved in the magma ocean bubbled away into the super-heated steamy atmosphere, hundreds of times heavier than the puny atmosphere of our current Earth. For the few million years the magma ocean was molten at the surface\footnote{Hamano, K., Abe, Y., \& Genda, H. (2013). Emergence of two types of terrestrial planet on solidification of magma ocean. Nature, 497(7451), 607.}, and continually pumped water into the atmosphere. But 4 to 5 Myr after the giant collision, the magma ocean solidified and the atmosphere began to cool. Although there is no direct evidence (yet!) for when this happened, computer models suggest that within a few \emph{hundreds to thousands} years, the atmosphere had cooled enough for the first---ever--rains to fall on the newly solidified Earth\footnote{Ikoma et al., $\S 4.3$}. The hydrosphere was born. These earliest oceans probably contained---within a factor of two---about as much water as the contemporary oceans do.
\subsubsection{How we characterize ancient oceans}
The chronology of Earth's hydrosphere nudges against the very start of the solar system, at least 4.547 billion years ago. But the first direct evidence---evidence that one can hold in one's hand---is younger, \emph{only} 4.374 billions years old, and comes in the form of fragments of even older rocks, deposited on Earth's early crust as sedimentary rocks.
\paragraph{Zircons and inclusions}
Sometimes the littlest packages have the largest surprises. This is certainly true of the zircon crystals found in the Jack Hills area of Western Australia. The rocks there (Figure \ref{fig:jhzirc} are ``only'' 3 Ga old, and consist of sandstones and conglomerates clearly deposited in water on an ancient continent. Within those rocks are grains of zircon, a mineral typically formed in granitic magmas. Zircons are tough little minerals, and can be precisely dated using U-Pb dating methods\footnote{Bowring, S. (2014). Early Earth: Closing the gap. Nature Geoscience, 7(3), 169 presents a brief review.}. Hundreds of thousands of zircons from the Jack Hills have been dated, and the oldest\footnote{Wilde, S. A., Valley, J. W., Peck, W. H., \& Graham, C. M. (2001). Evidence from detrital zircons for the existence of continental crust and oceans on the Earth 4.4 Gyr ago. Nature, 409(6817), 175.} (See Figure \ref{fig:jhzirc}) formed at \num[separate-uncertainty = true]{4.404(8)} Ga, less than 100 My after the Earth-Theia collision! The zircons have another surprise: the oxygen isotopic ratios indicate that the magmas from which they formed clearly interacted with surface waters: an early ocean. While it is no bigger than the width of a human hair, the zircon in Figure \ref{fig:jhzirc} it has planet-wide implications as the oldest direct evidence for Earth's hydrosphere.    
\paragraph{Sedimentary rocks}
Many sedimentary rocks form by the erosion, transport and deposition of rock fragments in oceans, so they too are direct evidence of an active hydrosphere. The oldest sedimentary rocks on Earth are found in and around western Greenland and eastern Canada, and date to ~3.75 Ga. These sedimentary rocks include cherts and banded iron formations formed on the ocean floor soon after the the last dregs of meteorite bombardment had stopped (Figure \ref{fig:bif}). Sedimentary rocks of every age are found since that time, so it is extremely likely that Earth has had stable oceans since then.
%\paragraph{Genomics and paleontology}

\subsection{Archean and Proterozoic hydrosphere}
In an influential essay, Ariel Anbar noted the odd truth of the modern oceans. You'd expect that the vast majority of the surface oceans would thrive with life: basking in sunlight streaming through the clear, blue water; provided with abundant $CO_2$, and slurping up required nutrients dissolved in the water. In fact, most of the tropical oceans are a lifeless desert, because the water lacks trace metals (mostly metals like Fe, Mn, and Co) as well as nitrogen\footnote{Lyons, T. W., Reinhard, C. T., \& Planavsky, N. J. (2014). The rise of oxygen in Earth's early ocean and atmosphere. Nature, 506(7488), 307-315. https://doi.org/10.1038/nature13068}. This is paradoxical: why after some 4 Ga of evolution should life in the oceans be so ill-adapated to life in the oceans! The explanation is that the core machinery of life evolved not in the modern oceans, but in those from 4 Ga, and that ocean was far different chemically from the one we have now. Understanding the evolution of the oceans is essential for understanding how the basic machinery of life evolved as well. Biosphere, hydropshere and geosphere are  intertwined not only today, but through time. Understanding the evolving compositions of the oceans is essential to understanding the evolution of Earth's climate.     
\subsubsection{The Archean: Hot, anoxic and iron-y}
Just about every aspect of the ancient oceans was different than the modern oceans, even such simple features as temperature. We saw in Section \ref{modern_sst} that typical sea surface temperatures in the modern ocean average about \SIrange{15}{20}{\celsius}. Although a fierce debate continues\footnote{Garcia, A. K., Schopf, J. W., Yokobori, S. I., Akanuma, S., \& Yamagishi, A. (2017). Reconstructed ancestral enzymes suggest long-term cooling of Earth's photic zone since the Archean. Proceedings of the National Academy of Sciences, 114(18), 4619-4624 is a recent discussion; see references to Figure \ref{fig:historic_sst} for a more complete list.} on this, SST at 4.0 Ga might have been as hot as \SIrange{55}{85}{\celsius}, or $130$ to $180^\circ F$, hot enough to safely cook chicken! As Figure \ref{fig:historic_sst} illustrates, SSTs have slowly decreased since then, arriving near modern temperatures only in the last 250 Ma. Many scientists disagree with this ``hot ocean'' model\footnote{Kasting, J. F., Howard, M. T., Wallmann, K., Veizer, J., Shields, G., \& Jaffr\'es, J. (2006). Paleoclimates, ocean depth, and the oxygen isotopic composition of seawater. Earth and Planetary Science Letters, 252(1-2), 82-93.}, arguing that ice ages, weathering rates, the dimmer Sun and even the unlikely ability of large animals to live in such hot water all support a cooler ocean for all of its history. Clearly someone is making incorrect assumptions about their science, and this debate will continue as new tools become available. Don't mistake this apparent ignorance of the ``facts'' as a weakness; as Stuart Firestein\footnote{Firestein, S. (2012). Ignorance: How it drives science. Oxford University Press, pg 15-16.} notes, ``Want to be at the cutting edge [of science]? Well, it's all or mostly, ignorance out there. Forget the answers, work on the questions.''\\
All data agrees that the Archean ocean had a substantially different chemistry than today's ocean. Prior to the Great Oxygenation Event at 2.5 Ga, oxygen would have been largely absent from the oceans, except for brief, local episodes\footnote{Kendall, B., Creaser, R. A., Reinhard, C. T., Lyons, T. W., \& Anbar, A. D. (2015). Transient episodes of mild environmental oxygenation and oxidative continental weathering during the late Archean. Science advances, 1(10), e1500777.}. Iron and chemically similar metals (Ni, Co and Mn, for example) readily dissolved into the anoxic ocean as reduced ions like $Fe^{2+}$, so the earliest oceans were 10,000 times more enriched in Fe than we see today. In such ``ferrungous'' or iron-rich environments, it is no wonder that life evolved to use these abundant metals. Life then as now depends upon gaining energy through reduction-oxidation reactions. The oldest of the molecular machines that conduct these reactions all have metal atoms at their core, metals abundant in the Archean---but not contemporary---oceans. While the deep oceans of the Archean were uniformly hot, anoxic and ferrungous, the continental slopes and shelves had their own peculiarity. Isotopic data from shales---mud-rich sedimentary rocks deposited on the shallow and well-lit shelves---indicate that relatively large portions\footnote{Reinhard, C. T., Planavsky, N. J., Robbins, L. J., Partin, C. A., Gill, B. C., Lalonde, S. V., ... \& Lyons, T. W. (2013). Proterozoic ocean redox and biogeochemical stasis. Proceedings of the National Academy of Sciences, 110(14), 5357-5362.} of the shelves may have been anoxic and rich in hydrogen sulfide, $H_2S$, the gas that gives rotten eggs and sewers their distinctive aroma. Such ``euxinic'' areas are exceptionally rare in the modern oceans]footnote{the Black Sea is largest such area in the world today}. Significant changes in ocean chemistry had to wait until oxygenic photosynthesis had established itself in earnest: at the GOE.  
\subsubsection{Proterozoic: Great Oxidation Event and the Boring Billion}
As we saw in Chapter 6, the evolution of photosynthesis\footnote{Planavsky, N. J., Asael, D., Hofmann, A., Reinhard, C. T., Lalonde, S. V., Knudsen, A., ... Rouxel, O. J. (2014). Evidence for oxygenic photosynthesis half a billion years before the Great Oxidation Event. Nature Geoscience, 7(4), 283-286. https://doi.org/10.1038/ngeo2122} at 2.95 Ga began the slow process of oxygenating Earth's surface. The bacteria conducting all this photosynthesis lived principally in the oceans, so one might expect the oxygen released in this photosynthesis would have oxidized the ocean before the atmosphere. But $>99\%$ of all the $O_2$ produced in photosynthesis is consumed by the same organism which produced it. The net result is the slow oxygenation of the well-lit surface layers of the Proterozoic Ocean, but not the deep portions, which remained anoxic and iron-rich (Figure \ref{fig:ocean_comp}). In fact, for the nearly billion year period from the end of the GOE to the NOE, known informally as the ``boring billion,'' represent a period of stasis in the Earth's evolution: geosphere, hydrosphere and biosphere all seemed to find a comfortable ``middle age'' of unchanging complacency. \footnote{amongst others, see Reinhard, C. T., Planavsky, N. J., Olson, S. L., Lyons, T. W., \& Erwin, D. H. (2016). Earth's oxygen cycle and the evolution of animal life. Proceedings of the National Academy of Sciences, 113(32), 8933-8938.} Well, with the exception of complex multicellular life, known collectively as eukaryotes. Subtle but persuasive evidence \footnote{Knoll, A. H., Javaux, E. J., Hewitt, D., \& Cohen, P. (2006). Eukaryotic organisms in Proterozoic oceans. Philosophical Transactions of the Royal Society B: Biological Sciences, 361(1470), 1023-1038; Parfrey, L. W., Lahr, D. J., Knoll, A. H., \& Katz, L. A. (2011). Estimating the timing of early eukaryotic diversification with multigene molecular clocks. Proceedings of the National Academy of Sciences, 108(33), 13624-13629.} all indicate that the branch of life that includes essentially all naked-eye life forms---algae, seaweed, plants, fungi and animals---evolved around 1.8 Ga, and underwent slow evolution during this time. Eukaryotes, big and multi-cellular, demand far more oxygen than the relatively tiny and uni-cellular prokaryote that had dominated the biosphere since life first evolved.    
\paragraph{End Proterozoic changes and birth of a ``modern'' ocean}
That boring period ended around 0.8 Ga, when dramatic changes rippled through Earth's spheres. Figure \ref{fig:ocean_comp} (pane 3) indicates a sudden and long-lasting increase in global ice cover; abrupt diversification of marine eukaryotes (pane 4) and large perturbations to the C cycle (pane 2), all ultimately caused by the reorganization of continents at the time\footnote{Shields-Zhou, G., \& Och, L. (2011). The case for a Neoproterozoic oxygenation event: geochemical evidence and biological consequences. GSA Today, 21(3), 4-11.}. For the first time in its history, Earth would have looked familiar from space, with blue skies, blue oceans,and green life (thin land-dwelling algae sliming the rocks) on land\footnote{Wellman, C. H., \& Strother, P. K. (2015). The terrestrial biota prior to the origin of land plants (embryophytes): a review of the evidence. Palaeontology, 58(4), 601-627.}. In the oceans, the important change was the rapid and inexorable penetration of dissolved oxygen into the deepest oceans. Once the bottom of the ocean was oxygenated, animal life quickly radiated across the world. After 4 billion years of history, all of the major forms of animals---from sponges and worms to chordates, our ancestors---evolved and spread across the world's oceans in the relatively brief Cambrian ``explosion'', from approximately 540 to 522 Ma\footnote{Maloof, A. C., Porter, S. M., Moore, J. L., Dud\'{a}s, F. O., Bowring, S. A., Higgins, J. A., ... \& Eddy, M. P. (2010). The earliest Cambrian record of animals and ocean geochemical change. Bulletin, 122(11-12), 1731-1774.}. A long journey for Earth and reader, but ending in a comfortable place: a recognizable and comfortable Earth.\\

\section{Contemporary Circulation}  \label{Circulation}
\subsection{Introduction: The oceans move heat from the equator to the pole.}
The oceans are in unceasing and tumultuous motion. It's one of the features that make the oceans so mesmerizing to watch, and one of the features that make the oceans most dangerous to live with. But the motions that most captivate human observers---waves and tides come to mind---aren't important to the ocean's long-term climatological roles. More subtle horizontal and vertical circulations in the oceans are what will concern us in this chapter. Like the overlying atmosphere, the fluid oceans are constantly slung about by Earth's rotation and heated by the Sun. Unlike the atmosphere, the oceans are dense, penned in by the continents, and heated from above, so circulation of the oceans is qualitatively different then circulation of the atmosphere. Currents in the ocean are 1000 times slower (\SI{0.01}{\metre\per\second}) than those in the atmosphere (\SI{10}{\metre\per\second}), so the first difference we should expect is the time it takes for the oceans to circulate (years to millennia) will be much longer than the those for the atmosphere (days to months). The second difference is that density changes within the oceans is controlled by temperature and salinity, both of which change only at the surface.\\
Circulations in the oceans are driven by both winds and density. We'll find that winds in the atmosphere drive circulation in the shallow, warm oceans above the thermocline. These \textbf{gyres} move water horizontally in ocean-wide circulation cells. Once dragged to polar regions by circulation in gyres, water cools, becomes saltier and so denser. This dense water sinks to abyssal depths, forming the \textbf{meridional overturning circulation}. The MOC is a vertical convective circulation of the oceans, and the primary way oceans move heat from the equator to the poles. A full trip around the MOC takes thousands of years, most it spent in the dark, isolated depths of the deepest parts of the oceans. We'll examine each of these circulations in sections \ref{gyres} and \ref{MOCs}.\\
Both gyres and the MOC are global circulations, with examples found in every ocean basin. In section \ref{ENSO} we'll examine the \textbf{El Ni\~no-Southern Oscillation} (ENSO), a unique and culturally-important circulation in the tropical Pacific Ocean and atmosphere. The ENSO is a quasi-periodic but natural variation in Pacific Ocean sea surface temperature and height, accompanied by reversals in the circulation of the overlying atmosphere. While the ENSO itself is limited to the tropical Pacific Ocean, changes there drive changes in weather across the world, often leading to substantial storms and billions of dollars of damage in the United States, Australia and South America. A particularly strong El Ni\~no in 1982-3 caused over 1000 deaths and \$13 billion in damages world-wide\footnote{whoi, see email}, and brought wide-spread attention to the phenomenon for the first time. We'll examine the ENSO in section \ref{ENSO}.      
\subsection{Gyres}\label{gyres}
Tiny Henderson Island is about as far as you can get from other humans. About halfway between Chile and New Zealand (its location is shown by the star in Figure \ref{fig:gyre}) is \SI{5000}{\kilo\metre} (3000 miles) from any substantial human habitation\footnote{Lavers, J. L., \& Bond, A. L. (2017). Exceptional and rapid accumulation of anthropogenic debris on one of the world's most remote and pristine islands. Proceedings of the National Academy of Sciences, 114(23), 6052. \href{https://doi.org/10.1073/pnas.1619818114}{Link}}. Despite this remoteness, the beaches of this uninhabited island are covered with plastic waste (Figure \ref{fig:henderson}, so dense one would think there had been a boisterous beach party the day before. All of the plastic waste in Figure \ref{fig:henderson} was bourne to the island by currents of the South Pacific gyre, which carries debris from as far away as Japan, Chile, and China\footnote{Lavers and Bond, op cit.}. Less visible is that gyres are the dominant circulation of the shallow oceans; we need to understand their geometry and the forces that drive them.\\
As you can see in Figure \ref{fig:gyre}, gyres are a series of currents forming elliptical circulations across each ocean basin. Figure \ref{fig:gyre_schematic} shows a schematic view of the layout of gyres. Northern Hemisphere oceans, penned in by continents to high latitudes, all have two sets of gyres, a clockwise circulating sub-tropical gyre and a counter-clockwise gyre centered (except in the Indian Ocean) at sub-polar latitudes. Across the equator, circulation reverses so sub-tropical gyres circulate counter-clockwise. The Southern Ocean is not hemmed by continents, so the Antarctic Circumpolar Current surrounds Antarctica, effectively shutting off the continent from the rest of the world's oceans. The map colors in Figure \ref{fig:gyre} show that the western, poleward moving portions of the sub-tropical gyres are narrow, fast moving streams of water coursing through the oceans; these \emph{western boundary currents} are the primary way oceans move heat from the equator toward the poles. The arrow colors in Figure \ref{fig:gyre} show the relative temperature of each current. The odd east-west dichotomy of SSTs from Figure \ref{fig:sst_anom_map} is clearly the effect of the warm and cool currents in the gyres.\\

\subsubsection{Driving Forces}
Winds alone are not enough to drive the shallow circulations seen in gyres. As always, we need to remember that all this motion occurs on a spinning Earth, and so Coriolis effects will sooner or later become important in explaining the motion. In the northern hemisphere, the Coriolis effect deflects any fluid motion to the right, and the same will be true for the motion of the oceans driven by winds. This was first noted by the famous Norwegian oceanographer, adventurer and Nobel peace prize-winning Fridtjof Nansen. In 1893 CE, he and his crew of 11 purposely locked themselves in Arctic Sea for nearly 2 years in an attempt to get to the North Pole. During this time, Nansen npticed that the ice would drift not with th ewind, but to the right. After his return, he mentioned this observation to Vagn Ekman, then a young student in graduate school. Ekman later set down and in a single night of remarkable creativity derived the complicated mathematics explaining Nansen's observation. Ekman found that Earth's rotation causes the ocean's surface to move \SIrange{30}{60}{\celsius} to the \emph{right} of the wind direction in the Northern Hemisphere (and a similar angel to the \emph{left} in the Southern Hemisphere). But this upper layer of the ocean drags the next layer down, which also then moves \SIrange{30}{60}{\celsius} to the \emph{right} of the force acting on it (Figure \ref{fig:ekman}. This pattern repeats, with each layer moving to the right (or left, in the Southern Hemisphere) of the one above: the lowest layers might be moving opposite to the wind's direction! Friction and turbulence between all these moving layers reduces the current velocity with depth, so this ``Ekman spiral'' is rarely more than \SIrange{10}{100}{\metre} deep. The result is that water moves at at right angles to the wind.\\
We can use the North Atlantic as an example. In the sub-tropics, the north-east trades induce circulation to the west and north, forming the North Equatorial Current. Trapped by winds and the eastern boundary of Norht America, the current pushes into the Florida Straits as the Gulf Stream, after which the prevailing westerlies of the temperate belt take over, forming the North Atlantic Current. The net motion is clockwise, and as a result the Ekman flow slowly pumps water into the center of the gyre, forming a low, wide dome in the sea surface. At the center of the dome, sea water slowly (\emph{very} slowly, on the order of just \SI{1}{\metre} per week) pumping warm salty water a few hundred meters deep.\\ At the equator and in the sub-polar gyres, this pump runs in reverse, sucking colder, fresher water from depth and drawing to the surface. The gyres, driven by the Ekman pump and fueled by global winds, determine the large scale shallow circualtion of the oceans.
\paragraph{Winds}
\paragraph{Coriolis}
\paragraph{Time scales}
\paragraph{Role as plastic waste collectors}
\subparagraph{SEA results and floating trash rafts}
\subparagraph{Henderson Island \href{http://www.pnas.org/content/114/23/6052.short?ssource=mfr&rss=1}{PNAS}}


\subsection{Meridional Overturning Circulation}\label{MOCs}
Like the atmosphere, the vertical circulation of the atmosphere happens through convection. In the atmosphere, air warmed by Earth's surface becomes less dense, and rises, and forms the great convective systems---the Hadley and Ferrel  Cells we saw in Chapter 6. In the oceans, a different process drives large-scale vertical convection known as the Meridional Overturning Circulation, or MOC. Warm ocean water at the poles looses heat and gains salt from evaporation and formation of ice. This colder, saltier and denser water sinks deeply, circulates through the ocean depths, and then returns to the surface and eventually to the poles. The MOC plays two roles of interest to us\footnote{Buckley, M. W., \& Marshall, J. (2016). Observations, inferences, and mechanisms of the Atlantic Meridional Overturning Circulation: A review. Reviews of Geophysics, 54(1), 5-63. \href{https://doi.org/10.1002/2015RG000493}{Link}
}. One is that it moves heat from the southern hemisphere to the northern, cooling the southern polar areas and warming the northern polar areas. The MOC are also important for the role they play in moderating the effects of climate change, by ``burying'' heat deep within the oceans. Fully 90\% of all the excess energy Earth has stored since we started affecting the climate is stored in the deep oceans by the MOC. Besides melting of the cryosphere, the formation and circulation of the MOC may be the most sensitive parts of the climate to human alteration.    
\subsubsection{Connection to NA sub-tropical and sub-polar gyres}
MOCs are driven by the formation of cold, salty, dense water. The conditions for this to happen---warm water underlying intensely cold air---are most likely where warm western boundary currents meet sub-polar gyres, as shown by figure \ref{fig:dwf}. While many areas around Antarctica contribute to MOC formation, only a small area of the Atlantic has the correct combination of temperature, depth and salinity to start the formation of dense water\footnote{Burls, N. J., Fedorov, A. V., Sigman, D. M., Jaccard, S. L., Tiedemann, R., \& Haug, G. H. (2017). Active Pacific meridional overturning circulation (PMOC) during the warm Pliocene. Science Advances, 3(9), e1700156. https://doi.org/10.1126/sciadv.1700156}. In all these areas, warm surface water cools by evaporation into the overlying colder air (Figure \ref{fig:moc_xssection}, red upward-pointing arrows). This cools the water and increases its salinity, both of which make the water denser. Formation of sea ice in the winter also makes the remaining water denser. This dense water then sinks rapidly toward the bottom, as shown schematically in Figure \ref{fig:moc_formation} by the downward pointing loops in the polar areas. The dense water sinks until it reaches water of the same density, typically a few \SI{1000}{\metre} below the surface. These newly formed ``deep water'' masses tend to maintain their temperature, salinity and hence density for centuries. The combinations are characteristic, and waters formed in different locations have names and can be tracked across the globe. Two of these deep waters are particularly important for our story: North Atlantic Deep Water is made in the Greenland and Labrador Seas, while Antarctic Bottom Water is made in the Ross and Weddell Seas (Figure \ref{fig:dwf}). Each of these deep waters sink and then flow equator-ward at depths, crossing the equator and then (Figure \ref{fig:moc_xssection}) rising to the surface again at about \SIrange{50}{60}{\celsius} S latitude.\\
This upwelling is a key piece of the MOC circulation---after all what goes down must eventually come back up. As we saw in Chapter 6, the prevailing winds of the Southern Ocean are westerlies, and \footnote{NOAA. What are the Roaring Forties? National Ocean Service website, 6/25/18,  \href{https://oceanservice.noaa.gov/facts/roaring-forties.html}{Link} accessed July 3, 2018} the gale-force winds between \SIrange{50}{60}{\celsius} S are known as the ``furious fifties!'' As Figure \ref{fig:moc_xssection} illustrates, these east-moving winds cause (via the Ekman current) surface currents to move northward, drawing both the NADW and the AABW toward the surface. Once there, the cold but nutrient rich waters exchange gasses with the atmosphere, and continue their convective motion back to their formation areas. \\

Along the way the former NADW waters warm and eventually join the Gulf Stream, transporting heat from the southern hemisphere and depositing it the sub-polar areas of the North Atlantic. The MOC caries about 30\% of all the heat passing \SI{30}{\celsius} N, with the rest carried by the atmosphere. This warms the northern and cools the southern hemisphere sub-polar areas---ocean and atmosphere---by\footnote{Buckley, M. W., \& Marshall, J. (2016). Observations, inferences, and mechanisms of the Atlantic Meridional Overturning Circulation: Reviews of Geophysics, 54(1), 5-63. https://doi.org/10.1002/2015RG000493, Figure 3c} \SIrange{1}{3}{\celsius}. Because the Pacific Ocean doesn't contribute to the MOC, the polar Atlantic ocean is also warmer than the Pacific Ocean, again by about \SI{3}{\celsius}. All of this combines to make western Europe warmer than it otherwise would be, although the exact degree of this warming due to MOC is vigorously debated\footnote{see, for example, Seager, R., Battisti, D. S., Yin, J., Gordon, N., Naik, N., Clement, A. C., \& Cane, M. A. (2002). Is the Gulf Stream responsible for Europe's mild winters? Quarterly Journal of the Royal Meteorological Society, 128(586), 2563-2586. https://doi.org/10.1256/qj.01.128}.  \\

The complete trip around the MOC can take centuries, which means over these periods the MOC is the primary mixer of the global oceans, bring water from the North Atlantic to the Southern Ocean and back again. The MOCs---particularly the Atlantic branch---thus controls how fast and how much anthropogenic heat and GHG will be mixed into the oceans. The atmosphere has so little mass that it warms quickly when exposed to the extra energy trapped by anthropogenic GHGs. The land's surface is much more massive than the atmosphere, but the land doesn't circulate, so a relative thin portion of Earth's surface warms rapidly when exposed to extra heat. The oceans are both massive \emph{and} circulating, so they can store extensive heat throughout their depth, \emph{if} the Atlantic MOC is operating well and mixing the oceans. Since 1960 CE, about 90\% of all the solar heat trapped by man-made GHGs has been trapped in the oceans, and the deep oceans share of that is increasing rapidly. On the short term---less than a human lifetime---that's good news. The oceans have on average only warmed by a tiny \SI{0.06}{\celsius} in the past 60 years\footnote{Cheng, L., Trenberth, K. E., Fasullo, J., Boyer, T., Abraham, J., \& Zhu, J. (2017). Improved estimates of ocean heat content from 1960 to 2015. Science Advances, 3(3), e1601545. https://doi.org/10.1126/sciadv.1601545. They have total ocean heat gain since 1960 CE as $\Delta Q_{1960}=\SI{33.5e22}{\joule}$; given the mass of the oceans $M_o=\SI{1.3e21}{\kilogram}$ and the heat capacity of sea water $C_{p, sw}=\SI{3.985e3}{\joule\kilogram\per\kelvin}$, $\Delta T=\frac{\Delta Q}{M_o\cdot C_{p, sw}}$} This tiny change is an average, but it makes the point that the massive oceans can store a great deal of heat without changing temperature. Had all that heat been left in the atmosphere, we'd have seen nearly 10 times more warming than actually occurred, for a total of \SI{8}{\celsius}, or $16\cdot F$. As Buckley and Marshall\footnote{op. cit} note, the MOC is now ``essentially setting the pace of global warming well into the 22nd century'' But on long terms, the increasing temperature of the oceans threatens to slow down the MOC, and this could have significant global consequences. The MOC's global power is tied directly to the formation of NADW in the Labrador and Greenland Seas, right where the two North Atlantic gyres meet, at the southern apex of Greenland.\\
\subsubsection{Climate change and slowing the MOC}
Most models and some recent observations\footnote{Moore, G. W. K., Vage, K., Pickart, R., \& Renfrew, I. (2015). Open-ocean convection becoming less intense in the Greenland and Iceland Seas. Nature Climate Change, 5, 877-882; Smeed, D. A., McCarthy, G., Cunningham, S. A., Frajka-Williams, E., Rayner, D., Johns, W. E., ... \& Bryden, H. L. (2014). Observed decline of the Atlantic meridional overturning circulation 2004-2012. Ocean Science, 10(1), 29-38; Rahmstorf, S., Box, J. E., Feulner, G., Mann, M. E., Robinson, A., Rutherford, S., \& Schaffernicht, E. J. (2015). Exceptional twentieth-century slowdown in Atlantic Ocean overturning circulation. Nature climate change, 5(5), 475.} all suggest that the Atlantic branch of the MOC will weaken as the world warms. This surprising result is due to the sensitivity of deep water formation in the Labrador and Greenland Seas. As air and sea surface temperatures increase, surface waters in these areas cool less, and hence become less dense. Also, all that warm air and water increases melting of ice on Greenland\footnote{Box, J. E., \& Colgan, W. (2013). Greenland ice sheet mass balance reconstruction. Part III: Marine ice loss and total mass balance (1840-2010). Journal of Climate, 26(18), 6990-7002; Watson, C. S., White, N. J., Church, J. A., King, M. A., Burgette, R. J., \& Legresy, B. (2015). Unabated global mean sea-level rise over the satellite altimeter era. Nature Climate Change, 5(6), 565.}, leading to increased fluxes of freshwater into the Seas. All this fresher, warmer water tends to stay fresher and warmer for longer than was (until recently) usual, so it sinks less readily, chocking off the very beginning of MOC formation. This change could have planet-wide implications, including\footnote{Vellinga, M., \& Wood, R. A. (2002). Global climatic impacts of a collapse of the Atlantic thermohaline circulation. Climatic change, 54(3), 251-267.} \emph{cooling} western Europe by \SI{4}{\celsius}, increasing sea ice in the Atlantic Ocean\footnote{Cheng, W., Bitz, C. M., \& Chiang, J. C. (2013). Adjustment of the global climate to an abrupt slowdown of the Atlantic meridional overturning circulation, pg. 295-, in Ocean Circulation: Mechanisms and Impacts-Past and Future Changes of Meridional Overturning, eds. Andreas Schmittner, John C. H. Chiang, \& Sidney R. Hemming, Volume 173 Geophysical Monograph Series}, moving\footnote{Timmermann, A., Okumura, Y., An, S. I., Clement, A., Dong, B., Guilyardi, E., ... \& Stouffer, R. J. (2007). The influence of a weakening of the Atlantic meridional overturning circulation on ENSO. Journal of climate, 20(19), 4899-4919.} the ITCZ southward in the Atlantic (and hence changing rainfall patterns in the tropics) and even affecting atmospheric circulation on the Pacific Ocean\footnote{ibid.}, making El Ni\`nos more likely (see next section). Some of these changes---reduced temperatures and increased sea ice in particualr---would reduce the impacts of climate change in at least some areas of the globe. Are these ``good,'' in the ethical sense? Are the other changes? whether they are or not, they will be coming sooner than most other changes associated with our experiment with Earth's climate, as they will happen in our lifetime. 
\subsection{ENSO}\label{ENSO}
\subsubsection{Introduction}
The two broad categories of ocean currents we just examined (Section \ref{gyres} and Section\ref{MOCs}) clearly showed that ocean currents are driven by winds at the ocean surface. The influence was one way: winds blow, water flows. Now we turn our attention to the El Ni\~no-Southern Oscillation (ENSO), a coupled see-saw motion of ocean and atmosphere in the tropical Pacific Ocean. The ENSO is the largest (natural) source of climatic variability over 2-10 year periods\footnote{McPhaden, M. J., Zebiak, S. E., \& Glantz, M. H. (2006). ENSO as an Integrating Concept in Earth Science. Science, 314(5806), 1740. \href{https://doi.org/10.1126/science.1132588}{link}}, and even affects the global atmospheric concentration of $CO_2$. Anomalously warm sea surface temperatures in El Ni\~no periods and cold anomalies during La Ni\~na episodes are accompanied by whole-scale reorganizations of atmospheric pressure and winds in the Pacific and weather across the globe. ENSO cycles typically develop in April to June, reach their full strength between October-February, and then fade away, replaced by normal conditions\footnote{For example, the IRI \href{http://iri.columbia.edu/our-expertise/climate/forecasts/enso/current/}{Link} accessed 05 April 2018, and Clarke, A. J. (2014). El Ni\~no physics and El Ni\~no predictability. Annual review of marine science, 6, 79-99.}, until the next event begins in 3 years or so.\\
The changes to winds and SST during an ENSO are of fundamental importance to  our planet's climate, and it is all-too-easy to focus one's attention of just these patterns. Equally important for us though is understanding how tightly coupled feedbacks between sea surface temperature, atmospheric pressure and the resulting winds in the equatorial Pacific Ocean combine to produce an ENSO. To understand this, we'll first look at ``normal'' conditions in the equatorial Pacific Ocean, and then use this to derive the forces behind the ENSO cycle.

\subsubsection{Pacific SSTs}
Figure \ref{fig:enso_norm} shows average sea surface temperatures in the Pacific Ocean, and it's hard to miss the enormous pool of warm water (greater than \SI{27}{\celsius}, shown with a light-pink color) in the western Pacific basin. On the other edge of the basin, cool deep water upwells along the coast of South America (shown in blues and greens). This water is rich in nutrients and $CO_2$, and once it reaches the surface it fuels remarkable biologic productivity, beginning with phytoplankton and ending (until recently) with consumers such as anchovy and other fish. This relatively cold water is dragged by the South Equatorial Current (see Section \ref{gyres}) to the west; the SEC is in turn largely driven by south-easterly trade winds (the large open arrows on Figure \ref{fig:enso_norm}.) As it travels across the Pacific, the water warms, and eventually trade winds pile the warm water against the western edge of the Pacific. The Western Pacific Warm Pool, propped up by trade winds and heated by intense sunlight, sits \SI{30}{\centi\metre} higher than its cooler counterparts in South America.  
\subsubsection{Walker Circulation}
Water in the warm pool readily evaporates into the overlying atmosphere. Humid air is less dense than dry air (Chapter 6), so it rises, cools and forms towering thunderstorms (Figure \ref{fig:wpwp_tstorms}) as part of a regional low pressure system (Figure \ref{fig:Walker}, lower panel.  Off at the east edge of the Pacific the relatively cold waters induce sinking in the overlying atmosphere, forming a regional high pressure system with clear skies and little rain. At the Equator the Coriolis effect is vanishingly small, so winds flow from the high to the low pressure areas, augmenting the trade winds and reinforcing the formation of the Western Pacific Warm Pool. This Walker Cell completes a circuit high in the atmosphere, connecting the low level westward flow with a high level eastward flow, as shown in Figure \ref{fig:Walker}, lower panel. This is a classic positive feedback system, with trade winds establishing the Western Pacific Warm Pool, which then initiates circulation in the atmosphere, which then strengthens the trade winds, amplifying the Warm Pool. This elegant description was first made by Bjerknes in 1969 CE\footnote{Atmospheric teleconnections from the equatorial Pacific. Mon Weather Rev 97:163-172}, only 14 years before the term \textit{El Ni\~no} became a household term, and the ``Bjerknes Feedback'' is named in his honor.
\subsubsection{The ENSO cycle} \label{enso_cycle}
But positive feedbacks work both ways: whether a signal grows stronger or weaker, a positive feedback amplifies the change. So if---for whatever reason---trade winds weaken for a period, the Western Pacific Warm Pool will start moving westward, dragging the associated low with it. As the east-west difference in SST grows smaller, so too does the Walker circulation, weakening winds and allowing the warm water to slosh eastward. In a classic El Ni\~no, the warm water pool surges all the way to South America, forcing the thermocline to sink in the eastern Pacific and stifling the upwelling of nutrient-rich waters there. Rain follows the warm pool, and moves from the Western Pacific to eastward of the International Date Line. The Walker Cell effectively reverses, and a new temporarily stable pattern---the El Ni\~no---is born. A typical cycle starts in April, reaches a peak in the (southern hemisphere) summer from December to February, and fades a year latter. This is all nicely shown by Figure \ref{fig:oni}, which shows the departures of central Pacific SST from the long-term average (The four circles on the graph indicate the times shown in Figure \ref{fig:enso_stages}). El Ni\~nos---reflecting higher SST---are shown as red spikes, with taller, broader spikes indicating stronger and longer-lasting episodes. An El Ni\~no is defined by a departure of $>$\SI{0.5}{\celsius} from the norm, with increasingly stronger episodes at \SIlist{1.0; 1.5; 2.0}{\celsius}. Take the time to count the number of El Ni\~no during the 1950-2020 period. Depending on how you count a few close calls, you should find 23 events in $2018-2050=68$ years. So the average period is
\begin{equation}
	Period=\frac{68\ years}{23\ events}\approx 3\ years/event
\end{equation}
Figure \ref{fig:oni} also shows that most El Ni\~nos are followed by La Ni\~nas, episodes of colder-than-normal water in the mid-Pacific (Figure \ref{fig:enso_stages}, bottom panel). Although just as frequent as their warm counter=parts, La Ni\~nas are generally weaker and less globally-important than El Ni\~no's they follow. They physics behind this is still not well-understood\footnote{McPhaden, et al., op. cit.} An excellent way to conceive of all this motion is to imagine the Warm Pool as the crest of a warm, high wave at the ocean's surface. Once released by weakening trade winds, this wave propagates over a few month period across the Pacific Ocean, reflects off the western coast of South America, then travels back across the Pacific\footnote{Wang, C., and J. Picaut, (2004) Understanding ENSO physics - A review, in Earth's Climate: The Ocean-Atmosphere Interaction, Geophysical Monograph Series, Volume 147, edited by C. Wang, S.-P. Xie, and J. A. Carton, pp. 21-48, AGU, Washington, D. C.}. Waves have both an crest and a trough; the trough following the warm pool is cool, and produces the La Nin\~na, which slowly dissipates as it reaches the mid-Pacific. As an La Ni\~na episode ends, the Pacific is back to a normal condition, awaiting the beginning of the next episode.\\

\begin{table} 
\centering
\caption{El Ni\~nos and La Ni\~na's since 1950}
\label{tab:enso_occurences}
%\begin{minipage}
\begin{tabular}{@{}cccccccc@{}} \toprule
\multicolumn{4}{c}{El Ni\~nos} &\multicolumn{4}{c}{La Ni\~nas}\\ \cmidrule(r){1-4} \cmidrule(r){5-8}%\footnote{The data and arrangement for this table was provided by Jan Null at \href{http://ggweather.com/enso/oni.htm}{ggweather link}
Weak		&Moderate	&Strong		&Very Strong	&Weak			&Moderate	&Strong		&Very Strong\\ \midrule
1954-55	&1955-56	&1973-74	&							&1952-53	&1951-52	&1957-58	&1982-83\\
1964-65	&1970-71	&1975-76	&							&1953-54	&1963-64	&1965-66	&1997-98\\
1971-72	&1995-96	&1988-89	&							&1958-59	&1968-69	&1972-73	&2015-16\\
1974-75	&2011-12	&1998-99	&							&1969-70	&1986-87	&1987-88\\
1983-84	&					&1999-00	&							&1976-77	&1994-95	&1991-92\\
1984-85	&					&2007-08	&							&1977-78	&2002-03	\\
2000-01	&					&2010-11	&							&1979-80	&2009-10	\\
2005-06	&					&					&							&2004-05	\\
2008-09	&					&					&							&2006-07\\
2016-17	&					&					&							&2014-15\\
2017-18\\ \bottomrule
\end{tabular}
%\end{minipage}
\end{table}


The obvious question---why do the trade winds weaken at the start an ENSO cycle---is still unsolved!\footnote{Thual, S., Majda, A. J., Chen, N., \& Stechmann, S. N. (2016). Simple stochastic model for El Ni\~no with westerly wind bursts. Proceedings of the National Academy of Sciences, 113(37), 10245. https://doi.org/10.1073/pnas.1612002113}. The fact that no two ENSO cycles are identical---varying in strength, duration, period and location---suggests that relatively small and random fluctuations can occasional combine to initiate an ENSO cycle. This variability also means that forecasting an El Ni\~no is devilishly difficult more than a few months before its peak\footnote{Ludescher, J., Gozolchiani, A., Bogachev, M. I., Bunde, A., Havlin, S., \& Schellnhuber, H. J. (2013). Improved El Ni\~no forecasting by cooperativity detection. Proceedings of the National Academy of Sciences, 110(29), 11742. \href{https://doi.org/10.1073/pnas.1309353110}{Link}; see their following paper, Ludescher, J., Gozolchiani, A., Bogachev, M. I., Bunde, A., Havlin, S., \& Schellnhuber, H. J. (2014). Very early warning of next El Ni\~no. Proceedings of the National Academy of Sciences, 111(6), 2064. \href{https://doi.org/10.1073/pnas.1323058111}{link} for a spectacular example of non-predictability!}. This is unfortunate, as strong El Ni\~nos wreck havoc across the globe, and some fore-warning would be welcome. 
\subsubsection{Teleconnections and ENSO's global reach}
Just from looking at the changes in rain fall accompanying an El Ni\~no (Figure \ref{fig:Walker} you can probably guess (correctly) that Indonesia and Australia tend to suffer severe drought at these times, while the island nations of the central Pacific and counties of western South America suffer flooding\footnote{McPhaden, et al., op cit.}. These local effects are sever but hardly surprising. But ENSO affects global weather through \textbf{teleconnections}, the propagation of atmospheric changes over long distances. The affect is much like that involving multiple people bouncing on the same trampoline. If one person changes the location and strength of their jumping, they'll change the patterns of all those around them.\\
A strong El Nin\~no reorganizes atmospheric circulation across the entire Pacific basin. Importantly for those of us in North America, the mid-latitude jet stream moves southward during the Northern Hemisphere winter\footnote{Climate Prediction Center Internet Team, El Nin\~no-Related Changes in Atmospheric Circulation in the Tropics and Mid-latitudes (2005) \href{http://www.cpc.ncep.noaa.gov/products/analysis_monitoring/ensocycle/enso_circ.shtml}{Link}, accessed Summer Solstice, 2018}. This southward swing to the Jet Stream brings the strong storm systems typical of the Alaskan Gulf to the shores of California. It was these storms that led the Los Angeles Times to publish over 1000 articles on El Ni\~no during the 1997-1998 episode! That year flooding caused landslides, floods (contributing to the deaths oof over 60 people in California alone) and other damages of more than \$1 billion. Flooding and heavy rains propagated across the sothern United States that year, including Texas (eight dead) and Florida. As Figure \ref{fig:enso_affects}, this pattern of changes is typical, with increased precipitation along the entire southern tier of the U.S. The polar jet stream in North America move northward at the same time, allowing warmer and drier air to invade the northwestern US and SW Canada. The winter of 1997-1998 was so warm in the northern U.S. that fatalities due to cold fell by 98\% that year. Partly because it was well-predicted, well-covered, and well-planned for, the 1997-1998 El Ni\~no actually brought net economic benefits and saved lives in the U.S. Other global consequences of ENSO variations are given in Table \ref{tab:enso_consequences}. \\

\begin{sidewaystable}
\centering
\caption{Global effects of El Ni\~nos}
\label{tab:enso_effects}
\begin{tabular}{@{}lp{15cm}l@{}} \toprule
Feature									&Description	&Effected Area\\ \midrule
Atlantic Hurricanes			&Fewer and lower intensity storms	& North Atlantic Ocean\\
Pacific Hurricanes			&More intense and more frequent typhoons (aka hurricanes) 	&Central Pacific\\
Carbon Cycle						&Reduced oceanic upwelling leads to lower $[CO_2]_{atm}$, but latter terrestrial effects lead to increased $[CO_2]_{atm}$, both by $\pm ~1ppm$.	& Global \\ 
Coral Reefs							&Bleaching and local die-off of corals	&\\  
Marine Animals					&Increased mortality and decreased reproduction	&West coast of South America\\
Human Life							&Global deaths can exceed 10,000, as in 1997/1998 &Global \\
Economic Impact					&Global economic damage can exceed \$35 billion US, as in 1997/1998	&Global \\
Agriculture							&Flooding and drought cause local carnage, globally El Ni\~no probably increases some crop yields (like soybeans) but decrease yields of others (corn, rice and wheat)	&Global \\ \bottomrule
\end{tabular}
\end{sidewaystable}



\subsubsection{History and future of ENSO}
The ENSO is not a by=product of global climate change, nor is there evidence that ENSO will change significantly as Earth continues to warm. Since the beginning of the Holocene ($\approx 11,700$ ya) the ENSO cycle has slowly strengthened\footnote{Liu, Z., Lu, Z., Wen, X., Otto-Bliesner, B. L., Timmermann, A., \& Cobb, K. M. (2014). Evolution and forcing mechanisms of El Ni~no over the past 21,000 years. Nature, 515(7528), 550-553. \href{https://doi.org/10.1038/nature13963}{Link}} by $\approx 15\%$, with the ``modern'' ENSO first appearing  \SIrange{4500}{3000}{\year} ago\footnote{Carre, M., Sachs, J. P., Purca, S., Schauer, A. J., Braconnot, P., Falcon, R. A., ... Lavallee, D. (2014). Holocene history of ENSO variance and asymmetry in the eastern tropical Pacific. Science, 345(6200), 1045-1048. \href{https://doi.org/10.1126/science.1252220}{Link}}. So ENSO is, at least by human scales, an ancient phenomenon, and clearly pre-dates even the beginnings of human's climate modifications. And while the warming oceans will bring a host of changes to the planets, it is still not clear if a changing ENSO pattern is one of them. Figure \ref{fig:oni} shows that the strength of the strongest EL Ni\~nos has increased linearly since 1950 CE, this is just one of many ways to measure the intensity of the ENSO, and the data set is still too brief to see a strong climate-change signal in the noisy natural variations of ENSO. Studies using digital climate models have both found\footnote{for example, Cai, W., Santoso, A., Wang, G., Yeh, S. W., An, S. I., Cobb, K. M., ... \& Lengaigne, M. (2015). ENSO and greenhouse warming. Nature Climate Change, 5(9), 849, and references therein} and not found\footnote{Collins, M., An, S. I., Cai, W., Ganachaud, A., Guilyardi, E., Jin, F. F., ... \& Vecchi, G. (2010). The impact of global warming on the tropical Pacific Ocean and El Ni\~no. Nature Geoscience, 3(6), 391.} evidence for changes in ENSO in a globally-warmed world. This uncertainty is the hallmark of the very-best aspects of science: a vigorous debate and continual improvement in hypotheses, models and data eventually leading to greater understanding. As Stuart Firestein\footnote{Firestein, S. (2012). Ignorance: How it drives science. New York, Oxford University Press, pg. 56} notes, this ignorance of important ideas ``is both the beginning of the scientific process[, ] and the result.'' 

\section{Recent changes to the oceans} \label{Climate change in the Oceans}
[[placeholder]]
\subsection{Warming}
[[placeholder]]
\subsection{Sea Level Rise}
[[placeholder]]
\subsection{Acidification}
[[placeholder]]
\subsection{Decreases in SST in NW Atlantic/MOC slowdown}
[[placeholder]]

\section{Exercises} \label{Hydrosphere_Exercises}

[[placeholder]]
 

%\begin{table} 
%\caption{Types of Resources}
%\label{tab:goods}
%\centering
%\begin{tabular}{@{}lll@{}} \toprule
%Extent & \multicolumn{2}{c}{Access}\\ \cmidrule(r){2-3} 
 %&\emph{Excludable} & \emph{Non-excludable}\\
 %\emph{Finite} & Private & Common Pool\\
 %\emph{Non-finite} & Club    & Public\\ \bottomrule
%\end{tabular}
%\end{table}
%
\section{Figures} \label{Hydrosphere_Figures}

\newpage
\begin{figure}[p]
\centering
  \includegraphics[width=5.5 in]{3_phase_water}%
\caption{Water is a unique---some might say miraculous---substance. Water naturally occurs in three phases at Earth's surface: liquid (as shown by the ocean), solid (as shown by the floating sea ice) and gas (as shown by the ``sea smoke,'' or fog, wisping through the early morning air in Helsinki harbor. (The air temperature at the time was a frosty $\SI{-33}{\degreeCelsius}$ [$-28^{\circ} F$]!) This behavior, coupled with water's unique chemical features, makes water a fundamental force affecting Earth's evolution. I do NOT yet have permission to use this photograph, arcived at \href{http://inktank.fi/these-surreal-photo-show-what-its-like-to-sail-through-the-sea-at-minus-28-degrees/}{this link}.}
\label{fig:3_phase_water}
\end{figure}

\newpage
\begin{figure}[p]
\centering
  \includegraphics[width=6.5 in]{Water_weird}%
\caption{Water is just weird, and weirdness not shared by substances chemically similar to water. The upper panel shows the melting and boiling points of water ($H_2O$) and the four analogs to water made with elements in the same period as $O$: sulfur ($S$), selenium ($Se$), tellurium ($Te$), and polonium ($Po$), plotted by the weight of each molecule. Predicting the melting and boiling points of water based on these analogs is straightforward: just estimate the trend of each line to the atomic mass of water, 18 amu. While the prediction is easy to do, it is \emph{wildly} wrong! The actual melting and boiling points are \SI{83}{\degreeCelsius} and \SI{170}{\degreeCelsius} warmer than the predictions. Why? (Data from CRC Handbook, 74th Edition, 1993) The \textbf{lower panel} shows the density of liquid water at temperatures of \SIrange{0}{10}{\degreeCelsius}.Note the density reaches a maximum near \SI{4}{\degreeCelsius}, not \SI{0}{\degreeCelsius}. Ice has a density of \SI{916.2}{\kilogram\per\metre^3}, which on scale of the graph would be \SI{17}{\metre} (56 feet) below the graph.}
\label{fig:wiw}
\end{figure}


\newpage
\begin{figure}[p]
\centering
  \includegraphics[width=5.5 in]{H_bonding}%
\caption{Water's weird behavior is due to hydrogen bonding between hydrogen and oxygen atoms in adjacent water molecules. Within molecules, the $H$ and $O$ atoms are held together with strong covalent bonds, shown as the cylinders connecting the red $O$ atoms and white $H$ atoms. Oxygen's stronger affinity for electrons gives it a partial negative charge $\delta^{-}$, while the hydrogen develops a partial positive charge ($\delta^{+}$). These partial charges form hydrogen bonds (yellow), only one twentieth the strength of the covalent bonds. Each hydrogen atom establishes one hydrogen bond, and each oxygen atom accepts two hydrogen bonds, so most water molecules have a total of 4 hydrogen bonds. The kinetic energy of room-temperature (and body-temperature) water is about the same as the strength of hydrogen bonds, so a given bond breaks and reforms about \SI{1e12}{s^{-1}}, or one trillion times per second. The figure was  made with Jmol.}
\label{fig:H_bonding}
\end{figure}

\newpage
\begin{figure}[p]
\centering
  \includegraphics[width=5.5 in]{w_tetra}%
\caption{At a microscopic level, water exists not as isolated, individual molecules of $H_2O$, but as chains of hydrogen-bonded ``super-molecules.'' Despite the overwhelming importance of water to every aspect of science, culture and economy, the microscopic structure of water is still an area of intense research. But at room temperature and below, liquid water resembles the structure shown above, with 4 water molecules arranged in a tetrahedron (gray outline) around a central fifth molecule. Each of molecules at the corner of this tetrahedron in turn will be arranged in a similar way, such that water consists of a interwoven and extensive matrix of interconnected molecules. The figure was made with Jmol.}
\label{fig:w_tetra}
\end{figure}


\newpage
\begin{figure}[p]
\centering
  \includegraphics[width=5.5 in]{where_water}%
\caption{Water is an integral part of every Earth sphere. Each column shows the relative proportion of water in (\textbf{left panel}) the entire Earth, (\textbf{middle panel}) the near-surface Earth, and (\textbf{right panel}) the Hydrosphere.}
\label{fig:where_water}
\end{figure}

\newpage
\begin{figure}[p]
\centering
  \includegraphics[width=5.5 in]{water_scales}%
\caption{Another way of visualizing the volume of near-surface water on Earth. Each sphere represents the volume of water in each of the labeled reservoirs. The top row shows the Hydrosphere, with the oceans occupying a sphere roughly \SI{1400}[{\kilo}{\metre}] (860 miles) across. Groundwater clearly and surprisingly makes up the majority of the non-ocean water in the Hydrosphere, with lakes, wetlands and rivers making up increasingly smaller portions. Water in the other spheres are shown in the lower portion of the Figure, with the Cryosphere dominating. Clearly the vast majority of fresh water on the planet---stored in the Cryosphere and in groundwater---is accessible to human societies only with great difficulty. This figure is based on one by Jack Cook, Woods Hole Oceanographic Institution.}
\label{fig:water_scales}
\end{figure}


\newpage
\begin{figure}[p]
\centering
  \includegraphics[width=5.5 in]{ice_soil}%
\caption{A model for soil water, the water table, and groundwater is easy to make with a glass of ice and water, here dyed red to make it visible. Near the top of the glass, air occupies all the void spaces between the ice particles. Lower in the glass, water fills an increasingly larger proportion of the void spaces, until the voids between ice particles are fully occupied by water. The red food coloring in the water makes the transition from the under-saturated upper layer to the fully saturated groundwater layer particularly striking. Inserting a straw into the unsaturated zone wouldn't help quench your thirst; only air would be drawn uo the straw. So too with real groundwater. Wells for extracting groundwater must extend at least to the water table to gain access to the resource. }
\label{fig:ice_soil}
\end{figure}

\newpage
\begin{figure}[p]
\centering
\includegraphics[width=5.5 in]{GW_sketch.pdf}%
\caption{A schematic view of the formation, flow and fate of groundwater. The \textbf{upper panel} (modified from Figure 4.1 of Berner, E. K., \& Berner, R. A. (1987). Global water cycle: geochemistry and environment. Prentice-Hall) shows rain falling on the recharge area. A small proportion of the rain percolates into the ground as soil water, eventually reaching the water table to become groundwater. (The remainder of the rain runs off along the surface, evaporates, or is absorbed and later ``transpired'' by plants back to the Atmosphere.). Wetlands, which require long-term wet environment, form where the water table approaches the surface. Rivers, lakes and other standing bodies of water form from the baseflow provided by the intersection of the water table with the surface, as shown on the far right of the diagram. The \textbf{lower panel} (modified from Figure 3 of Winter, T. C. (Ed.). (1998). Ground water and surface water: a single resource. Denver, Colo: U.S. Geological Survey, Circular 1139.) shows a larger view of groundwater. Three different aquifers (volumes of rock and sediment which host and store groundwater) each contain groundwater of different age, with deeper aquifers holding older water.}
\label{fig:gw_sketch}
\end{figure}

\newpage
\begin{sidewaysfigure}[ht]
\centering
\includegraphics[width=7 in]{HPA_final.pdf}%
\caption{The state of groundwater in the High Plains Aquifer, which supplies nearly a third of all irrigation water in the US. The \textbf{left panel} shows the depth to the water table (in feet) measured in 2007 CE. In general, shallower water tables indicates areas with greater recharge. Areas with deeper water tables have slower recharge and older groundwater. (Modified from Dennehy, K. F., Gurdak, J. J.,  McMahon, P. B., Stanton, J. S. \& Qi, S. L. (2007) Sources-Quality of Recently Recharged  Water in the High Plains Aquifer, Chapter 1 of Water-Quality Assessment of the High Plains Aquifer, 1999-2004. USGS Professional Paper 1749 Retrieved July 22, 2017, from \href{https://pubs.usgs.gov/pp/1749/downloads/pdf/P1749ch1.pdf}{Link}). The \textbf{right panel} shows the percentage change in the thickness of the groundwater layer, between 1950 CE (when large scale pumping began) and 2013 CE. Note that areas with deeper groundwater, such as western Texas and south-western Kansas, have lost more of their groundwater than areas with a shallow water table. This puts farmers in those areas in a double bind: they need the water more than other areas do, but it is more expensive to pump, slower to recharge and more rapidly depleted than groundwater in other parts of the aquifer. These will be the first places to exhaust the aquifer, unless the users can find a way to reduce their pumping to sustainable levels. (Modified from McGuire, V. L. (2014). Water-level changes and change in water in storage in the High Plains aquifer, predevelopment to 2013 and 2011-13 (No. 2014-5218). US Geological Survey.}
\label{fig:hpa}
\end{sidewaysfigure}

\newpage
\begin{figure}[p]
\centering
  \includegraphics[width=5.5 in]{ro_comp}%
\caption{The oceans are not simply ``concentrated'' river water. Rivers are the primary, but not only, factor contribution to the ocean's salinity. rivers (in the blue shading) and oceans (in pink) have distinctly different ratios of ions. If the oceans were simply concetrated river water, the proportions of ions in each would be identical. As the graph shows, this is not true. Other sources and sinks---which reside in the Geosphere and Biosphere---must modify sea water's composition. }
\label{fig:or_ions}
\end{figure}


\newpage
\begin{sidewaysfigure}[ht]
\centering
    \includegraphics[width=8 in]{global_bathymetry}
    \caption{Earth's topography (height of land above sea level) and bathymetry (depth below sea level) reveals a dynamic planet. Mountain belts line the edges of continents, but the middle of oceans, attesting to the different plate tectonic forces forming them. The map was made with GeoMapApp (http://www.geomapapp.org) and uses the GMRT data assembled by Ryan, W.B.F., et al. (2009), Global Multi-Resolution Topography synthesis, Geochem. Geophys. Geosyst., 10, Q03014, doi:10.1029/2008GC002332. }
    \label{fig:global_bathy}
\end{sidewaysfigure}

\newpage
\begin{sidewaysfigure}[ht]
\centering
    \includegraphics[width=8 in]{sea_floor_detail.pdf}
    \caption{A detailed view of the bathymetry of the ocean floor, with important features highlighted. Red lines mark divergent plate boundaries; yellow convergent boundaries, and green major transform boundaries. The map was made with GeoMapApp (http://www.geomapapp.org) and uses the GMRT data assembled by Ryan, W.B.F., et al.,  (2009), Global Multi-Resolution Topography synthesis, Geochem. Geophys. Geosyst., 10, Q03014, doi:10.1029/2008GC002332. Plate boundaries are modified from Bird, P. (2003), An updated digital model of plate boundaries, Geochem. Geophys. Geosyst., 4, 1027, doi:10.1029/2001GC000252, 3.}
    \label{fig:bathy_detail}
\end{sidewaysfigure}

\newpage
\begin{sidewaysfigure}[ht]
\centering
    \includegraphics[width=8 in]{hypso}
    \caption{The distribution of height and depth of every area on Earth's surface. In the panel on the left is a histogram of heights and depths. The continents are, on average, relatively low at an elevation of about 800 m (2650 ft, about one half mile) with a tiny proportion made up by mountains higher than 3000 m (10,000 ft). Below sea level, the most common depths are those of the deep abyssal plains. At 3700 m (11,000 ft) deep the oceans are, on average, far deeper than the continents are high. In the right panel, the cumulative distribution of area gives the hypsometric curve. The continental plains (where the weathered and eroded remnants of mountains are deposited) give way to the continental shelves, submerged portions of the continents. The continental slope marks the transition from continental to oceanic crust, and the beginning of abyssal depths. Trenches, though spectacularly deep (the deepest at 11,911 m [35,800 ft, or 6.8 miles!])  make up a tiny proportion of Earth's surface. The intersection of sea level and the blue curve marks the proportion of Earth coverd by land: currently about 30\%, and the oceans about 70\%.  But knowing the shelves are really temporarily submerged parts of the continents suggests a different value: 35\%  for the continents and 65\% for the oceans. Data are from the ETOPO1 ''Ice Surface data set,'' and is inspired by Eakins, B.W. and G.F. Sharman, Hypsographic Curve of Earth's Surface from ETOPO1, NOAA National Geophysical Data Center, Boulder, CO, 2012.}
    \label{fig:hypso}
\end{sidewaysfigure}

\newpage
\begin{sidewaysfigure}[ht]
\centering
    \includegraphics[width=8 in]{smokin}
    \caption{Two examples of hydrothermal vents, spectacular examples of interactions between the oceans and the hot volcanic rocks found at mid-ocean ridges. On the left panel, super-heated water emerges from natural ``chimneys.''  The white solids form when hot fluids mix with cooler seawater, forming the chimneys. The white precipitates include compounds of barium, calcium and silicon. In the right panel, water at \SI{340}{\celsius} ($650\circ F$) emerges from a vent at a depth of \SI{2200}{\metre} ($7200\ ft$). The black precipitates are metal sulphides ($FeS$), which likely include iron, copper, manganese and zinc. When these ``black smoker'' vents were first discovered, it became clear that the world's most important copper deposits formed in similar locations in Earth's distant past. Credits: While from NOAA web sites, the copyright for thee mages is unclear. I do NOT yet have certain rights to these photographs. Left panel URL points to \href{http://oceanexplorer.noaa.gov/okeanos/explorations/10index/logs/slideshow/ex_july_highlights/gallery/hires/white_plumes_hires.jpg}{NOAA}, as does the right panel \href{http://oceanexplorer.noaa.gov/okeanos/explorations/10index/background/hires/boardwalk_black_smoker_hires.jpg}{URL}.}
    \label{fig:smokers}
\end{sidewaysfigure}

\newpage
\begin{figure}[p]
\centering
  \includegraphics[width=5.5 in]{Salinity_Map}%
\caption{The annual average salinity of the surface ocean reveals that exchange of water with the atmosphere is the principle force determining surface salinity. Salinity is highest at $25\deg$ N and S latitude, a few degrees poleward of the zone of maximum evaporation of water from the oceans to the atmosphere. Salinities are highest in the Atlantic Ocean because of the net movement of water by evaporation out of the Atlantic Basin and to the Pacific Basin. }
\label{fig:surf_salinity}
\end{figure}


\begin{figure}[p]
\centering
  \includegraphics[width=5.5 in]{ODV_sal_vs_Lat}%
\caption{Salinity variations below the surface layer are far less than those in the surface layer, as shown in this cross section of the Atlantic ocean. Below the \SI{600}{\metre} (2000 feet) deep surface layer, salinities are uniformly within the narrow range of $34.5-35$ \textperthousand.}  
\label{fig:salinity_sect}  
\end{figure}

\newpage
\begin{figure}[p]
\centering
  \includegraphics[width=5.5 in]{zonal_sst}%
\caption{Warmer at the equator, cooler at the poles, the average sea surface temperature reflects the balance of heat flowing in and out of the oceans. Abundant sunlight and warm air in the tropics warms the surface waters of the ocean there, while less sunlight and relatively colder air in the extra-tropics cools the oceans. Circulation of the oceans (Section \ref{gyres}, like the atmosphere, transports heat from the equator to the poles, moderating Earth's climate extremes. Data are the 1961-1990 SST from the NOAA Optimum Interpolation (OI) Sea Surface Temperature (SST) V2, available at \href{https://www.esrl.noaa.gov/psd/data/gridded/data.noaa.oisst.v2.html}{NOAA PSD} accesssed pi day, 2018.}
\label{fig:zonal_sst}
\end{figure}

\newpage
\begin{figure}[p]
\centering
  \includegraphics[width=5.5 in]{sst_map}%
\caption{The annual average surface temperature of the oceans.}
\label{fig:sst_map}
\end{figure}


\begin{figure}[p]
\centering
  \includegraphics[width=5.5 in]{sst_anom_map}%
\caption{Anomaly map}
\label{fig:sst_anom_map}
\end{figure}



\newpage
\begin{figure}[p]
\centering
  \includegraphics[width=5.5 in]{ODV_Temp_vs_Lat}%
\caption{The average temperature of the deep oceans, here in the Atlantic. Below the thin surface layer ~600 meters (2000 feet) deep, the oceans are monotonously cold, averaging between \SIrange{1}{3}{\celsius} ($34-37 \deg$F). When combined with the monotonous salinity variations (Figure \ref{fig:salinity_sect}, and the fact that sea water density is largely controlled by salinity and density, we conclude that density variations in the deep ocean must be subtle. Tiny difference in temperature and salinity can have profound effects on density, and hence circulation, of seawater.}
\label{fig:ODV_Temp_vs_Lat}
\end{figure}

\newpage
\begin{figure}[p]
\centering
  \includegraphics[width=5.5 in]{ph_map}%
\caption{Ocean acidity determines marine organisms ability to construct shells. Lower pH means higher acidity and more difficulty making shells. Global variations in pH are relatively small compared to the variations we've seen in other ocean properties, partly because the pH scale is logarithmic, and partly because of natural buffering of pH by the many forms of $CO_2$ dissolved in the oceans. In the tropics and temperate areas, pH is lower (more acidic) in summer and higher in winter. This pattern is reversed in the poles, with lower pH in winter and higher pH in summer. This odd pattern is due to the interplay of light and nutrient supplies affecting photosynthesis in the water column. The map shown here is for 2005 CE; since then pH has dropped roughly 0.03 units, equivalent to a 7\% increase in acidity just since this mimage was made. (Takahashi, T., Sutherland, S. C., Chipman, D. W., Goddard, J. G., Ho, C., Newberger, T., ... \& Munro, D. R. (2014). Climatological distributions of pH, pCO2, total CO2, alkalinity, and CaCO3 saturation in the global surface ocean, and temporal changes at selected locations. Marine Chemistry, 164, 95-125. [I DO NOT HAVE (yet) PERMISSION TO USE THIS FIGURE. the data are available...])  }
\label{fig:ph_map}
\end{figure}


\begin{figure}[p]
\centering
  \includegraphics[width=5.5 in]{omega_aragonite_map}%
\caption{All marine organisms that secrete shells must expend energy to do so. Part of this energy goes toward removing the various chemical components (calcium  $[Ca^{2+}]$ and bicorbonate $[HCO_3^{1-}]$ ions) from the ocean. The relative ease with which this process takes place is given by the ``saturation state.'' Higher saturation states lead to easier shell building; lower states indicate more difficult shell building and even shell dissolution into the oceans. Warm, tropical oceans have higher saturation states than colder temperate waters. The map shown here is for 2005 CE; since then saturation states have dropped roughly 0.12 units (Takahashi, T., Sutherland, S. C., Chipman, D. W., Goddard, J. G., Ho, C., Newberger, T., ... \& Munro, D. R. (2014). Climatological distributions of pH, pCO2, total CO2, alkalinity, and CaCO3 saturation in the global surface ocean, and temporal changes at selected locations. Marine Chemistry, 164, 95-125.) in the 15 years since then. [I DO NOT HAVE (yet) PERMISSION TO USE THIS FIGURE. the data are available...]). Coral reefs include various types of calcium carbonate, including both the skeletons and shells of living organisms, and sediments from the eroded remains. The mud is more susceptible to dissolution, and Eyre, B. D. Cyronak, T., Drupp, P., De Carlo, E. H., Sachs, J. P., \& Andersson, A. J. (2018). Coral reefs will transition to net dissolving before end of century. Science, 359(6378), 908-911 argue that this mud will begin to dissolve at saturation states around 2.9, a situation that we'll see globally around 2050-2070 CE.  }
\label{fig:arag_map}
\end{figure}



\newpage
\begin{figure}[p]
\centering
  \includegraphics[width=5.5 in]{ocean_acidification}%
\caption{Ocean acidification is driven by absorption of $CO_2$ by the oceans. The resulting change in acidity of the shallowest parts of the ocean is shown for three sites across the world. All three sites show about the same rate of change: -0.02 pH units per decade, equivalent to about +4\% in acidity each decade. The annual cycle of pH change is also evident in the data; the range of these oscillations depends on the location of the station. The Canary and Bermuda data are plotted 0.15 and 0.30 pH units lower than their actual for clarity. Data are from the websites associated with: Bermuda---Bates, N.R., Michaels, A.F., Knap, A.H., 1996. Seasonal and interannual variability of oceanic carbon dioxide species at the US JGOFS Bermuda Atlantic Time-series Study (BATS) site. Deep-Sea Res. II 43, 347-383. http://dx.doi.org/10.1016/0967-0645(95)00093-3 (Corrigendum: 43, 1435-1435, 1996); Canary Islands---Santana-Casiano, J.M., Gonz'alez-D'avila, M., Rueda, M.-J., Llin'as, O., Gonz'alez-D'avila, E.-F.,2007. The interannual variability of oceanic CO2 parameters in the northeast Atlantic subtropical gyre at the ESTOC site. Glob. Biogeochem. Cycles 21. http://dx.doi.org/10.1029/2006GB002788 (GB1015); Hawai'i---Dore, J.E., Lukas, R., Sadler, D.W., Church, M.J., Karl, D.M., 2009. Physical and biogeochemical modulation of ocean acidification in the central North Pacific. PNAS 106, 12235-12240. http://dx.doi.org/10.1073/pnas.0906044106. Files: ph\_plot.m and ph\_ocean\_acid.
}
\label{fig:oa}
\end{figure}

\newpage
\begin{figure}[p]
\centering
  \includegraphics[width=5.5 in]{light_in_seas}%
\caption{The ocean is divided into three zones based on depth and light level. The upper \SIrange{100}{200}{metre} (300 to 650 feet) of the ocean is called the euphotic, or ``sunlight,'' zone. Photosynthesis is confined to the euphotic zone, so it contains the vast majority of commercial fisheries and is home to most of the protected marine mammals and sea turtles. Sunlight entering the oceans may travel about \SI{1000}{metre}into the ocean under the right conditions, but there is rarely any significant light beyond SI{200}{metre} (650 feet). The zone beneath the euphotic zone but above \SI{1000}{metre} (3,300 feet) is usually referred to as the ``twilight'' zone, but is officially the dysphotic zone. In this zone, the intensity of light rapidly decreases as depth increases. The aphotic, or ``midnight,'' zone exists at depths below 1,000 meters (3,280 feet). Sunlight does not penetrate to these depths and the zone is bathed in darkness. Figure and caption modified from the National Oceanic and Atmospheric Administration, https://oceanservice.noaa.gov/facts/light\_travel.html, accessed 17 May, 2018}
\label{fig:light_in_seas}
\end{figure}

\newpage
\begin{figure}[p]
\centering
  \includegraphics[width=5.5 in]{N_H_isotope_ratios_planets}%
\caption{Earth's water is from far-flung sources: beyond the orbit of Jupiter. Hydrogen has two stable isotopes; their ratio in water reflects the ancient history of the water. A similar observation applies to nitrogen: the ratio of its two stable isotopes gives complimentary information on the history of a object's nitrogen. This history largely reflects how far from the sun the water formed. Earth (blue circle and line, the Moon (grey rectangle), Mars (red rectangle) and even the largest asteroid Ceres (white rectangle) all have hydrogen and nitrogen isotopic ratios near 0. Comets formed 1000s of AU from the sun in the Oort cloud (yellow box), and comets formed near 30 AU (green box) have hydrogen and nitrogen ratios distinct from Earth, and clearly are not the source of Earth's water. Of all the objects we have measured, only a few types of primitive carbon- and water-rich carbonaceous chondrite meteorites have the right composition to source Earth's water. These meteorites likely formed as asteroids near and beyond the orbit of Jupiter, at 5 to 10 AU. See Figure \ref{fig:jupiter_source_water} for illustrations of how Jupiter scattered these asteroids into Earth's neighborhood.} 
\label{fig:N_h_isotopes}
\end{figure}

\newpage
\begin{figure}
\centering
\includegraphics[width=5.5 in]{Formation_hydrosphere}%
\caption{The hydrosphere's early history. Delivery of carbon- and water-rich asteroids to the inner solar system began as early as 8 to 20 My after CAIs, long before the final moon-forming collision between Earth and Theia. The wet asteroids from the outer solar system delivered the equivalent of 3 to 5 times the mass of today's oceans to the still-forming Earth. The magma ocean resulting from the collision delivered much of this water to the hot, steam-rich atmosphere surrounding our nascent planet. Once the magma ocean solidified, steam condensed to water and the very first rain fell to the surface, forming the hydrosphere. This early liquid ocean was most likely repeatedly boiled back into the atmosphere by meteorites striking the surface, a process that may have continued until 4 Ga.(Marchi, S., Bottke, W. F., Elkins-Tanton, L. T., Bierhaus, M., Wuennemann, K., Morbidelli, A., \& Kring, D. A. (2014). Widespread mixing and burial of Earth's Hadean crust by asteroid impacts. Nature, 511(7511), 578.) The oldest pieces of Earth yet found---tiny grains of the mineral zircon from outcrops in the Jack Kills area of Australia---have clear O isotopic evidence that abundant surface water existed at 4.374 Ga, less than 100 My after Earth had a solid crust. The oceans are \emph{old}. Dark-hued rectangles show best estimates for events, while light-hued rectangles give permissible (typically 95\%) intervals; text in \textit{italics} indicate events inferred from computer models, not measurement.}   
\label{fig:water_timing}
\end{figure}


\newpage
\begin{figure}[p]
\centering
\includegraphics[width=5.5 in]{Jup_Sat_water}%
\caption{Sophisticated computer models illustrate how a growing Jupiter and Saturn could have flooded the inner solar system with water-rich asteroids. One such model mathematically grows Jupiter and Saturn over a brief interval a few million years after solar system formation began. The horizontal axis shows the orbital radius of each object plotted, while the vertical axis gives the eccentricity of the orbit. Asteroids (the colored dots) are color coded by their distance to the Sun. Red and green asteroids are relatively dry, blue and cyan asteroids are relatively wet. The Earth will eventually form at an orbital radius of 1 AU. As Jupiter (the black circle at 5.4 AU) and Saturn (at 7.2 AU) grow, their gravity perturbs the orbits of nearby asteroid, forming ``wings.'' Eventually the planets grow big enough to force asteroids into highly eccentric orbits, which carry them into the inner solar system (indicated by the dashed purple line). Here they can collide with the growing terrestrial planets, providing water to the previously dry planets. This illustration is modified from Figure 2 of Raymond, S. N., \& Izidoro, A. (2017). Origin of water in the inner Solar System: Planetesimals scattered inward during Jupiter and Saturn's rapid gas accretion. Icarus, 297, 134-148. I DO NOT YET HAVE PERMISSION TO USE THIS FIGURE.}   
\label{fig:jupiter_source_water}
\end{figure}


\newpage
\begin{sidewaysfigure}[ht]
\centering
\includegraphics[width=7 in]{jhzircon}%
\caption{The oldest bits of our planet known are found in an ancient part of western Australia (pane a). From space (b) the desolate Jack Hills region shows ~3 Ga sedimentary rocks that have repeatedly been heated an deformed. On the ground (c), the rocks include conglomerates clearly formed by and in running water, hardly surprising in such ``young'' rocks. Zircon grains (d) found in these rocks are substantially older than their host rocks: the lower left corner of the zircon here is \num[separate-uncertainty = true]{4.404(8)}. Oxygen isotope ratios in the zircon indicate that the rocks from which the zircon formed had interacted with liquid water. Earth's hydrosphere was up and running over 4.4 billion years ago. Pane (a) and (b) are courtesy of NASA Earth Observatory (https://earthobservatory.nasa.gov/Features/Zircon/Images/australia\_map.jpg; https://eoimages.gsfc.nasa.gov/images/imagerecords/6000/6331/jackhills\_etm\_1999208\_lrg.jpg). Panes (c; http://www.geology.wisc.edu/zircon/Earliest\%20Piece/Images/3.jpg) and (d; http://www.geology.wisc.edu/zircon/Earliest\%20Piece/Images/5.jpg) are from John Valley's web site, and I DO NOT YET HAVE PERMISSION TO USE THIS THEM. }   
\label{fig:jhzirc}
\end{sidewaysfigure}

\newpage
\begin{figure}[ht]
\centering
\includegraphics[width=7 in]{BIF_usgs}%
\caption{The famous banded iron formations of Hamersley Range, Western Australia, a mere 2.46 Ga. These extensive deposits from ancient shallow seas are found world-wide, and are the primary source of iron ore to society. They are are also the oldest-known sedimentary rocks (although the ones pictured here are younger by 1.2 billion years!) Figure courtesy of the USGS and photographer https://www.usgs.gov/media/images/banded-iron-formation-hamersley-range. Accessed 29 May 2018.}   
\label{fig:bif}
\end{figure}


\newpage
\begin{figure}[ht]
\centering
\includegraphics[width=7 in]{ocean_temps_archean}%
\caption{Debate still rages on the ancient ocean's temperature. One school of thought (exemplified in the model results shown in yellow in pane a) argues that SST were roughly the same as those now. Well-documented ice ages (pane c), average weathering rates, the faint-young sun, even the existence of life (pane d) all argue for low SST over geologic history. But the evidence for hot oceans (isotopic results shown in red, green and purple; ezymatic evidence in blue in pane a) are consistent and are based on a wide variety of observations. Archean and Proterozoic oceans were hot, as high as \SIrange{55}{85}{\celsius}, and have cooled slowly over time. There is some evidence of temperature increases following both oxygenation steps (2.5 and 0.7 Ga) of the atmosphere. The net cooling of the surface oceans in this case is on the order of \SI[per-mode=symbol]{15}{\celsius\per\giga\year}. Which school is the correct one is still impossible to tell. Data sources: blue rectangles, Garcia, A. K., Schopf, J. W., Yokobori, S. I., Akanuma, S., \& Yamagishi, A. (2017). Reconstructed ancestral enzymes suggest long-term cooling of Earth's photic zone since the Archean. Proceedings of the National Academy of Sciences, 114(18), 4619-4624; red and green bands, Robert, F., \& Chaussidon, M. (2006). A palaeotemperature curve for the Precambrian oceans based on silicon isotopes in cherts. Nature, 443(7114), 969; purple, Knauth, L. P., \& Lowe, D. R. (2003). High Archean climatic temperature inferred from oxygen isotope geochemistry of cherts in the 3.5 Ga Swaziland Supergroup, South Africa. Geological Society of America Bulletin, 115(5), 566-580; yellow, Krissansen-Totton, J., Arney, G. N., \& Catling, D. C. (2018). Constraining the climate and ocean pH of the early Earth with a geological carbon cycle model. Proceedings of the National Academy of Sciences, 201721296; deep purple, Hren, M. T., Tice, M. M., \& Chamberlain, C. P. (2009). Oxygen and hydrogen isotope evidence for a temperate climate 3.42 billion years ago. Nature, 462(7270), 205; orange, Blake, R. E., Chang, S. J., \& Lepland, A. (2010). Phosphate oxygen isotopic evidence for a temperate and biologically active Archaean ocean. Nature, 464(7291), 1029; oither arguments for low temperature from Kasting, J. F., Howard, M. T., Wallmann, K., Veizer, J., Shields, G., \& Jaffr'es, J. (2006). Paleoclimates, ocean depth, and the oxygen isotopic composition of seawater. Earth and Planetary Science Letters, 252(1-2), 82-93.}   
\label{fig:historic_sst}
\end{figure}


\newpage
\begin{figure}[ht]
\centering
\includegraphics[width=7 in]{ocean_comp_time}%
\caption{The oxidation state and trace element content of the oceans over time. Prior to the Great Oxidation event, the vast majority of the oceans---from surface to bottom) were poor in dissolved oxygen (anoxic) but rich in dissolved  iron, $Fe^{2+}$ (ferruginous), as shown by the brown shaded regions. Occasional oxygen-rich oases may have briefly bubbled-up in shallow oceans after the evolution, ~3 Ga, of oxygen-producing photosynthesis. The oceans above the continental slopes were anoxic, ferruginous and rich in $H_2S$, or euxinic, a condition recorded by abundant black shales from the period. Following the GOE, an oxygen-rich surface layer (in blue) spread globally, but anoxic and ferruginous continued in the deep ocean. Only after the NOE did bottom waters of the ocean become oxygenated, allowing soon thereafter for the evolution of large animals for the first time. Note (bottom panel) the sudden expansion of metazoan animals at the time of the NOE. Euxinic areas, though rare now, are still found in some areas of intense biological activity. (Lyons, T. W., Reinhard, C. T., \& Planavsky, N. J. (2014). The rise of oxygen in Earth's early ocean and atmosphere. Nature, 506(7488), 307; Knoll, A. H., \& Nowak, M. A. (2017). The timetable of evolution. Science Advances, 3(5), e1603076; Chen, X., Ling, H. F., Vance, D., Shields-Zhou, G. A., Zhu, M., Poulton, S. W., ... \& Archer, C. (2015). Rise to modern levels of ocean oxygenation coincided with the Cambrian radiation of animals. Nature Communications, 6, 7142.}   
\label{fig:ocean_comp}
\end{figure} 

\newpage
\begin{figure}[ht]
\centering
\includegraphics[width=7 in]{enso_norm}%
\caption{Normal sea surface temperatures (SSTs) in the equatorial Pacific (for the period 1961-1990). This figure is identical to Figure \ref{fig:sst_map}, except for its focus on the Pacific and the different color scale. The ``Western Pacific Warm Pool,'' with SSTs over \SI{27}{\celsius} stretches from the western edge of the Pacific basin to the International Date Line at $180\circ$E longitude. The locations of Darwin and Tahiti are labeled; they become important locations for understanding the ENSO phenomena. The ``cold tongue'' along the equator from the western coast of South America to the IDL is also noticeable. Figure \ref{fig:sst_anom_map} nicely highlights the east-west temperature differences nicely.} 
\label{fig:enso_norm}
\end{figure} 

\newpage
\begin{figure}[ht]
\centering
\includegraphics[width=7 in]{ENSO_stages}%
\caption{The ENSO includes three end-member stages, here illustrated by maps of SST differences from normal conditions. At normal conditions (third panel from top) SST differences are small (white areas) and the few areas of larger differences (yellow and blue areas) are small and scattered. During the classical El Ni\~no (second panel from top) the warm pool sloshes across the Pacific Ocean, piling up against South America. This is attended by a deeper thermocline and reduced upwelling, both detrimental to the productive areas along the coast. The ``Central Pacific'' form of the El Ni\~no also has the warm pool moving eastward, but it stalls in the central Pacific ocean. After return to normal conditions following an El Ni\~no, a La Ni\~na can form (bottom panel) where the strengthened cold tongue surges westward at the equator. The La Ni\~na SST cooling is generally much less than the warming associated with an El Ni\~no. (Wang, C., Deser, C., Yu, J. Y., DiNezio, P., \& Clement, A. (2012). El Ni\~no and southern oscillation (ENSO): a review. In Coral Reefs of the Eastern Pacific, P. Glymn, D. Manzello, and I. Enochs, Eds., Springer Science Publisher, 85-106.)}    
\label{fig:enso_stages}
\end{figure} 

\newpage
\begin{figure}[ht]
\centering
\includegraphics[width=7 in]{Walker}%
\caption{Caption}    
\label{fig:walker}
\end{figure} %\newpage


\newpage
\begin{figure}[ht]
\centering
\includegraphics[width=7 in]{wpwp_tstorms}%
\caption{[This is a low res version of a much larger file.] ``Late afternoon sun casts long shadows from high thunderhead anvils over southern Borneo. Crews aboard the International Space Station have recently concentrated on panoramic views of clouds--taken with lenses similar to the focal length of the human eye. These images reveal the kinds of views crews see -- huge areas of the planet, with a strong three-dimensional sense of what it is like to fly \SI{400}{\kilo\metre} (250 miles) above the Earth. High-altitude winds are clearly sweeping the tops off the many tallest thunderclouds, generating long anvils of diffuse cirrus plumes that trail south. Storm formation near the horizon -- more than 1000 km distant (image center) -- is assisted as air currents rise over the central mountains of Borneo.'' The inset shows the approximate view as seen from the ISS (http://www.isstracker.com/historical) Image and caption courtesy of NASA, https://eol.jsc.nasa.gov/SearchPhotos/photo.pl?mission=ISS040\&roll=E\&frame=88891, accessed on Summer solstice, 2018 (Northern Hemisphere).}    
\label{fig:wpwp_tstorms}
\end{figure} 



\newpage
\begin{figure}[ht]
\centering
\includegraphics[width=7 in]{ONI}%
\caption{The El Ni\~no-Southern Oscillation since 1950, as shown by departures of SST from the average in the ``Nino3.4 region'' of the central Pacific Ocean. El Ni\~nos and La Ni\~nas are indicated by the red and blue spikes respectively; greater and longer departures indicate stronger events. Note that El Ni\~nos are in generally stronger than La Ni\~nas. A quick count of El Ni\~nos indicates (see Section \ref{enso_cycle}) an average period of 3 years, although there is significant variation around this average. The diagram gives the maximum amplitude of El Ni\~os is rising with time. This plot, modified from one created by Kevin Trenberth (https://climatedataguide.ucar.edu/climate-data/nino-sst-indices-nino-12-3-34-4-oni-and-tni),  is constructed from NOAA\_ERSST\_V4 data provided by the NOAA/OAR/ESRL PSD, Boulder, Colorado, USA, from their Web site at https://www.esrl.noaa.gov/psd/.}    
\label{fig:oni}
\end{figure}%\newpage


%\newpage
%\begin{figure}[p]
%\centering
%\subfloat{%
  %\includegraphics[width=5.5 in]{npp_jul.jpg}%
	%}
%
%\subfloat{%
  %\includegraphics[width=5.5 in]{npp_jan.jpg}%
%}
%
%\caption{Caption}
%\label{fig:npp}
%\end{figure}
%
%
%\newpage
%\begin{figure}[p]
%\centering
  %\includegraphics[width=5.5 in]{file_name}%
%\caption{Pablum}
%\label{fig:groups_and_circ}
%\end{figure}
%


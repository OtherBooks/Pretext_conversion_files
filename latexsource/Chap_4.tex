\chapter{The Greenhouse Effect and Earth's Many Temperatures}\label{chap:GHE}

\section{Core concepts}
\begin{itemize}
\item Greenhouse gasses in Earth's atmosphere absorb infra-red radiation emitted by Earth's surface. The energy is re-emitted by the gasses, warming Earth's surface. This is the greenhouse effect.
\item Naturally occurring greenhouse gasses warm Earth's surface by \SI{33}{\kelvin} or nearly ${60}^{\circ}$ F. This natural greenhouse effect is crucial to the maintenance of life-friendly conditions on the planet.
\item Feedback loops connect the concentration of greenhouse gasses in the atmosphere to processes in the geosphere, hydrosphere and biosphere. These feedback loops maintain Earth's climate on timescales from centuries to billions of years.
\item Changes in insolation and albedo with latitude influence the climate at each latitude. Earth's orbital obliquity imposes seasonal changes on climate as well. 
\end{itemize}

\section{Opening problem}
What happened? At the conclusion of Chapter 3, we proudly declared that Earth's equilibrium temperature $T_{eq}$, was \SI{255}{\kelvin}, \SI{-18}{\celsius} or ${-1}^{\circ}$ F, and then admitted that this temperature was \SI{33}{\kelvin} or ${60}^{\circ}$ F colder than the true average temperature of Earth's surface. Why is Earth's surface temperature so much warmer than we expect?

\section{The Greenhouse Effect}
\subsection{A brief explanation }
Observation 3 from the previous chapter noted that ``anything that slows the radiation of heat from a warm object will keep the object warmer, longer.'' Just like a blanket keeps you warmer, so too can a planet's atmosphere keep its surface warmer and longer than otherwise. This \emph{Greenhouse Effect} is a crucial aspect of Earth's climate, now, in Earth's past, and in your future. Four naturally occurring greenhouse gasses in Earth's atmosphere---water vapor ($H_2O$), carbon dioxide ($CO_2$), methane ($CH_4$) and nitrous oxide ($N_2O$)---conspire to warm Earth's surface above $T_{eq}$. By absorbing and re-radiating the infrared radiation emitted from Earth's surface, these gasses naturally retard the loss of heat from Earth. A portion of this re-radiated energy strikes Earth's surface, warming it by \SI{33}{\kelvin}. Carbon dioxide, although it is neither the primary gas contributing to the Greenhouse Effect, nor a major constituent of the atmosphere, acts as a thermostat for Earth's climate by regulating the amount of water vapor in the atmosphere. 
\subsection{Empirical evidence}
\subsubsection{An hypothesis: Atmospheres matter to the Greenhouse Effect} 
At the end of Chapter 3 we successfully calculated an equilibrium surface temperature ($T_{eq}$) for each of the terrestrial planets, but we found rather poor agreement between these temperatures and the actual surface temperatures, as recorded in Table 3.04. That table is reproduced as Table \ref{tab:atmo_press_ghg}, with the addition of an extra parameter: the atmospheric pressure on the surface of each planet. (Pressure is given in \textit{bars}, which is the surface pressure of Earth's atmosphere at sea level.) Neither Mercury nor the Moon have appreciable atmospheres, while Mars' thin, dry one rates a paltry 1\% of Earth's surface pressure. The Venusian atmosphere, on the other hand, is almost 100 times the pressure of Earth's atmosphere! The atmospheric pressure on the surface of Venus is equivalent to that at a depth of \SI{920}{\metre} (3000 ft) in Earth's ocean-some \SI{90}{\metre} (300 ft) greater than the height of the world's tallest building\footnote{According to the Council on Tall Buildings and Urban Habitat, \href{http://www.skyscrapercenter.com/}{link}, the Burj Khalifa, Dubai, at \SI{828}{\metre} has been the world's tallest building since 2008 CE}. A careful examination of Table \ref{tab:atmo_press_ghg} illustrates an excellent correlation between atmospheric pressure and the difference between the observed average and equilibrium temperatures. A saying in science is that correlation between two things doesn't imply that one causes the other. But as Edward Tufte noted, ``Correlation is not causation but it sure is a hint''\footnote{Tufte, E. R. (2006). Beautiful Evidence. Cheshire, CT: Graphics, p. 159}. A good hypothesis is that the amount of atmosphere surrounding a planet not only predicts this temperature difference, but causes it as well.\\

\begin{table} 

\centering
\caption{Atmospheres of the terrestrial planets and the greenhouse effect}
\begin{tabular}{@{}cccccccc@{}} \toprule
Parameter	&Symbol	&Units	                      &Mercury	&Venus	&Earth	&Moon	&Mars\\ \midrule
Albedo	  &A		  &	 -                          &0.07     &0.90   &0.30   &0.11 &0.25\\
Insolation&S	&$\SI{}{\watt\per\m^{2}}$	              &9130	    &2610	  &1361	  &1361	&590\\
Equilibrium Surface Temperature	&$T_{eq}$	 &K	  &440	    &184	  &255	  &270	&211\\
Average Surface Temperature	    &$T_{avg}$ &K	  &440	    &737	  &288	  &270	&210\\
Greenhouse effect $T_{avg}- T_{eq}$    &$\Delta T_{GHE}$ &K &0   &553    &33     &0    &1\\
Atmospheric Pressure &$P_{atm}$ &bar            &0        &92     &1      &0    &0.01\\ \bottomrule
\end{tabular}
\label{tab:atmo_press_ghg}
\end{table}

\subsubsection{Testing the hypothesis with a different kind of graph}
An excellent way to test this hypothesis is to graph the difference between the observed average surface temperature and the calculated equilibrium temperature ($T_{av}- T_{eq}$) against atmospheric pressure ($P$). Most graphs you've seen in textbooks use arithmetic axes: each division on an axis is separated by a constant value. The left-hand pane of Figure \ref{fig:tghe_P} shows the temperature difference and atmospheric presure on typical arithmetic axes, with increments of 20 bar on the horizontal axis and 100 K on the vertical axis. The problem with these axes is that data are difficult to see, because most of the data points are squeezed into the tiny corner of the graph near the origin. This is a common problem in studying Earth: the scale of measurement can range over factors of thousands, even millions. Data like that just don't fit on arithmetic axes.\\

A better way to display such data is to use axes where each division is separated by a constant factor, or multiple. This factor is typically 10, as it is on both axes of the right-hand pane of Figure \ref{fig:tghe_P}. Each major division on both axes is a factor of 10 greater than the preceding division. This ``log-log'' graph expands the lower left corner of the arithmetic graph, allowing us to clearly see the majority of the data. The downside of such a graph is that we can't plot any points which are equal to 0, such as atmospheric pressure for Mercury and the Moon. The benefits are that a clear pattern emerges, one that supports our hypothesis: the pressure-temperature difference for Mars, Earth and Venus follows a spectacular line. Following Tufte, we conclude that the atmosphere of these three planets is somehow warming the surface.\\

\subsubsection{Natural warming from the Greenhouse Effect}
This unexpected behavior by planetary atmospheres is called the \emph{greenhouse effect}, a central concept of this text and our study of Earth's Systems. The greenhouse effect has profound influence on the climate of Venus, Earth, and just about any planet with a dense atmosphere\footnote{for example, page 35 of Pierrehumbert, R. T, 2011. Infrared radiation and planetary temperature, Physics Today, Vol. 64(1),pg 33-38}. Driven by a remarkable greenhouse effect (Table \ref{tab:atmo_press_ghg}) the average surface temperature of Venus is an astonishing \SI{737}{\kelvin}, or roughly ${870}^{\circ}$ F. This is hotter than the melting point of about a third of the naturally occurring elements, including zinc and lead, both of which would melt to puddles of liquid on Venus' surface. On Earth, the greenhouse effect naturally raises the average surface temperature of the planet from well below the freezing point of water to a comfortable \SI{288}{\kelvin} or \SI{15}{\celsius} (${60}^{\circ}$ F). Our planet is habitable (for life as we know it, including us!) because of this natural greenhouse effect. The magnitude of the greenhouse effect is sensitive to even small variations in the concentration of greenhouse gasses in the atmosphere, which is how tiny inputs of carbon dioxide from the Anthroposphere to the Atmosphere can cause dramatic global climate change. We need to understand how trace gasses in the atmosphere cause the greenhouse effect, the implications of the greenhouse effect to Earth's climate, and how human activities are nudging the greenhouse effect higher.

\subsection{Light and molecules}
Light, we decided in Chapter 3, is how electrons interact. All matter with a temperature above absolute zero vibrates constantly, and vibrating electrons continuously radiate energy as light into their environment. This light travels across the Universe until it interacts with electrons which can vibrate with the same energy as the light contains. The energy in light is inversely proportional to wavelength and hence color, so different electrons absorb different colors of light. Some gasses absorb light with energies in the visible portion of the spectrum (Figure \ref{fig:iodine}), but all gasses absorb and emit some color of light. This is the basis for the sometimes arcane science of spectroscopy, and for the success of John Tyndall, the most important scientist of whom you've never heard.\\ 
\subsubsection{Tyndall discovers the ability of carbon dioxide to absorb infrared light} \label{tyndal}
In the 1820s---45 years before the American Civil War---Joseph Fourier\footnote{See Spencer Weart's remarkable \textit{The Discovery of Global Warming} website, hosted by the American Institute of Physics, at \href{https://history.aip.org/climate/index.htm}{this link}, last accessed 7 May 2019.} had speculated that visible light from the Sun could easily pass through Earth's atmosphere, while infrared light from Earth could not. Fourier's first hypothesis should come as no surprise to you; otherwise you wouldn't be able to see distant buildings, mountainsides or even the setting sun. But his second hypothesis was far more speculative. After all, no one could see infrared light, or measure it quantitatively. Not until 1859---the same year Oregon became the $33^{rd}$ state---did John Tyndall experimentally confirm\footnote{Weart, op cit, \href{http://www.aip.org/history/climate/co2.htm}{this page}} Fourier's second hypothesis. In a justly famous set of experiments, Tyndall measured the surprisingly robust ability of both water vapor and carbon dioxide gas to absorb infrared light. He also discovered that the two primary constituents of Earth's atmosphere, nitrogen ($N_2$) and oxygen ($O_2$) gasses, do not absorb infrared light to any appreciable degree.\\

We now know that the ability of water vapor and carbon dioxide to absorb infrared light is due to the shape and composition of these molecules. Both molecules have three atoms connected by shared electrons that form bonds. Infrared light happens to have exactly the same energy as that which excites the electrons in the bonds, causing the molecules to vibrate and oscillate. Figure \ref{fig:bending_modes}: shows a simple model of how water and carbon dioxide vibrate and rotate. These movements are possible only because the molecules have three or more atoms of two or more elements. Nitrogen and oxygen gas ($N_2$ and $O_2$ respectively) lack both these properties, so cannot absorb infrared radiation to any appreciable magnitude. Tyndall knew and explained to the global scientific community these behaviors, if not their explanation, over 160 years ago.\\ 
\subsubsection{Raising the temperature}
The energy to vibrate and rotate bonds comes from infrared light hitting the molecules. Within a few millionths of a second after the light is absorbed, the newly energized molecule bumps into another, lower temperature molecule in the atmosphere, transferring some of its energy to that molecule\footnote{Pierrehumbert, R. T, op cit}. In this manner every gas in the atmosphere---even those that can't absorb infrared light---gains energy, and so gets hotter. The greenhouse effect raises the temperature of the entire atmosphere.\\
Within a second of absorbing the light, the warmer greenhouse gas molecule will re-emit some infrared light, its vibration or rotation decreasing in response. The molecule has no ``memory'' of the direction the original light was traveling, so it emits the new infrared light in a random direction, effectively sending half of the light toward space, and half back down toward the ground. The ground absorbs some of that radiation and warms in response, which is why Earth's average surface temperature is warmer than the equilibrium temperature. The greenhouse effect raises the surface temperature, too.\\
\subsection{Earth's energy balance}\label{ghe_1}
If you look back at Figure 3.21, you'll see that the sun radiates energy in the ulatra-violet, visible and infra-red portions of the spectrum. Because ultraviolet and visible light both have shorter wavelengths than infrared light, we can lump both together as \textit{short wave radiation}, or \textit{SWR}. The longer-wavelength infrared light is termed \textit{long wave radiation}, or \textit{LWR}. We also learned in Chapter 3 that each square meter of Earth's upper atmosphere facing the Sun receives on average $\SI{340}{\watt}$ of solar power.  Let's follow the SWR from the Sun as it makes its way from the top of the atmosphere to the Earth's surface (Figure \ref{fig:erb_a}), in three stages of increasing complexity.\\ 
Figure \ref{fig:erb_a} shows $\SI{340}{\watt\per\m^{2}}$ of incoming SWR (yellow arrow in the center) hitting Earth. About 30\% of this light is reflected as SWR (yellow arrows on left), most by clouds and gasses in the atmosphere, some by Earth's surface. This reflected portion is due to Earth's \emph{albedo}, $A$, and amounts to
\begin{align}
	E_{reflected}=Albedo\cdot E_{SWR}\\
	E_{reflected}=0.30\cdot \SI{340}{\watt\per\m^{2}}\\ \label{eq:eref}
	E_{reflected}=\SI{102}{\watt\per\m^{2}}
\end{align}
of SWR reflected back into space. ``Reflect'' is a surprisingly precise word. Look simultaneously at an object and its reflection in a mirror. Note that reflection doesn't change the color, and hence energy, of the reflected light. True to Fourier's hypothesis and Tyndall's discoveries (Section \ref{tyndal}), the incoming \textit{SWR} from the Sun doesn't strongly interact with atmospheric gasses as it travels downward toward the surface. After it reflects from the surface, with its energy, wavelength and color unchanged, the SWR passes unperturbed and unchanged back through the atmosphere and into space. (The tiny portion of the Sun's \textit{SWR} in the ultra-violet spectrum is absorbed about \SI{20}{\kilo\metre} (12 miles) above the surface by ozone, a process we'll cover in more detail in Chapter 8.)\\
 
The 70\% of the Sun's SWR \emph{not} reflected back to space must be absorbed by Earth's surface. This warms the surface to a temperature you already know, $T_{eq}$, \SI{255}{\kelvin}. But Observation 1 in Chapter 3 says that Earth at equilibrium must emit into space just as much energy it receives from the Sun. Hence Earth must emit \SI{238}{\watt\per\m^2} to stay in balance with the incoming solar radiation. 
\begin{align}
	E_{absorbed}=E_{SWR}-E_{reflected}\\
	E_{absorbed}=\SI{340}{\watt\per\m^{2}}-\SI{102}{\watt\per\m^{2}}\\ \label{eq:eabs}
	E_{absorbed}=\SI{238}{\watt\per\m^{2}}
\end{align}
But what color is the light emitted by Earth? The light emitted by blackbody radiators like Earth is directly related to their equilibrium temperature, $T_{eq}$. We can determine the wavelength of Earth's maximum emission using Wein's Law, equation 3.05:
	 
	 \begin{align}
\lambda_{max}=\frac{\SI{2.9e6}{\nano\metre\kelvin}}{T}=	\frac{\SI{2.9e6}{\nano\metre\kelvin}}{\SI{255}{\kelvin}} =\SI{1.14e4}{\nano\metre}=\SI{1.14e-5}{\metre} \label{eq:lambda_earth}
	 \end{align}

Look back to Figure 3.16, and you can see that this wavelength is deep in the infra-red portion of the spectrum. Figure \ref{fig:sun_earth_spectrum} shows the Sun's and Earth's spectra on a single diagram. Note that Earth (in the red tint) emits all of its energy as infrared light, or LWR! Figure \ref{fig:erb_a} shows this radiation (in red as well) heading off to space. This LWR fought a tough battle to escape Earth's atmosphere. We know this because we have thousands of weather stations across Earth's surface, and dozens of satellites in Earth's orbit, measuring the planet's average temperature. This temperature is $T_{av}=\SI{288}{\kelvin}$, far warmer than the $T_{eq}$ we calculated back in Chapter 3. Figure \ref{fig:erb_b} illustrates the problem: At a temperature of \SI{288}{\kelvin}, Earth's surface radiates $E=\sigma (\SI{288}{\kelvin})^4 = \SI{390}{\watt\per\m^2}$, far, far more than it receives from the Sun. What is this source of this energy?\\

Tyndall's discovery explains the origin of this energy. The details are shown in Piece 3, Figure \ref{fig:erb_c}. As the LWR emitted from the surface travels upward, it is absorbed by water vapor, carbon dioxide and other greenhouse gasses in the atmosphere. As we saw above, these gasses hold on to that energy, warming the atmosphere, until they re-emit the energy in a random direction. Roughly half of this energy will head up to space, and roughly half will head down, toward the ground, shown in the figure by the large, ``U''-shaped arrow with magnitude \SI{326}{\watt\per\m^{2}}. Eventually the surface absorbs a good portion of this re-emitted energy, and warms in response. \emph{This is the Greenhouse Effect: Earth's surface is warmed by infrared radiation re-emitted by the atmosphere.} As climate scientist David Randall notes\footnote{Randall, D. (2012). Atmosphere, clouds, and climate. Princeton University Press. Amongst the finest general interest books on climate, the book is an excellent reference for those wanting more details on this aspect of Earth's climate.}, ``It is a very important and perhaps counter-intuitive fact that \textit{Earth's surface actually absorbs more energy from...the atmosphere, than it gains from the Sun!}'' (Randall's well placed emphasis).

\subsection{Three different ways of understanding the Greenhouse Effect}
Understanding the greenhouse effect is central to every topic we examine in this book, and so we'll work through three different, but complimentary, explanations for it. Each is physically correct, and each provides insight into the processes behind this crucial concept.
\subsubsection{Explaining the Greenhouse Effect through radiation from the atmosphere}
You've already read the first explanation in the previous section \ref{ghe_1}: the greenhouse effect as radiation from the atmosphere to the surface. In this view, Earth's surface is warmer than one expects because \emph{both} the Sun and the atmosphere warm the surface. This second, unexpected source of energy to the surface leads to a unexpectedly warmer surface temperature. 
\subsubsection{Explaining the Greenhouse Effect as an atmospheric blanket}
Another way to understand the Greenhouse Effect is to consider Observation 3 from the previous chapter: ``Anything that slows the loss of heat from a warm object will keep the object warmer, longer.'' Greenhouse gasses in the atmosphere do just this. They delay the flow of heat (the LWR) off the planet just as a blanket delays heat from escaping a body. A difference exists between the atmosphere and a blanket in this analogy. A blanket is equally good at keeping heat out as in, as anyone who has used an oven mitt when picking up a hot pan can attest (the mitt slows the flow of heat from the pan to your hand). The atmosphere's quality as a blanket is asymmetric, allowing incoming SWR through, but blocking outgoing LWR, illustrated in Figure \ref{fig:sun_earth_spectrum}.\\
The middle panel shows the relative amount of light emitted by the Sun (in yellow) and Earth (in red) as a function of wavelength. The higher the curve, the more light (and energy) is emitted at that wavelength. The figure shows that the Sun radiates largely SWR, and Earth LWR. The gray areas in the middle panel depict how effectively Earth's atmosphere transmits radiation at each wavelength: the lower the gray area, the less absorbing---and more transparent---is Earth's atmosphere to that wavelength of light. The long slow upswing in absorption seen at the blue end of the visible spectrum is due to scattering of light by gas molecules in the atmosphere, and largely explains why the sky appears blue and the setting sun red. Our atmosphere is of course transparent to visible light, but is totally opaque to ultraviolet light with wavelengths less than \SI{300}{\nano\metre}. At the other end of the visible spectrum, a series of absorption bands (gray in the figure) appear at about \SI{750}{\nano\metre}, and become higher and wider to \SI{5000}{\nano\metre}. The upper panel shows these absorption bands are due to water vapor molecules in the atmosphere. Carbon dioxide contributes as well, including the significant absorption band near \SI{15000}{\nano\metre}. This band is particularly important, as it lies on the ``shoulder'' of Earth's emission and absorbs a significant portion of Earth's LWR. All those absorption bands add up to the Greenhouse ``blanket,'' effectively slowing the movement of heat from the surface to space, warming the surface \textit{via} the Greenhouse effect.\\

\subsubsection{Explaining the Greenhouse Effect as a result of an atmospheric escape threshold}
In Chapter 3 we carefully constructed the idea of an equilibrium temperature, which we calculated is $\SI{255}{\kelvin}$ for Earth. This equilibrium temperature has to be correct: the energy Earth receives from the Sun has to be the same energy Earth emits back to space. How can this happen if the surface is warmer than $\SI{255}{\kelvin}$? The answer must be that somewhere in the atmosphere Earth has an apparent temperature of 255 K, and that this level of the atmosphere is what ``sees'' space. This \emph{emission level} in the atmosphere will be where infrared light leaving a greenhouse gas molecule has a decent chance of escaping the atmosphere, of reaching space before it is absorbed by another greenhouse gas. The situation is exactly analogous to walking through a thick forest. As you look toward the edge of the forest (Figure \ref{fig:forest_trees}, Panel A), your sight lines to the meadow beyond are blocked by tree trunks: you can't see the edge of the forest for the trees. But as you get closer to the forest's edge (Figure \ref{fig:forest_trees}, Panel B), and fewer trees lay between you and the edge, your chances of glimpsing the sky beyond the forest increase. Eventually the trees are sufficiently rare that you can see the sky. This is the emission level of the forest, the distance into the forest that you can see the sky beyond. The same idea applies to the atmosphere: sufficiently high above Earth's surface, infra-red light can ``see'' space and escape the planet.\\

Figure \ref{fig:emission_heights} shows the variation in emission heights for infra-red light in a mid-latitude winter's atmosphere, when the sky is cloudless and blue, much like it was when this sentence was written. Between \SI{7500}{\nano\metre} and \SI{12500}{\nano\metre} Earth's atmosphere is transparent to most radiation and so emission from a few hundred meters above the surface can make its way to space. In essence, the forest here is thin. But the large \textit{plateau} at 15,000 nm again marks the strong absorption peak for $CO_2$. Such is the ability of $CO_2$ to absorb this particular range of infrared radiation that light can't escape to space until it is above 75\% of the atmosphere. Here the forest is dense. The emission level in the atmosphere averages about \SI{5.5}{\kilo\metre} (3 miles), although in the figure it lies at about \SI{6.6}{\kilo\metre} km (3.6 miles). Above this level infrared light largely escapes the planet; below this level we expect infrared light is absorbed and warms the air. This warmer air in turns keeps the surface warmer and once again we understand the Greenhouse Effect. 

\subsection{Earth's Average Surface temperature}
We can now write the most important equation in this book, one so simple that its importance may seem misplaced:
\begin{align}
	T_s&=T_{eq}+\Delta T_{GHE}\\
	\SI{288}{\kelvin}&=\SI{255}{\kelvin}+\SI{33}{\kelvin}
\end{align}
The equation states that Earth's average surface temperature, $T_av$ is equal to the sum of the equilibrium temperature, $T_{eq}$, and an increment from the Greenhouse Effect, $\Delta T_{GHE}$. The current magnitude of the natural Greenhouse Effect is $\Delta T_{GHE}=\SI{33}{\kelvin}$, or $60^{\circ}$F. Earth's equilibrium temperature is about $0^{\circ}$F, the average temperature of Fairbanks, Alaska. In February\footnote{Mean temperature climate normals from the 30-years 1981-2010, averaged over Fairbanks AP\#2, International Airport, and Midtown stations, provided by the National Climatic Data Center, \textit{via} \href{http://climate.gi.alaska.edu/Climate/Normals}{link}, accessed June 30, 2015.}. Although the feedbacks between life and climate are complex, the natural Greenhouse Effect undoubtedly makes the planet habitable for life as we know it.\\ 
We saw in Chapter 3 that the equilibrium temperature is itself due to the insolation $S$ we receive from the Sun and Earth's albedo $A$, and we saw above that the Greenhouse Effect is a function of the amount and type of greenhouse gasses in our atmosphere:
\begin{align}
	T_s&=T_{eq}+\Delta T_{GHE}\\ \label{eq:Ts}
	T_s&=\left[\frac{S(1-A)}{4 \sigma}\right]^{1/4} + f(H_2O, CO_2, CH_4, N_2O)_{atm}
\end{align}
The Sun's output $S$ and Earth's albedo $A$ change naturally on scales of billions of years to months. The type and amount of greenhouse gasses in the atmosphere naturally change on similar timescales, a variation we will look at closely over the course of this book.\\ 
Humans can't change the Sun's output, and can change Earth's average albedo only slowly, but we sure can change the amount and kind of greenhouse gasses in the atmosphere! Over the past 250 years or so, we have increased the effective concentration of such gasses in the atmosphere\footnote{http:\/\/www.esrl.noaa.gov\/gmd\/aggi\/aggi.html, accessed 9 May, 2019, and extrapolated to 2020 CE using a linear regression of equivalent $CO_2$ concentration over time. Uncertainty in the estimate is $<1\%$ point.} by 80\%. In response, Earth's average surface temperature has increased\footnote{Field, Christopher B., et al. ``Summary for policymakers.'' Climate change 2014: impacts, adaptation, and vulnerability. Part a: global and sectoral aspects. Contribution of working group II to the Fifth Assessment Report of the Intergovernmental Panel on Climate Change (2014): 1-32.} by at least \SI[separate-uncertainty]{0.92(07)}{\kelvin}, or \SI[separate-uncertainty]{1.7(4)}{} $^{\circ}$F .

\section{$CO_2$ as Earth's thermostat}
The four principal GHG are water vapor, carbon dioxide, methane and nitrous oxide. All of them are minor components of the atmosphere, but produce the vast majority of the Greenhouse Effect. While water vapor will turn out to be the largest contributor, we'll see that it is not the most climatologically important greenhouse gas. That role belongs to carbon dioxide. Table \ref{tab:ghgconcs} and Figure \ref{fig:annghg} show the more important aspects of these greenhouse gasses, including their recent (and projected future) changes. 

\subsection{The four principal greenhouse gasses}

\subsubsection{Water vapor}
Water is unique: no other substance can be found in solid, liquid and gaseous form on Earth's surface. Water vapor is the gaseous form of water, and it generally exists dissolved into air, just like salt dissolves in water. On average, water vapor makes up just 0.2 to 0.5\% of the atmosphere\footnote{The lower limit is an area-weighted global annual average 1000-300 mb specific humidity, 1981-2010 Climatology from NOAA NCEP-NCAR CDAS-1: Climate Data Assimilation System I; NCEP-NCAR Reanalysis Project, via IRI maproom http://iridl.ldeo.columbia.edu/SOURCES/.NOAA/.NCEP-NCAR/.CDAS-1/.mc8110\/, accessed June 30, 2015; upper limit is from T.J. Blasing, Recent Greenhouse Gas Concentrations, http://cdiac.ornl.gov/pns/current\_ghg.html, accessed 30 June 2015.}, more in the tropics, and less in the poles. As that observation suggests, the concentration of water vapor in the atmosphere is strongly proportional to temperature, rising about 6\% for every \SI{1}{\kelvin} temperature increase (3.5\% for every $1^{\circ}F$ increase). Water vapor constantly moves across the globe (Figure\ref{fig:wv_vis}) due to atmospheric circualtion, and just as constantly cycles from the atmosphere to the oceans, cryosphere and geosphere, and back in to the atmosphere. Because of its raw abundance, water vapor provides about half of the total greenhouse effect (precise values are given in Table \ref{tab:ghgconcs}).

\subsubsection{Carbon Dioxide}
Macbeth is the star of Shakespeare's shortest and darkest play, \textit{The Tragedy of Macbeth}. During the play Macbeth savagely slaughters his King, his best friend, the wife and child of his kinsman, a few of his own guards, and drives his Queen to suicide. He's an odd ``hero'' for a play. Carbon dioxide is the Macbeth of this book. It's the central character, the moody and ever-present actor, in climate science, and though of course it lacks Macbeth's moral flaws, carbon dioxide certainly can play the antihero. Carbon dioxide is ubiquitous, found in various forms in the atmosphere, hydrosphere, geosphere, biosphere and even---trapped as tiny, ice-surrounded bubbles---in the cryosphere. We'll spend all of Chapter 11 understanding the crucial role of carbon in Earth's climate. For now, we need only to know that carbon dioxide makes up just [409, 412, 414, 416, 419] parts per million (abbreviated ppm) of the atmosphere [in mid-year 2019, 2020, 2021, 2022, 2023 respectively]\footnote{I don't intend that the book actually have this terminology; rather I include this so the book is up-to-date when published, and I don't have to continually recalculate the value! The extrapolation here uses NOAA's global $CO_2$ ``trend'' data (\href{ftp://aftp.cmdl.noaa.gov/products/trends/co2/co2_mm_gl.txt}{Link here}, accessed 11 May 2019, complete from Jan 1980 to Feb 2019), fit to a quadratic model in time, and extrapolated to mid years. 95\% CI on the extrapolations are less than 0.3 ppm}.\\
A \emph{part per million} is a common way of expressing the concentration of trace constituents, and in this case refers to the number of molecules of carbon dioxide found in one million molecules of ``air.'' One way of envisioning the ppm unit is to think of \$10,000, all in pennies. Take \$4 worth of those one million pennies, paint them red, then mix them into the other \$9,996. Those red pennies are 400 ppm of the total. The terminology is so common in climate studies we need a handy way of writing ``the concentration of this molecule in this place is this,'' which we do this way:
\begin{equation}
	[CO_2]_{atm}=412\ ppm
\end{equation}
The square brackets are read as ``concentration of,'' the chemical formula in the brackets identifies the compound of interest, and the subscript gives the particulars of where (or when) the concentration was measured.


\begin{table} 
\centering
\caption{Greenhouse gasses in Earth's Atmosphere}
\label{tab:ghgconcs}
\begin{tabular}{@{}lccccc@{}} \toprule
Parameter				&Units		    &Water		  &Carbon Dioxide		&Methane	&Nitrous Oxide\\ \midrule
Formula					&             &$H_2O$		  &$CO_2$      			&$CH_4$		&$N_2O$\\
Concentration, 2020	  &ppm	  &2000-5000	&412	            &1.880	  &0.332\\
Concentration, 1750	  &ppm	  &2000-5000	&280	            &0.722	  &0.27\\
Average Annual Change	&\%/yr  &0	        &0.6	            &0.4	    &0.3\\
GWP (over 100 yrs)	  &-	    &0.2	      &1	              &28	      &265\\
Lifetime	            &years	&0.03	      &$\sim 100-1000$	&$\sim10$	&$\sim120$\\
Proportion Greenhouse Effect	&\%		&57	  &23	              &6	      &2\\ 
Proportion US emissions	&\%	  &0	        &82	              &10	      &5\\
Condensation Temperature& $^{\circ}C$ &0	  &-78	            &-187	    &-88\\ \bottomrule
\end{tabular}
\end{table}

The lifetime of carbon dioxide in the atmosphere is anywhere from 100 to 1000 years, depending on how one defines the lifetime. We'll define it as the time before a molecule is removed from the atmosphere by any process. The implications of this should surprise you: next time you use a gasoline-powered vehicle, step back and look at the exhaust before you get in the vehicle. On average\footnote{http://www.rita.dot.gov/bts/sites/rita.dot.gov.bts/files/publications/national\_transportation\_statistics/html/table\_01\_26.html\_mfd accessed 30 January 2016 shows that average age of US cars is 11.4 years in 2014, unchanged for 3 model years.  Alson, J., Hula, A., \& Bunker, A. (2014). Light-Duty Automotive Technology, Carbon Dioxide Emissions, and Fuel Economy Trends: 1975 through 2013. Appendix F, US Environmental Protection Agency, Ann Arbor, Michigan, available at: http://www.epa.gov/oms/fetrends.htm\# accessed 30 January 2016 report that average $CO_2$ emissions for the US average fleet is (for the 1990-2014 period, weighted by a triangular distribution with mean = 2014-11.4=2002.6) is 420 g $CO_2$\/mile. Assume average speed of 33 mph=0.55 mile\/minute=230 g\/minute}, for every minute you drive, that car will emit about 230 g---roughly \emph{one half pound}---of carbon dioxide. That gas will remain in the atmosphere at least until your great-grandchildren are breathing it, and probably much longer. The growth of carbon dioxide in the atmosphere-largely due to human-caused emissions-amounts to about 0.6\%, or roughly 2.5 ppm, per year\footnote{This is the exponentially smoothed 2018 growth rate (alpha=0.25) for $CO_2$ based on the NOAA's ``measured global annual mean dry-air mole fractions used in deriving the radiative forcing values provided in ... the AGG,'' available at https://www.esrl.noaa.gov/gmd/aggi/NOAA\_MoleFractions\_2019.csv}.\\

Despite its tiny abundance, carbon dioxide is responsible for about one quarter of the entire greenhouse effect. This is because Earth happens to radiate infrared light at precisely the same wavelengths carbon dioxide absorbs it (Figure \ref{fig:sun_earth_spectrum}), and because carbon dioxide is an efficient absorber of that infra-red radiation. A decent way of measuring the ability of a greenhouse gas to influence climate is through the \emph{global warming potential}, or GWP. This is a relative measure, and by convention the GWP for carbon dioxide is set at 1. The GWP of water, although it is hard to measure accurately\footnote{Because their absorbencies overlap, the relative contributions of water and carbon dioxide to the total GHE is indeterminate. Students find this enormously frustrating. So I calculate and approximate measure by dividing actual GHG concentrations by the CDAIC AGGI contributions from each GHG, then normalizing those ratios to that for $CO_2$. These values were within factor of ~2 of IPCC 100 year GWP. The corresponding ratio for H2O is 0.2, relative to $CO_2$. I fully acknowledge this is at best an approximation, and at worst a pedagogical extrapolation. But I think the cost in terms of precise physics is offset by the increased accuracy of students' understanding.}, is about one fifth that of carbon dioxide. Roughly speaking, one molecule of carbon dioxide has 5 times the potency of one water molecule to warm the climate. Water is more important overall because there is ten times more water vapor in the atmosphere than carbon dioxide.\\
\subsubsection{Methane}
Methane---a common molecule found from Earth's deep interior, to the surface to the atmosphere---is a simple molecule consisting of a central carbon atom and 4 surrounding hydrogen atoms. The majority of natural methane emissions come from wetlands\footnote{Anderson, Brian, et al. \textit{Methane and nitrous oxide emissions from natural sources.} United States Environmental Protection Agency, Office of Atmospheric Programs, Washington (2010).}, where it is released as a byproduct of oxygen-free decay of organic matter. Bubbles of methane rising from such places has the wonderful name of ``swamp gas.'' But these natural emissions now represent only 40\% of global emissions. The other 60\% comes from human activity, including the production of natural gas and coal, the gaseous discharge from the intestines of domesticated animals (largely cattle), and from rice paddies. Natural gas, which is an increasingly important source of energy in the United States, is largely methane. While the contemporary concentration of [1.87, 1.88, 1.89, 1.90, 1.91] ppm in mid-year [2019, 2020, 2021, 2022, 2023 respectively] has increased recently (around 0.4\% per year.\footnote{NOAA, ibid.}), the rate of increase is quite variable over 5 to 10 year periods. The reasons for this are still hotly debated, but include increases  in the methane released by microbes (often found in rice paddies and the guts of cattle), from increased fossil fuel production, and a decrease in the rate methane is removed from the atmosphere, or a combination of all three\footnote{Nisbet, E. G., et al. (2019), Very Strong Atmospheric Methane Growth in the 4 Years 2014-1017: Implcations for the Paris Agreement. Global Biogeochemical Cycles 33(3), \href{http://doi.org/10.1029/2018GB006009}{This doi}}. Over the past 250 years, humans have more than doubled atmospheric methane concentrations, Methane has a GWP of 28, so despite its low concentrations it is a powerful greenhouse gas, fueling about 6\% of the greenhouse effect. Once in the atmosphere, methane is destroyed over a 10 year period by reactions with naturally-occurring compounds in the atmosphere. \\
\subsubsection{Nitrous oxide}
Yes, laughing gas is a greenhouse gas, the least influential of the top four gasses we'll consider in detail. This gas is still rare-only [0.332 0.332 0.333 0.334 0.335] ppm in mid-year [2019, 2020, 2021, 2022, 2023 respectively\footnote{NOAA, ibid.}] and has only increased about 25\% since pre-industrial times. This low increase reflects the fact that almost two thirds of all nitrous oxide reaching the atmosphere is from natural sources, mostly from bacterial activity in soils and the oceans. This also explains the gas's relatively slow increase of 0.3\% annually. Growing food\footnote{Donald J. Wuebbles, Nitrous Oxide: No Laughing Matter, Science 2 October 2009: 326 (5949), 56-57. [DOI:10.1126/science.1179571]}, aided by nitrogen-rich application of fertilizers and manures, is the primary source of anthropic nitrous oxide, although burning of fossil fuels contributes a small amount as well. Nitrous oxide is removed from the atmosphere by bacteria, or destroyed by chemical reactions and ultraviolet light. With a long lifetime (120 years) and remarkably high GWP (265), nitrous oxide provides a small but important 2\% of the GHE.\\
\subsubsection{Greenhouse Gasses and Human Needs}
Buried in the dry recitation of facts above are hints of ideas we will revisit often, but should start considering now. Look back and notice that both methane and nitrous oxide are released through agricultural practice: bovine emissions and rice paddies for methane, fertilizer for nitrous oxide. Of course in most parts of the developed world, these crops and animals are cared for, harvested and transported by vehicles consuming fossil fuels and emitting carbon dioxide, the other important GHG. Something as necessary and fundamental to existence as growing food is a primary source of the very gasses that are threatening our way of life. This could be quite a dilemma: could it really come down to choosing to heat or to eat?\\

\subsection{Feedbacks}
Back in Chapter 3 we defined a system as a collection of interacting parts. One important type of interaction in climate systems are feedbacks, in which one part of the system affects another part, and that one yet another, and so on until the initial part itself is affected. Such feedback loops are common in any complicated system, including living organisms, human societies, and of course climate.\\
\subsubsection{Block diagrams and mechanics}
A handy way of visualizing such a loop is a block diagram, as shown in Figure \ref{fig:feedback_loops_intro}. Block diagrams represents parts of the system as blocks, and interactions between the parts as lines. Inputs to a block represent interaction signals from other parts of the system, while outputs are resulting signals sent to other parts. The signal represents how change in one part of the system will change the next part along the loop. Signals have one of two values: positive or negative. A positive signal means the two parts will change in the same sense: if one goes down, the other will, too. In this book, positive signals are shown in block diagrams with green lines ending in a filled circle. A negative signal means the parts change in the opposite sense: if one increases, the other will decrease, and \textit{vice versa}. Negative signals are shown with red dashed lines ending in an unfilled circle, as shown in the center of Figure \ref{fig:feedback_loops_intro}. (A signal of ``no change'' is possible, but irrelevant; such a signal would mean the parts really don't interact!) 

\subsubsection{Positive and Negative Feedback Loops}
Interaction is the primary reason we'll study feedback loops, because loops interact to either amplify or dampen changes that occur to a system. Two examples illustrate this crucial concept. Collegiate wrestlers compete in narrowly-defined weight classes, so a wrestler typical monitors her weight closely. A very simplified feedback loop of the system \textit{Wrestler} is shown in the Panel A of Figure \ref{fig:feedback_loops_intro}. Jump in to the loop at \textit{Calories Consumed} and follow the positive signal output to \textit{Body Weight}. This indicates that if a wrestler were to increase her caloric intake while keeping all other factors constant, her weight would increase. The opposite is also true: fewer calories would lead to weight loss. This is a \emph{positive} interaction, because the changes have the same sign. Follow the negative output from \textit{Body Weight} to \textit{Caloric Intake} and you'll see that in response to her weight gain, the wrestler would need to reduce her subsequent caloric intake, which (follow the loop around!) would eventually lead to weight loss. Note how this loop acts to dampen, or reduce, weight changes and leads to the maintenance of a steady state weight. Any complicated system which remains near a steady state condition for the long term is dominated by such \emph{negative feedback loops}. Don't let the word \textit{negative} fool you: it refers not to the goodness or desirability of the outcome, but to the fact that the scale of change is reduced on each passage through the loop.
Not surprisingly, positive loops amplify change on each passage. In the upper right of Figure \ref{fig:feedback_loops_intro} is a simplified block diagram of a Student. The student, for whatever reason, reduces his exercise intensity for a period of time. As you may have experienced, this leads to a decrease in the student's conditioning, which in turn leads to a further reduction in the student's exercise intensity. Eventually the student may find himself in pretty poor shape, and stop exercising all together. Despite the unfortunate outcome, this is a positive feedback loop, because the change grows larger with each trip around the loop. Positive loops are inherently unstable: even a small initial change can lead to arbitrarily large changes over time.\\
Determining the nature of a loop is easy: count the number of negative signals. If a loop has an odd number of negative signals, the loop is itself negative. The lower panel of Figure \ref{fig:feedback_loops_intro} shows changes in the wrestler's weight (in red) and student's conditioning (in green) over time. While the wrestler's weight increases suddenly and dramatically, her weight returns quickly to the initial weight after a few trips around the loop, even if it overshoots that steady state value on the return. On the other hand, the student's conditioning initially drops only slightly, but that change is rapidly amplified.\\
For nearly 4 billion years, Earth's climate, while varying widely for short periods of time, has been sufficiently stable and comfortable for life to flourish through the oceans, across the surface, into the atmosphere and even tens of kilometers deep in the geosphere. What a remarkable example of long-term maintenance of a steady state in a complicated system! Many strong, persistent and overlapping negative feedbacks have maintained a habitable climate on Earth for billions of years. Oddly enough, though, the first feedback we need to look at is a positive one, and involves carbon dioxide, water and the greenhouse effect.

\subsection{The Role of Carbon Dioxide and other Non-Condensable Gasses in the greenhouse effect}
Figure \ref{fig:schmidt_hills} was taken in the Schmidt Hills of Antarctica, and shows the frozen Polar Plateau leading off to the Ronne Ice Shelf in the background. At the time the photograph was taken, the air temperature was a balmy \SI{-5}{\celsius} ($20^{\circ}$ F), pretty mild for Antarctica. Today the ice there is a thousand meters (more than 3000 feet) thick, but the smoothness of the rocky ridge in the background suggests that the ice at one time covered the ridge as well. The ice is nothing more than compacted snow, but to the surprise of most people, snow in Antarctica is actually pretty rare, simply because the air is too cold to hold much water. Antarctica is a unique and spectacular part of planet Earth, but much of the planet's surface would look like the landscape in Figure \ref{fig:schmidt_hills} if it weren't for the effects of carbon dioxide as a \emph{non-condensable gas}: a gas that doesn't freeze to a solid on the surface. 
\subsubsection{Specific humidity \textit{vs.} temperature}
To act as a greenhouse gas, water must dissolve into the air. This process is similar to the dissolution of salt in water. The dissolved salt is invisible in the water, at least to the naked eye. There is also a limit to how much salt a given amount of water can dissolve, which defines the solubility of salt in water. Water in air behaves much the same, both invisible to the naked eye and having a well-defined maximum solubility. Humidity is a measure this amount, and it is a strong function of temperature. Warmer air has a higher solubility for water than colder air, with a $\sim 6\%$ increase in solubility for each \SI{1}{\kelvin} increase in temperature\footnote{Lalibert\'{e}, F., Zika, J., Mudryk, L., Kushner, P. J., Kjellsson, J., \& D\"{o}\"{o}s, K. (2015). Constrained work output of the moist atmospheric heat engine in a warming climate. Science, 347(6221), 540-543; Allan, R. P., \& Soden, B. J. (2008). Atmospheric warming and the amplification of precipitation extremes. Science, 321(5895), 1481-1484.}. Figure \ref{fig:relative_humidity} shows the relationship between air temperature and water solubility. The blue curve shows the maximum amount of water dissolved in air, and defines air at 100\% relative humidity. Relative humidity, that standard of weather forecasts, is just the proportion of the maximum possible water actually present in the air. For example, the red dashed line on Figure \ref{fig:relative_humidity} shows 50\% relative humidity, a reasonably comfortable day. The orange dotted line shows 25\% relative humidity, which most people would find uncomfortably dry. Note how water solubility plummets as air temperature decreases. As air temperature approaches \SI{-40}{\celsius}, solubility approaches 0, which means that no water vapor exists in the air. At these cold temperatures, typical of Antarctica for example, there can be no snow: no water can get into the air to freeze!\\

At Earth's $T_{eq}$, (\SI{-18}{\celsius} left arrow on the figure), water vapor in the air is exceptionally small, less than a tenth of the average amount today\footnote{To good approximation the saturation water vapor pressure is given by $p_{wv}=A \cdot 10^{\frac{B \cdot T}{C+T}}$, where $p_{wv}$ is in\si{\pascal}, T is temperature in \si{\celsius}, and $A,\ B \ and \ C$ are \SI{611}{\pascal}, 7.5, and \SI{237.7}{\celsius} respectively. Thus the ratio is $\frac{10^{\frac{B \cdot T_{eq}}{C+T_{eq}}}}{10^{\frac{B \cdot T_{av}}{C+T_{av}}}}\sim 11.5$}. This poses a problem. If water is responsible for over half of the greenhouse effect, how did it get into the air in the first place? Wouldn't temperatures be too cold to get any water into the air? Enter the non-condensable gasses. As Table \ref{tab:ghgconcs} shows (bottom row) carbon dioxide, methane and nitrous oxide have condensation temperatures far colder than Earth's $T_{eq}$. Those gasses remain in the atmosphere regardless of surface temperature, and produce a sufficient greenhouse effect to get \emph{some} water vapor into the air (Figure \ref{fig:wvco2}). This in turn increases the greenhouse effect, raising the temperature even more. This positive feedback loop, sketched in Figure \ref{fig:wvco2} is a fundamental and crucial mechanism in the maintenance of Earth's climate.

\subsubsection{And a powerful feedback it is}
Think of carbon dioxide as Earth's thermostat, a wonderful term\footnote{The first use of which I can find is Oestreicher, D. (1973). \textit{Growing, Crowding Population Hiking Earth's Thermostat}. Detroit Free Press, Tuesday, June, 12(1), 973.} that evocatively summarizes the importance of carbon dioxide to climate, society, and this book. Thermostats don't heat or cool a room, they merely control the mechanism that does. So too with carbon dioxide and the other non-condensable greenhouse gasses we discussed, which control the greenhouse effect through the water vapor feedback.
Ideally we would conduct careful, controlled experiments on this relationship, but of course this is impossible. To do a controlled experiment, as might be done in chemistry or biology, you need to prepare at least two identical systems. One system is left unchanged and acts as a control, a measure of the normal behavior of the system. The other systems are then modified, and the difference between the control and experimental systems reveals the effects of the modifications. Earth sciences are unique in that such controlled experiments are often impossible: we have, after all, only one Earth. Earth scientists often result to using natural experiments to understand how Earth works. Over Earth's history $[CO_2]_{atm}$ has varied from 1000s to perhaps 200 ppm, and that enormous range has been accompanied by equally enormous changes in surface temperatures. These natural experiments confirm the strength of the water vapor feedback, but don't provide a precise view of the relationship between $[CO_2]_{atm}$ and the average surface temperature.\\

For the past 250 years, we have been conducting a real, but uncontrolled, experiment on the relationship between $[CO_2]_{atm}$ and the average surface temperature., by increasing GHG concentrations by about 80\%. The resulting global warming provides clear evidence of the water vapor feedback, but using this relationship seems premature. Fortunately there is another way. A third and powerful technique for understanding how carbon dioxide drives climate is to use computer models. Such models range from simple ones using a single mathematical approximation to global models using over half a million points at dozens of levels in atmosphere and ocean, connected by thousands of mathematical relationships. The strength of these models is the ability to do exactly the controlled experiments we can't do on the real Earth. In a computer model, changing the concentration of a gas, or the albedo of a continent, is straightforward, although determining the accuracy of the results is not. In a beautiful example of the usefulness and power of this method, researchers from NASA\footnote{Lacis, A. A., Schmidt, G. A., Rind, D., \& Ruedy, R. A. (2010). Atmospheric CO2: Principal control knob governing Earth's temperature. Science, 330(6002), 356-359. Figures are reproduced from http://www.giss.nasa.gov/research/briefs/lacis\_01/, accessed 3 August, 2015} twiddled Earth's thermostat by removing (in the model) non-condensable greenhouse gasses from the atmosphere. 

\subsubsection{Why water vapor is a response, not a driver, of global climate change}
Figure \ref{fig:noco2earth} shows the remarkable results, with colors showing the average annual temperatures at each latitude. Areas in reds and yellows have average annual temperatures above freezing; areas in blues have annual temperatures below freezing. Results to the left of ``0'' time nicely reproduce actual global temperatures.  As soon as condensable gasses are removed at 0 years, surface temperatures plummet, quickly, globally, and solely because carbon dioxide is gone. Average global temperatures decrease by about what you should expect: \SI{35}{\celsius} ($63^{\circ}$F), and do so with a half-life of just 3.5 years. In the first 3.5 years, this model predicts that global temperatures would decrease by \SI{17}{\celsius} ($31^{\circ}$F), or roughly \SI{5}{\celsius} ($9^{\circ}$F) per year! Water vapor may be the most important of the greenhouse gasses in our atmosphere, but it holds that position only at the invitation of the other, non-condensable, greenhouse gassess, like carbon dioxide. As suggested by the box model in Figure \ref{fig:wvco2}, and confirmed in the computer models illustrated in Figure \ref{fig:noco2earth}, water vapor in our atmosphere is a \emph{response}, not a driver, of global climate change. This is a topic to which we will return: as you can imagine, increase in atmospheric water vapor will propagate into changes in precipitation and water supply as well. 

\subsection{How a tiny bit of human-emitted $CO_2$ can change a planet's temperature}
Typically people do not use huge blocks of ice or open fires to change the temperature of a room. True, you might open a window, but most of us live in buildings where the temperature is controlled by a thermostat, sitting quietly in a corner of the room. One merely flicks a button, and the (now) computer-controlled thermostat will send signals to the heating or cooling system of the building, which heats or cools air which is subsequently blown into the room. Remarkably little energy is expended by you in the act of changing the room's temperature. The thermostat acts as a (metaphorical) lever, magnifying the forces you exert into room-wide changes in climate. Anthropic release of carbon dioxide works in precisely the same way. Because of its power as a greenhouse gas, carbon dioxide and other gasses we emit into the atmosphere directly change Earth's surface temperature. But these effects are leveraged by the $CO_2-H_2O$ feedback into even larger changes, changes which have led to a roughly \SI[separate-uncertainty]{0.92(6)}{\celsius} ($1.7^{\circ}$F) increase in global surface temperatures\footnote{IPCC, 2013: Summary for Policymakers. In: Climate Change 2013: The Physical Science Basis. Contribution of Working Group I to the Fifth Assessment Report of the Intergovernmental Panel on Climate Change [Stocker, T.F., D. Qin, G.-K. Plattner, M. Tignor, S. K. Allen, J. Boschung, A. Nauels, Y. Xia, V. Bex and P.M. Midgley (eds.)]. Cambridge University Press, Cambridge, United Kingdom and New York, NY, USA., page 3, updated to the 2009-2018 decade by averaging the GISS (GISTEMP Team, 2019: GISS Surface Temperature Analysis (GISTEMP). NASA Goddard Institute for Space Studies. Dataset accessed 2019-05-29 at https://data.giss.nasa.gov/gistemp/) average surface data. I took the difference between the average GISS 2003-2012 temperatures (the period used in the SPM) and the GISS 2009-2018 temperatures, and added it to the SPM value of 0.78 C.} since the 1850-1900 CE period. These and other changes will continue in the future; understanding the tempo and mode of which will occupy the final few chapters of the book.\\

Before that, we need to have a baseline from which we can measure and understand differences and changes in climate. So, let's turn to our final topic for this chapter, why the Tropics are hot and the Poles are cold.

\section{Why are the Tropics Hot, and the Poles Cold}
\subsection{A Brief Review}
\subsubsection{Two important characteristic temperatures}
In the Chapters and Sections above we found that we could predict an equilibrium temperature for the Earth as a whole using the Stefan-Boltzman Law, finding $T_{eq}=\SI{255}{\kelvin} = \SI{-18}{\kelvin}= -1^{\circ}F$. This temperature is a manifestation of the proportion of the Sun's insolation---short wave radiation---absorbed by Earth. We realized this temperature was far lower than Earth's actual surface temperature, and discovered (Equation 4.04) that the Greenhouse Effect raises Earth's average global temperature to $T_{av}=\SI{288}{\kelvin} = \SI{15}{\kelvin}= 60^{\circ}F$. This change was due to the interrupted flow of heat---as long wave radiation---from Earth to space by greenhouse gasses in the atmosphere. As you might have guessed, these two temperatures, important as they are, are insufficient to describe Earth's climate. Unless you live in a sealed, air conditioned room, you know that there are seasonal and diurnal variations in temperature at one place, as well as temperature variations at the same time from place to place. These temporal and spatial variations are due to temporal and spatial variations in the amount of energy received and absorbed by Earth; understanding their origin and scale will prepare us for understanding the crucial role the atmosphere and oceans play in transporting heat and moderating climate across the globe.
 
\subsection{Climate Averages}
\subsubsection{A brief geographical interlude: latitude and longitude}
Understanding Earth's climate requires that we understand Earth's natural directions and geography. As a rotating sphere, Earth naturally has two Poles, North and South, and an Equator half-way between. Figure \ref{fig:globe} is a map of the Earth in Robinson projection, which nicely presents all parts of the globe with relatively little distortion, except near the poles. On the Robinson projection, the North and South Poles, which are really points, are spread into the upper and lower blues lines on the map. Lines of latitude are actually circles (shown in blue), parallel to the Equator and indicate the angle from the equator to the given parallel of latitude measured north or south of the Equator. The contiguous 48 states of the United States fall, for example, between $\ang{49.5}$ N and $\ang{24.5}$N, although Alaska and Hawaii are respectively much further north and south of these limits. The North and South Poles are at latitudes $\ang{90}$N and $\ang{90}$S, and the Equator is at $\ang{0}$. A number of parallels have significant climatological and cultural significance, and are shown and labeled on the map. \\
Establishing an east-west coordinate system is more difficult; lines of longitude run from pole to pole, but one such line is no more natural a starting point than another. By international agreement, one fraught with intrigue and mathematics, 25 nations decided somewhat arbitrarily in 1884 CE to set the line of longitude running through a particular telescope at the Greenwich Observatory, near London, England, as the starting point for longitude\footnote{Steel, D. (2001). Marking time: The epic quest to invent the perfect calendar. John Wiley \& Sons., pg. 269}. Longitudes increase from $\ang{0}$ to $\ang{180}$E eastward from London, and from $\ang{0}$ to $\ang{180}$W to the west of London. The contiguous United States fall between $\ang{67}$W (West Quoddy Head, Maine) and $\ang{125}$ W (Cape Alava, Washington).\\
\subsubsection{``Latitudes however, are more significant for the diversity of lands than longitudes''}
In terms of locating a place on Earth, latitude and longitude are equal. Not so in terms of climate. Over 2500 years ago, the Greek philosopher Parmenides\footnote{G\'{o}mez, N. W. (2008). The tropics of empire: why Columbus sailed south to the Indies. Cambridge, MA: MIT Press, pg. 72} was aware that climate varied more by latitude than by longitude, an observation best summarized by the great Muslim polymath Averro{\:e}s (1126-1198 CE) in the quote opening this section\footnote{Ibid, pg. 1}. Directions on Earth are climatologicaly important.\\
Parmenides called areas of similar latitude \emph{zones}, and so east-west directions and averages are termed \emph{zonal}, as shown on Figure \ref{fig:globe}. For the subsequent 25 centuries western science has divided Earth into 5 major zones, symmetric to the Equator. They are labeled in bold letters on the map: two Polar zones, two Temperate zones, and a single Tropical zone. Parmenides, and a long train of geographers after him, used the Sun's apparent position at key seasonal points to define the boundaries. In the polar zones, the Sun does not rise at least one day per year, and does not set at least one day per year. The Arctic and Antarctic circles (dashed lines at $\ang{66.6}$ N and S) mark the equator-ward extent of these areas. In the Tropics, the Sun passes directly overhead at least one day during the year. The Tropic of Cancer and Capricorn (dashed lines at $\ang{23.4}$N and S) bound the Tropics. Precise latitudes for the climatological boundaries between the zones are neither necessary nor possible, but if you have to have a number, choose $\ang{30}$ and $\ang{60}$, N and S, for the climatological boundaries between the zones. These aren't the traditional boundaries identified by Parmenides, but as we'll see they reflect the actual behavior of Earth's climate system better than the solar boundaries. The other direction, parallel to lines of longitude, is termed meridional, after meridian, another name for a line of longitude. Figure \ref{fig:Apollo17} shows the zonal nature of climate nicely. \\
\subsubsection{The power of ignoring detail}
Semester grades at most colleges and universities are based on some form of average, typically an average over time and over different types of assignments. That average grade is a remarkably compact number, summarizing an entire semester's worth of last minute cramming and late night writing into a single number or letter. Averages eliminate the need to slog through every grade on every quiz and homework; they illustrate the big picture without the distraction of detail. Climate is a ``big picture'' view of average weather over time. This time scale varies, but for the purposes of our studies, climate is generally considered to be the average over multiple decades, and very often 30 years. This averaging removes the (sometimes interesting) short-term detail, but also removes the distractions of too much detail. Climate can be averaged over space, as well as time. Zonal averages, averages at a given latitude or range of latitudes, taken over all longitudes, will be particularly useful. We don't need too much detail to get a great understanding of Earth's climate.\\

\section{Global and Zonal Average Temperatures}
Although you may not have realized it when we first defined them, $T_{eq}$ and $T_{avg}$ are both global annual average temperatures. They have to be: they lack any information about time of year or location. They provide the least detailed, but most broadly useful, measure of Earth's temperature. But as our discussion of Parmenides' zonal climate belts suggested, we expect that both insolation (S) and albedo (A) most likely vary with latitude. If insolation and albedo are functions of latitude, then Earth's surface temperature should vary with latitude as well.
\subsection{Albedo variation across Earth}
Figure \ref{fig:Apollo17} shows one of the more famous photographs ever taken: Earth viewed from NASA's Apollo 17 spacecraft on its way to the Moon. The inset superimposes the climate zones from Figure \ref{fig:globe} on the photograph (which is far too stunning to hide under unsightly labels!) Particularly on the continents, you can easily see how albedo varies with latitude. Around $\ang{30}$N and S, bright reflective deserts dominate the land surface, while in the Tropics dark rainforest predominate, except where hidden by a broken band of clouds along the equator. Farther south, in the Southern Hemisphere Temperate zone, a merry-go-round of large, comma-shaped high-albedo cloud systems rotate around highly reflective polar areas of Antarctica.\\
Thanks to another NASA mission, we have excellent measurements of albedo for the globe. The left-hand panel of Figure \ref{fig:albedomap} shows annual average albedos (including the effect of clouds), while the panel to the right shows zonal averages. The zonal pattern of albedo variation glimpsed in Figure \ref{fig:Apollo17} is readily visible. Where clouds are absent, albedo is controlled by the type of material at Earth's surface. Snow and ice have albedos of 0.4 to 0.95, depending on age; water from 0.03 to 0.10; and grasslands and forests around 0.25. High albedos in the poles are---or at least have been until recently---the norm. On the other hand, much of the equatorial region is ocean, with relatively low albedos. The band of slightly elevated albedo just north of the Equator is due to clouds associated with an important atmospheric region called the Inter Tropical Convergence Zone (ITCZ); you can see clouds associated with the ITCZ in the photograph of Figure \ref{fig:Apollo17}. Rapid transitions from low (dark blue on Figure \ref{fig:albedomap}) to moderate albedo (aqua on Figure \ref{fig:albedomap}) over the oceans at $\ang{30}$N and S mark the limits of high pressure systems in the atmosphere; these systems and the ITCZ will play crucial roles in understanding the movement of heat in the atmosphere.
\subsubsection{The Reasons for Seasons}
All else being equal, you'd expect that Earth's less reflective Tropics would be warmer than the Poles, simply because the former reflect far less heat back to space than the latter. Of course, all else here means insolation, and insolation is not equal across Earth's surface, or over time. Insolation varies by day, as Earth spins around its axis each 24 hours, and annually, as Earth orbits the Sun each year. Earth's motions through space are fundamental to our understanding of Earth's spheres, climate and weather. This seems like a good as time as any to review them. Figure \ref{fig:season1} shows the schematic relationship between Earth and two points of reference in the sky: our Sun, and the star Polaris. Relative to the Sun, Earth completes one rotation around its axis every 24 hours, or one day. We can imagine Earth's rotation axis as pointing out of the North Pole and into space, where it currently points near Polaris, or the North Star. Visualizing this rotation is easier using a time lapse photograph of the night sky. The spectacular result in Figure \ref{fig:startrails} shows trails caused by the apparent motion of stars as Earth rotates beneath the vault of the heavens. Look carefully at the bulls-eye of stars above and to the left of the volcano: the star with the shortest trail is Polaris, low on the horizon from tropical Indonesia, where the photograph was taken. Even as it rotates around its axis, Earth is orbiting around the Sun. The plane of Earth's orbital motion is called the ecliptic plane, and is shown by the blue dashed line in Figure \ref{fig:season1}. On average, Earth moves about 66,000 mph along its orbit\footnote{The tropical year is 365.24 days long, and Earth's average distance from the sun is roughly 1 AU. Thus, skipping units, $v_{orbit}=\frac{c_{orbit}}{t_{orbit}}=\frac{2\cdot\pi\cdot(150x10^6/1.62)}{365.24\cdot24}\sim 66400$ mph. The JPL Horizons server gives components of Earth's orbital velocity; averaging daily values of these for 2019 CE gives  $v_{orbit}=\sim\SI{66628.1}{\kilo\metre\per\sec}$. I note that the approximate value is within 0.3\% of the actual value, took a few seconds to calculate, and doesn't return the velocity of the devil.}, a little more in the northern hemisphere winter, and little less in northern hemisphere summer. Summer and winter exists, however, only because Earth's rotational axis is tilted relative to the orbital plane.
This tilt---Earth's \emph{obliquity}---is currently $\ang{23.4}$, and this angle is what defines the classic limits of Paramendies' climate zones in Figure \ref{fig:globe}. Obliquity is also causes Earth's seasons. Because Earth's axis is (currently) fixed in the direction of Polaris, the relative relationship of Earth and Sun changes during the year, as Earth rumbles along its orbit. Figure \ref{fig:season2} shows Earth's orbit around the sun from far above (in the upper panel) and from the side (in the lower panel). Again this is a schematic illustration, with Earth's size wildly exaggerated relative to that of the orbit. (At the scale of the orbit, Earth's diameter would be $\frac{1}{8}^{th}$ the width of a human hair.) The arrows on the Earth represent the direction to Polaris. The first day of Spring, technically called the Vernal Equinox, occurs when the angle from the Sun, to Earth, and then to Polaris, is exactly \ang{90}. For various fascinating reasons\footnote{Steel, D. (2001). Marking time: The epic quest to invent the perfect calendar. John Wiley \& Sons.. pg. 16}, this event occurs on March 20 in our civil calendar.\\
The first day of Northern Hemisphere Summer begins when Earth, Sun, and Polaris lie in a straight line, with Earth and Polaris are on opposite sides of the Sun. As it happens, Earth is actually farthest from the Sun at about this time: seasons have little to do with distance \label{distance_cite} between Earth and Sun. The real cause of seasons is of course heat flow! Look back to Figure ??? Imagine two ``tubes'' of sunlight, each carrying the exact same energy from Sun to Earth. One tube happens to hit Earth near the Tropic of Cancer (the one coming directly from the sun image); the other hits near the Antarctic Circle. At this season, Earth's surface near the Tropics is more or less perpendicular to the Sun's rays: the energy in the tube is thus concentrated over a small area (the heavy blue line at the end of the tube), and so temperatures will be higher. Near the Antarctic Circle an equivalent energy ``tube'' hits nearly parallel to Earth's surface. The energy is spread over a much wider area, and so the area is colder. Northern hemisphere autumn begins when the Sun-Earth-Polaris angle is $\ang{-90}$, and finally winter rolls in when the Earth, Sun and Polaris are again in a straight line, but with the Earth and Polaris on the same side of the Sun. Here the arrangement of relative heating we saw in northern hemisphere summer is reversed, with energy spread over smaller areas in the southern hemisphere and larger areas in the northern hemisphere. Seasons on Earth are simply due to the way obliquity changes the location of maximum solar heating of Earth's surface.\\
\subsection{Variation of Insolation with Latitude}
As Figure \ref{fig:insolat} shows, the amount of sunlight hitting the top of Earth's atmosphere varies more strongly with latitude than albedo does, but in the same sense. The solid line gives the annual average, with the other lines showing daily totals for the first day of summer (summer solstice) and winter (winter solstice) in the northern hemisphere. Two features of the graph are particularly interesting. One is the impressive symmetry of the annual average around the equator. We expect from our understanding of the seasons that the average insolation should be about the same for each hemisphere. The second feature is that---to little surprise---the Tropics receive more insolation than the Poles. More energy mean, most likely, higher temperatures. Measured at the top of the atmosphere, far above the clouds and Earth's surface, the Tropics receive over twice the solar energy as the Poles. Combined with the Tropics lower albedo, the reasons for the Tropics warmth relative to the Poles is becoming increasingly clear. Another interesting feature is the minima in monthly insolation near the Arctic and Antarctic Circles. These minima are due to the continuous summer daylight near the summer pole, which adds to the annual totals for areas poleward of the Arctic and Antarctic circles (an effect clearly visible in Figure \ref{fig:Apollo17}, where you can see thta all of Antarctica, on both sides of the South Pole, is bathed in sunlight). Continuous winter darkness in the same zones explains the areas of 0 winter-time insolation as well. The gray shading on the left-hand side of Earth represents the night-time hemisphere. Note that Antarctica remains in darkness, while all of the areas above the Arctic Circle are bathed in continuous sunlight.\\ 
We saw on page \pageref{distance_cite} that Earth's seasons are caused by Earth's obliquity, and you read that the seasons have ``little'' to do with Earth's distance from the Sun. Little, but not nothing. Earth is closest to the sun (the technical term is ``at perihelion'') on January 4 or so, just after the southern hemisphere's summer solstice. Earth is farthest from the sun (``at aphelion'') on July 4 or so, just after northern hemisphere's summer solstice. The distance difference is only 3.4\%, but because energy decreases as the square of distance from the sun, the northern hemisphere receives about 6.8\% \emph{less} insolation during its summer than the southern hemisphere does, despite the fact that northern hemisphere summer is 4.5 days longer. The net result of this is that the Northern Hemisphere gets a little less insolation than the Southern Hemisphere, as you can see on Figure \ref{fig:insolat}. Changes in these subtle relationships between seasons and proximity to the Sun are the primary drivers of Ice Ages, a subject we'll examine in chapter 13.\\
  
\subsubsection{Earth's Predicted \textit{versus} Actual Zonal Temperatures: Where's the Heat Going?}
So we now have a good understanding of how both insolation and albedo vary with latitude over Earth. Using a variant of Equation \ref{eq:ts}, we can find the expected zonal average temperatures, that is the temperature at each latitude of Earth, based on the insolation received at each latitude, and the average albedo at each latitude:
\begin{align}
	T_{lat,\ expected}&=T_{eq,\ zonal}+\Delta T_{GHE}\\ \label{eq:Tlat1}
	T_{lat,\ expected}&=\left[\frac{S_{lat}*\left(1-A_{lat}\right)}{4 \sigma}\right]^{\frac{1}{4}}+ 33\ K
\end{align}
This equation isn't all that different than the form we've already used, except here we're using multiple values for insolation (S) and albedo (A), each keyed to a different latitude, as indicated by the ``lat'' subscripts. Figure \ref{fig:avgsurftemp} shows the resulting pattern of expected (as calculated from Equations \ref{eq:Tlat1}) and the actual annual average temperatures measured over Earth's latitudes. The two horizontal dotted lines are old friends: $T_{eq}$ and $T_{av}$. These are both global annual average temperatures. They have to be: they lack any information about time of year or location. Both characteristic temperatures are horizontal lines because, by definition, they are independent of latitude. Note that $T_{av}$ is \SI{33}{\kelvin} greater than $T_{eq}$ because of the GHE.\\
The zonal average temperatures \textit{calculated} using Equation \ref{eq:Tlat1} are shown by the blue dot-dash line, while the \textit{measured}\footnote{Actually this the cos(2*lat) model of measured surface temperatures of North, G. R., \& Coakley Jr, J. A. (1979). Differences between seasonal and mean annual energy balance model calculations of climate and climate sensitivity. Journal of the Atmospheric Sciences, 36(7), 1189-1204.} zonal average temperatures are shown by the solid green curve. Yet again it seems we have done a great deal of work to little end. Our calculated temperatures (blue dashed-dot) are far too cold compared to observed values. But if we add in the greenhouse effect, $\Delta T_{GHE}$, we get the model shown by the red dashed line, $T_{calculated, GHE}$, which does a great job predicting temperatures across Earth's surface. This model, labeled $T_{calculated, GHE}$ in the figure isn't perfect. Compared to the actual temperature patterns the model is still too hot in the tropics and too cold in the poles, with a ``cross-over'' at about $\ang{39}$N and $\ang{39}$S, near the boundary of the Temperate zones with the Tropics. Fortunately we are now old hands at heat and temperature. The calculated average temperatures don't account for heat \emph{flowing} from the warmer tropics to the cooler poles. The heat pulled from the tropics lowers the temperatures there, while raising temperature at the poles. So all our calculations were helpful: they allow us to hypothesize that heat is flowing from the Tropics toward the Poles. This thermal imbalance between the Poles and the tropics, set up by differences in albedo and insolation, is the primary driver of weather, and hence climate, on the planet.\\
This hypothesis is beautifully confirmed by NASA's Earth Radiation and Budget Experiment satellite. This mission measured both incoming short-wave radiation (SWR) from the Sun, and the long-wave radiation (LWR) emitted from Earth for most of the globe. Figure \ref{fig:erb} shows that the tropics receive more SWR from the Sun then they emit as LWR. Were such an imbalance to continue, the tropics would be steadily warming (which they are not). The reverse observation happens in both the temperate and polar areas: they actually emit more LWR energy than they receive as SWR. Were such an imbalance to continue, the polar regions would be cooling (which they are not). The green, solid line in Figure 22 is the difference in net radiation. The ``excess'' energy between $\ang{30}$ N and $\ang{30}$S does what heat naturally does: flows to cooler areas, the temperate and polar zones. Imagine the ``hill'' of positive net energy in the Tropics oozing out to fill in the ``valleys'' on either side. Figure 23 shows the enormous scale of this heat flow. Energy flow peaks at \SI{6}{\peta\watt}, the equivalent energy of 100 Hiroshima-sized bombs exploding \emph{every second}.\\
This meridional (north-south) movement of heat is the primary driver of weather on Earth's surface. Most of this heat moves by advection, some by convection, primarily through motions of the atmosphere, a little by motions of the oceans. We will have much more to say about this energy flow, eventually using it to explain such disparate facts as the European ``discovery'' of the Americas, why Canada is the largest supplier of lumber to the United States\footnote{All timber products, as reported by the Foreign Agricultural Service  http://apps.fas.usda.gov/gats/default.aspxm accessed February 04, 2016.}, and why 21 Bostonians died in the early afternoon of January 15, 1919, drowned in a flood of molasses\footnote{Park, Edwards (November 1983). \textit{Without Warning, Molasses in January Surged Over Boston}. Smithsonian 14 (8): 213-230}. But before we do that, we need to understand each of Earth's natural spheres, how they formed, how the move, and how they have evolved over the long 4.567 billion years of Earth's history. 

\section{Figures}
.
\newpage

\begin{figure}[p]
\centering
\includegraphics[width=7 in]{ghe_pressure}%
\caption{The cause of the greenhouse effect, revealed by the relationship between atmospheric pressure (on the horizontal axis) and the difference between actual surface temperature and the blackbody equilibrium temperature (vertical axis). On the left is a graph with standard arithmetic axes: each tick mark is a fixed amount greater than the previous. This type of graph can't show the often enormous range of values typical in studying climate. The log-log graph on the right has a constant factor between tick marks. Each one is 10 times larger than the previous. These axes expand the lower left corner of a typical graph, and reveal that pressure and temperature difference are closely related. }   
\label{fig:tghe_P}
\end{figure}

\begin{figure}[p]
\centering
\includegraphics[width=3 in]{IodoAtomico.JPG}
\caption{Iodine (from the Greek, \textit{ioeides}, violet colored) gas has a beautiful purple color, because the electrons in this molecule absorb much of the red through green light, allowing the blue and purple wavelengths to pass through. Do not try this at home: $I_2$ is a noxious and potentially toxic gas. This image is courtesy of and by Matias Molnar - Laboratorio Quimica Inorganica II UBA, Argentina. Licensed under CC BY-SA 3.0 via Wikimedia Commons,  https://commons.wikimedia.org/wiki/File:IodoAtomico.JPG\#/media/File:IodoAtomico.JPG.}   
\label{fig:iodine}
\end{figure}

\begin{figure}[p]
\centering
\includegraphics[width=5 in]{Bending_modes_GHG.jpg}%
\caption{Molecules with three or more atoms can vibrate and bend in ways that absorb infrared light. The light's energy causes bonds to stretch or rotate, as shown above. The three gasses making up 99\% of the (dry) atmosphere (Nitrogen gas, $N_2$; oxygen gas, $O_2$, and argon $Ar$) have one or two atoms. They do not absorb infrared energy in any significant way.}   
\label{fig:bending_modes}
\end{figure}

\begin{figure}[p]
\centering
\includegraphics[width=6 in]{ERB_a}%
\caption{Earth's energy balance in three easy pieces. Piece 1: Over the long term, Earth must emit as much energy as it receives from the Sun. On average, each square meter of Earth's surface receives energy at a rate of \SI{340}{\watt}. Some of this energy is reflected back to space, a proportion called Earth's albedo, A. Earth must emit the rest of the energy, \SI{340}{\watt}, back to space. This amount of energy corresponds to a $T_{eq} = \SI{255}{\kelvin}=\SI{-18}{\celsius}=-1^{\circ}$F.}   
\label{fig:erb_a}
\end{figure}

\begin{figure}[p]
\centering
\includegraphics[width=6 in]{ERB_b}%
\caption{Piece 2: Earth's actual surface temperature is $T_{av} = \SI{288}{\kelvin}=\SI{15}{\celsius}=60^{\circ}$F, \SI{33}{\kelvin} hotter than $T_{eq}$. Hotter objects emit more energy than cooler ones, so Earth's surface must emit more energy than it receives from the Sun alone! How is this possible? The Greenhouse Effect.}   
\label{fig:erb_b}
\end{figure}

\begin{figure}[p]
\centering
\includegraphics[width=6 in]{ERB_c}%
\caption{Step 3: The solution is that each square meter of Earth's surface receives \SI{326}{\watt} of energy re-emitted from the atmosphere (the large red arrow on the right of the diagram). This energy comes from energy absorbed by the clouds and atmosphere directly from the Sun (the yellow arrow labeled ``absorbed''), from energy contained in thermals of rising air (think cumulus clouds) and from water evaporation and plant growth (evapotranspiration, in green), and from the surface itself. Earth's surface absorbs more energy from the atmosphere than from the Sun!}   
\label{fig:erb_c}
\end{figure}

\begin{figure}[p]
\centering
\includegraphics[width=6 in]{sun_earth_spectrum}%
\caption{The Sun's and Earth's spectra and the origin of the greenhouse effect. The lowest panel shows the important bands of the electromagnetic spectrum, with short wavelength ultra-violet to the left, visible in teh middle, and long-wavelength infra-red on the right. Note the log scale on the wavelength axis. The middle panel shows the relative amount of energy emitted at each wavelength by the Sun (in yellow) and Earth (in red). The gray shading indicates how effectively Earth's atmosphere absorbs or scatters light of each wavelength. The lower the gray line, the more transparent the atmosphere is to that color of light. Portions of the spectrum (for example, most of the infra-red with wavelengths above \SI{10,000}{\nano\metre}) where the gray fill extends to ``0 transmission'' indicates total absorption by the atmosphere. Light of that wavelength never makes it through the atmosphere, in \textit{or} out. Portions (such as the visible band) where the atmosphere is transparent to light are called ``windows''. By cosmic coincidence, the Sun's peak insolation exactly overlaps a big, wide window in the atmosphere. Earth, on the other hand, emits infra-red radiation that is largely absorbed, except for the narrow window right at \SI{10,000}{\nano\metre}. The upper panel identifies the greenhouse gasses responsible for absorption features. Note that water vapor is a strong absorber of infra-red radiation, and causes the spikes of absorbance all through the infra-red. Note how water and carbon dioxide conspire to make the atmosphere around Earth's peak emission relatively opaque. This traps the light and its energy in in the atmosphere, warming it and causing the greenhouse effect.}   
\label{fig:sun_earth_spectrum}
\end{figure}

\begin{figure}[p]
\centering
\includegraphics[height=6 in]{forest_trees}%
\caption{Seeing the sky for the trees and emission heights in the atmosphere. In Panel A (upper), you stand within a dense forest, and the trunks effectively block your ability to see the horizon. Greenhouse gas molecules in the atmosphere act like trees in a forest, preventing infra-red light from ``seeing'' space. But in Panel B (lower) you are standing at the edge of the forest, and there are so few tree trunks between you and the forest's edge you can see the horizon. Analogously, in Earth's atmosphere infra-red light can only escape when there relatively few greenhouse gas molecules left left to run into. This is the emission height of the atmosphere.}   
\label{fig:forest_trees}
\end{figure}

\begin{figure}[p]
\centering
\includegraphics[width=7 in]{emission_heights.jpg}%
\caption{Seeing through the forest, as applied to Earth's atmosphere. The line shows the typical height at which light (of the wavelength given on the horizontal axis, on a log scale to easily compare with Figure \ref{fig:sun_earth_spectrum}) reaches space without further absorption in Earth's atmosphere. The global average for all infra-red light is around \SI{5.5}{\kilo\metre}, although for the atmosphere modeled here, this height is closer to \SI{6.6}{\kilo\metre}, about 22,000 feet. This is about half the typical flight level of a large passenger jet. The peak at \SI{15,000}{\nano\metre} is due to $CO_2$, and the height of this emission (\SI{11}{\kilo\metre}, about 36,000 feet) is at the flight level of that same jet. Note how the emission height shown here is qualitatively the same as the transmission of the atmosphere depicted in the central panel of Figure \ref{fig:sun_earth_spectrum}.}   
\label{fig:emission_heights}
\end{figure}

\newpage
\begin{sidewaysfigure}
\centering
\includegraphics[width=8 in]{ann_ghg}%
\caption{Changes in the concentrations of the three anthropogenic greenhouse gasses since 1979 CE. Each gas is shown in a separate row (carbon dioxide, $CO_2$; methane, $CH_4$; and nitrous oxide $N_2O$, from top to bottom) with atmospheric concentrations (in parts per million ppm) in the left hand column and annual changes (\%) on the right hand column. Measured vales are shown in unfilled symbols, and projections in the smaller, filled symbols. The abundances of all three gasses has increased in the past 40 years, with typical growth rates 0.3 to 0.6 percent per year. Methane (middle panel) is a clearly different. Before 2000 CE methane concentrations grew rapidly, then leveled off (even falling) for 7 years, only to begin increasing again since 2008 CE. The reasons for the recent increase are still debated, but include changes in the methane released by microbes (often found in rice paddies and the guts of cattle), from increased fossil fuel production, a decrease in the rate methane is removed from the atmosphere, or a combination of all three. }
\label{fig:annghg}
\end{sidewaysfigure}

\begin{figure}[p]
\centering
\includegraphics[width=6 in]{wv_vis.jpg}%
\caption{Water vapor dissolved in the air is not the same as visible clouds. In Panel A (on the left), is a water vapor image of North America, taken at 11:15 AM EDT on 30 June, 2015 using infra-red light emitted by water dissolved in the atmosphere. Brown to black areas are quite dry, with low concentrations of water vapor, while areas colored white to blue have increasingly higher concentrations of water vapor. In Panel B (on the right) is a visible light image taken at the same time, showing just clouds. Note how water vapor is present in areas where clouds are not, indicating the water vapor really is dissolved in the air, and not (in most places) forming clouds. For example, North Carolina, Kansas, and Massachusetts (outlined in green) have high concentrations of water vapor, but are cloud free. Images are courtesy of the National Weather Service.}   
\label{fig:wv_vis}
\end{figure}

\begin{figure}[p]
\centering
\includegraphics[width=6 in]{feedback_loops_intro.png}%
\caption{Systems can act in complicated ways when their parts interact. One important class of interactions are feedbacks, where change in one part of a system produces change in another. Panel A (upper left) shows a competitive collegiate wrestler whose weight must stay within a narrow target range. If the wrestler's weight increases or decreases, the wrestler decreases or increases respectively their calorie intake. Such \emph{negative} signals, shown by the red dashed line connecting the two parts of the system, reflect the opposite signs of the changes. An increase or decrease in caloric intake leads to weight increase of decrease respectively, a positive signal (shown by the solid green line). Systems, such as the wrestler, with an odd number of negative signals are negative feedback systems: over time, small changes in the system get squashed and the system tends to stay in steady state (Panel C, green solid line). Systems with no or an even number of negative signals are positive loops (Panel B, upper right). Over time, small changes in the system get amplified and the system tends to diverge (sometimes catastrophically) from steady state. Earth's climate, stable for billions of years, is a negative feedback system. But anthropogenic signals, from greenhouse gas emissions to use of fertilizers to land use change, are pushing the climate into new realms.}   
\label{fig:feedback_loops_intro}
\end{figure}


\begin{figure}[p]
\centering
\includegraphics[width=6 in]{schmidt_hills.jpg}%
\caption{This view of the Schmidt Hills of Antarctica, shows the frozen Polar Plateau leading
off to the Ronne Ice Shelf in the background. All that ice is nothing more than compacted snow, but
to the surprise of most people, snow in Antarctica is actually pretty rare, simply because the Antarctic air
is too cold to hold much water. Antarctica is a unique and spectacular part of planet Earth, but
much of the planet's surface would look like the landscape there if it weren't for the effects
of carbon dioxide as a non-condensable gas: a gas that doesn't freeze to a solid on the surface.}   
\label{fig:schmidt_hills}
\end{figure}


\begin{figure}[p]
\centering
\includegraphics[width=6 in]{relative_humidity.pdf}%
\caption{Water vapor is the dominant greenhouse gas in our atmosphere, but the amount of water in the atmosphere is a strong function of temperature. Humidity, which measures the amount of water in air, increases by about 6\% for every \SI{1}{\kelvin} increase in temperature, as shown by the solid blue line. As air cools, the humidity drops so fast that substantial snow falls are essentially impossible. The blue solid line gives the maximum humidity possible, while the red dashed line and orange dotted lines show relative humidities (RH) of 50\% and 25\%. Typical relative humidities [Citation here] in Hawai'i are 70\% to 50\%, in Arizona 50\% to 25\%. Most people feel pretty comfortable at of 50\% RH, but uncomfortably wet at 75\% RH and uncomfortably dry at 25\% RH. Since 1750 AD, average surface temperatures have increased by about \SI{1}{\kelvin}, increasing water vapor concentrations by $\sim 6\%$. In one of the most graphic signs of climate change, this has increased the likelihood of heavy rains in both tropical [Donat, M. G. Ang\'{e}lil and Ukkola, A. M. (2019) Intensification of precipitation extremes in the world's humid and water-limited regions. Environmental Research Letters 14(6), 14 065003] and temperature areas, exclusive of dry areas like deserts, where the data are sparse.}   
\label{fig:relative_humidity}
\end{figure}

\begin{figure}[p]
\centering
\includegraphics[width=6 in]{watervapor_co2.pdf}%
\caption{The positive feedback loop involving carbon dioxide, water vapor and the greenhouse effect plays a key role in moderating Earth's climate. Even at Earth's cold equilibrium temperature, the three greenhouse gasses other than water vapor---carbon dioxide, methane and nitrous oxide---all remain gasses in the atmosphere. These non-condensable gasses (Box 1) increase the greenhouse effect (Box 2) which in turn increases the surface temperature of the planet (Box 3). This leads to an increase in the maximum humidity of the atmosphere (Box 4), and hence the evaporation of water (Box 5), and hence more water vapor in the atmosphere (Box 6). This, finally, leads to another increase in the greenhouse effect (back to Box 2), and the loop repeats. Because there are only positive signals in this system, this is a positive feedback loop.}   
\label{fig:wvco2}
\end{figure}

\begin{figure}[p]
\centering
\includegraphics[width=6 in]{noco2climate}%
\caption{The results of a computer model demonstrates how carbon dioxide and other ``non-condensable'' gasses control Earth's greenhouse effect. Colors indicate (see the scale on the bottom of the image) the average annual mean surface temperature for each latitude. The researchers removed all the $CO_2,\ CH_4\ and\ N_2O$ from the modeled atmosphere at year ``0,'' and then ran the model forward. Within 10 years of the digital removal, the model predicts drastic changes in climate. Without $CO_2$ and other non-condensable gasses, the atmosphere cools so much that little water vapor remains in the atmosphere. Due to the water vapor---carbon dioxide feedback loop (Figure \ref{fig:wvco2}), this lowers the greenhouse effect, lowers the surface temperature and lowers the humidity even father. Only 50 years after the experiment begins, average surface temperatures have dropped \SI{35}{\kelvin}, almost half the ocean is covered in ice, and Earth has an albedo of 0.42 (which acts to coll the planet even more). Ice occurs at sea level in the oceans except the narrow band (indicated in yellow) north of the equator. Figure is courtesy of NASA.}   
\label{fig:noco2earth}
\end{figure}

\begin{figure}[p]
\centering
\includegraphics[width=6 in]{globe}%
\caption{You can't tell the planet without a program, so this figure shows the important geographical features and boundaries of our fair planet. Blue latitude and red longitude lines are shown every $\ang{30}$ and $\ang{60}$ respectively. Places associated with climate are shown in \textbf{\textsc{Bold}}, while those features associated with positions of the Sun are shown in \textit{italic} lettering. The two arrows show meridional and zonal directions. The four hemispheres are shown in the margins. Robinson projection.}   
\label{fig:globe}
\end{figure}



\newpage
\begin{sidewaysfigure}
\centering
\includegraphics[width=7 in]{Apollo_17}%
\caption{Earth, in all its splendor, as seen from the Apollo 17 spacecraft a few hours after leaving Earth for the Moon on December 7, 1972. About the original photograph (Panel B, on the right) Ben Cosgrove noted that this \textit{Blue Marble} image wasn't the first picture of Earth from space, ``but no other photograph ever made of planet Earth has ever felt at-once so momentous and somehow so manageable, so companionable.'' Note the enormous range in albedo across the globe: the highly reflective ice covering polar Antarctica, equally reflecting swirls of clouds forming storm systems in the Southern Hemisphere Temperate zone, the unreflective Atlantic (on the left) and Indian (right) oceans, the dark rainforests of Tropical Africa, and the surprisingly bright and reflective Sahara Desert. Note the discontinuous band of clouds near the equator, a sign of the Inter-tropical Convergence Zone. As Carl Sagan famously pointed out, there's very little evidence of humans visible on this photograph. But since this photograph was taken, the human population on the planet has nearly doubled. Panel A (on the left) gives place names and important parallels of latitude. (Photograph courtesy of NASA: http:\/\/www.nasa.gov/topics/earth/earthday/gall\_whole\_earth.html)}
\label{fig:Apollo17}
\end{sidewaysfigure}

\newpage
\begin{sidewaysfigure}
\centering
\includegraphics[width=7 in]{AlbedoMap}%
\caption{Earth's average annual albedo, or reflectivity. Panel A (left) shows the average in map view, while Panel B (right) shows the zonal averages. In Panel A the the highly reflective polar regions (with albedo ranging from 55\% to 75\% stand out, but they actually cover relatively little of the surface. Instead, look at the vast tropical oceans, with albedo of 10\% to 15\%, which dominate Earth's average albedo. The thin light blue areas around the equator (noticeable as the ``bulge'' in Panel B at $\ang{5}$ N) is due to the clouds associated with the Inter-tropical Convergence Zone, about which we'll see more in Chapter 7. Note how the albedos displayed on this map reflect the conditions shown in the photograph from Apollo 17, Figure{fig:Apollo17}. The map in Panel A is in Robinson projection; the ordinate in Panel B is weighted by the sine of latitude to reduce  }
\label{fig:albedomap}
\end{sidewaysfigure}

\begin{figure}[p]
\centering
\includegraphics[width=6 in]{season1}%
\caption{A schematic diagram of the orientation of Earth, the Sun, and Polaris (the North Star). The orientation is correct for the first day in the Northern Hemisphere Summer, June 21. ``Schematic'' in this sense means ``a diagram showing general arrangement, but without a consistent scale,'' and that certainly applies to this diagram. At the scale of the Earth shown in the diagram, the Sun would be 25 feet across and almost 3800 feet away from the model Earth! Polaris-the North Star-is about 430 light years away.To put Polaris at the correct distance from our model Earth, we'd need a sheet of paper that extends beyond the orbit of Pluto! Let's accept these distortions for now. Earth spins a full $\ang{360}$ relative to the Sun on it's axis once every 24 hours. Over the course of a year, Earth orbits the sun in our ``orbital plane'' shown by the dashed green line. Earth's axis is titled $\ang{23.4}$ relative to its orbital plane; this \emph{obliquity} is why Earth has seasons. Because Earth is a sphere, some portions of the Earth face the Sun directly during some part of the day; others face away from the Sun for the entire day. The solid blue lines show equal amounts of solar energy (think of them as ``tubes'') from the Sun hitting Earth. The tube of energy falling on Africa is concentrated over a small area, shown by the blue rectangle, and hence leads to higher temperatures. The same energy is spread over a much larger area in the lower tube, leading to cooler temperatures. This is one of the reasons the Tropics are warm and the Poles cold.}   
\label{fig:season1}
\end{figure}

\newpage
\begin{sidewaysfigure}
\centering
\includegraphics[width=7 in]{Startrails}%
\caption{A spectacular time lapse photograph of the northern polar region of the sky. The circular star trails record Earth's rotation (counter-clockwise is this orientation), with stars rise on the eastern horizon (right hand side) rotating about the north celestial pole (in the center) and setting in the west. The inset shows Polaris, the North Star, in the center; note the tiny trail, indicating that Earth's North Pole ``points'' close to Polaris. http://apod.nasa.gov/apod/ap140818.html  Attribution-NonCommercial 2.0 Generic (CC BY-NC 2.0). (http://mydarksky.org/). This is a placeholder only; I do not yet have rights to use this picture. }
\label{fig:startrails}
\end{sidewaysfigure}

\begin{figure}[p]
\centering
\includegraphics[width=6 in]{season2}%
\caption{Another schematic image of the relationships between Earth, Sun and stars. The upper panel shows earth orbiting counter-clockwise around the sun. The arrow shows the orientation of Earth's rotation, spinning around the North Pole and pointing toward Polaris. As in Figure {fig:season1}, the scale here is wildly wrong. The constant direction of the arrow reminds us that Earth's orientation in space is effectively fixed over the year. The lower panel shows that Earth's obliquity causes seasonal differences in the directness of insolation from the Sun. Summer happens when a hemisphere is tilted toward the Sun, increasing the intensity and duration of solar radiation. The increased heating leads to warmer temperatures. Note that the northern hemisphere summer happens when Earth is farthest from the Sun in its orbit.}   
\label{fig:season2}
\end{figure}

\begin{figure}[p]
\centering
\includegraphics[width=6 in]{insolation_lat_good}%
\caption{Annual variations and averages of the insolation received in each latitude on Earth. The solid black line gives the annual average, roughly symmetric about the equator. The blue dotted line shows the insolation on the northern hemisphere's Winter Solstice. Note the area north of the Arctic Circle (light gray line at \ang{66.6} N) is perpetually dark at this time, while the areas south of the Antarctic Circle (light gray line at \ang{66.6} S) is perpetually light. This situation is reversed at the Northern Hemisphere's Summer Solstice.}   
\label{fig:insolat}
\end{figure}

\begin{figure}[p]
\centering
\includegraphics[width=6 in]{avgsurftemp}%
\caption{The many temperatures of Earth. The two horizontal gray lines give the equilibrium $T_{eq}$ and global average $T_{av}$ temperatures: their only difference is the greenhouse effect. These are global averages, so their value doesn't change with latitude. The solid green line is the observed average surface temperature, $T_{observed}$, which clearly changes with latitude, colder at the poles, warmer in the tropics, as we'd expect. The blue dot-dash line, $T_{calculated}$, is the zonal average temperature one calculates from zonal insolation and albedo, as shown in Equation \ref{eq:Tlat1}, and is clearly does a terrible job estimating Earth's actual surface temperatures (green). Adding the greenhouse effect in helps, this is shown by the dashed red line labeled $T_{observed, GHE}$. This is a pretty good model, a little too warm in the tropics and a little too cool in the poles, but the overall agreement is good. We clearly understand the basic forces controlling Earth's average surface temperatures. But our model doesn't include a crucial element, the flow of heat from the warm tropics to the cooler pole. This heat flow cools the tropics and warms the poles, explaining the discrepancies in our model.}
\label{fig:avgsurftemp}
\end{figure}

\begin{figure}[p]
\centering
\includegraphics[width=6 in]{ERBE}%
\caption{NASA's ``Earth Radiation Budget Experiment'' measured the incoming energy from the Sun, and the total outgoing energy leaving, each part of the Earth's surface. The measured incoming shortwave radiation (from the Sun, shown by the blue dot-dashed lines) and outgoing longwave radiation (emitted by Earth's atmosphere, shown by the red dashed line). The line marked ``net'' represents the difference between the two. Latitudes with net$>0$(areas between of \ang{39}N and \ang{39}S) receive more energy than they emit; areas with net$<0$ (poleward of \ang{39}) emit more energy than they receive. Note that these boundaries are the same as those we saw in Figure \ref{fig:avgsurftemp}. How can this be? What does this imply? }   
\label{fig:erbe}
\end{figure}

\begin{figure}[p]
\centering
\includegraphics[width=6 in]{ERBE}%
\caption{NASA's ``Earth Radiation Budget Experiment'' measured the incoming energy from the Sun, and the total outgoing energy leaving, each part of the Earth's surface. The measured incoming shortwave radiation (from the Sun, shown by the blue dot-dashed lines) and outgoing longwave radiation (emitted by Earth's atmosphere, shown by the red dashed line). The line marked ``net'' represents the difference between the two. Latitudes with net$>0$(areas between of \ang{39}N and \ang{39}S) receive more energy than they emit; areas with net$<0$ (poleward of \ang{39}) emit more energy than they receive. Note that these boundaries are the same as those we saw in Figure \ref{fig:avgsurftemp}. How can this be? What does this imply? }   
\label{fig:erbe}
\end{figure}


\begin{figure}[p]
\centering
\includegraphics[width=6 in]{heat_flow}%
\caption{Advective heat transport from the equator to the poles moderates surface temperatures across the planet. Lines indicate the net heat flow from the equator towards the poles at each latitude; areas above the \SI{0}{\peta\watt} heat flow line have heat moving northward, while areas below have heat flowing southward . The units of heat flow at petawatts, or \SI{10e15}{\watt}. One \si{\peta\watt} is the equivalent energy of 16 of the atomic bombs that destroyed Hiroshima exploding every second. [The explosion had a yield of 15 kton, or approximately \SI{63}{\tera\joule}, thus $\SI{1000}{\tera\joule\sec}/\SI{63}{\tera\joule}\approx \SI{16}{\sec}$]The atmosphere accounts for the vast majority of this transport, and the ramifications of this are felt globally as ``weather.'' Weather is the result of large scale heat flow on Earth, and climate is the long term average of weather.}   
\label{fig:heatflow}
\end{figure}



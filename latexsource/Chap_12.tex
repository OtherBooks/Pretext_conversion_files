\chapter{Quaternary Glaciations}\label{chap:Ice_Ages}

\section{Core Concepts}
\begin{itemize}
	\item	Earth is currently in its sixth period of profound continental glaciation, the Quaternary Ice Age.  
	\item This Ice Age is marked by periodic growth and then retreat of ice sheets across continents of the northern hemisphere. This cyclic pattern is initiated by periodic variations in Earth's orbit, which change the insolation incident on the polar areas of the northern hemisphere. These Milankovich cycles are due to the gravitational effects of other planets on Earth's orbit.
	\item Positive feed backs between ice cover, albedo, ocean temperature and atmospheric $CO_2$ deepen the severity of stadials, and contribute to their termination as well. 
	\item The most recent glacial period ended roughly 20 kya; our current inter-stadial period is long by recent standards, but it is very unlikely a new stadial will begin within the next 50-100 kyr. A new glacial cycle will not save human societies from global warming. 
\end{itemize}

\section{Opening Problem}
Charles August de Sales, the Bishop of Geneva, had a problem. His parishioners had entreated him to save them from a tireless scourge which for 30 years had destroyed houses, farmlands and even entire villages: rapidly growing glaciers. It was June of 1644 CE, and glaciers---yes, glaciers---were rampaging across the countryside. In the Medieval period a few centuries before, warm climate and shrinking glaciers in the European Alps allowed farmers to expand their fields high into alpine valleys. But by 1610 CE, colder climates and growing glaciers of the Little Ice Age had brought ruin to these villages and farm lands, conditions well-documented by authorities assessing the farms for taxes. The area surrounding Mont Blanc is now an idyllic, beautiful, and wealthy part of Europe, but in 1616 CE, one assessor found that in this part of the Alps 
\begin{quotation}
the great glacier of La Rosi\`erc every now and then goes bounding and thrashing or descending; for the last five or six years...it has been impossible to get any crops from the places it has covered....Behind the village of Les Rousier, by the impetuosity of a great terrible glacier which is above and just adjoining the few houses that remain, there have been destroyed [40 acres\footnote{Conversion from the imprecisely known journal, (French \textit{journeux}), the area of land worked in a day by a man and a horse; see Russ Rowlett's `How Many?'' \href{http://www.ibiblio.org/units/dictJ.html}{web site}, accessed 28 November 2018. I used the geometric mean of the conversion limits} of land] with nothing but stones and little woods of small value, and also eight houses, seven barns, and five little granges have been entirely ruined and destroyed.
\end{quotation}
He found the all-too-human cost of climate change, when he
\begin{quotation} 
went to the village of Ch\^atelard where there are still about six houses, all uninhabited save two, in which live some wretched women and children, although the houses belong to others. Above and adjoining the village, there is a great and horrible glacier of great and incalculable volume which can promise nothing but the destruction of the houses and lands which still remain.
\end{quotation}
Charles' problem was same one spreading across Europe and Asia: how to ameliorate famine, plagues, social unrest and war due to rapid climate change. He did the best he could, exorcising four different glacier and ordering them to retreat\footnote{Grove, J. M. (2012). The little ice age. Routledge, pages 108-10. The quotes are as translated by Laudrie, 1971, pages 148-9, as cited in Grove.}. But the Little Ice Age was hard to stop, eventually leading to the death of millions from disease, famine and war. How can societies mitigate, perhaps even avoid, crises caused by cliamte change?
\section{Introduction}
Our goal in this (final) chapter of Part II is to understand recent climate variations, where ``recent'' is in the geologic sense of the past 3 My or so. While such a span is long compared to a human life time---even to our species' life time---the past 3 My represent only 0.07\% of Earth's history, equivalent to the time of two seconds in an hour long period. The very end of this period, the past 6000 years or so, is unique in Earth history. Over that period we, \textit{Homo sapiens sapiens}, have come to dominate the planet, and extended our control over the very machinery of climate. This has profoundly altered all five planetary spheres.\\

\section{Measuring past climates: the power of proxies and programs}
Understanding recent climate changes requires precise and accurate estimates of surface temperature, insolation, albedo and atmospheric composition over those same times. There are few to no such measurements at all until the beginning of the Enlightenment Period (~1650 CE), and only since 1900 CE or so has a global network of weather stations been measuring temperatures. We need stand-ins, something we can measure now that precisely and accurately records both the climatic conditions and the age of past environments. These climate \emph{proxies} are ``stand-ins'' for actual measurements of past climate. A good climate proxy can be as seemingly simple as the width of tree rings. Careful counting of tree rings establishes their age, while their width is a proxy for temperature and precipitation at the time of growth. \\ 

\subsection{How we know ``what:'' proxies}
A climate proxy\footnote{See Sam White's masterful 2013 paper, The Real Little Ice Age. Journal of Interdisciplinary History, 44(3), 327-352. \href{https://doi.org/10.1162/JINH_a_00574}{Link}.} is a
physical, chemical and biological ``record that correlates reasonably well to a climate variable, and that researchers can take reliable high-resolution measurements of that record and calibrate them against modern instrumental measurements.'' Paleo-climatologists use a wide variety of records as proxies, including those as familiar as tree-rings; as standard as isotopic ratios of stable elements in ice cores, corals, speleothems \footnote{the general terms for stalagmites and stalactites, deposits formed in caves from minerals dissolved in water}, and as obscure as the relative proportion of fatty molecules containing 86 carbon atoms made by a particular marine prokaryote\footnote{Kim, Jung-Hyun, et al., (2008), Global sediment core-top calibration of the TEX86 paleothermometer in the ocean, Geochimica et Cosmochimica Acta, 72 (4), 15 February 2008, Pages 1154-1173, http://dx.doi.org/10.1016/j.gca.2007.12.010.}. The most common proxies for paleoclimatology are listed in table \ref{tab:proxies}. All of them share three components. Each has a relatively tough physical substance that holds the record, and  might be sediment, ice, wood, or rock. All proxies have some way of accumulating the substance in layers that allow the age of the substance to be calculated. For example, in glaciers and trees, annual variations in snow fall and tree growth lead to annual layering, while in sediments variations in Earth's orbit might produce layers spanning hundreds of thousands of years\footnote{Hinnov, L. A. (2013). Cyclostratigraphy and its revolutionizing applications in the earth and planetary sciences. Geological Society of America Bulletin, 125(11-12), 1703-1734. \href{https://doi.org/10.1130/B30934.1}{Link}}. Finally, there must be something in the record that is sensitive to climatological aspects, the proxy itself. Glacial ice contains sulphate compounds, which reflect the volume of volcanic eruptions, while the relative abundance of different species of single-celled foraminifera reflect the temperature and salinity of the oceans in which they grew. Stable isotope ratios in those same shells are also good proxies for water temperature and acidity, and can even constrain the mass of ice on the continents.    

\begin{sidewaystable} 
\caption{Paleoclimate Proxies}
\label{tab:proxies}
\centering
\begin{tabular}{@{}llllllllll@{}} \\ \toprule
         &	&\multicolumn{8}{c}{Proxies} \\ \cmidrule(l){3-10}
				 &  &\multicolumn{2}{c}{Physical} &\multicolumn{4}{c}{Biological}&\multicolumn{2}{c}{Chemical}\\ \cmidrule(lr){3-4}  \cmidrule(lr){5-8} \cmidrule(l){9-10}
Recorder & Period (y)	&Form	&Dust &Fossils	&Charcoal	&Pollen, spores	&Biomarkers	&Stable isotopes	&Chemistry	 \\ \midrule
Sediments		& 1-0.5 Ma	&Sed. character	&\checkmark	&\checkmark	&\checkmark	&\checkmark	&\checkmark	&\checkmark	&\checkmark \\
Ice					&1-100 y		&	&\checkmark	&	&	&	&	&O, H, N &Na, SO4\\
Speleothems	&1-20 y		&\checkmark	&	&	&	&	&	&O &Th, U\\
Trees				&1-10 y		  &Ring width	&	&	&	&	&	&O, C, N &Zn, Pb\\
Corals			&1-20 y		  &\checkmark	&	&	&	&	&	&O, C &Th, U\\ \bottomrule
\end{tabular}
\end{sidewaystable}
% Sources include https://www2.usgs.gov/landresources/clu_rd/paleoclimate/proxies.asp, https://www.ncdc.noaa.gov/news/what-are-proxy-data

\subsection{How we know ``when:'' chronology}
The power of proxies rests on their ability to constrain the climate at a certain point in the past. For ice cores and rings from living trees, finding the age of a proxy is as straightforward as counting annual layers, although the inevitable complexities of nature lead to errors of 1-2\% in the ages\footnote{Svensson, A., Andersen, K. K., Bigler, M., Clausen, H. B., Dahl-Jensen, D., Davies, S. M., ... \& Steffensen, J. P. (2006). The Greenland ice core chronology 2005, 15-42 ka. Part 2: comparison to other records. Quaternary Science Reviews, 25(23-24), 3258-3267, \href{https://pdfs.semanticscholar.org/273c/c5454f8f3fef1e698bb3d1da55fff48ccc42.pdf}{Link}}. Sediments, speleothems and corals are typically dated radiometrically, using the decay of radioactive isotopes measured in the samples to date a few key horizons, and then assuming a constant rate of formation between the dated horizons\footnote{for one of thousands of examples of this, see Zhang, P., Cheng, H., Edwards, R. L., Chen, F., Wang, Y., Yang, X., ... \& An, C. (2008). A test of climate, sun, and culture relationships from an 1810-year Chinese cave record. science, 322(5903), 940-942. My nature requires that I mention Gould, S. J. (1965). Is uniformitarianism necessary?. American journal of science, 263(3), 223-228 as a persuasive counter to such assumptions.} Regardless of how a record is dated, the accuracy of the age is generally inversely proportional to the age, so as the age increases, so too does the uncertainty in that age. Natural processes tend to ``smear'' proxies across their record as well, which essentially averages a proxy signal over many time periods.\\     
\subsection{How we know ``where:'' field work}
The final facet of proxies we need to consider is their location, their place. For some proxy records, location is straight forward: a tree ring clearly must be from the place the tree grew. But in other cases, place is not so straightforward. Most sediment cores are collected from the sea floor. As we learned in Chapter 6, tectonic plates are constantly on the move, so sediments even a few million years old have moved since they were deposited. The climate signals recorded in speleothems reflect the average climate conditions over a large area proximate to the cave in which it is found, not just at the cave itself. Regardless of the area over which a proxy record records climate, scientists still need to get into the field to collect it. Field work is for many scientists a great joy of the job, even though it's often exhausting, filthy and occasionally dangerous work. Despite the travails, being ``in the field'' allows scientists to determine the setting and regional context of the record they are extracting from nature. Even cursory visual inspection in the field of an ice core, coral sample, or tree section can reveal interruptions, damage or alteration that reduce the sample's utility, allowing collection of additional samples.\\ 

\subsection{How we test hypotheses: models}
Dr. Wubbo Octels was a physicist, astronaut and keen observer of our planet\footnote{http://www.esa.int/Our\_Activities/Human\_Spaceflight/Astronauts/Wubbo\_J.\_Ockels, assessed 10 December 2018} who once noted that ``There's only one earth. And there's no spare.'' While Octels meant that statement to be a call to environmental action, it is also a cause to pause for a moment and ask how climatologists can do hypothesis-driven science if we only have one climate. You probably learned in high school or college classes that science good science involves studies with control runs and experimental runs, all rigidly testing single hypotheses. Real science doesn't run like that, particularly in the sciences (geology, meteorology, astronomy, evolution) that study enormous, complicated, or extinct systems. In these fields, we turn to \emph{natural experiments} provided by nature to test hypotheses\footnote{see the spectacular Understanding Science. 2018. University of California Museum of Paleontology. Accessed 10 December 2018; specifically \href{https://undsci.berkeley.edu/article/0_0_0/howscienceworks_08}{page 8} of their site. This brilliant site is good reading for all undergraduate students}. We have, fortunately, no way of generating tornadoes; instead meteorologists use their natural occurrence over time and space to test hypotheses about their formation and behavior\footnote{Perkins, Sid, Extreme tornado outbreaks are getting worse, but why? Dec. 1, 2016, Science Magazine doi:10.1126/science.aal0449 \href{https://www.sciencemag.org/news/2016/12/extreme-tornado-outbreaks-are-getting-worse-why}{link}, accessed 10 December 2018}. As we'll see below, the last 65 My of Earth's history provides exactly these kinds of natural experiments, which is why understanding Earth's past behavior is crucial to understanding soon-to-be climates. After all, we \emph{are} unintentionally running an experiment with climate: for 250 years we've been pumping greenhouse gasses into the atmosphere, and climate has in fact changed. But how can we be sure this correlation involves cause, and not coincidence?\\

Enter computer models, which attempt to approximate those enormous, complicated systems with mathematics. Some models are quite simple, doable with pencil and paper. Arguably the very first climate model was done in 1896 by Sweedish chemist Svante Arrhenius\footnote{Wearth, Spencer. The Discovery of Global Warming, February 2018
\href{shttps://history.aip.org/climate/co2.htm\#S1}{link}, accessed 10 December 2018}, who with pencil and paper correctly predicted that doubling the atmospheric concentration of $CO_2$ would raise average surface temperatures by \SIrange{5}{6}{\celsius} ($9-11 ^\circ\,F$). Arrhenius's work demonstrated the key aspect of climate models: use math to approximate the physics driving climate, and then change some part of the system and watch what happens to the rest. Energy balance models are a bit more complicated, using incoming and outgoing radiation, greenhouse gas concentrations and albedo to predict surface and atmospheric temperatures. You used just such a model in Chapter 4 to calculate Earth's average surface temperature for a variety of conditions! Intermediate complexity models add details\footnote{Harper, Lauren, 18 May 2018. What Are Climate Models and How Accurate Are They? Earth Institute at Columbia Unversity \href{https://blogs.ei.columbia.edu/2018/05/18/climate-models-accuracy}{link}, accessed 10 December 2018} of land, oceans, atmosphere and cryosphere to the model. More accurate are Atmosphere-Ocean General Circulation Models, which can resolve movement in the ocean and atmosphere, so provide insight into how the circulation of the spheres change over time. These complex models divide the atmosphere and ocean into cells, typically 100 km across and a few hundred meters thick (Figure \ref{modelgrid}). AOGCMs are accurate for seasonal to decadal time scales \footnote{IPCC AR5, Chapter 9, section 9.1.2.1}, because they lack an important part of the climate: feed backs between the spheres and the cycles of the elements, particularly the carbon cycle. Earth System Models do just this, incorporating an array of feedback processes in the models. If you or a friend have used gaming consoles, you are aware how games' complexity and realism grows as computers get faster and more powerful. So too with climate models, the most complex of which require hours-long runs on supercomputers (Figure \ref{modelgrid}). Additional details are avialalbe at CarbonBrief (\href{https://www.carbonbrief.org/qa-how-do-climate-models-work}{Link})\\
But the results are worth the effort: they are the very experimental Earths we need. A particularity important demonstration of the power of models is shown in Figure \ref{fig:2models}. In the upper pane, the actual average surface temperature is shown by the heavy black line. Swarming around it are light yellow lines, each showing the temperature produced by roughly 14 AOGCM models run 58 times, driven by all natural and anthropogenic forcings, including human-produced greenhouse gasses and pollutants. The scattered results are due to slight differences between the models and the data input in to them. One way to reduce this noise is to average the results, shown by the red line, which nicely matches the observed temperature trend, including the episodic, sudden, but brief drops in temperature caused by volcanic eruptions (shown by the dashed vertical lines). In the lower pane are 19 experiments from 5 models (in light blue), but here the anthropogenic forcings are omitted from the models. Note how the modeled temperatures (in dark blue) veer from actual temperatures in the early 1960s, with the differences increasing as the models approach the $21^{st}$ Century. This is the power of models: to test hypotheses about climate's past, present and future, and to attribute observed changes to actual causes.        

\section{Setting up the present: Earth's climate over the past 65 Ma}
\subsection{A generally warmer Earth}
We take for granted that the world we live in, with ice-covered poles and warm tropical oceans, is ''normal.'' But as we've seen repeatedly throughout this book, Earth's climate oscillates between two stable states: a warmer greenhouse world and a colder icehouse world\footnote{Anagnostou, E., John, E. H., Edgar, K. M., Foster, G. L., Ridgwell, A., Inglis, G. N., ... \& Pearson, P. N. (2016). Changing atmospheric CO 2 concentration was the primary driver of early Cenozoic climate. Nature, 533(7603), 380; particularly their reference 16}. The difference between these states is starkly evident in the Cenozoic Era, the part of Earth's history that began roughly 66 Ma (with giant impact that extirpated the non-bird dinosaurs) and continues to today. Earth sat firmly in the hothouse state for the first 30 My of the Cenozoic, with global temperatures at least 5 K warmer than now, and reptiles living comfortably near the ice-free poles. All this warmth was driven by extraordinarily high concentrations of greenhouse gasses in the atmosphere\footnote{Zachos, J. C., Dickens, G. R., \& Zeebe, R. E. (2008). An early Cenozoic perspective on greenhouse warming and carbon-cycle dynamics. Nature, 451(7176), 279.}, with $CO_2$ estimated from proxy data at 3000-4500 ppm, over ten times contemporary levels! The highest temperatures were during the Early Eocene Climatic Optimum, roughly 51-53 Ma, when Earth's average surface temperature soared \SI{14(3)}{\celsius} higher than our pre-industrial average, driven by the high (~1400 ppm) concentration of $CO_2$ in the atmosphere. Over millions of years (Figure \ref{fig:cz_climate_from_zachos}), plate tectonic cycles and weathering of the continents slowly withdrew the $CO_2$ from the atmosphere and deposited it in marine sediments. \\ 
\subsubsection{Hyperthermal episodes}
Those slow, majestic cycles produced slow, majestic changes to Earth's climate. But the Cenozoic period was also time of numerous \emph{hyperthermals}, brief episodes of extraordinary warmth and high $[CO_2]_{atm}$ that provide natural examples for how Earth will respond to our continuing release of greenhouse gasses into the atmosphere. Humans currently release roughly 10 GtC into the atmosphere every year\footnote{Cite as: Boden, T. A., Marland, G., and Andres, R. J.: Global, Regional, and National Fossil-Fuel CO2 Emissions, Oak Ridge National Laboratory, U.S. Department of Energy, Oak Ridge, Tenn., U.S.A., doi 10.3334\/CDIAC\/00001\_V2017, 2017; available at: \href{http://cdiac.ess-dive.lbl.gov/trends/emis/overview\_2014.html}{This link} and (land use emissions) average of two bookkeeping models: Houghton, R. A. and Nassikas, A. A.: Global and regional fluxes of carbon from land use and land cover change 1850-2015, Global Biogeochemical Cycles, 31, 456-472, 2017;  Hansis, E., Davis, S. J., and Pongratz, J.: Relevance of methodological choices for accounting of land use change carbon fluxes, Global Biogeochemical Cycles, 29, 1230-1246, 2015.}. The best-studied hyperthermal, the Paleocene-Eocene Thermal Maximum at ~55.8 Ma, followed the release of $<10,000$ Gt C into the atmosphere in less than 10,000 years\footnote{Ivany, L. C., Pietsch, C., Handley, J. C., Lockwood, R., Allmon, W. D., \& Sessa, J. A. (2018). Little lasting impact of the Paleocene-Eocene Thermal Maximum on shallow marine molluscan faunas. Science advances, 4(9), eaat5528.} While the source of all that carbon is still hotly debated\footnote{Gutjahr, M., Ridgwell, A., Sexton, P. F., Anagnostou, E., Pearson, P. N., P\"alike, H., ... \& Foster, G. L. (2017). Very large release of mostly volcanic carbon during the Palaeocene-Eocene Thermal Maximum. Nature, 548(7669), 573.}, it \emph{is} clear is that it was released far more slowly than we are currently releasing C into the environment\footnote{Zeebe, R. E., Ridgwell, A., \& Zachos, J. C. (2016). Anthropogenic carbon release rate unprecedented during the past 66 million years. Nature Geoscience, 9(4), 325.}. Global temperatures spiked upward by \SIrange{5}{8}{\celsius} during the PETM, with global changes to precipitation, acidification of the oceans, and invasive migration of mammals across the continents\footnote{McInerney, F. A., \& Wing, S. L. (2011). The Paleocene-Eocene Thermal Maximum: A perturbation of carbon cycle, climate, and biosphere with implications for the future. Annual Review of Earth and Planetary Sciences, 39, 489-516.}. Earth's systems needed almost 200 ka to recover from this release, which is in good accord with climate models. Earth's response to the PETM C surge teaches us three things. First, because humans most likely won't release 10,000 Gt of C into the atmosphere\footnote{McGlade, C., \& Ekins, P. (2015). The geographical distribution of fossil fuels unused when limiting global warming to 2 C. Nature, 517(7533), 187.}, we will see smaller changes than observed in the PETM. Second, our rate of C release into the atmopshere is 10 times that of the PETM, so climate change will be faster now than during the PETM. Many species, on land ind in the oceans, will be unable to adapt to changing conditions fast enough to avoid extinction. Finally, our study of the PETM shows that humanity's experiment with twiddling the control knobs of the climate system will persist for tens, even hundreds, of thousand of years into the future.  

\subsubsection{Beginnings of the chill: Antarctic ice develops}
Soon after the PETM, Earth's surface temperature peaked at \SI{14(3)}{\celsius} above pre-industrial temperatures, which works out to an astonishing \emph{average} surfaace temperatuer of $84\pm 6 ^\circ\,F$! \footnote{Anagnostou, E., John, E. H., Edgar, K. M., Foster, G. L., Ridgwell, A., Inglis, G. N., ... Pearson, P. N. (2016). Changing atmospheric CO2 concentration was the primary driver of early Cenozoic climate. Nature, 533(7603), 380-384. \href{https://doi.org/10.1038/nature17423}{Link}}. During this time, $CO_2$ concentrations were around 1400 ppm (Figure CZ-climate), about what Earth will have if humans burn all of the known, and most of the suspected, fossil fuels stored in the crust\footnote{Zachos et al., op cit. I assumed consumption of all fossil fuel reserves and resources (Rogner, H.-H. (1997). An Assessment of World Hydrocarbon Resources. Annual Review of Energy and the Environment, 22(1), 217-262. \href{https://doi.org/10.1146/annurev.energy.22.1.217}{Link}), contributing a total of 5000 GtC.}. But after peaking 52 Ma, atmospheric $CO_2$ began to fall slowly, and so too did surface temperatures. Just prior to 33 Ma, tectonic plate motion finally separated Antarctica from both South America and Australia, allowing the Antarctic Circumpolar Current to form, essentially isolating Antarctica from warm tropical seas. Glaciers rapidly spread across eastern Antarctica, plunging Earth into a new icehouse state\footnote{DeConto, R. M., \& Pollard, D. (2003). Rapid Cenozoic glaciation of Antarctica induced by declining atmospheric $CO_2$. Nature, 421(6920), 245.}. Only around 10 Ma did temperatures fall low enough to allow ice sheets to spread across Greenland and higher places on the continents. Icehouse was coming, but only after a last, brief warm period, one which has significant similarities to the Earth we are creating now through global climate change.\\ 
\subsection{The mid-Pliocene: An analog for our near future?}
This was the Mid-Pliocene Warm Period (MPWP), which lasted from 3.3 to 3.0 Ma. During this period (about as long as modern \textit{Homo sapiens} have existed\footnote{Hublin, J. J., Ben-Ncer, A., Bailey, S. E., Freidline, S. E., Neubauer, S., Skinner, M. M., ... \& Gunz, P. (2017). New fossils from Jebel Irhoud, Morocco and the pan-African origin of Homo sapiens. Nature, 546(7657), 289.}) $CO_2$ concentrations were in the range 350-450 ppm, and global average temperatures were \SIrange{2.7}{4}{\celsius} warmer than today. That climate was quite different than ours, with seas \SIrange{15}{25}{\metre} (50-80 feet) higher than now, a more energetic water cycle, strong polar amplification and a much reduced crysophere\footnote{ Haywood, A. M. et al. Integrating geological archives and climate models for the mid-Pliocene warm period. Nat. Commun. 7:10646 doi: 10.1038/ncomms10646 (2016).} The warmer, wetter atmosphere actually steered Hadley cell circulation toward the poles, resulting in changed precipitation patterns and smaller deserts. As we've seen with recent climate change, the poles warmed the most, with both proxy data and models suggesting increases of greater than \SI{10}{\celsius} in both the Arctic and Antarctic. You may have noticed that even though the MPWP climate was far different from ours today, atmospheric $CO_2$ concentrations in the MPWP were about what we have now. This is an accurate and crucial observation. The MPWP gives us insight into the \emph{equilibrium} climate Earth will have once our ``experiment''is over and Earth has fully adjusted to the changes we've imposed. Earth's climate needs thousands\footnote{Hansen, J., Sato, M., Russell, G., \& Kharecha, P. (2013). Climate sensitivity, sea level and atmospheric carbon dioxide. Phil. Trans. R. Soc. A, 371(2001), 20120294.}, even tens of thousands\footnote{Archer, D., Eby, M., Brovkin, V., Ridgwell, A., Cao, L., Mikolajewicz, U., ... \& Tokos, K. (2009). Atmospheric lifetime of fossil fuel carbon dioxide. Annual review of earth and planetary sciences, 37, 117-134.}, of years to adjust to the enormous changes we're imposing on it, so the current climate is only a transitional one. The choices our species makes now will affect Earth for as long as our species has been on the planet.\\ 

\section{Quaternary Glaciation: the (harsh) cradle of human culture}
One way that scientists measure the quality of their work is through citations, the number of times one of their papers is cited by authors' of other papers. By that index, a 2005 CE paper by Lorraine E. Lisiecki and Maureen Raymo ranks as one of the more important papers published this century, having been cited (according to a Google Scholar search in December 2018) an astonishing 4301 times, an average of nearly \emph{once per day} for 13 years. Their seminal contribution was using the $\delta O^{18}$ proxy to build a precise chronology of climate change for the past \textasciitilde 5 Ma. Their \emph{stack}, and the proxy behind it, are sufficiently important to deserve a careful explanation. Only after that can we fully appreciate the insights the proxy provides to us.
\subsection{The $\delta O^{18}$ proxy}
Most of the oxygen atoms (see Chapter 3) in the Solar System are the \isotope[16][8]{O} isotope: 8 protons (like all oxygen atoms, by definition) and 8 neutrons. But two other isotopes are also stable, \isotope[17][8]{O} and \isotope[18][8]{O}, as shown in Table \ref{tab:oxisos}. The vast majority of water molecules have \isotope[16][]{O}(we'll drop the ``8'' from now on), which is considered ``normal'' water. But roughly 0.2\% of water molecules have \isotope[18]{O}, making ``heavy'' water. Generally, different isotopes of the same element are chemically identical, because all isotopes of an element have the same electron configuration, which mediates chemical reactions. But relatively light isotopes, like those near the top of the periodic table (including oxygen) have substantially different masses. Note in Table \ref{tab:oxisos} that \isotope[18]]{O} is 16\% heavier than the more common \isotope[16][]{O}, so heavy water is about 11\% heavier than its normal cousin, and its physical behavior is ever-so-slightly different. Heavy water evaporates from the oceans less readily, and precipitates out of the atmosphere more readily, than normal water. The net effect of all this is to enrich the world's oceans with heavier water, and enrich the cryosphere in the normal isotope. This \emph{fractionation} effect is greater in warmer climate, less in cooler ones (Figure \ref{fig:isoexplainer}). The ratio of light to heavy water is measured by the $\delta O^{18}$, with more positive numbers indicating relatively more \isotope[18]{O}, and more negative numbers representing relatively less \isotope[18]{O}. Fortunately for climatology, single-celled shell-building foraminifera are plentiful in the oceans, and their $CaCO_3$ shells contain O atoms stripped from sea water. Foram shells record the ocean's $\delta O^{18}$, and provide an excellent proxy for sea temperature and the amount of continental glaciation. Because ice sheets preferentially store light O, in colder climate the oceans have more positive $\delta O^{18}$, and so too will forams growing in the that ocean. (Most people tend to think of colder temperatures as being lower than higher temperatures, which is why $\delta O^{18}$ is plotted in reverse on all the graphs in this book: higher ratios represent colder, icier conditions.)

\begin{table} 
\label{tab:oxisos}
\centering
\caption{Oxygen Isotopes}
\begin{tabular}{@{}llllll@{}} \toprule
Symbol						 &Number	&Neutrons			&Mass (amu) 	&Mass diff. from 16 (\%)&Abundance(\%)	\\ \midrule
\isotope[16][8]{O} &16			&8						&15.99491464	&0													 &99.759 	     \\
\isotope[17][8]{O} &17			&9						&16.99913060	&6													 &0.037			   \\
\isotope[18][8]{O} &18			&10						&17.99915939	&13													 &0.037        \\ \bottomrule
\end{tabular}
\end{table}


\subsection{Description of the ice ages/glacial-interglacial cycles (GIGC)}\label{GIGC_intro}
\subsubsection{Three climate patterns over the past 5 Mya}
The $\delta O^{18}$ record (Figure \ref{fig:raymostack}) shows three important climate trends over the past 5 My. The first is nicely demonstrated by the long, slow, downward curve in $\delta O^{18}$, indicating an increasingly cooler and icier Earth. The relative warmth of mid-Pliocene Warm Period stands out (as we saw above) as a distinctly warm pulse in the overall cooling trend, which accelerated after the MPWP. The Southern Hemisphere cooled first, with pulses of glacial growth and retreat around the edges of Antarctica\footnote{Raymo, M. E., Lisiecki, L. E., \& Nisancioglu, K. H. (2006). Plio-Pleistocene Ice Volume, Antarctic Climate, and the Global $\delta$ 18O Record. Science, 313(5786), 492-495. https://doi.org/10.1126/science.1123296}. Not until 2.5 Ma did glaciers spread across the Northern Hemisphere continents. Temperatures continued to fall (and ice cover increased) on average to the present day. But this decrease was ragged, which is the second climate pattern, evident as the ``saw-tooth'' pattern in the data. At the top of each oscillation, Earth was in an \emph{inter-glacial} period, with temperatures, ice cover and sea level all similar to our contemporary climate. But for the remainder of each oscillation, Earth was colder, ice more abundant, and sea level lower: these are \emph{glacial} intervals (more colloquially known as ``Ice Ages''), the standard state for Earth over much of the past 2.5 My.  From 2.5 Ma to roughly 1.5 Ma, these oscillations have periods averaging ~41 ky, the same period as Earth's obliquity oscilations. Around ear 1.5 Ma, the period changes to a much longer ~100 Ky, about the period of eccentricity oscillations in Earth's orbit. Clearly Earth's climate over the past 2.5 Ma is responsive to changes in insolation driven by changes in Earth's orbit, which was Milankovich's original hypothesis. The third and final pattern is the increasing magnitude of the glacial-interglacial cycles, which are now about 4 times larger than at 2.5 Ma.\\
The lower panel on Figure \ref{fig:RaymoStack} shows the past 800 ky in more detail. Glacial and interglacial periods are shown by the blue and pink shading at the top of the panel. Note the similarity in each cycle: a slow, bumpy descent to colder and icier conditions, punctuated by a rapid termination and entry to a new interglacial period. Larger bumps in the the descent are separated by ~41 ky (the obliquity period), and super-imposed on these are smaller, briefer bumplets separated by 23 ky, the precession period. For the past ~11.5 ky, Earth has sat firmly in an interglacial period, but this is rare. Over the past 800 kyr, Earth is in interglacials only 20\% of the time, with the other 80\% dominated by glacial cycles. The climate in which our culture has thrived is rather rare.\\
     

\subsection{Geographic distribution and climatic of glacial periods}
More common are times when half-mile thick sheets of ice covered half of North America and portions of Europe and Asia, as shown in Figure \ref{fig:LGMmap}.  At glacial maximum, The Laurentide Ice Sheet covered most of North America above 45\textdegree N latitude (with the curious exception of Alaska), and merged with the much expanded Greenland Ice Sheet. The Eurasian ice sheet stretched from the the British Isles, through fenno-scandanavia and into the far northern limits of Siberia, largely sparing the vast continent from ice. In the Southern Hemisphere, the relatively small ice sheet in the southern Andes was dwarfed by the extensive Antarctic Ice Sheet, most of which still remains today. \\
All that additional ice cover had a profound effect on global climate. The ice's high albedo drove Earth's average albedo up as well, reducing average surface temperatures across the globe. As we'll explore in detail below, atmospheric concentration of $CO_2$ plummeted during glacial episodes as well, roughly doubling the cooling effects of the reflective ice. Global temperatures during the most recent glacial periods were roughly \SI{8}{\celsius} (14\textdegree F) colder than pre-industrial times. Lower insolation, colder temperatures, the ``blocking'' effects of large glaciers all lead to significant changes in atmospheric circulation and precipitation across the globe\footnote{Ludwig, P., Schaffernicht, E. J., Shao, Y., \& Pinto, J. G. (2016). Regional atmospheric circulation over Europe during the Last Glacial Maximum and its links to precipitation. Journal of Geophysical Research: Atmospheres, 121(5), 2130-2145; \href{https://agupubs.onlinelibrary.wiley.com/doi/pdf/10.1002/2015JD024444}{Link}.} Colder air is drier air (Chapter 5), and so global precipitation was also lower, altering the global hydrosphere. The hydrosphere was noticeably smaller during glacial maxima. All the water locked in the ice sheets ultimately had to come from the oceans, and so glacial-interglacioal cycles drove changes in sea level as well (Figure \ref{fig:GISealevel}. Sea levels at hte last Glacial Maximum were an astonishing \SI{130(10)}{\metre} lower than currently, exposing (Figure \ref{fig:LGMmap}) wide swaths of the continental shelves. One of these formed a natural land bridge between Asia and North America, allowing multiple species, including \textit{H. sapiens} to migrate into North America. During the previous interglacial period, sea levels were\footnote{Dutton, A., \& Lambeck, K. (2012). Ice volume and sea level during the last interglacial. Science, 337(6091), 216-219, /href{http://science.sciencemag.org/content/337/6091/216}{Link}} \SIrange{6}{9}{\metre} (18 to 30 feet) higher than contemporary sea levels. That is far more than frozen in the mountain glaciers and continental ice sheets, and requires that substantial portions of the Greenland and Antarctic ice sheets melted as well. That interglacial period was about as warm as Earth is now, roughly \SI{1}{\kelvin} warmer than pre-industrial times, so it provides a natural experiment on how much sea level rise we can expect in the next few centuries.   
\subsection{Humans evolved during this icy stage of Earth's history} 
\subsubsection{We are inextricably linked to glacial Earth}
By any accurate measure, humans evolved physically and socially in a climate defined by repeated and sometimes rapid climate change: the Quaternary Period, one dominated by cold but also cyclical often rapid climate change. In some ways, our species' global success reflects our ability to thrive under change.
But as we'll see, the speed with which our activities are driving new climate change---to conditions not seen on earth since long-before we evolved---far exceed those that we've experienced in our evolutionary past. 
\begin{quotation}
	From AR 4 , Ch 6.4.1.1: ``The present atmospheric concentrations of CO2, CH4 and nitrous oxide (N2O) are higher than ever measured in the ice core record of the past 650 kyr (Figures 6.3 and 6.4). The measured concentrations of the three greenhouse gases fluctuated only slightly (within 4\% for CO2 and N2O and within 7\% for CH4) over the past millennium prior to the industrial era, and also varied within a restricted range over the late Quaternary. Within the last 200 years, the late Quaternary natural range has been exceeded by at least 25\% for CO2, 120\% for CH4 and 9\% for N2O. All three records show effects of the large and increasing growth in anthropogenic emissions during the industrial era.''
	``Variations in atmospheric CO2 dominate the radiative forcing by all three gases (Figure 6.4). The industrial era increase in CO2, and in the radiative forcing (Section 2.3) by all three gases, is similar in magnitude to the increase over the transitions from glacial to interglacial periods, but started from an interglacial level and occurred one to two orders of magnitude faster (Stocker and Monnin, 2003). There is no indication in the ice core record that an increase comparable in magnitude and rate to the industrial era has occurred in the past 650 kyr.''
\end{quotation}

\section{Causes}\label{GIGC_causes}
\subsection{Introduction: Only three ways to change $\overline{T}_{surf}$}
The glacial climate Earth's experienced over the past 2.8 Ma---the climate our species knows and loves---turns out to be unusually cool compared to climates of the past. How? What causes these cycles of global climate? Fortunately, we already know that changing global average surface temperature involves relatively few players:
\begin{align}
	\overline{T}_{surf}&=\left(\frac{S(1-A)}{4\sigma}\right)^{1/4}+\Delta T_{GHE} \label{eqTsa}\\
	\overline{T}_{surf}&=\left(\frac{S(1-A)}{4\sigma}\right)^{1/4}+\quad f([GHG]_{atm}) \label{eqTsb}
\end{align}
where $\overline{T}_{surf}$ is the global average surface temperature, $S$ is the global average insolation (the short wave radiation arriving from the Sun), $A$ is the global average albedo, and $\Delta T_{GHE}$ is the temperature increase from the greenhouse effect, which in turn is a function of the concentration of green house gasses in the atmosphere, $\quad f([GHG]_{atm})$. So making Earth cold enough to grow glaciers must involve changes to one or more of the three variables in \ref{eqTsb}. Insolation can change from internal variations of the Sun, as well as changes to Earth's distance from Sun or orientation relative to the Sun. Albedo, which controls the reflected short wave radiation, can vary due to changes in surface type (replacing low-albedo ocean with more reflective land), the atmosphere's cloudiness, even from volcanic eruptions. Finally changing the concentration of greenhouse gasses in the atmosphere changes the outgoing long wave (or infra-red) radiation, which also changes the surface temperature. Of course all these factors may change, and many of them will, through feedbacks, affect each other. We need to examine the influence of each factor in turn.\\

\subsection{Milankovich's insightful hypothesis: Orbital variations and insolation changes}
In Chapter 3 we noted that, ``with few exceptions,'' our Earth System included only Earth and the Sun. Understanding GIGC cycles is one of those few exceptions. To fully understand the driving forces responsible behind GIGC, we need to broaden our system to other objects in our Solar System. Jupiter, Saturn, Venus and the Moon all exert tiny gravitational torques on Earth, subtly swaying Earth's orbital path and orientation relative to the Sun. Planetary orbits were human's original clocks: the year is one orbit of Earth about the Sum, a month one orbit of the moon about Earth, and a day one rotation of Earth relative to the Sun. The clock-like regularity of GIGC suggest that orbital variations driven by graviatational perturbations are the pace-makers of the Ice Ages\footnote{Hays, J. D., Imbrie, J., \& Shackleton, N. J. (1976, December). Variations in the Earth's orbit: pacemaker of the ice ages. Science Magazine, 194(4270):1121-32.}. 
\subsubsection{The shape of orbits}
Take a piece of string, tie a light weight to it, and start twirling. Eventually you'll reach a point where the string exerts a noticeable force on your hand---you'll feel it pulling at your skin. Relax your grip for a moment and the weight will fly across the room; slow the twirling and the object will droop to a lower position. Keeping that weight moving consistently requires a perfect balance between the force you exert on the string and the weight's velocity. This applies to the orbits of objects in the Universe: a stable orbit requires a perfect balance between two forces\footnote{Ignoring relativistic effects!}. The mutual gravitational attractive force between objects is perfectly balanced by the tendency of an object to move in  straight line. These forces are universal, and so all orbits of all objects in the Universe follow the same fundamental rules, first worked out (through decades of exhausting effort) by Johannes Kepler in the $17^{th}$ century CE\footnote{see, for example, Dreyer, J. L. E. A History of Astronomy from Thales to Kepler, 1905, reprinted Dover Publications Inc, 1953. ISBN 0-486-60079-3}. For convenience, we'll focus on orbits of planets around the Sun, but the rules below are generally applicable to all orbits.\\ 

Kepler's first discovery was that orbits are ellipses lying in a plane with the Sun at one focus (Figure \ref{fig:ellipse}, Panel A) of the ellipse. Drawing a circle with a loop of string is easy: place a finger at one end of the loop and a pencil in the other, and drag the pencil around the finger keeping the loop taut. To make an ellipse, place two fingers in the loop, and have a friend hold the pencil. Now separate your fingers and have your friend draw while holding the loop taut. Each finger is a focus of the ellipse; moving the fingers farther apart makes a skinnier ellipse, moving them closer together makes a more circular ellipse. The Sun occupies one focus of the orbit, while the other is empty. The average distance between a planet and the Sun is the \emph{semi-major axis}, the distance from the long end of the ellipse to the center (Figure \ref{fig:ellipse}, Panel A). This distance is a fundamental factor determining a planet's average temperature, as the amount of sunlight hitting a planet changes inversely as the square of the distance: planets closer to the Sun receive far more insolation than more distant planets (recall Table 4.1). While the size of an orbit is given by the semi-major axis, the shape is given by the flattening of the ellipse, termed the \emph{eccentricity}. A circle has an eccentricity of 0 (more or less a circular character!), while a cigar-thin ellipse has an eccentricity of 0.9 (Figure \ref{fig:ellipse}, Panel B). Most planets in our Solar System have low eccentricity orbits---Earth's orbital eccentricity is currently 0.017\footnote{From NASA/JPL Horizons ephemeris, https://ssd.jpl.nasa.gov/horizons.cgi\#top, accessed 25 September 2018}, so small that you wouldn't be able to distinguish from 0. The ellipse in figure \ref{fig:ellipse} has an eccentricity of 0.76, some 40 times larger than Earth's orbit. As planets orbit the Sun on their eccentric orbits, their distance from the Sun continuously changes. At \emph{perihelion}, the planet is closest to the Sun, while at \emph{aphelion}, the planet is farthest from the Sun. \\

The time for a planet to complete one orbit about the Sun is the planet's period. Kepler's third discovery was that a planet's period depends only on the semi-major axis. Surprisingly the eccentricity plays no role setting a planet's period. But eccentricity does change how fast the planet moves in various parts of its orbit.  This is nicely shown by the filled dots on Figure \ref{fig:ellipse}, Panel A, which shows the location of a planet with a 1 year period at two-week intervals. Planets move faster than average at perihelion, and more slowly near aphelion. The reasons for this are easier to understand if you consider the orbit as a roller-coaster ride. As a planet approaches aphelion, it is moving farther from sun, working against gravity, essentially traveling uphill on the tracks. Conservation of energy requires that this gain in orbital potential energy is balanced by a loss of orbital kinetic energy: the planet moves more slowly along the orbit. This just what happens on a roller-coaster ride, with the cart slowing (but not stopping) as it reaches the crest of a hill. After passing the top, the cart accelerates downhill, converting potential to kinetic energy and speeding up until it reaches the bottom (Figure \ref{fig:ellipse}, Panel C). Planets on eccentric orbits spend more time farther from the Sun than nearer. While changing distance from the Sun isn't the cause of seasons, it does produce insolation changes during the year, the magnitude of which is proportional to the eccentricity.  At perihelion, Earth is 1.7\% closer to the Sun than on average, while at aphelion Earth is 1.7\% farther from the Sun. Because energy goes as the square of distance, Earth receives only 97\% of the average insolation at aphelion, but 103\%  at perihelion. Oddly enough, perihelion falls on January 4, in the middle of northern hemisphere winter.\\

Earth's seasons are caused by our planet's orientation in space, not our location on the orbit: this is a fundamental point and deserves careful description. As we saw above, Earth's orbit around the Sun is an ellipse lying in a plane. Earth's rotational axis is tilted from the perpendicular to this plane by \ang{23.4}. This \emph{obliquity} is the ultimate cause of the seasons. The northern axis points (for now) near the North Star, Polaris, and remains pointed there as Earth trundles around the Sun during the year. Because of the obliquity, Earth displays different aspects to the Sun throughout the year. For example, at the northern hemisphere \emph{summer solstice} typically on June 21, the entire northern hemisphere is tilted toward the Sun. During this time, the Sun appears high in the northern hemisphere sky, leading to intense insolation and longer daylight, both of which contribute to the higher temperatures we associate with summer. At the same time, areas in the southern hemisphere experience exactly the opposite conditions, with the Sun low in the sky, leading to low insolation, shorter daylight, and the colder temperatures of winter. Just three months later, at the \emph{autumnal equinox} on September 23, both hemispheres receive roughly equal insolation, beginning the fall and spring seasons in the northern and southern hemispheres respectively. At the \emph{winter solstice} on December 21, the northern hemisphere is tilted away from the sun, and we begin our long trudge into winter, finally emerging on the \emph{vernal equinox} on March 20 into a much-needed springtime. The annual variation of directness of insolation is the ultimate cause of the seasons. Planets with near-0 obliquity (Mercury, for example) don't have seasons, while planets with high obliquities (such as Uranus) have more extreme seasons than we do.\\  

To fully understand GIGC, we have to superpose Earth's orientation onto Earth's orbit, as in Figure \ref{fig:dummy}. Currently, Earth is near perihelion during the northern hemisphere winter, and at aphelion during summer (the relationship is revered in the southern hemisphere), and demonstrate nicely that the seasons are about obliquity, not distance! But recall that distance does produce $\pm 3\%$ changes in the intensity of sunlight. So northern hemisphere winters are moderated by our closeness to the Sun, while at the same time southern hemisphere summers are amplified. Likewise, temperatures in northern hemisphere summer are moderated by our greater distance from the Sun, but southern hemisphere winters are amplified, nudged a little colder than otherwise by the attenuated sunlight way out at aphelion. The combination of eccentricity and obliquity controls the magnitude of \emph{seasonality}, the variation in insolation during a year. The global annual average insolation depends only on the fixed semi-major axis of our orbit, and doesn't change as obliquity and eccentricity change. And they do change, slowly, as does the direction of Earth's axis. These changes are the pace-makers of the ice ages.\\

\subsubsection{Orbital mechanics and changes to earth's orbit}
Due to periodic alignments of Earth with Venus, Jupiter and Mars\footnote{Hinnov, L. A. Cyclostratigraphy and Its Revolutionizing Applications in the Earth and Planetary Sciences. Geological Society of America Bulletin 125, no. 11-12 (November 1, 2013): 1703-34. \href{https://doi.org/10.1130/B30934.1}{Link}.}, Earth's eccentricity varies smoothly with periods of roughly 100 and 405 kya, changing the orbit from circular to noticeably elliptical (as shown in Table \ref{tab:milank_values} and Figure \ref{fig:milank_draw}). At times of low eccentricity, Earth's seasonality is low, and \textit{vice versa}. Different gravitational interactions between Earth, the Moon and Mars drive faster variations in obliquity, with a period of 41 ky and an amplitude of only \ang{2}. Currently Earth's orbit is near the middle of these ranges, so our climate has only moderate seasonality. We saw above that Earth's axis points near the star Polaris, but this direction is only temporary. Gravitational forces from the Moon and Sun slowly torque the Earth, causing the direction of the North Pole to swing in a grand circle over a period of ~23 ky. Imagine we could extend Earth's poles to the stars. Over 23 ky, precession sweeps out a cone in space. This means that our current North Star, Polaris, is only a temporary one. About 12,000 years ago Vega, a bright star in the constellation Lyra, was the pole star, and it will be again 11,000 years hence. At that time, the relationship between seasons and perihelion was exactly reversed, with northern hemisphere summer at perihelion (leading to amplified summer heat) and winters at aphelion (and thus cold and long winters). All three of these cycles---in eccentricity, obliquity, and precession---happen simultaneously, but at different periods and amplitudes. And all of them affect seasonality and the distribution of insolation over Earth's surface. Fortunately, the gravitational forces at work here are well known, and computer models of Earth's all three are both precise and accurate, even as far back (or forward) as 50 My, so we can actually calculate the affect of all these variations on seasonality and Earth's climate with great precision. Figure \ref{fig:milank_wide} shows the past 1 My of orbital parameters and Oxygen isotope ratio, our proxy for iceiness. The orbital data are from a modern model\footnote{Laskar, J., Fienga, A., Gastineau, M., \& Manche, H. (2011). La2010: a new orbital solution for the long-term motion of the Earth. Astronomy \& Astrophysics, 532, A89.}, and the proxy data from studies all done within the past 30 years. Note how the subtle variations in eccentricity time the start of the ice ages, which start when eccentricity is high. It is remarkable that long before the invention of modern computers, Milutin Milankovitch had worked out the mathematics of how orbital changes and insolation variation, and predicted how these variations drive the ice ages.\\

\subsubsection{Milankovich's hypothesis}
Milankovich hypothesized in 1941 CE that ice ages begin when eccentricity, obliquity and precession all conspire to produce cool summers and intensely cold winters in the northern hemisphere. He correctly predicted that the cool summers would allow at least some of the previous winter's snow to remain unmelted, thus allowing the sudden build up of continental glaciers. Like many visionary scientists, he lacked the precise data he needed to adequately test his hypotheses at the time he made them, but we have that data now, and his explanation is fundamentally correct. Ice ages begin when high eccentricity happens about the same time precession aligns northern hemisphere summer with aphelion. The vagaries of geography are to blame for the northern hemisphere's predominate role in starting the ice ages. Look at a map (or better a globe) of Earth, and note that the band around \ang{60}-\ang{70}N latitude is largely dry land, a perfect place for snow and ice to grow. This is the core of the ``Milankovich'' hypothesis, and explains the clear periodic timing of glaciation revealed in Figure \ref{fig:o18_stack}: Quarterany ice ages are initiated by the rhythmic oscillations of Earth's orbit in response to gravitational nudges of other objects in the Solar System.  Two examples show this well. Figure \ref{fig:milank_closeup} shows two glacial-interglacial cycles, the most recent in blue solid lines, and the sixth-most recent in red dashed lines. Note that the peaks in eccentricity correspond to the origination of each Ice Ages. In both cases, the precession is high, indicating that northern hemisphere summer occurs near aphelion, as Milankovich predicted, although obliquities are not in terribly good agreement. (Few hypotheses are perfect!)\\

What Milankovich could not have predicted is that these changes in insolation are not enough to drive Earth into ice ages\footnote{AR 4 Ch 6.4.1.6; Abe-Ouchi, Ayako, Fuyuki Saito, Kenji Kawamura, Maureen E. Raymo, Jun'ichi Okuno, Kunio Takahashi, and Heinz Blatter. Insolation-Driven 100,000-Year Glacial Cycles and Hysteresis of Ice-Sheet Volume. Nature 500, no. 7461 (August 2013): 190-93. \href{https://doi.org/10.1038/nature12374}{Link} and references therein.}, even after accounting for the expected ice/snow albedo feedbacks. Sophisticated climate models clearly show that the insolation changes produced by Milankovich cycles are responsible for only \emph{half} of the temperature changes during  glacial cycle. Other factors are needed, factors which amplify the climatological muscle of orbital changes to insolation.\\
	
%\begin{quotation}
	%From AR 4 Ch 6.4.1.6 ``The Milankovitch theory proposes that ice ages were triggered by reduced summer insolation at high latitudes in the NH, enabling winter snowfall to persist all year and accumulate to build NH glacial ice sheets (Box 6.1). Continental ice sheet growth and associated sea level lowering took place at about 116 ka (Waelbroeck et al., 2002) when the summer incoming solar radiation in the NH at high latitudes reached minimum values. The inception took place while the continental ice volume was minimal and stable, and low and mid-latitudes of the North Atlantic continuously warm (Cortijo et al., 1999; Goni et al., 1999; McManus et al., 2002; Risebrobakken et al., 2005). When forced with orbital insolation changes, atmosphere-only models failed in the past to find the proper magnitude of response to allow for perennial snow cover. Models and data now show that shifts in the northern treeline, expansion of sea ice at high latitudes and warmer low-latitude oceans as a source of moisture for the ice sheets provide feedbacks that amplify the local insolation forcing over the high-latitude continents and allow for growth of ice sheets (Pons et al., 1992; Cortijo et al., 1999; Goni et al., 1999; Crucifix and Loutre, 2002; McManus et al., 2002; Jackson and Broccoli, 2003; Khodri et al., 2003; Meissner et al., 2003; Vettoretti and Peltier, 2003; Khodri et al., 2005; Risebrobakken et al., 2005). The rapid growth of ice sheets after inception is captured by EMICs that include models for continental ice, with increased Atlantic Meridional Overturning Circulation (MOC) allowing for increased snowfall. Increasing ice sheet altitude and extent is also important, although the ice volume-equivalent drop in sea level found in data records (Waelbroeck et al., 2002; Cutler et al., 2003) is not well reproduced in some EMIC simulations (Wang and Mysak, 2002; Kageyama et al., 2004; Calov et al., 2005).''
%\end{quotation}
\subsection{Albedo changes and other immediate feedbacks from growing ice sheets}
As we saw in Chapter 8, adding ice and snow to the planet lowers Earth's global albedo, which then positively feeds back to lower temperatures and contributes to more snow. Once ice sheets start forming in the polar northern hemisphere, this feedback contributes to rapid thickening and spreading of glaciers across the continents. All that water falling as snow and ice on the continents lowers sea level, replacing dark oceans with relatively reflective dry land and further raising albedo. All of these changes typically add up to an increase of 0.01 in albedo\footnote{AR4, Chapter 6}, a remarkable 3\% increase during glacial periods. Just the albedo change alone is enough to drop the global $T_e$ by a full \SI{1}{K}. As ice sheets thicken and spread, they get higher and colder, which again accelerates their growth. Finally, all that ice changes ocean and atmospheric circulation, which ends up providing additional ocean-derived moisture to the ice sheets, nudging Earth toward a new glacial cycle. \\
\subsection{C cycle feedbacks}
Nudge, but perhaps not push. All of the direct changes associated with insolation changes we disucssed above aren't enough to get Earth into a full glacial stage. Perhaps you aren't surprised to learn that the concentration of $CO_2$ changes in time with insolation changes, but scientists were: how could the carbon cycle and insolation be connected? The connection is excellent, too. Note how carefully (Figure \ref{fig:milank_wide})the concentration of $CO_2$ parallels insolation during GIGC \\  
Contribute about half the forcing, But not really understood how!\\
Probably involves changes in deep ocean, due to (1) Biology changes and/or\\
(2) Solubility of CO2 in oceans from T decrease and salinity increase \\
\begin{quotation}
	AR4 FAQ 6.1: ``Although it is not their primary cause, atmospheric carbon dioxide (CO2) also plays an important role in the ice ages. Antarctic ice core data show that CO2 concentration is low in the cold glacial times (~190 ppm), and high in the warm interglacials (~280 ppm); atmospheric CO2 follows temperature changes in Antarctica with a lag of some hundreds of years. Because the climate changes at the beginning and end of ice ages take several thousand years, most of these changes are affected by a positive CO2 feedback; that is, a small initial cooling due to the Milankovitch cycles is subsequently amplified as the CO2 concentration falls. Model simulations of ice age climate (see discussion in Section 6.4.1) yield realistic results only if the role of CO2 is accounted for. ''
\end{quotation}
\begin{quotation}
	AR 4, FAQ 6.2.  ``Most explanations propose changes in oceanic processes as the cause for low glacial CO2 concentrations. The ocean is by far the largest of the relatively fast-exchanging (<1 kyr) carbon reservoirs, and terrestrial changes cannot explain the low glacial values because terrestrial storage was also low at the Last Glacial Maximum (see Section 6.4.1). On glacial-interglacial time scales, atmospheric CO2 is mainly governed by the interplay between ocean circulation, marine biological activity, ocean-sediment interactions, seawater carbonate chemistry and air-sea exchange. Upon dissolution in seawater, CO2 maintains an acid/base equilibrium with bicarbonate and carbonate ions that depends on the acid-titrating capacity of seawater (i.e., alkalinity). Atmospheric CO2 would be higher if the ocean lacked biological activity. CO2 is more soluble in colder than in warmer waters; therefore, changes in surface and deep ocean temperature have the potential to alter atmospheric CO2. Most hypotheses focus on the Southern Ocean, where large volume- fractions of the cold deep-water masses of the world ocean are currently formed, and large amounts of biological nutrients (phosphate and nitrate) upwelling to the surface remain unused. A strong argument for the importance of SH processes is the co-evolution of antarctic temperature and atmospheric CO2.\\
	One family of hypotheses regarding low glacial atmospheric CO2 values invokes an increase or redistribution in the ocean alkalinity as a primary cause. Potential mechanisms are (i) the increase of calcium carbonate (CaCO3) weathering on land, (ii) a decrease of coral reef growth in the shallow ocean, or (iii) a change in the export ratio of CaCO3 and organic material to the deep ocean. These mechanisms require large changes in the deposition pattern of CaCO3 to explain the full amplitude of the glacial-interglacial CO2 difference through a mechanism called carbonate compensation (Archer et al., 2000). The available sediment data do not support a dominant role for carbonate compensation in explaining low glacial CO2 levels. Furthermore, carbonate compensation may only explain slow CO2 variation, as its time scale is multi-millennial.\\
	Another family of hypotheses invokes changes in the sinking of marine plankton. Possible mechanisms include (iv) fertilization of phytoplankton growth in the Southern Ocean by increased deposition of iron-containing dust from the atmosphere after being carried by winds from colder, drier continental areas, and a subsequent redistribution of limiting nutrients; (v) an increase in the whole ocean nutrient content (e.g., through input of material exposed on shelves or nitrogen fixation); and (vi) an increase in the ratio between carbon and other nutrients assimilated in organic material, resulting in a higher carbon export per unit of limiting nutrient exported. As with the first family of hypotheses, this family of mechanisms also suffers from the inability to account for the full amplitude of the reconstructed CO2 variations when constrained by the available information. For example, periods of enhanced biological production and increased dustiness (iron supply) are coincident with CO2 concentration changes of 20 to 50 ppm (see Section 6.4.2, Figure 6.7). Model simulations consistently suggest a limited role for iron in regulating past atmospheric CO2 concentration (Bopp et al., 2002).\\
	Physical processes also likely contributed to the observed CO2 variations. Possible mechanisms include (vii) changes in ocean temperature (and salinity), (viii) suppression of air-sea gas exchange by sea ice, and (ix) increased stratification in the Southern Ocean. The combined changes in temperature and salinity increased the solubility of CO2, causing a depletion in atmospheric CO2 of perhaps 30 ppm. Simulations with general circulation ocean models do not fully support the gas exchange-sea ice hypothesis. One explanation (ix) conceived in the 1980s invokes more stratification, less upwelling of carbon and nutrient-rich waters to the surface of the Southern Ocean and increased carbon storage at depth during glacial times. The stratification may have caused a depletion of nutrients and carbon at the surface, but proxy evidence for surface nutrient utilisation is controversial. Qualitatively, the slow ventilation is consistent with very saline and very cold deep waters reconstructed for the last glacial maximum (Adkins et al., 2002), as well as low glacial stable carbon isotope ratios (13C/12C) in the deep South Atlantic.\\
	In conclusion, the explanation of glacial-interglacial CO2 variations remains a difficult attribution problem. It appears likely that a range of mechanisms have acted in concert (e.g., K\"ohler et al., 2005). The future challenge is not only to explain the amplitude of glacial-interglacial CO2 variations, but the complex temporal evolution of atmospheric CO2 and climate consistently.''
\end{quotation}

\subsection{Variations in Solar Luminosity}
Long term changes from Sun's core not applicable here (too long, plus only one direction!)\\
\subsubsection{But the sun undergoes shorter-term cycles}
Very short term cycles as measured by satellites. (Too short to matter for GIG cycles!)\\
The 11-year magnetic cycle revealed by sunspot cycle Too short!\\
Centennial to millennial cycles as measured by B and C isotopes still too short\\

\section{The current interglacial}
Humans have been busy since the end of the last glacial cycle. At the end of the most recent (it won't be the last) glacial interval 12 kya, there were 1-10 million humans on the planet. To put this in perspective, more people live today in Los Angeles, California than lived on the entire planet 12,000 years ago. But as ice retreated, the globe warmed, and the atmosphere grew wetter, small bands of humans around the world invented agriculture\footnote{Ruddiman, W. F., Fuller, D. Q., Kutzbach, J. E., Tzedakis, P. C., Kaplan, J. O., Ellis, E. C., ... \& Lemmen, C. (2016). Late Holocene climate: Natural or anthropogenic?. Reviews of Geophysics, 54(1), 93-118.}. Modern humans---accustomed to the seemingly abundant food provided by modern industrial agriculture---take it for granted that the invention of agriculture was a natural and beneficial progression of human societies, but little evidence suggests this. Instead, before agriculture and after, human population crept upward, doubling every 2000 years or so from 12 to 2 kya\footnote{Zahid, H. J., Robinson, E., \& Kelly, R. L. (2016). Agriculture, population growth, and statistical analysis of the radiocarbon record. Proceedings of the National Academy of Sciences, 113(4), 931-935.}. By the dawn of the common era, 2 kya, global population had grown to 200 to 400 million souls, or about the population of the U. S. Global population is now approximately 7.6 billion and increasing at 1\% per year. All of this growth has happened over the current interglacial period, a brief respite from the usual glacial conditions which have dominated the Northern Hemisphere for the past 1 My. In this section, we examine how the climate of this interglacial has affected human culture, and perhaps equally important, how has human culture affected climate of the interglacial.\\

 
And then there is us, Homo sapiens sapiens, and the remarkable way we have powered our global influence through our unique ability to use energy. Energy not just to run our bodies, but our societies. As of 2013, the average global per capita energy consumption was over 25 times the daily nutritional requirement of 2000 kcal/day; for the United States this ratio is closer to 100. We are the only species that can accomplish this remarkable trait: use more energy than the body can consume to do work upon the environment. This work has changed all five spheres of the planet.

\subsection{Temperatures variations}
The end of the most recent glaciation was marked by a rapid rise in average surface temperatures, driven by increasing $CO_2$ concentrations and (in yet another instance of the ice-albedo feedback) decreasing extent of ice in the northern hemisphere. Extensive research and collection of proxy records allows us to construct a accurate temperature record for the past 12 ky, as shown in Figure \ref{fig:HTR} (Panel A). Temperatures rose by over \SI{4}{\kelvin} ($7\circ F$) during two long periods from 15,000 BCE to about 2000 BCE. After this, temperatures began a slow but steady fall. From 500 CE to the present (\ref{fig:HTR}, Panel B) the density of proxy records increases, allowing us to see shorter and smaller temperature changes, reflected in the increased ``bumpiness'' in the record. Three features in that graph are important to our story, and we'll examine them in detail in following sections and chapters. The \emph{Medieval Climate Anomaly}, from roughly 950 CE to roughly 1250 CE, was a period of generally stable climate with warm winters, shown with orange highlighting in Figure \ref{fig:HTR}, Panel B. The ``MCA'' was followed by a cool period of highly variable weather and cold winters, the Little Ice Age, that had profound and generally negative effect on European culture in particular.\\
The third trend is unprecedented increase in temperatures since 1850 CE, a feature famously called the``hockey stick,'' following its first publication in 1998 CE \footnote{Mann, M. E., Bradley, R. S., \& Hughes, M. K. (1998). Global-scale temperature patterns and climate forcing over the past six centuries. Nature, 392(6678), 779.}. The graph, the data behind it, and the climatologists who first published it have all been the subject of extra-ordinary and extra-unjust controversy, including death threats and congressional inquiries\footnote{See Robin McKie Sat 3 Mar 2012, \textit{Death threats, intimidation and abuse: climate change scientist Michael E. Mann counts the cost of honesty}, The Guardian, \href{https://www.theguardian.com/science/2012/mar/03/michael-mann-climate-change-deniers}{This The Guardian article}.} In the 20 years since its first publication, the trends demonstrated by Figure \ref{fig:HTR} have been repeatedly confirmed\footnote{National Research Council. (2007). Surface temperature reconstructions for the last 2,000 years. National Academies Press; \href{https://www.nap.edu/read/11676/chapter/1}{Link}}: average surface temperatures have increased in an unprecedented way since roughly 1850 CE, due largely to emission of GHG, land use changes and the manufacture of cement by humans. The remarkable way we have powered our global influence is through our unique ability to use energy. Energy not just to run our bodies, but our societies. As of 2014 (the most recent year for which data are available\footnote{World Bank Open Data Bank, \href{https://data.worldbank.org/indicator/EG.USE.PCAP.KG.OE}{link} accessed 18 October 2018}, the average global per capita energy consumption was over 25 times the daily nutritional requirement of 2000 kcal/day; for the United States this ratio is closer to 100. We are the only species capable of this remarkable ability: use more energy than the body can consume to do work upon the environment. This work has changed all five spheres of the planet, and may even have stalled the beginning of the next Ice Age.   \\

\subsection{What might have been: natural progression w/o humans}
We saw in Section \ref{GIGC_causes} that the primary drivers of GIGC---insolation changes from orbital oscillations, and variations in $[CO_2]_{atm}$---are both well constrained for the past 800 kya. We also saw that most interglacial periods are pretty short, geologically speaking, ranging from 7 to 30 ky, with a median of about 14 ky. Our interglacial is about 12 ky old, and based on the recent past, it would seem our interglacial might be almost over. How much longer will it last? A reasonable (even hopeful) question is whether the natural GIGC cycle might counteract global warming and usher in a new Ice Age. The idea that Earth could be on the precipice of another ice age might seem ridiculous to people raised in an ever-warmer planet, but the entire point of this book is to use the past as a guide for understanding the present and future. Figure \ref{fig:holo_temp} shows the Northern Hemisphere land temperatures changes since the Last Glacial Maximum. Temperatures rise in two stages \footnote{Shakun, J. D., Clark, P. U., He, F., Marcott, S. A., Mix, A. C., Liu, Z., ... \& Bard, E. (2012). Global warming preceded by increasing carbon dioxide concentrations during the last deglaciation. Nature, 484(7392), 49.} with a sudden and deep return to colder temperatures in between. Over the past 11,000 years, temperatures peaked, and then begin a slow, gradual decline until the beginning of the industrial Revolution, around 1750 CE.

As we saw above, high eccentricity and precession forcing northern hemisphere summer at aphelion are both necessary to kick off a glacial cycle. Notice on Figure \ref{fig:milank_wide} that our contemporary eccentricity (Panel B, blue line) is actually quite low compared to other initiations, although precession is exactly right for initiation of a new glacial cycle. According to some models\footnote{Ganopolski, A., Winkelmann, R., \& Schellnhuber, H. J. (2016). Critical insolation-$CO_2$ relation for diagnosing past and future glacial inception. Nature, 529(7585), 200.}, it was a close call: we may have just missed falling in to a new glacial cycle 900 years ago\footnote{Ruddiman, W. F., Fuller, D. Q., Kutzbach, J. E., Tzedakis, P. C., Kaplan, J. O., Ellis, E. C., ... \& Lemmen, C. (2016). Late Holocene climate: Natural or anthropogenic?. Reviews of Geophysics, 54(1), 93-118.} or so, the result of a little too much insolation and relatively high $CO_2$ concentrations in this interglacial (see Section \ref{ag_GHG}). The same models suggest that, even without anthropogenic greenhouse gas emissions, Earth will spend the next 50 to 100 ky in the current inter-glacial, the longest interval in the past 2.75 Ma. A new glacial cycle will not save human societies from global warming.          
\subsection{Agriculture and GHG forcings during the past 7 ky}\label{Ag_GHG}
And we may very well have the foundation of modern society---agriculture---to thank for that near miss. Some aspects of human behavior, like music, counting, and story-telling\footnote{Brown, D.E. 1991. Human universals. New York: McGraw-Hill}, are universal, and appear in every society. Agriculture is not a ``universal,'' but pretty close. It was invented independently by dozens of different societies around the globe, mostly after 12 kya\footnote{Price, T. D., \& Bar-Yosef, O. (2011). The origins of agriculture: new data, new ideas: an introduction to supplement 4. Current Anthropology, 52(S4), S163-S174; Ruddiman et al. op cit. and references therein} as warmer, wetter and more stable climates fostered easier domestication of plants and animals\footnote{Zahid, H. Jabran, Erick Robinson, and Robert L. Kelly. Agriculture, Population Growth, and Statistical Analysis of the Radiocarbon Record. Proceedings of the National Academy of Sciences 113, no. 4 (January 26, 2016): 931-35. \href{https://doi.org/10.1073/pnas.1517650112}.}. Prior to agriculture, small bands of hunter-gatherers roamed freely over a home territory, utilizing found resources in an area on the order of \SI{100}{\kilo\metre\square}\footnote{Tallavaara, M., Eronen, J. T., \& Luoto, M. (2017). Productivity, biodiversity, and pathogens influence the global hunter-gatherer population density. Proceedings of the National Academy of Sciences, 201715638, assuming a band size of ~50 people, and using the breakpoint on figure 2 as a ``typical'' population density.}. Their use of resources, and the waste products they produced, were no more than other non-human organisms produced. But the spread of domesticated plants and animals changed both the nature and magnitude of waste production, and the evidence for this comes from the same ice cores we've already used.\\

\emph{Land use changes} required by agriculture include burning, clearing, and even submergence of natural plant cover to make land amenable to our needs. Raising domesticated plants and animals fundamentally changes the way landscapes retain and cycle nutrients, including carbon, so we shouldn't be surprised that agriculture's early growth and spread affected the carbon cycle, and thus changed the atmospheric concentrations of important greenhouse gasses, such as $CO_2$ and methane ($CH_4$). Figure \ref{fig:ruddiman} shows (in blue and green) the typical trajectory of GHG during the 8 previous times Earth moved from inter-glacial to glacial conditions, lined up using the cyclical nature of the GIGC. In all cases, both $CO_2$ and $CH_4$ steadily increase as ice ages end, reaching a peak at the warmest and least-glacial time. The most recent cycle (in orange and red) shows the same pattern, as would be expected from the regular drum beat of glaciatial cycles. After the peak, GHG concentrations in the previous 8 cycles all slowly decrease, nudging Earth into the subsequent glacial cycle. But the most recent cycle---the one and only to include the development and spread of agriculture---is different. Notice in Figure \ref{fig:ruddiman}, Panel B that rather than falling, $CO_2$ concentrations begin to rise about 6 kya during the present cycle, the same time agriculture began spreading across the planet. Methane concentrations also rise beginning about 4 kya, when rice cultivation began in southeastern China. (Rice paddies emit prodigious amounts of methane, due to anaerobic respiration of bacteria in the muds of the flooded paddies). In a decades-long series of studies\footnote{Ruddiman, William F. (2013). \textit{Earth Transformed} (1st ed.). W.H. Freeman. ISBN 978-1-4641-0776-4 is encyclopedic, but Ruddiman, W. F., Fuller, D. Q., Kutzbach, J. E., Tzedakis, P. C., Kaplan, J. O., Ellis, E. C., ... \& Lemmen, C. (2016). Late Holocene climate: Natural or anthropogenic?. Reviews of Geophysics, 54(1), 93-118 is briefer and equally compelling}, William Ruddiman has convincingly argued that human agriculture caused these changes, and that our species has been actively modifying Earth's climate for 6000 years. By the beginning of the Industrial Revolution around 1750 CE, $CO_2$ concentrations were 25 ppm higher than would have been expected, warming Earth by ~\SI{0.3}{\kelvin} during the very time insolation reached a minimum. Earth barely escaped descending into an ice age in the past 1000 years, and our early---and inadvertent---climate manipulation through agriculture may be part of the reason for that escape. \\
    
\subsection{Culture and Climate Collide: Climate Change in the Historic Period}
As we slowly approach the present day, we should expect two facets of our study of climate to change. One is that paleo-temperatures and other climatological features will have more precision and more accuracy. Our proxies will be more numerous\footnote{For example, Figure 2 (bottom panel) in Ahmed, M., Anchukaitis, K. J., Asrat, A., Borgaonkar, H. P., Braida, M., Buckley, B. M., ... \& Curran, M. A. (2013). Continental-scale temperature variability during the past two millennia. Nature Geoscience, 6(5), 339.} and so the records will have much better time resolution. Both of these attributes are nicely displayed in Figure \ref{fig:HTR} Panel A, where data from 500 CE onward is finely resolved on an annual basis, and rather subtle temperature changes become apparent. The second change is that finally we can \emph{do} something with all this data, because in modern times we have cultural records, actual written and physical objects which allow us to finally see how human societies reacted in the past to climate change. After all, the real goal of this book is to help you decide if and what you want to do about Earth's changing climate.

\subsubsection{Medieval Climate Anomaly}
Perhaps the most famous society that dealt with climate change in the past millennium are the Norse (the ancestors of Vikings) who settled islands (Figure \ref{fig:norse_map} in the North Atlantic Ocean, including the Faroe Islands (around 800 CE)\footnote{Arge, S. V., Sveinbjarnard\'ottir, G., Edwards, K. J., \& Buckland, P. C. (2005). Viking and medieval settlement in the Faroes: people, place and environment. Human Ecology, 33(5), 597-620.}, Iceland (around 973 CE), and Greenland (around 985 CE)\footnote{for both Iceland and Greenland: Hartman, S., Ogilvie, A. E. J., Ingimundarson, J. H., Dugmore, A. J., Hambrecht, G., \& McGovern, T. H. (2017). Medieval Iceland, Greenland, and the New Human Condition: A case study in integrated environmental humanities. Global and Planetary Change, 156, 123-139. https://doi.org/10.1016/j.gloplacha.2017.04.007
}. Yes, Greenland, where a few thousand Norse men, women and children\footnote{Hartmen et al. op cit.} successfully farmed, hunted and traded in two different colonies for 500 years. What drove these settlers to journey in open boats across hundreds of miles of featureless ocean? How could Greenland be a place where farming and animal husbandry was possible? Drought and warmer winters seems to be the key.\\

The Norse began their settlement of the Atlantic at the start of the Medieval Climate Anomaly, a period\footnote{Stine, S. (1994). Extreme and persistent drought in California and Patagonia during mediaeval time. Nature, 369(6481), 546.} where much of the northern hemisphere was relatively warm (particularly in the winter)\footnote{Trouet, V., Esper, J., Graham, N. E., Baker, A., Scourse, J. D., \& Frank, D. C. (2009). Persistent Positive North Atlantic Oscillation Mode Dominated the Medieval Climate Anomaly. Science, 324(5923), 78-80. https://doi.org/10.1126/science.1166349} but experienced ``frequent, extensive and persistence summer time drought''\footnote{Cook, E. R., Seager, R., Kushnir, Y., Briffa, K. R., B\"untgen, U., Frank, D., ... \& Baillie, M. (2015). Old World megadroughts and pluvials during the Common Era. Science Advances, 1(10), e1500561.}, but relatively mild winters\footnote{Diaz, H. F., Trigo, R., Hughes, M. K., Mann, M. E., Xoplaki, E., \& Barriopedro, D. (2011). Spatial and temporal characteristics of climate in medieval times revisited. Bulletin of the American Meteorological Society, 92(11), 1487-1500.} and unusually low climatic variability. The MCA was\emph{in some parts of the Northern Hemisphere} the most recent period of warmth rivaling the 20th Century's warmest periods, and thus provides a natural experiment for how societies react to climate change. The Norse are interesting not because of who they were or their ancestry, but because they illustrate the vulnerability of a society at the edge of sustainability to both changing climates, and unchanging cultural practice\footnote{Hartman, S., Ogilvie, A. E. J., Ingimundarson, J. H., Dugmore, A. J., Hambrecht, G., \& McGovern, T. H. (2017). Medieval Iceland, Greenland, and the New Human Condition: A case study in integrated environmental humanities. Global and Planetary Change, 156, 123-139. https://doi.org/10.1016/j.gloplacha.2017.04.007
, Nelson, M. C., Ingram, S. E., Dugmore, A. J., Streeter, R., Peeples, M. A., McGovern, T. H., ... Smiarowski, K. (2016). Climate challenges, vulnerabilities, and food security. Proceedings of the National Academy of Sciences, 113(2), 298-303. https://doi.org/10.1073/pnas.1506494113}. Ultimately, the Norse abandoned their settlements in Greenland because climate and the world economy changed faster than their culture could. \\

The Norse were a sea-going people who farmed, raised domesticated animals (including cattle, sheep and goats)\footnote{Dugmore et al. op cit.} and traded as a way of life. The droughts accompanying the MCA were centered over their Scandinavian homeland (Figure \ref{fig:drought}, and may have contributed to their westward expansion. In Greenland the Norse lived in small, largely isolated farms, where they spent summers growing food for themselves and their animals. Growing fodder for their animals was an existential prblem: despite the MCA's warmth, Greenlandian winters were long, so animals were kept indoors, and fed fodder by hand, for 9 months of the year\footnote{Hartman et al., op cit.}. Theirs was a precarious existence, so they supplemented their crops by hunting and collecting wild food, most of land-based animals, such a reindeer, with the occasional seal. By 1125 CE the Norse has started a new---and lucrative---way of making a living\footnote{Star, B., Barrett, J. H., Gondek, A. T., \& Boessenkool, S. (n.d.). Ancient DNA reveals the chronology of walrus ivory trade from Norse Greenland, 9.}: trading in walrus ivory. Nearly every Norse farm have abundant fragments of walrus skulls. No tusks, no bones of any other part of the animals, just the skulls. These fragments are testaments to the communal hunts members of both the Eastern and Western settlements would conduct each summer, as they rowed and sailed the 800 km or more in open boats from their settlements to the \textit{nor\dh rsetur} or Northern Hunting Ground near Disko Bay, Greenland. There they harvested walrus ivory for later shipment to Norway and Europe, where it was a luxury good and in demand in an increasingly urban and expanding populace\footnote{Star et al. op. cit}. The walrus trade bought in enough money and goods to establish and outfit exquisite churches (Figure ) and may well subsidized the settlements pastoral economies. But beginning around 1250 CE, as the MCA drew to a close, the diet, climate and culture of the Greenland Norse began to change. \\

We know this from another proxy: the isotopic composition of C and N from the settler's remains. Terrestrial and marine animals eat foods with distinctly different ratios of C and N isotopes. Consumers of these animals---the Norse, in this case---retain their food's isotopic composition even after their deaths. Near 1250 CE isotopic signals indicate a sudden, dramatic and permanent change from primarily land-based to marine-based food for the majority of Greenlanders. Seals migrating south in the spring were hunted communally, and these marine mammals became a staple for all classes and genders in the population. The shift is good evidence of crop failure in an increasingly chilly and stormier climate. Those same storms made seal hunting more dangerous, even fatal, to the very people who were increasingly needed to work the farms\footnote{Hartman et al. op cit.}. But the archaeological record indicates that the small, outlying farms that predominated in the  settlements were abandoned, with their farmers taking up as tenants in large central farms as a way of concentrating the work force. But other changes, far outside the control of Greenlanders, were afoot. Europe was changing, and walrus ivory dropped in value over the 1400s, while fish and wool became important trading goods. Greenlanders choose to provide neither, so trade dried up, depriving them of much needed supplies from Europe at the very time their own environment was providing less and less. The smaller Western Settlement was finally abandoned around 1360 CE\footnote{Young et al., op cit.}, just before the last regular shipping between Greenland and Europe around 1370 CE. The last known voyage to or from the colony was made around 1420, and by 1450 CE, the Eastern Settlement was also abandoned. The Norse had lived successfully on the shores of Greenland for 450 years, nearly twice as long as the US has been a country. They successfully adapted much of their lifestyle to conditions imposed on them by the increasingly cold and harsh climate of the period from 1350-1145 CE, but these changes were too small and too late to prevent their society's collapse. They were not the only society to suffer from the climate and cultural changes of the Little Ice Age, but they may well have been the first.\\

\subsubsection{Little Ice Age}
\paragraph{Geographic extent and magnitude}
The relative warmth and stability of the MCA led to unparallelled growth of populations across the northern hemisphere. Both slowly but profoundly eroded after 1250 CE, until by ~1450 CE the Northern Hemisphere had entered a Little Ice Age. This 4 century-long period of cold centered on the Northern Hemisphere, and was strongest across the temperate and sub-polar land areas of the Northern Hemisphere (Figure AR 5. chapter 5 figure 5.9), from Japan, across northern Asia, through Europe and onto most of North America. All of the southern continents and much of southern Asia were largely spared the colder climates of the LIA\footnote{AR 5, Chapter 5, Figure 5.09, panels c, e}. Average temperatures in the Northern Hemisphere were only \SI{0.5(2)}{\celsius} (0.9 $^\circ$F) colder than the $20^{th}$ century average, which doesn't sound terribly cold! But even this moderate change was enough to cause glacier growth throughout Europe, Asia and North America, bring snowier and longer winters, freeze rivers, and to make farming impossible in many northern areas (as we saw in Greenland) and higher-elevation areas. To see how the LIA really affected societies, we should consider changes between the MCA and the LIA, as that was the transition that societies actually experienced. Europe and North America both cooled by \SI{1(2)}{\celsius} (2 $^\circ$F) from the MCA to LIA, which stressed an already shaken continent. The LIA to produced profound challenges and changes to societies throughout the hemisphere, changes that led to colonization of North and South America by Europeans, to global and long-lasting effect.

\subsubsection{Societal consequences}
A series of non-climate disasters shook Europe and China in the centuries between the end of the MCA in 1250 CE and the beginning of the LIA in 1450 CE\footnote{White, op cit.}. Widespread crop failure led to equally-widespread famine across Europe in 1310 CE. The Black Death---a plague caused by bacteria spread by rats---raced across Europe between 1347 and 1351 CE, leaving 30 to 50\% of the population dead. So much of Europe was already suffering when temperatures dropped during the LIA. But the cold, long winters nucleated a raft of cultural, political, social and even artistic movements in Europe that, as Sam White put so beautifully, \footnote{White, op cit, p 339} ``may reflect the direct or indirect influence of extreme weather and climate'' of the LIA. These included, as incredible as it may seem to those of us accustomed to winter, the development of architecture and clothing appropriate for colder weather, ``gloomy winter poetry'' and even art. \\

\paragraph{Art}
Hannah Stamler\footnote{Stamler, Hannah (2018) How Curators Are Finding the Climate in Art History, } contends that ``All works of art contain... an environmental perspective... we need only retrain our eyes to look for them.'' Following that edict, The earliest known portrayal of snow in European art is Ambrogio Lorenzetti's \textit{Effects of Good Government in the City and Countryside} from 1337-1339 CE\footnote{Philip McCouat, "The emergence of the winter landscape: Bruegel and his predecessors", Journal of Art in Society, href{http://www.artinsociety.com/the-emergence-of-the-winter-landscape.html\#}{Link}, accessed 26 October 2018}. Lorenzetti's painted in one small panel a rather dapper man standing in a heavy snowfall (Figure \ref{fig:winterart}, Panel A) and holding a well-made snowball. From the look in his eyes, and the quality of the snowball, this isn't his first snowball fight. But it is the oldest representation of snow in European painting. Nearly a century latter, the wealthy of Europe were commissioning ``Book of Days,'' essentially private prayer books. One of these is \textit{the Tres Riches Heures du Duc de Berry} (``The Very Rich Hours of the Duke of Berry''), commisioned by the Duke himself between 1412 and 1416 CE but not finished until 1486 CE\footnote{Alexander, J. (1990). Labeur and paresse: Ideological representations of Medieval peasant labor. The Art Bulletin, 72(3), 436-452.} well into the LIA. The book is richly and beautifully illustrated, with each month receiving an appropriate scene. The February scene (Figure \ref{fig:winterart}, Panel B) is interesting to us, as it shows how newly strong winters were experienced (through the eyes of artists working for a greedy ruler) by peasants. The realistic image features snow above all else, snow which covers the entire scene under leaden skies. Although the farm seems to be a wealthy one (note the pen full of sheep). The three peasants (two figures to the left; the women in pink skirts on the right) lack the clothes and shoes needed to keep them warm (note the condensed breath of the woman in pink!). In the right background, a better-dressed man languidly chops wood for a fire, while another peasant (upper right) trudges through the heavy snow with a mule's load of wood. All of the people depicted are singularly under-prepared for cold weather the artists clearly thought typical of the new European climate.\\

The best-known example of art inspired by the Little Ice age are winter landscapes, a genre of painting apparently invented (at least in Europe) by Pieter Brueghel following the powerfully cold winter of 1564-1565 CE. Brueghel's \textit{Hunters in the Snow} powerfully evokes the hardships and joys of wintertime in the LIA. Clearly Brueghel isn't painting a real scene (he lived in what is now Northern Belgium, which conspicuously lacks the mountains depicted in the painting). But the painting is rich in carefully rendered details of winter living in the LIA. In the left foreground, a trio of clearly tired, cold and dejected hunters stump home, carrying only (slung on the back of the central figure) a pitifully small fox or rabbit, hardly enough to feed the pack of hungry dogs behind them. To the left, workers at an inn (note the broken sign hanging above them) are singeing the hairs from hogs prior to preparing a meal\footnote{McCouat op cit.} Despite the obvious cold and the threat it presents to the villagers, not all of life at this time was toil. In the background of the painting, Brueghel carefully depicts villagers enjoying the cold weather, by skating, playing hockey, curling and even ice fishing. But overall, the turmoil created by the LIA was the greatest Europe had seen in a millennium, inducing centuries of war, famine and migrations.\\

\paragraph{Crop failure, war, and population contraction.}
The coldest part of the LIA, from 1560 to 1660 CE, might be called The \textbf{General Crisis of the $17^{th}$ Century}\footnote{Zhang, D. D., Lee, H. F., Wang, C., Li, B., Pei, Q., Zhang, J., \& An, Y. (2011). The causality analysis of climate change and large-scale human crisis. Proceedings of the National Academy of Sciences, 108(42), 17296-17301. \href{https://doi.org/10.1073/pnas.1104268108}{link}}, or \textit{GCSC}. In this period, economic distress, civil unrest, war and population declines all spread across Europe. (Other cultures were affected as well, including China\footnote{Zhang, op cit.}, and Japan \footnote{White, op cit.}, but we won't cover them for brevity.) Particularly noteworthy is the dramatic increase in the number, longevity and lethality of war during this period. The \textit{Thirty Year's War}, fought across central Europe by powers of northern Europe against those in central Europe, resulted in the death of 8 million people, at a time when all of Europe had only 100 million people\footnote{Zhang, op cit.}. (One of these casualties was none other than Johanes Kepler, the discoverer of Kepler's Laws, which are so crucial to understanding the Ice Ages. He was forced to move multiple times across central Europe to avoid the various armies waging the Thirty Years' War. He died in 1630 CE, but his grave was desecrated and destroyed by members of the Swedish army. Kepler's final resting place is unknown.). Migration, both across the European continent and for the first time across ocean basins also peaked at his time. Figure \ref{fig:migration_rate} shows the rate of European emigration and colonization of North and South America, an dth efirst peak in this corresponds exactly with the GCSC. The social and political changes caused by climate change in the LIA quite literally reshaped the world.\\

That's a pretty bold statement: \emph{caused} implies that climate change itself set in to motion social disturbance, war, famine and migration and eventually significant population collapse throughout Europe. And if climate change has done this in the past, it seems reasonable to ask if current climate change might produce similar turmoil. Fortunately, the GCSS has been well studied, and the assertion of causality is well documented\footnote{Zhang et al, op cit.; Zhang, D. D., Brecke, P., Lee, H. F., He, Y. Q., \& Zhang, J. (2007). Global climate change, war, and population decline in recent human history. Proceedings of the National Academy of Sciences, 104(49), 19214-19219.} Figure \ref{fig:zhang_cause} is one of our typical systems diagrams, but this one focuses on human culture. Climate change, in this case far colder weather in the northern hemisphere, starts the chain of causality. In response to the shorter growing season and reduction in arable land, plant growth goes down, which reduces agricultural production. The vast majority of the population of Europe subsisted on grains grown on farms, so as grain production fell, costs went up, and on average the amount of food per person plummeted. These economic troubles degraded human health (yellow-filled boxes) including increased famine, nutritional stress (adult heights dropped by an average of ~\SI{2}{\centi\metre} ($3/4$ inch) during the $16^{th}$ century) and epidemics. All this personal stress contributed to the growth of societal stress as well (gray-filled boxes), including increased frequency of war, migrations and other social disturbances. All of these factors decimated Europe's population, reducing the population by 15\%---20 million souls---effectively delaying population growth in Europe by 50 years---two generations. The only question is why? What caused the LIA?

\subsubsection{Causes (notice the plural) of the MCA and the LIA}
Slow, long-term changes in insolation clearly drive the initiation of Ice Ages. But such changes can't be responsible for the relatively brief, local and oscillatory changes in climate observed during the MCA and LIA. Instead, other ephemeral forces may be driving brief changes in Earth's energy balance. Or, perhaps the MCA and LIA were just the result of unusual (but perfectly normal) climatic variations possible in the complex climate system. The first hypothesis is testable using proxies, the second by understanding models of Earth's climate. In this section, we'll examine how proxies are constructed and used to constrain Earth's past climate. 
\paragraph{Insolation}
Could natural natural variations in the sun's output over decades to centuries contribute to the climatological variations of the MCA and LIA? Figure \ref{fig:ssn_all} shows two data sets illustrating the relatively small variations in the Sun's output. The total solar insolation (or TSI) is the energy output of the sun at all wavelengths, and is shown on Figure \ref{fig:ssn_all} (Panel A) for the period over which Earth-orbiting satellites have directly measured the output. The original data (in blue and orange) are ``noisy,'' they vary rapidly over many time scales. So the black line shows the ``smoothed'' values, by averaging the data over a 4 month time period. Both the noise and the averaging are typically done with data, to extract a real signal from the data. (The same process is used to determine you semester grade in classes. By averaging over a few exams and homeworks, you awarded a grade for your average accomplishments, not that second exam when you had the flu! The inset graph shows exactly the same data, but shown at true scale, with \SI{0}{\watt\metre\square} as the lower limit on the ordinate. Clearly the sun's output varies, but only by roughly \SI{\pm 0.5}{\watt\metre\square} over the 45 years of measurement, a whopping
\begin{equation}
	proportional\ variation=\frac{variation}{mean}=\frac{\SI{0.5}{\watt\metre\square}}{\SI{1361}{\watt\metre\square}}\cdot100\%=0.037\%
\end{equation}
This is equivalent to varying the length of a typical 45 minute-long class by 1 second: to tiny for anyone to notice. Regardless of the magnitude, the Sun's output clearly varies over a roughly 11-year cycle, with peaks in 1980, 1991, 2002 and 2014 CE. This well-known pattern is most-likely due\footnote{Charbonneau, P. (2010). Dynamo models of the solar cycle. Living Reviews in Solar Physics, 7(1), 3; the precise mechanism of the variations is one of the great unknowns in solar physics!} to the evolution of the Sun's internal magnetic field, driven by changes deep in the interior of the Sun and largely irrelevant to our story. Combined with their small size, these rapid variations in the Sun's output couldn't have contributed to the centuries-long MCA or LIA.\\
What about variations in the Sun's output prior to the satellite era, and over longer periods? Fortunately, changes in the Sun's magnetic field also change the number of \emph{sunspots} on the Sun's visible surface. Figure \ref{fig:sunspots} Panel A is a visible-light image of the sun, taken from the SOHO satellite on January 23, 2014. The large brown objects marring the surface are sunspots, regions of relatively lower temperature (and hence lower energy output) on the photosphere of the Sun, shown enlarged in Panels B and C. Panel D shows a spectacular view of the Sun's limb, with arcing plasma tracing the magnetic fields stretching between the sunspot groups. Given that total solar insolation \textit{and} the number of sunspots on the Sun's surface (SSN)are driven by the same process, it seems reasonable to hypothesis that insolation and sunspot number should be correlated. Take a minute to hypothesis about the relationship: Do you expect maxima in sunspot number to correlate with maxima or minima in insolatiuon? Why? Figure \ref{fig:ssn_all} shows that the number of sunspots not only varies over time, but does so in good synchronicity with the total solar insolation, but (perhaps to your surprise) insolation is \textit{highest} at times of high sunspot number. These solar maxima are marked by abundant sunspots, but also by even more abundant \textbf{faculae}, the bright areas surrounding sunspots. These brighter (and hence hotter) areas release enough energy to counter the dark sunspots. Figure \ref{fig:ssn_all}, Panel C shows that the number of sunspots and the Sun's total insolation are reasonably well correlated: Sunspots are a proxy for total solar insolation.\\

Sunspots have been observed and recorded since at least 165 BCE\footnote{Usoskin, I. G. (2017). A history of solar activity over millennia. Living Reviews in Solar Physics, 14(1). \href{https://doi.org/10.1007/s41116-017-0006-9}{Link} is a readable and thorough review.}, when Chinese and Korean astronomers recording sunspots observed with the naked eye. (Warning: \emph{do not} look at the Sun with the naked eye, except at sunset and sunrise, or you will suffer permanent blindness). For the past 400 years we have a good and nearly continuous record of telescopic observations and counts of sunspots. The data are rich in detail and pattern, as shown on Figure \ref{fig:ssn_all2}, Panel A. The horizontal lines in green show the likely total solar insolation, based on the relationship we derived in Figure \ref{fig:ssn_all}. The 11-year solar cycle is apparent, but so too are \textit{Grand Minima}, periods of little or no sunspots. These are nicely revealed by the smoothed count (shown in black) in the Figure.  The \emph{Maunder Minimum} is particularly interesting as it occurred in the middle of the LIA, suggesting that low insolation may have contributed to colder temperatures. \\

Alas, our sunspot record only goes back to the middle of the LIA; we have no sunspot information before 1600 CE. Fortunately we have another insolation proxy. The sun constantly emits a stream of charged particles, delightfully named the solar wind, into space. These particles induce aurorae in our atmosphere, and inflate a bubble of plasma around the solar system, extending to at least twice the orbital radius of Pluto (Figure \ref{fig:heliosphere}). The size of the heliosphere depends upon the strength of the solar wind, which in turn depends directly upon the Sun's total insolation, and so varies in the same way as the TSI. Think of the heliosphere as our solar system's part of space, a little ``island'' in the vast realm of inter-stellar space. Cosmic rays (mostly fast-moving protons and He nuclei) are constantly traversing this inter-stellar space, and the edge of the heliosphere---the heliopause---deflects most of these charged particles away from the solar system, reducing the amount of radiation hitting Earth. Earth's magnetic field also deflects cosmic rays, so the flux of radiation hitting Earth's atmosphere is modulated by changes in both the sun and Earth.
The power of cosmic rays is impressive: the most energetic particle yet measured, the  ``Oh my God'' particle, blasted into the atmosphere above Utah traveling 99.9999999995\% the speed of light, and had the same kinetic energy as a baseball traveling 100kph. (The particle has its own Wikipedia page: http://en.wikipedia.org/wiki/Oh-My-God\_particle). Once in the atmosphere, cosmic rays collide spectacularly with atmospheric molecules, creating ``showers'' of sub-atomic particles cascading through the sky. These particles have enough energy to cause nuclear reactions, and one of these reactions produces that most famous of all isotopes, carbon 14, or  \isotope[14][6]{C} (Equation \ref{eq:c14_prod}). Once produced, the newly-minted \isotope[14][6]{C} is oxidized into carbon dioxide (Equation \ref{eq:c14_ox})and enters the carbon cycle (Equation \ref{eq:c14_incorp}). Somewhat more than half is incorporated into the biosphere via photosynthesis. The remainder enters the oceans on the inorganic part of the C cycle.
 
\begin{align}
	Production: & n+\isotope[14]{N} & \rightarrow & \isotope[14]{C}+p^+ \label{eq:c14_prod} \\
	Oxidation:  & \isotope[14]{C}+O_2 & \rightarrow & \isotope[14]{C}O_2 \label{eq:c14_ox} \\
	Incorporation: & \isotope[14]{C}O_2+H_2O & \rightarrow & \isotope[14]{C}H_2O+O_2 \label {eq:c14_incorp} \\
	Decay: & \isotope[14]{C}&\rightarrow& \isotope[14]{N} + e^- +\bar{\nu}_{e^-}
\end{align}

Precise dating of \isotope[14][6]{C}-containing samples (counting annual growth rings in trees, or annual layers in lake sediments or ice cores, for example) permits construction of a record of \isotope[14][6]{C} production, $\Delta\isotope[14]{C}$ through time, and is shown in Figure \ref{fig:ssn_all2} Panel B. Before you examine the graph, work through the physics and hypothesize: Will \isotope[14][6]{C} rise with total solar irradiation, or fall? As the Figure shows, $\Delta\isotope[14]{C}$ is inversely correlated with insolation: as insolation increases, the heliopause also strengthens, so cosmic ray bombardment decreases and with it \isotope[14]{C} production. So we now have a proxy for insolation that is useful for approximately 50,000 years into the past. And at first glance the results are consistent with insolation changes contributing to both the LIA and thee MCA: insolation is lower in the LIA and (somewhat) higher than average during the MCA. But the changes are tiny-about \SI{1}{\watt\per\metre\square}. Perhaps another force was at work? 

\paragraph{Volcanic Eruptions}
Large volcanic eruptions inject massive volumes of dust and gas into the atmosphere (Figure \ref{fig:pinatubo}. While the dust can change the local weather for a few days, and cause wide-spread damage for hundreds of kilometers, the dust is relatively heavy and falls out of the atmosphere within minutes to weeks after the eruption\footnote{Robock, A. (2000). Volcanic eruptions and climate. Reviews of Geophysics, 38(2), 191-219. \href{https://doi.org/10.1029/1998RG000054}{link}}. But gasses emitted by the volcano can stay in the atmosphere far longer. The more commom gasses emitted by volcanoes are our old friends $H_2O$ and $CO_2$. But the amount emitted is vanishingly small compared to the amounts already present in the atmosphere. The real climate culprit in eruption is sulfur dioxide, or $SO_2$. Particularly large eruptions inject $SO_2$ into the stratosphere, where it reacts with water to form small drops of sulphuric acid, $H_2SO_4$. These drops are about the same size as the wavelength of visible light, and so they are spectacularly efficient at reflecting sunlight back into space. Withing two to three weeks, these clouds can circle the globe, and spread across a hemisphere. Even in the stratosphere, atmospheric circulation across the equator is weak, so eruption tends to increase albedo, decrease in-coming energy, and cool only one hemisphere, and can do so for 1 to 2 \textit{years} following an eruption. Eruptions are powerful, if brief, drivers of climate. The 1815 CE eruption of Tambora, on Sulawesi island in Indonesia led to the infamous \textbf{year without a summer} in 1816 CE, a last swift cold spell in the LIA. Hard frosts every month of summer plagued New England \footnote{1816: The Year Without a Summer, New England Historical Society, \href{http://www.newenglandhistoricalsociety.com/1816-year-without-a-summer/}{Link}}, snow fell throughout Europe and North America, and unusually frequent rainfall sequestered people indoors, including who wrote a novel about a certain Dr. Frakenstein and his monster\footnote{Robock, op cit.}. Crop failures led to sever price increases, with social unrest(Figure \ref{fig:Zhang}) famine, illness and large scale unemployment following in its wake\footnote{Post, J. D. (1977). The last great subsistence crisis in the Western World. Johns Hopkins Univ Pr.}. Fortunately, the effects of even large eruptions, like Tambora, are brief. Within a year or two the droplets suspended in the stratosphere either chemically degrade or fall to the surface. Ice cores in particular can preserve this sulfur signal, providing a convenient proxy for volcanic reduction of in coming solar energy.\\

\paragraph{Solar and volcanic affects on the MCA and LIA} 
Figure \ref{fig:forcings} shows the result of decades of research quantifying the driving fores of climate during the past millennium. The ordinate shows annual \emph{differences} from  average forcings for insolation (green), volcanic eruption (red) and greenhouse gasses (blue). Positive differences lead to net warming, while negative differences lead to net cooling. Clearly our hypothesis that the MCA and LIA were caused by higher and lower insolation is \emph{not} supported. The insolation differences are negative for both periods, and the magnitudes are too small to cause the roughly \SIrange{-0.5}{-1}{\kelvin} temperature changes during the LIA. Don't be disappointed (or upset) that all the work we did on insolation proxies was ``wasted.'' Negative findings, where an hypothesis is disproved, is actually the most common result of good science, it just isn't as flashy as positive results\footnote{PLOS Collection, Positively Negative: A New PLOS ONE Collection focusing on Negative, Null and Inconclusive Results, February 25, 2015,  \href{https://blogs.plos.org/everyone/2015/02/25/positively-negative-new-plos-one-collection-focusing-negative-null-inconclusive-results/}{Link}, accessed 13 November 2018}. And remember too that discovering insolation is \emph{not} a factor in recent climate change just means another factor is present. That factor is not $CO_2$ either. Although the concentration of $CO_2$ did fall in the coldest portions of the LIA, the change of roughly -8 ppm\footnote{J.M. Barnola, D. Raynaud, C. Lorius, N. I. Barkov, Historical CO2 Record from the Vostok Ice Core, \href{http://cdiac.ornl.gov/ftp/trends/co2/vostok.icecore.co2}{Link} accessed 13 November 2018} is not enough to cause substantial temperature changes\footnote{MacFarling Meure, C., Etheridge, D., Trudinger, C., Steele, P., Langenfelds, R., Van Ommen, T., ... \& Elkins, J. (2006). Law Dome CO2, CH4 and N2O ice core records extended to 2000 years BP. Geophysical Research Letters, 33(14).}, which you can confirm using the ERB, our simple climate model (see the end of chapter problems). Volcanic forcing, in contrast, is a potential factor. Note that the MCA ends with a long series of eruptions, and concludes in 1257 CE with the massive eruption of Samalas in Indonesia\footnote{Lavigne, F., Degeai, J.-P., Komorowski, J.-C., Guillet, S., Robert, V., Lahitte, P., ... Belizal, E. de. (2013). Source of the great A.D. 1257 mystery eruption unveiled, Samalas volcano, Rinjani Volcanic Complex, Indonesia. Proceedings of the National Academy of Sciences, 201307520. \href{https://doi.org/10.1073/pnas.1307520110}{Link}}, which brought a cold summer, heavy rains, floods and crop failures across the northern hemisphere. THe LIA begins with the substantial eruption of Kuwae volcano (also in Indonesia\footnote{Erik Klemetti 24 May 2012, Kuwae Eruption of the 1450s: Missing or Mythical Caldera? \href{https://www.wired.com/2012/05/kuwae-eruption-of-the-1450s-missing-or-mythical-caldera/}{Link}, accessed 13 November 2018}. Large eruptions and the attendant decrease in solar energy reaching the surface are frequent during the LIA, particularly in the coldest (and most strife-ridden) portions coincident with the General Crises of the Seventeenth Century. 

   
\subsection{The Industrial revolution and the acceleration of anthropogenic climate change}

\clearpage 
\section{Figures}

\begin{figure}
\begin{center}
	\includegraphics[width=6 in]{norse_map}%	
	\caption{Medieval Climate Anomaly in action: the Norse expansion into the North Atlantic.}
	\label{fig:norse_map}
\end{center}
\end{figure}



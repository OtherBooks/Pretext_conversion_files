\documentclass[amstex,12pt]{book}
\usepackage{amsmath}
\usepackage{amsfonts}
\usepackage{verbatim}
\usepackage{wasysym}
\usepackage[margin=2cm]{geometry}
\usepackage{hyperref}
\usepackage{epigraph}
\usepackage{graphicx}
\usepackage{siunitx}
\usepackage[caption=false]{subfig}
\usepackage{booktabs}
\newcommand{\textsub}[1]{$_{\text{#1}}$}
\newcommand{\textsup}[1]{$^{\text{#1}}$}
\setcounter{chapter}{1}
\setcounter{section}{-1}
\pagestyle{plain}
\graphicspath{ {figures/} }
\sisetup{separate-uncertainty}%

\begin{document}
\title{Climate Change and the Science of Ethics} \label{The Ethics of Climate Change}
\date{\today}
\author{Larry McKenna\\Department of Physics and Earth Sciences\\Framingham State University}
%\title{}
\maketitle
\section*{Epigraph}\label{Epigraph}
\epigraph{From him who has been given much, much will be demanded---from someone to whom people entrust much, they ask still more.}{Luke 12:48 from the CJB, Stern} 

\section{Core Concepts} \label{Core Concepts}
\begin{itemize}
	\item	As Stephen Gardiner notes, ``we cannot get very far in discussing why climate change is a problem without invoking ethical considerations.''
	\item	Climate change is a long-term global-scale problem caused by short-term, local-scale actions. 
	\item	Some problems, like climate change, have no technical solutions, but only behavioral ones
	\item	Some problems, like climate change, require collective actions to solve. 
\end{itemize}

\section{Opening Problem} \label{Who owns the air?}
Who owns the air? The brief answers most often given are ``no one'' or ``all of us,'' which are oddly similar answers. In the United States, courts in different states have made different decisions regarding  governments' responsibility to caring for its citizens' atmosphere. In May of 2015, Judge Karsten H. Rasmussen of the Oregon Circuit Court questioned
``whether the atmosphere is a `natural resource' at all, much less one to which the'' state had any responsibility \footnote{Chernaik v. Brown, Circuit Court of The State of Oregon For Lane County, case Case No. 16-11-09273, 2015}. The state Appeals Court of New Mexico \footnote{Sanders-Reed v. Martinez, Court of Appeals of the State of New Mexico, 2015} found that the state government \emph{is} responsible for the atmosphere's care, but decided that those responsibilities were limited \footnote{Ibid, pg 10, lines 14-21}. Clearly something about the atmosphere---its size, its transience, its ubiquity---distinguishes it from something as small, permanent and personal as a cell phone. Answering the question``Who owns your iPhone?'' is pretty simple. So too is answering the question ``Who is harmed if I break your iPhone?'' But can you, personally, be aggrieved if some one ``harms'' your atmosphere? Is it unethical to harm the atmosphere?
 
\section{No one owns the air, but everybody can use it} \label{Air is a common property}
You've probably never lived through the type of killer smog that settled over Donora, Pennsylvania in October, 1948. For a few days before Halloween, an atmospheric temperature inversion created a ``lid'' over the valley, trapping the exhaust from the mills lining the town's river. Over 4 days the concentrated pollutants from those mills killed 20 mostly elderly people and sickened nearly half the residents\footnote{Hamill, S. D. (2008). Unveiling a museum, a Pennsylvania town remembers the smog that killed 20. New York Times, page A22, November 1, 2008}. A similar event in London in December of 1952 killed over 4000 people in just one week\footnote{The Great Smog of 1952, http://www.metoffice.gov.uk/learning/learn-about-the-weather/weather-phenomena/case-studies/great-smog, accessed 05 September 2016}. These disasters are essentially absent now in the United States and other developed nations because of laws, such as the US \emph{Clean Air Act}\footnote{Title 42 U.S.C., ch. 85, subch. I \S 7401 \textit{et seq.}}, that limit the type and quantity of pollution emitted into the atmosphere. Since its original passage in 1963, and a visionary strengthening signed into law by President Richard Nixon in 1970, the CAA has saved thousands of lives\footnote{Chay and Greenstone, M. (2003). Air quality, infant mortality, and the Clean Air Act of 1970. Working Paper 10053, National Bureau of Economic Research} and billions of dollars in health care costs and even increased American's productivity to the tune of over \$100 billion \footnote{Isen, A., Rossin-Slater, M., \& Walker, W. R. (2014). Every Breath You Take-Every Dollar You'll Make: The Long-Term Consequences of the Clean Air Act of 1970 (No. w19858). National Bureau of Economic Research, page 30}. The US Environmental Protection Agency (EPA) has even devised an Air Quality Index (AQI), which indicates the relative unhealthiness of polluted air, as determined from the concentrations of five pollutants regulated (in the U.S.) by the \emph{Clean Air Act}. An AQI of less than 50 is considered ``good'' quality air, while a reading over 300 indicates that the air is harmful to everyone, even healthy adults. The US Environmental Protection Agency AQI scale ends at 500.

\subsection{Air pollution in Eastern China}
The residents of developing nations aren't so lucky as those in wealthier countires. Rohde and Muller report that ``Air pollution is a problem for much of the developing world and is believed to kill more
people worldwide than AIDS, malaria, breast cancer, or tuberculosis'' \footnote{Rohde, R. A., \& Muller, R. A. (2015). Air pollution in China: Mapping of concentrations and sources. PloS one, 10(8), e0135749}. Residents of eastern China suffer from air pollution so thick and pervasive it is visible from Earth orbit (Figure \ref{fig:china_pollution_ab}) and can easily be mistaken for fog (Figure \ref{fig:china_pollution_c}). Just as in Donora and London, the air in Beijing is filled with the byproducts of combustion and industrial production.

\subsubsection{Local issues}
At the time Figure \ref{fig:china_pollution_ab} was taken, the AQI in this part of China was 341. Two days before, on January 12, 2013, AQI in Beijing was 775.  Air pollution in Beijing is literary off-scale awful. Recall the London ``killer fog'' which contributed to the death of 4000 Londeners in just one week? Over 4000 Chinese citizens die \emph{every day} from the pollution in Chinese skies\footnote{Rhode and Muller, \textit{op cit.}}. This is equivalent to roughly 1.6 million Chinese deaths per year, at least during the period of Rhode and Muller's study in 2012. This mortality is largely due to microscopic particulate matter, less than \SI{10}{\micro\meter} (approximately $\frac{4}{10000}$ inches) across (see Figure \ref{fig:EPA_pm}). Such particles, common in the exhaust plumes emitted by accelerating diesel-powered buses or trucks, lodge in the lungs and increase the risk of stroke, heart disease, lung cancer and respiratory diseases\footnote{Rhode and Muller, \textit{op cit.}}. Over 80\% of all Chinese people are exposed to particulate matter concentrations that exceed the levels healthy for ``sensitive groups'' in the population, such as the elderly and the young. Almost 40\% of Chinese people are subjected to air in which the concentration of particulate material is dangerous to all people.   

\subsubsection{Local causes}
The source of all this pollution is combustion of fuels, largely coal and wood, in China. The energy released by this burning is used to generate power for industrial production, to generate electrical power, to heat homes, and to provide transportation. Coal burning alone is responsible for about 40\% of all air pollution-related deaths in China,\footnote{Wong, Edward. Burned Coal Is Deadliest Part of China's Polluted Air, Study Says. The New York Times, Thursday August 18, 2016 Thursday, page 8, 805 words. LexisNexis Academic. Web. Date Accessed: 2016/09/13}, and accounts for approximately one quarter of all the pollution-related premature deaths in China in recent years. While the Chinese government has announced plans to strike ``heavy blows'' in the efforts to reduce air pollution\footnote{Liu, Qui. 13th Five-Year Plan is the first to include PM2.5 targets,ChinaDialogue, \href{https://www.chinadialogue.net/article/show/single/en/8696-13th-Five-Year-Plan-is-the-first-to-include-PM2-5-targets}{link}, accessed 2016/11/01.}, China still burns almost as much coal as th erest of the world combined. Chinese industries are also far less efficient than their western counterparts, emitting 6-30 \emph{times} more pollution when accounting for the total value of goods made in each country\footnote{Lin, J., Pan, D., Davis, S. J., Zhang, Q., He, K., Wang, C., ... \& Guan, D. (2014). China’s international trade and air pollution in the United States. Proceedings of the National Academy of Sciences, 111(5), 1736-1741. Table S1.}. This reflects a fundamental difference between the Chinese economy and that of the United States. In the United States, manufacturing creates only 12\% of the country's annual wealth, while in China this value is 3 times higher, approximately 38\%\footnote{Lin et al. quote 43\% in 2006. The World Bank \href{http://databank.worldbank.org/data/reports.aspx?source=2&series=NV.IND.MANF.ZS&country=}{Data Bank} has substantially lower values of 33\% in 2007. I quote the average.} This dominance is hardly surprising. China has become the largest manufacturer in the world, surpassing the United States in 2009. What might be surprising is that only 1/3 of all this manufacturing is actually consumed by the Chinese people: the rest is exported across the world\footnote{Berger, B., \& Martin, R. F. (2011). The growth of Chinese exports: An examination of the detailed trade data. Board of Governors of the Federal Reserve System, International Finance discussion papers, 1033.}, mostly to the well-developed countries of Europe and North America, including the United States.

\subsection{Transport to North America}
Manufactured goods aren't the only thing China exports to the United States. Winds carry significant quantities of the pollution in Chinese skies to Western North America.

\subsubsection{Local issues}
This pollution has measurably worsened air quality in the western United States, and even forced cities in California to pay fines for violating the \emph{Clean Air Act}\footnote{Hand, E. (2014). China blamed for US ozone. Science, 345(6202), 1233-1233.}! The precise pollutant responsible for this was \emph{not} PM, but ozone, O\textsub{3}, generated by industrial activity in China. The magnitude of this transport is remarkable. In the 20 years from 1990 to 2010, O\textsub{3} concentrations dropped extensively across rural areas of the Eastern United States, reflecting the successful limitation of local O\textsub{3} sources. In the western United States, 5 \emph{times} fewer rural stations recorded decreases in O\textsub{3}.

\subsubsection{Distant Causes}
Approximately 40\% of the difference is due to O\textsub{3} or its precursors produced in China and borne to the United States by westerly winds blowing across the Pacific. These winds and their trajectories are nicely shown in Figure \ref{fig:O3_pac_winds}, which traces particulate matter (most of it natural) being carried from eastern Asia to western North America. Despite the indirect trajectory, the illustration shows that air from China reaches the coast of North America in just days; ozone and other pollutants produced in China can last weeks in the lower atmosphere\footnote{Monks, P. S., Archibald, A. T., Colette, A., Cooper, O., Coyle, M., Derwent, R., ... \& Stevenson, D. S. (2015). Tropospheric ozone and its precursors from the urban to the global scale from air quality to short-lived climate forcer. Atmospheric Chemistry and Physics, 15(15), 8889-8973.}. Once they wash over North America, those pollutants inflict some costs on the local residents, both financial and personal. Chinese pollution is a hemispheric problem, an hypothesis well described as early as 2003 \footnote{Akimoto, H.: Global air quality and pollution, Science, 302, 1716–
1719, 2003.}.  

\subsubsection{``Trade'' pollution and what local and distant now mean}
The residents of Donora, Pennsylvania suffered a terrible tragedy in 1948. Both the immediate cause and suffering of that disaster were ``local'' by any standard: emissions in one small town caused death and injury in that  one small town. By 1952, the scale of the problem had grown: the entire city of London was bathed in smog, but still the cause of the pollution was well within the affected area.  If you can order a phone charger on Monday, and have it shipped to your house in a few days from China, how is China ``distant?'' If the pollution generated by the manufacture of that charger reaches your house a few days later, how is your ``local'' environment not affected? Even as modern transportation and communication infrastructure allows inter-continental ordering, manufacturing and shipping to flourish, we find that ``local'' and ``distant'' have become fuzzy concepts. And so too have the ideas of harm and responsibility. Who's to blame for pollution in your backyard caused by a Chinese manufacturing plant making a phone charger for you? Can it be that anyone and everyone owns the atmosphere, but no one is responsible for it?

\section{Global problem, individual causes} \label{Global problem, individual causes}
The ``ozone dilemma'' of the previous section is one example of a global problem with local causes. Pollution is something with which we have personal experience: we can see, feel even feel threatened by pollutants in our environment. But other significant global problems are more difficult for us to perceive. Some changes in Earth's environment are so slow we simply can't find enough change in our lifetimes. Other changes act over such enormous scales  of Earth's surface that we can't see them all at once. The ecologist Aldo Leopold noted that in dealing with Earth's environment, humans needed to ``think like a mountain.'' To understand global problems with local causes, we must understand that our wonderful ability to think and act have been honed over millions of years of evolution in small groups with limited geographic range. Those thoughts and actions are no longer enough in our globally-connected planet. 

\subsection{Global climate change is different}\label{GCC_diff}
Global climate change unfortunately has all the characteristics that make personal perception of the problem difficult. Climate change additionally suffers from the ``ozone dilemma,'' because linking causes and effects is difficult, even impossible. This is a point made famous by Steven Gardiner, in a series of papers and books all called a variation on \emph{Perfect Moral Storm: the Ethical Tragedy of Climate Change}. As we'll see in the next section, recent changes to Earth's climate are largely due to anthropic (human-caused) release of greenhouse gasses. People have been releasing these gasses as part of their daily practice of living for nearly 6000 years \footnote{Ruddiman, W. F., Fuller, D. Q., Kutzbach, J. E., Tzedakis, P. C., Kaplan, J. O., Ellis, E. C., ... \& Lemmen, C. (2016), Late
Holocene climate: Natural or anthropogenic?, Rev. Geophys., 54, doi:10.1002/2015RG000503.}


\subsection{What is and isn't at risk from Global Climate Change}\label{Risk}
\subsubsection{The Holocene climate is ending}
\paragraph{Climatic stability and human evolution}
Our species evolved in a climatically tumultuous period of Earth's history about 2 Ma. So much so that many evolutionary biologists have argued that \textit{Homo sapiens'} success stems form our ability to adapt to changing climates\footnote{Stewart, J. R., \& Stringer, C. B. (2012). Human evolution out of Africa: the role of refugia and climate change. Science, 335(6074), 1317-1321, deMenocal, P. B. (2011). Climate and human evolution. Science, 331, 540., deMenocal, P. B. (2004). African climate change and faunal evolution during the Pliocene–Pleistocene. Earth and Planetary Science Letters, 220(1-2), 3-24.}. For the past 11,000 years however\footnote{Barnosky, A. D., Hadly, E. A., Bascompte, J., Berlow, E. L., Brown, J. H., Fortelius, M., ... \& Martinez, N. D. (2012). Approaching a state shift in Earth/'s biosphere. Nature, 486(7401), 52-58.} Earth's climate has only slowly oscillated over a narrow range of conditions, in an epoch of Earth's history called \emph{the Holocene}. Our species' cultural evolution is intimately entwined with and largely conditioned by this consistent Holocene climate. The invention of agriculture 12,000 years ago--- ``one of the most significant cultural and evolutionary transitions in the 200,000-y history of our species'' \footnote{Larson, G., Piperno, D. R., Allaby, R. G., Purugganan, M. D., Andersson, L., Arroyo-Kalin, M., ... \& Doust, A. N. (2014) Current perspectives and the future of domestication studies. Proceedings of the National Academy of Sciences, 111(17), 6139-6146.}--- was possible largely because of the stability of the Holocene climate.\\

But the Holocene and its comfortable climate are at an end, largely because of our release of vast quantities of greenhouse gasses into Earth's atmosphere. Earth is even now moving into a new epoch, \emph{the Anthropocene}, when climate is changing rapidly and in one direction. We will spend the remainder of the book understanding the causes, effects, and implications of these changes. But to understand the ethical issues produced by these changes-by the individual actions of the billions of people on the planet in the past 100 years, those here now, and by the billions yet to come-we need some idea of the changes Earth's climate is experiencing.

\paragraph{Temperature changes}
Earth's surface temperature is increasing, and on average has done so since at least 1880 CE. The Intergovernmental Panel on Climate Change (IPCC) noted in 2013 that ``almost the entire globe'' has warmed during this period, by an average of \SI{0.78(7)}{\degreeCelsius} (Ref\footnote{IPCC, 2013: Summary for Policymakers. In: Climate Change 2013: The Physical Science Basis. Contribution of
Working Group I to the Fifth Assessment Report of the Intergovernmental Panel on Climate Change [Stocker,
T.F., D. Qin, G.-K. Plattner, M. Tignor, S. K. Allen, J. Boschung, A. Nauels, Y. Xia, V. Bex and P.M. Midgley (eds.)]. Cambridge University Press, Cambridge, United Kingdom and New York, NY, USA}). During the remainder of this century, temperatures will continue to rise in proportion to the mass of greenhouse gasses each of the billions of humans on the planet chooses to emit. Predicting this future is difficult, and in the past the IPCC has consistently \emph{underestimated} emissions, but by the best estimates, Earth's surface will be ~\SI{3(1)}{\degreeCelsius} degrees warmer in 2100 CE than it was in 1850-1900 CE. \\

Global maps (Figure \ref{fig:rcp_T}) of these predicted temperature changes are sobering. For a business as usual trajectory of greenhouse gas emissions (the ``RCP 8.5'' model), global temperatures \emph{increase} by an average of \SI{5}{\degreeCelsius} (about $9 \circ F$), above the average temperatures during 1986-2005. Much higher increases in the poles and over land are counterbalanced by smaller increases over the oceans. For a future where greenhouse gasses emissions are limited through concerted international efforts (the ``RCP 4.5'' model), global average temperatures increase only by \SI{2.5}{\degreeCelsius} (about $4.5 \circ F$), again with higher increases in the poles and on land.\\ 

This increase is \emph{virtually certain} to produce fewer cold days and nights, is \emph{virtually certain} to produce more frequent hot days and nights, and is \emph{very likely} to produce more warm spells by the end of the century. The scale of this heating is difficult to forecast without detailed modeling, but one study published in 2015\footnote{Pal, J. S., \& Eltahir, E. A. (2015). Future temperature in southwest Asia projected to exceed a threshold for human adaptability. Nature Climate Change 6, 197–200 doi:10.1038/nclimate2833.} found that end-of-the century temperatures in coastal areas of the Persian Gulf may might be lethally hot and humid to people outdoors. Parts of the planet's surface will literally become fatal to humans. At the other extreme, polar regions will experience widespread melting of ice and the permanently frozen ground (``permafrost'') on which is built much of the infrastructure of the polar regions (Figure \ref{fig:impermafrost}). All of this infrastructure will need to be rebuilt. 
 
\paragraph{Precipitation changes}
Those temperature changes will in turn modify precipitation patterns  (Figure \ref{fig:rcp_P}). Most importantly, by the close of this century it is \emph{very likely} much of the globe will see an increase in the frequency and intensity of heavy precipitation. The length and severity of drought will \emph{likely} increase over large regions of Earth, even while it is \emph{more likely than not} that hurricanes threatening North America's eastern seaboard will become more intense. There is some evidence that these changes have already begun, contributing for example to the 2012-2017 Californian drought\footnote{Diffenbaugh, N. S., Swain, D. L., \& Touma, D. (2015). Anthropogenic warming has increased drought risk in California. Proceedings of the National Academy of Sciences, 112(13), 3931-3936.}. The amount and quality of drinking water will also suffer. A combination of climate change and migration to cities could lead\footnote{McDonald, R. I., Green, P., Balk, D., Fekete, B. M., Revenga, C., Todd, M., \& Montgomery, M. (2011). Urban growth, climate change, and freshwater availability. Proceedings of the National Academy of Sciences, 108(15), 6312-6317.} to as many as 1.1 \emph{billion} more urbanites suffering from perennial water shortage  by 2050 CE.   
 
\paragraph{Sea level increase and acidification}\label{sl_acid}
The oceans will not escape change either. Sea levels will increase across the globe (Figure \ref{fig:slr}), with average increases of \SIrange{0.4}{1.8}{\metre} (Ref.\footnote{Sweet, William V., et al., 2017, Global And Regional Sea Level Rise Scenarios For The United States, National Oceanic and Atmospheric Administration, NOAA Technical Report NOS CO-OPS 083, Silver Spring, Maryland; and Oppenheimer, M., \& Alley, R. B. (2016). How high will the seas rise?. Science, 354(6318), 1375-1377, and references therein.}). While this rise may seem small (only 1-6 feet), its effect on our society will be enormous. Approximately 0.5\% to 2.4\% of the world's population lives at or below these elevations: they will all have to move away, or protect their dwellings, from the encroaching seas. By 2100 CE, this will be the equivalent of moving the \emph{entire population} of Spain or Mexico to higher ground! In poor areas, such as Bangladesh\footnote{Gardiner, Harris. 2014. As Seas Rise, Millions Cling to Borrowed Time and Dying Land, New York Times, March 29, 2014, page A1, \href{https://www.nytimes.com/2014/03/29/world/asia/facing-rising-seas-bangladesh-confronts-the-consequences-of-climate-change.html}{New York Times}}and Louisiana\footnote{Davenport, Carol and Robertsonm Campbell, 2016. Resettling the First American ‘Climate Refugees’, New York Times, May 3, 2016, page A1\href{https://www.nytimes.com/2016/05/03/us/resettling-the-first-american-climate-refugees.html}{New York Times}}, this migration has already begun, to the notice of few. In areas where coastal land is valuable, such as New York City, migration inland will be  more difficult, more expensive, and less likely. The experience of residents in the New York City after ``super-storm'' Sandy is instructive. Despite the evidence of their recently destroyed houses, people and businesses are rebuilding their city ``in place,'' by building walls, levees, and houses on 10 foot tall stilts\footnote{Feuerocy, Alan, 2014. Building for the Next Big Storm After Hurricane Sandy, New York Rebuilds for the Future, New York Times, October 25, 2014, page MB1 \href{https://www.nytimes.com/2014/10/26/nyregion/after-hurricane-sandy-new-york-rebuilds-for-the-future.html}{New York Times}}. \\

The oceans will also become more acidic and warmer in the future. By the end of the century, ocean water will be from 40 to 100\% more acidic than it is now. Simultaneously the oceans, particularity the deep oceans, will be warming, by as much as \SI {3}{\kelvin}. Both of these changes will make life more difficult for marine organisms, including corals and fin fish. Corals are a foundation of the marine ecosystem. But, as ocean acidity increases, corals have increasingly more difficulty making and maintaining their limestone shells. Already there is evidence of decreased coral growth around the world\footnote{Foster, T., Falter, J. L., McCulloch, M. T., \& Clode, P. L. (2016). Ocean acidification causes structural deformities in juvenile coral skeletons. Science advances, 2(2), e1501130.} and although coral extinction isn't imminent\footnote{Pandolfi, J. M., Connolly, S. R., Marshall, D. J., \& Cohen, A. L. (2011). Projecting coral reef futures under global warming and ocean acidification. Science, 333(6041), 418-422.}  noticeable decrease in coral growth and stability will very likely be seen by 2035 CE\footnote{Mora, C., Frazier, A. G., Longman, R. J., Dacks, R. S., Walton, M. M., Tong, E. J., ... \& Ambrosino, C. M. (2013). The projected timing of climate departure from recent variability. Nature, 502(7470), 183-187.}. Increases in ocean temperatures have already led to migration of valuable Cod in the Gulf of Maine\footnote{Pershing, A. J., Alexander, M. A., Hernandez, C. M., Kerr, L. A., Le Bris, A., Mills, K. E., ... \& Sherwood, G. D. (2015). Slow adaptation in the face of rapid warming leads to collapse of the Gulf of Maine cod fishery. Science, 350(6262), 809-812.}, and this trend of local migration and extinction is very likely to continue in the future. 

\subsubsection{Mass extinctions will increase}
Ah, dinosaurs! No other animals seem to inspire such delightful fear as these long-extinct cousins of birds. Many readers will know that the dinosaurs were driven to extinction about 66 Ma through climate change brought about by simultaneous asteroid impact and enormous, long-lived volcanic eruptions \footnote{Petersen, S. V., Dutton, A., \& Lohmann, K. C. (2016). End-Cretaceous extinction in Antarctica linked to both Deccan volcanism and meteorite impact via climate change. Nature Communications, 7.; Renne, P. R., Sprain, C. J., Richards, M. A., Self, S., Vanderkluysen, L., \& Pande, K. (2015). State shift in Deccan volcanism at the Cretaceous-Paleogene boundary, possibly induced by impact. Science, 350(6256), 76-78.} These twin catastrophes caused a \emph{mass extinction}, including not just the dinosaurs, but also the spectacular mosasaurs (think swimming dinosaurs), the pterosaurs (think flying dinosaurs) and as many as 40\% of the shell-producing animals in the oceans. This mass extinction is not unique: at least four others have happened in the past 450 My. These ``Big Five'' extinctions (Figure \ref{fig:mass_ext}) shaped life on Earth, paring and occasionally lopping off entire branches from the tree of life. Each of the``Big Five'' eliminated more than 75\% of the larger animal species extant at the time. While the primary cause(s) of these extinction are still being argued, all eventually were due to changes to Earth's climate, and all were due to natural events.\\ 

You live during the sixth mass extinction\footnote{Barnosky, A. D., Matzke, N., Tomiya, S., Wogan, G. O., Swartz, B., Quental, T. B., ... \& Mersey, B. (2011). Has the Earth's sixth mass extinction already arrived?. Nature, 471(7336), 51-57.}. This one is unique, as it is driven by diverse human activities including logging, mining, agriculture, conversion of forests to farmland, hunting, emission of waste products, and even climate change. In short, this mass extinction is caused by our use of the world to fuel human culture. The scale of this extinction is similar to those of the past, with the added ``luxury'' that we can measure the extinctions precisely and in real time. Over the past 500 years 1-1.5\% of vertebrate animals\footnote{Ceballos, G., Ehrlich, P. R., Barnosky, A. D., García, A., Pringle, R. M., \& Palmer, T. M. (2015). Accelerated modern human–induced species losses: Entering the sixth mass extinction. Science advances, 1(5), e1400253.} (fish, amphibians, reptiles, birds and mammals) have gone extinct, a rate \emph{800\% to 1200\% higher} than the expected extinction rates. As global climate change accelerates, so too will extinction rates. The actual rate depends on the total emissions of greenhouse gasses, so again we need to estimate a range of conditions in the future. A reasonable estimate\footnote{Urban, M. C. (2015). Accelerating extinction risk from climate change. Science, 348(6234), 571-573.} is that in the best case (the RCP 4.5 model) $5 \pm 1\%$ of vertebrate animals will be extinct by 2100 CE, increasing to $15 \pm 4\%$ in the ``business-as-usual'' model (RCP 8.5). Unfortunately, many of our closest evolutionary cousins-the primates-will be amongst the losers of this extinction lottery. Of the 500 or so non-human primates, $\approx 60\%$ are currently ``threatened with extinction,'' and $\approx 75\%$ have declining populations\footnote{A. Estrada, P. A. Garber, A. B. Rylands, C. Roos, E. Fernandez-Duque, A. Di Fiore,
K. Anne-Isola Nekaris, V. Nijman, E. W. Heymann, J. E. Lambert, F. Rovero, C. Barelli,
J. M. Setchell, T. R. Gillespie, R. A. Mittermeier, L. V. Arregoitia, M. de Guinea, S. Gouveia,
R. Dobrovolski, S. Shanee, N. Shanee, S. A. Boyle, A. Fuentes, K. C. MacKinnon, K. R. Amato,
A. L. S. Meyer, S. Wich, R. W. Sussman, R. Pan, I. Kone, B. Li, Impending extinction crisis of
the world’s primates: Why primates matter. Sci. Adv. 3, e1600946 (2017).}. One primate not likely to be in either category is \textit{Homo sapiens}. Us.\\
  
\subsubsection{Humans will survive}
Despite all of the changes Earth and its environment will face in the remainder of this century, our species will undoubtedly survive these changes. In fact, those scenarios we've been using in this section include population estimates of 9 to 12 billion humans at century's end, equivalent to a 20\% to 70\% increase from 2017 CE population estimates. Humans will definitely survive, but our current cultural will not. The combination of changes to climate, the oceans, the cryosphere, population increases, food production and mass extinctions will mean that human culture of 2100 CE will have to adapt to succeed.\\

The first societies to face necessary adaptation are all around us. But they tend to be societies that are distant, poor and unrelated to most citizens of wealthy countries like the United States, Canada and those of the European Union. Thirty five of these more wealthy countries belong to the Organisation for Economic Co-operation and Development. Together these countries are responsible for over 60\% of the global economy\footnote{World Bank's``GDP ranking (GDP)'' downloaded from http://datacatalog.worldbank.org/, accessed 08 February 2017}, but they make up only 18\% of the global population. Since 1960 CE, OECD countries have released about half of all greenhouse gasses emitted by humanity, with non-OECD countries making up the rest\footnote{Data for $CO_2$ FROM http://www.globalcarbonatlas.org/en/CO2-emissions, accessed 02 February 2017}. On average, OECD residents have released about 4.5 times as much greenhouse gasses per-person as non-OECD residents. And yet the consequences of those emissions are now and will in the future disproportionately affect the relatively poor people in relatively poor non-OECD countries. \footnote{Gardiner, S. M. (2011). A perfect moral storm: The ethical tragedy of climate change. New York: Oxford University Press; Nolt, J. (2011). Greenhouse gas emission and the domination of posterity, in \textit{The Ethics of Climate Change}, ed. Denis G. Arnold, 60-76, Cambridge, UK: Cambridge University Press.}. \\

\subsection{Social dilemmas and the need for ethical awareness}
\paragraph{Forty years of knowing.} Because no one really does own the air, individuals, companies and countries have for millennia discharged waste products, like greenhouse gasses, into the atmosphere. For the vast majority of that time, the consequences of those emissions were largely unknown. But that convenient innocence ended at least in 1977 CE, when climatologist Wally Broecker published``Are we on the Brink of a Pronounced Global Warming?'' in the journal \textit{Science}. In the article, he predicted that``the exponential rise in the atmospheric carbon dioxide content will ...by early in the next century will have driven the mean planetary temperature beyond the limits experienced during the past 1000 years.'' Writing now in 2017 CE, it iss unavoidably true that for at least 40 years we've known that releasing greenhouse gasses into the atmosphere was bound to cause harm to someone, somewhere, somewhen.\\

But even now it is difficult to precisely determine how much damage global climate change has caused to specific people, specific areas, or specific times. Our``standard moral calculus'' is that harms and their causes are personal, prompt and local. If I hit your car,  the damage was done by me, to you,  the damage happened as soon as I hit it, and we were clearly in the same place at the same time! But the``standard moral calculus'' isn't obviously applicable to the causes and consequences of climate change. Sure, residents of OECD countries have released 5 times the greenhouse gasses of non-OECD residents. But how am I responsible for the harm to a person in another part of the world, a person who perhaps won't be born until after I die? If we have essentially no interaction with the people harmed by our actions, how should we choose to act ethically? Are we really morally responsible for the welfare of people we'll never meet, for people born a generation after our own death?\\

\paragraph{Morals and ethics} Although often used inter-changeably, morals and ethics are distinctly different and complimentary aspects of species. Ethics is about the rules and practices individuals and societies invent to do the right thing. Morals are the personal opinions and beliefs that define what the right thing is. Ethics are how morals are put in to practice. Ian Welsh once once wagged that``Morals are how you treat people you know; ethics are how you treat people you don’t know.'' Even briefer:``morals are about me, ethics are about everyone else''.\\

The ethical aspect is what distinguishes the study of climate change (and this textbook) from most other scientific pursuits (and science textbooks). Were you enrolled in a course on Solar System Astronomy, for example, the class would likely be devoid of the talk of ethics and morals. You might touch on the affect of potential impacts on society, or on the philosophical aspects of exoplanets and alien life, but these are not moral problems. In fact, just about about every science professor you will ever have or had as an instructor will consciously and fervently avoid the topic in her class. Science, after all, is about the measurable and testable, two topics apparently not applicable to ethics and morals. But climate change is different. Human agency is irrelevant in the study of Solar System Astronomy: the planet Jupiter does not care how about your morals, and Kepler's Laws are unchanged by the ethical choices of OECD residents. The same is not true of climate. In his influential and precisely-titled book \textit{The Perfect Moral Storm, The Ethical Tragedy of Climate Change}, Stephen Gardiner\footnote{Gardiner, S. M. (2011). A perfect moral storm: The ethical tragedy of climate change. Oxford University Press, page 398} notes

\begin{quotation}\noindent\textit{At the most general level, ... we cannot get very far in discussing why climate change is a problem without invoking ethical considerations. If we do not think that our own actions are open to moral assessment, or that various interests (our own, those of our kin and country, those of distant people, future people, animals and nature) matter, then it is hard to see why climate change (or much else) poses a problem. But once we see this, then we appear to need some account of moral responsibility, morally important interests and what to do about both. And this puts us squarely in the domain of ethics.}
\end{quotation}
 
\subsubsection{The difference is the social dilemma}\label{Prisoners_D}
\paragraph{Me \textit{vs.} You or Me \textit{vs.} Me}
Ethical considerations are necessary when studying climate change because those changes are ultimately due to choices made by individuals. Individuals never exist in a society of one: we all interact with dozens, hundreds, even millions of people in our daily lives. Despite these social connections, most people understandably act in a way which benefits themselves in the short term. But often these choices lead to less good outcomes for others in the short term, and in the long term to less good outcomes even for themselves. This is the \emph{social dilemma of climate change}\footnote{Ostrom, Elinor (2009). A Polycentric Approach for Coping with Climate Change (October 1, 2009). World Bank Policy Research Working Paper Series, No. 5095. https://ssrn.com/abstract=1494833, accessed 9 February 2017}. The classic illustration of this concept is the Prisoner's Dilemma.\\

Connie and Blythe rob a bank, but are caught shortly afterward. Although accomplices, the two had never worked together previously. In jail they are placed in separate cells, and are unable to communicate with each other. The prosecutor visits each prisoner, and offers each the same deal:\\

''You have two choices: confess your actions or deny your guilt. If you confess, and your accomplice doesn't, you can go free and your accomplice will get a long sentence. If they confess and you don't, you'll get the long sentence and they walk free. If you both confess, you'll both do time, but you'll be paroled after awhile. If you both deny your guilt, you'll both be released with a fine. What do you choose to do?''\\

\paragraph{What would you do in this situation?} To start, think about the \emph{other} prisoner's choices. If they confess, and you deny, they go free. If they confess and you do too, they get some time. The very worst outcome for them is if they deny guilt and you confess. Then they do hard time. Your accomplice doesn't know or particularly trust you (you were robbing a bank, after all), so they expect you will do what is in your best self-interest, which is to confess. So they choose to do so, too. They make a perfectly rational self-interested decision which benefited them in the short-term, but which is detrimental to both you and them in the long-run! Had you both denied your guilt, you'd both be walking free. Apparently there are situations where being responsible for the welfare of others actually makes your situation better, too.\\

The difficulty so many scientists have with discussing ethical issues in the classroom is also well-illustrated by the Prisoner's Dilemma. Your professor could easily explore with you the fascinating dynamics of the issue. She might even explore the surprisingly deep mathematics of this and related``public goods games,'' even to the point of doing an experiment with grades on an assignment. She probably wouldn't even explicitly discuss the moral issues in the game, because most scientists are trained to test hypotheses with objective data. Few things are less objective than morals! But whether you or your professor want to deal with it explicitly or not, the choice you make in the prisoner's dilemma---or any real social dilemma---inevitably affects the welfare of others.\\

As this chapter demonstrates, even casual decisions we make in our daily lives has implications for the welfare of others---others who may live on the opposite side of the globe, even those not yet born. Our discussion of the deeply personal but important ethical issues involved in climate change is not to tell you what to do or how to think. Instead, these discussions should catalyze self-consideration of the ethical implications of your moral choices. As Matt Schlapp of the American Conservative Movement has noted \footnote{Interview of Matt Schlapp by Rachel Martin, 23 February 2017, 4:35 am EST. \textit{CPAC in the Trump Era}, Morning Edition, Washington, DC. National Public Radio. \href{http://www.npr.org/2017/02/23/516787809/cpac-in-the-trump-era}{link}, accessed 24 February 2017}, citizens should``get the ability to hear a vigorous debate that's raging across the country.... We think that helps us find the truth.''  \\

\subsection{Emission histories...transnational to individual}
\subsubsection{OECD to national emission histories}
\subsubsection{carbon foot prints}

\section{Hinterlands and Wastelands} \label{Hinterlands and Wastelands}
Even the casual decisions we make have global implications because of the way the global economy works. Few people have the time, ability or interest to gather, grow and hunt for all their needs. Instead, the vast majority of people (OECD and non-OECD residents alike) are part of a global economic system, with individuals devoting labor to a tiny part of the cycle that delivers food, goods and services, and removes trash, waste, and effluents to households. All this activity takes place at or near Earth's surface, and so inevitably affects all of Earth's natural spheres. The next time you peel back and enjoy the fruit of a banana, recall that growing, harvesting, transporting and consuming that banana is possible only through contributions of the biosphere, geosphere, atmosphere and hydrosphere. Oh, and a significant contribution from the economics of the Anthroposphere. \\
 
\subsection{The Circular Flow Model}
The``circular flow'' model is a simple but insightful way of visualizing this economic activity (See Figure \ref{fig:circ_flow}). Households and firms form the heart of the economy with individuals in the households providing labor to firms, which in turn pay wages to the individuals (as shown by the lower set of red and green arrows in the figure). Households then purchase goods and services from firms (as shown by the upper set of red and green arrows), thus completing two intersecting flows of goods, cash and work.\\

    
\subsubsection{Hinterlands Provide Resources}
But this model is insufficient: the firms need material to build goods, and households need food, not just wages, to live. So another, outer circle is needed\footnote{Harris, J. M., \& Codur, A. M. (2004). Macroeconomics and the Environment. Global Development and Environment Institute: Boston, MA, USA. \href{http://www.ase.tufts.edu/gdae/education_materials/modules/Macroeconomics_and_the_Environment.pdf}{link}, accessed 22 February 2017}. This circle illustrates the extraction of natural resources from the geosphere and biosphere that provides the raw materials for human culture. Some of these products (metal ores, coal) are used to make other products (automobiles and smart phones) while some (fish) are used directly.\\

For millenia these natural resources were found near human settlements. They had to be, as travel was difficult and transport of large volumes of material impossible. But the evolution of cities changed this. The high population density and built environment in a city required that natural resources be extracted not from the city itself, but from a  more distant \emph{hinterland}. This hinterland (literally``behind place'') had to be of sufficient size and accessibility to provide the urban areas with the raw materials, water, food and energy needed to fuel the city. Now, the``hinterland'' is global: that banana you ate last paragraph was likely grown in Guatemala, shipped in boats built in China, and fueled with oil mined in Russia.\\

Hinterlands can be messy places; extraction of even high grade resources generally requires enormous investments of energy and labor. Figure \ref{fig:hill_top} shows the results of over 30 years of mining coal in the Hobet-21 mine of West Virginia. This is a``hilltop removal'' mine; the valuable coal is exposed by scrapping off and dumping (in stream valleys) the overlying rock. The yellow line indicates the total area of hilltop removed over the 30 year life-time of the mine. The upper part of the image shows the island of Manhattan, to the same scale. The coal mined from Hebet-21 in 2007 CE alone would provide Manhattan with all its energy needs for less than 2 months. The Anthroposphere requires extensive hinterlands! This is made startlingly clear by Figure \ref{fig:pipelines}, which shows that of all the raw materials provided by hinterlands, water is by far the most important. \\

\subsubsection{Households and Firms Convert Resources and Energy to Products and Wastes}
After these natural resources are extracted from Earth's spheres, firms convert them to products and services consumed by other firms and households. The coal mined at Hebet-21, for example, was cleaned, crushed and transported by train cars to power plants, where the coal was burned to provide energy for offices, factories and homes. You can imagine that an office worker is designing a new generation of smart phone, destined to be manufactured at the factory, and eventually sold to an individual householder. All this activity is represented by the blue, red and green arrows of Figure \ref{fig:circ_flow}. While just about every OECD resident is part of this global circulation of economic activity, just 5 countries (China, the US, India, Brazil and Russia) consumed over 50\% of all globally extracted natural resources in 2009 CE\footnote{Giljum, S., Dittrich, M., Lieber, M., \& Lutter, S. (2014). Global patterns of material flows and their socio-economic and environmental implications: a MFA study on all countries world-wide from 1980 to 2009. Resources, 3(1), 319-339.}. On average, every person on the globe consumed about 11 metric tonnes (23 tons, about the weight of 4 full-grown elephants) of resources in 2017 CE\footnote{Giljum \textit{et al., op. cit.}}\\

\subsubsection{Wastelands Receive Waste Products}
But all of this activity also generates wastes. Some of this waste is a direct consequence of the activity, such as the smoke of burning coal escaping from the narrow smokestack in Figure \ref{fig:coal_powerplant}. By far though, the greatest waste produced by humans is sewage, which you can think of as the natural result of (and part of the need for) all that water we use.Figure \ref{fig:ins_and_outs} shows how all the inputs are used, and where the waste products go. About 96\% of those wastes are emitted, with various levels of pre-treatment, into the hydrosphere and atmosphere \footnote{Baccini, P., \& Brunner, P. H. (2012). Metabolism of the Anthroposphere: analysis, evaluation, design. MIT Press, page 45-46.} In parallel with the idea of``hinterlands,'' we'll call these dumping areas``wastelands.'' \\

\subsection{Hinterlands and Wastelands are the 5 spheres}
Clearly the hinterlands that supply natural resources to our civilization, and the wastelands in which we dilute the resulting waste products, are just Earth's five natural spheres. These spheres are clearly deeply ingrained in our global economy, where they play essential and irreplaceable roles. Resources are most valuable when they are highly concentrated. This makes their extraction, refining, and transport much easier to the firms or other entities which produce the resource. So resources---even those in the most remote hinterlands---tend to be owned by private concerns, or national or other governments. These same entities will go through enormous efforts---including removing entire hilltops (see Figure \ref{fig:hill_top}), to extract these concentrated resources. \\

But the whole the point of wastelands is to do just the opposite: to dilute the concentrated waste stream and remove it from the Anthroposphere. The least appealing of these waste streams---sewage---is an excellent (if disgusting) example. The infrastructure involved with containing, transporting, treating and diluting sewage is expensive: in 2014 CE alone, state and local governments in the United States spent approximately \$50 billion, roughly \$200 per person, on maintenance and construction of sewers and related infrastructure\footnote{Eskaf, Shadi (2015) Four Trends in Government Spending on Water and Wastewater Utilities Since 1956, University of North Caroline Environmental Finance Blog, \href{http://efc.web.unc.edu/2015/09/09/four-trends-government-spending-water/}{link} accessed 4 March 2017.} After being treated, the sewage is dumped into rivers, lakes and oceans, where currents both carry it away from the source and begin to dilute the sewage with cleaner water.\\

The same process happens to gaseous wastes, for example the smoke escaping the smokestack in Figure \ref{fig:coal_powerplant}. The waste products of combustion, including $CO_2$, eventually make thier way in to the atmosphere. This process is relatively inexpensive: unlike the hinterlands, the cost of using the wastelands is nearly free. After all, no one owns the air.\\ 

\section{The Atmosphere as a Common Pool Resource}  \label{The Atmosphere as a CPR}
But the desire to dilute effluent into the wastelands means that eventually they spread world-wide, like the smog from Chinese factories or the carbon dioxide from cars in the United States. The ability of the hydrosphere and atmosphere to hold this waste is yet another resource provided by Earth to human society. But this resource is different than most: this resource is freely available to anyone, often without direct cost. You can test this yourself: light a camp fire or offer to wash a friend's car. Chances are, no one will ask you for a license to add waste to the atmosphere or hydrosphere. Sure, outdoor fires may be banned by your local municipality, but \emph{not} because of the waste gasses. They are afraid of the open fire. If you live in an area of drought, you may not be allowed to use public water to wash that car, but letting the rinse water seep into the local stream is most likely \emph{not} unlawful. We all, it seems, use these spheres in common. \\  

\subsection{Common Pool Resources}\label{cprs}
The atmosphere's ability to act as a wasteland is an example of a \emph{common property resource}. You are most likely familiar with private ownership of resources: you own your phone, perhaps a car, maybe even a house or some land. You may also belong to an organization or club that provides resources to members only: gyms are a typical example. Finally, many resources are owned by the public, typically through some form of government. The United States government owns not only a splendid variety of National Parks, but also 27\% of all land in the US\footnote{Vincent, Carol Hardy, Laura A. Hanson, Jerome P. Bjelopera (2014) \textit{Federal Land Ownership: Overview and Data} Congressional Research Service 7-5700, R42346, \href{https://fas.org/sgp/crs/misc/R42346.pdf}{link}, accessed 06 March 2017}. All these things are resources, but the question of who``owns'' them is less clear. \\

While the Anthroposphere relies on thousands of different resources, most can be neatly and naturally divided by two complimentary characteristics (See Table \ref{tab:goods}). The first is \emph{excludability}. Clubs and private concerns can exclude``others'' from access to their resources. Common property and public resources are not easily excludeable---just about anyone can use them, freely. The other quality is the resource's \emph{extent}. Some resources are \emph{non-finite}. This class includes equipment in the gym (sure, there may be a line to use a machine, but the machine's usefulness isn't \textit{destroyed} by others' use), and some resources held by the public, such as park land. But for us, it is the many resources which are \emph{finite} that interest us. That resources are finite is obvious when applied to your dinner, but perhaps less obvious when applied to resources in the hinterlands, such as coal or gold ore, or wastelands, such as the ability of a dump to hold wastes.\\

\begin{table} 
\caption{Types of Resources}
\label{tab:goods}
\centering
\begin{tabular}{@{}lll@{}} \toprule
Extent & \multicolumn{2}{c}{Access}\\ \cmidrule(r){2-3} 
 &\emph{Excludable} & \emph{Non-excludable}\\
 \emph{Finite} & Private & Common Pool\\
 \emph{Non-finite} & Club    & Public\\ \bottomrule
\end{tabular}
\end{table}

\subsubsection{Definition}
Common property resources are by definition finite and freely available for use. Finite in this sense means that their use is a``zero-sum'' game: if you use some, there is less for me to use. Consider fish. Only a finite number of fish are available for taking at a given time, and if you extract a few thousand pounds from the ocean, there are fewer fish for me to take. Excluding others from taking common property resources is difficult---because of law, tradition, or nature of the resources. Fish for example can migrate across state and national boundaries, making them open to all.\\

\subsubsection{Examples}
CPRs are surprisingly (pardon the pun) common. As hinted above, fish are an excellent example of a common property resource. The European settlement of the New England region was driven by the spectacular abundance of fish in New England waters. At the time of the American Revolution, fish was the largest source of revenue to New England, and hence the war effort\footnote{Magra, C. P. (2007). The New England cod fishing industry and maritime dimensions of the American Revolution. Enterprise \& Society, 8(4), 799-806.}, and to this day a``sacred cod'' hangs in the Massachusetts state legislature as a reminder of the role cod fishing once played in the Commonwealth. As Figure \ref{fig:cod_lobstah} illustrates, a combination of over-fishing, climate change and bad policy decisions\footnote{Pershing, A. J., Alexander, M. A., Hernandez, C. M., Kerr, L. A., Le Bris, A., Mills, K. E., ... \& Sherwood, G. D. (2015). Slow adaptation in the face of rapid warming leads to collapse of the Gulf of Maine cod fishery. Science, 350(6262), 809-812, and Acheson, J., \& Gardner, R. (2014). Fishing failure and success in the Gulf of Maine: lobster and groundfish management. Maritime Studies, 13(1), 8} led to the utter collapse of the cod population in 1990s. The lobster fishery is still going strong. As we'll see below, the different fates of these resources has much to do with how fishers choose to act towards others.

Other common property resources can be found closer to home. Trash barrels in a park fit the definition:  freely available for use, but of finite capacity (Figure \ref{fig:trash}). Trash barrels are clearly the entry point of tash into a literal``wasteland'' for solid wastes. Eventually the barrels will be emptied and the trash transferred to an incinerator or landfill. Either way, the remnants of that trash will remain in the atmosphere or geosphere for 100s of years, as natural recycling(Figure \ref{fig:circ_flow} of such materials is dreadfully slow. Eventually even the largest dump, landfill or wasteland eventually fills up.\\

\subsubsection{Application to Climate}
Like a trash barrel, our atmosphere's ability to hold waste products is limited. Not by size, but because of how the very wastes we emit into the atmosphere are changing Earth's climate. Three fundamental actions at the very core of human culture---the burning of fossil fuels, land use change, and manufacuring if cement--- all release waste $CO_{2}$ gas into the atmosphere. In an irony known before the start of the $20^{th}$ Century, $CO_{2}$ is a potent greenhouse gas. Every kilogram we emit will eventually make Earth's climate warmer. The ability of the Atmosphere to absorb our waste $CO_{2}$ without changing Earth's climate is finite. Every kilogram of $CO_{2}$ you emit is one less kilogram I can emit without consuming the wasteland's resourcefulness.\\

But the atmosphere has been, and still is, a freely available wasteland for $CO_{2}$ and other GHG wastes. There are few direct costs (as of now) for emitting $CO_{2}$ into the atmosphere. So we are left with in inescapable conclusion:
\begin{quote}
	The ability of the Atmosphere to absorb the waste greenhouse gasses of human activities without changing Earth's climate is a common property resource.
\end{quote}
As a shorthand, we'll refer to this as the``atmospheric CPR.'' We've already seen that some common property resources, like cod from Maine, can be destroyed in years, while some, like lobster, can be well-managed for decades. A number of questions may occur to you immediately: Is collapse of a common property resource inevitable? Is there scientific research that provides insight into why some CPRs are well managed (like lobster) and some (cod) not? And finally, is there any evidence that humans are inclined to manage CPRs effectively? And perhaps most importantly, which path will the atmosphere, and climate, follow?\\
   
\subsection{The Tragedy of the Common: What do you choose to maximize}\label{collective_action}
CPRs are fragile because their fate is precariously balanced between their finite size and infinite availability. An added complexity is that CPRs are used by humans, humans who can make irrational decisions when they feel their best interests are at stake, as we saw in the Prisoner's Dilemma above. And most importantly, some problems raised by CPRs don't have a technological solution, so we can't``science the heck'' our way out of the problem\footnote{Huffam,  M., Kinberg, S., Schaffer, M., Scott, R., Sood, A. (Producer), \& Scott, R. (Director). (2015). The  Martian [Motion picture]. United States:  Twentieth Century Fox Film Corporation.}.\\
 
\subsubsection{Hardin and the Tragedy of the Common} 
These various threads surrounding CPRs were brought to public attention in a famous, controversial, quirky and brilliant paper written by ecologist Garrett Hardin over 50 years ago. In \textit{The Tragedy of the Commons\footnote{Hardin, G. (1968). The Tragedy of the Commons. Science, 162(3859), 1243. https://doi.org/10.1126/science.162.3859.1243}} Hardin noted that CPRs are often driven to extinction in a process he famously called the \textit{Tragedy of the Commons}. Because the CPR is freely available yet not owned by any individual, Hardin thought it was rational for users to extract as much from the resources as they could, as fast as they could, before someone else consumed the resource. The short-term rational self-interests of the users leads to destruction of the resource for all. This is the collective action dilemma: choice that may be good and rational for an individual user may be bad and catastrophic for a group of users. \\

Hardin used a simple but brilliant example of this problem. Many old villages in New England have a \emph{common}, (no 's') a publicly-owned, shared area. All landowners were allowed to graze cattle on the common, at no direct cost. Consider a single farmer: he has every reason to put as many cattle as he can on the common, where they can eat all the grass they want for free. Of course, all farmers in town have the same idea, and load their cattle on to the Common. Soon, the grass (the common resource) is completely consumed. James Acheson and Ray Gardner put the case succinctly when they wrote\footnote{Acheson, J., \& Gardner, R. (2014). Fishing failure and success in the Gulf of Maine: lobster and groundfish management. Maritime Studies, 13(1), 8. https://doi.org/10.1186/2212-9790-13-8} ``In a collective action dilemma, rational action by individuals brings disaster'' to societies.\\

After describing this result, Hardin correctly noted that collective action problems often``have no technical solutions.'' Individuals and societies can't science their way out of a collective action problem. They need to change social preferences, the way people, groups and societies act toward each other. Hardin thought the very first action societies had to take was choosing what facet of the resource to maximize.  Options might include short-term production, long-term sustainability, profitability, resource health, or a host of other properties. The problem is that only one facet can be maximized at a time. Hardin was so pessimistic about his fellow humans that he thought the Tragedy of the Commons was inevitable unless resources were owned by private concerns or governments. He was, fortunately, wrong about \emph{that}.\\

\subsection{A public goods game. Appendix}
\subsection{The Tragedy of the Commons isn't inevitable}
You can see for yourself (the lobster fishery in Figure \ref{fig:cod_lobstah}) that some CPRs avoid self-destruction just fine, thank you. Other famous examples can be found from local irrigation cooperatives in Tibet to wildlife sanctuaries in India to the atmosphere's ozone layer (see Chapter 6)\footnote{Dietz, T., Ostrom, E., \& Stern, P. C. (2003). The struggle to govern the commons. Science, 302(5652), 1907–1912.}. CPRs which have self-destructed are all too common. The cod fishery is one example, as are irrigation projects in California and, recently , the atmospheric CPR. Most people have lived through the collapse of the New England cod fishery without even knowing it, and the difficulties with California's irrigation system directly affects you only through the price of food. But self-destruction of the atmospheric CPR will inevitable lead to global climate change, and the planet-wide consequences discussed in Section \ref{Risk}. As Ostrom notes\footnote{Ostrom, Elinor (2009). A Polycentric Approach for Coping with Climate Change (October 1, 2009). World Bank Policy Research Working Paper Series, No. 5095. https://ssrn.com/abstract=1494833, accessed 9 February 2017}, global climate change is``potentially the largest dilemma the world has ever faced.'' How do individuals, groups and civilizations thrive if no seems to be looking out for each other?\\ 

\section{Altruism, Reciprocity and the Evolution of Morals} \label{The Evolution of Morals}
The answer is that people \emph{do} look out for each other. This altruism is the social glue holding human (and other primate) groups together. Altruism is the act of incurring some cost to the self, so that another benefits\footnote{Buss, D. (2012). Evolutionary psychology: The new science of the mind, 4th Edition Psychology Press, page 238.}. Altruistic behavior---towards one's kin, friends, fellow citizens, even probable enemies\footnote{Luke 10:25-37, the Parable of the Good Samaritan}---exists because it increases the fitness of societies. Altruism makes societies more likely to succeed, survive and thrive. 

\subsection{Users and their frequency determine the fate of CPRs}
As we saw in Section \ref{collective_action}, collective action problems often have no technical solution. Instead, their solution lies in changing individuals' rational choice to benefit themselves on the short term to actions which avoid disaster for societies in the long-term. Users of any CPR fall into four broad categories\footnote{Ostrom, E. (1999). Revisiting the Commons: Local Lessons, Global Challenges. Science, 284(5412), 278–282 \href{https://doi.org/10.1126/science.284.5412.278}{Link}}. \emph{Free-loaders} are those who never waiver from rational self-interest and never cooperate with others in solving the collective action dilemma. \emph{Conditional cooperators} will only help if guaranteed there is no free-loading by others. \emph{Provisional Cooperators} cooperate willingly, under the hope that others will reciprocate and cooperate as well. Finally, true \emph{altruists} willingly cooperate to solve the collective action problem without any conditions at all. Figure \ref{fig:CPR_users} shows the spectrum of user behaviors, and a rough estimate for the proportion of each user type in a society. 

Societies solve the collective action dilemma when their rules and behaviors increase the proportion of conditional cooperators over time, as shown by the blue, solid line in Figure \ref{fig:CPR_users}. Societies where free-loaders are too common, or where rules and behaviors allow free-loaders to increase, tend to spiral into the Tragedy of the Common (red dash-dot line)\footnote{Vollan, B., \& Ostrom, E. (2010). Cooperation and  the Commons. Science, 330(6006), 923–924. \href{https://doi.org/10.1126/science.1198347}{Link}}. These fates are schematically shown on the right panel of Figure \ref{fig:CPR_users} with more cooperation leading to long-term success, and less leading to, well, less. Note the similarity of these trajectories to those of lobster and cod in Figure \ref{fig:cod_lobstah}. The small proportion of altruists plays a unique role in nucleating cooperation amongst others in a society. Without them, cooperation is hard to initiate, and collective action dilemma lead to catastrophe.  

\subsection{Why would anyone choose altruism?}
But  being an altruist can be risky. Perhaps you've been in a long line---for a concert, a new product, or bus---and someone jumped in to line in front of you. Before you even decide how to react, someone ahead of her calls the line jumper out. This is a situation fraught with danger for that altruist who, with no expectation that any of the strangers in line with him will ever help him out in a similar situation, tries to prevent a free-loader from abusing the system. The free-loader could simply walk away, fight back, or just ignore the altruist. You and all the others in line are in that crucial middle group of potential``cooperators,'' whose reaction may determine the situation's outcome. Regardless, the altruist has little to gain, and much to loose from his actions. Altruism is a human trait, witnessed across cultures and across our history\footnote{Buss, D. (2012). Evolutionary psychology: The new science of the mind, 4th Edition Psychology Press, page 269.}, and yet the altruist's actions lead to no direct gains. Firefighters rush in to a burning building to save a child they've never met; strangers jump into rivers to rescue strangers; a soldier risks death to protect fellow citizens who will never hear of the sacrifice. How can altruism exist in human societies if it carries such risks? \\

Because it is evolutionarily helpful, not to the individual, but to the altruist's kin (relatives) and the rest of their society. Altruism makes societies more fit, and more likely to continue. Altruism evolved and remains part of human behavior because it's good for human societies. People don't choose to be altruistic. People \emph{are} altruistic. It's n their genes.\\

\subsection{The Science of Morality}\label{moral_science}
Charles Darwin himself thought morals were simply a response to evolutionary pressures. In The Descent of Man\footnote{Darwin, Charles (1871). The Descent of Man, and Selection in Relation to Sex. London: John Murray. 1st ed., pg 93. Facsimile available at \href{http://darwin-online.org.uk/EditorialIntroductions/Freeman_TheDescentofMan.html}{Darwin Online}, accessed 15 March 2017}, Darwin realized that``No tribe could hold together if murder, robbery, treachery, \&c., were common; consequently such crimes within the limits of the same tribe 'are branded with everlasting infamy'....'' But it isn't just altruism that evolved in this way. So too have morals, the very foundation of how humans decide between right and wrong. 
  
\subsubsection{Evolutionary Moral Psychology}
Contemplate your brain for a moment. One organ among six or seven dozen, but it consumes 20\% of all the energy (and oxygen) you consume in a day\footnote{Swaminathan, N. (2008). Why does the brain need so much power. Scientific American, 29(04), 2998.}. The human brain is a remarkable thing, but it isn't wholly human. Deep in the brain's interio (Figure \ref{fig:brain_affec_cog} lie the midbrain and the hindbrain. These are (evolutionarily speaking) the most ancient parts of the brain, as mammals, birds and most reptiles all have them. The fore brain---the part of your brain that did all that thinking about your brain is evolutionarily recent. As you just demonstrated, the forebrain is all about cognition: the act of making conscious, deliberative choices. The mid- and hind-brains are all about emotions, about the ingrained and hard-to-control \emph{feelings} we get when confronted with danger, injustice, or moral  violations\footnote{Haidt, J. (2007). The new synthesis in moral psychology. Science, 316(5827), 998–1002.}. The fore brain is the thoughtful, cognitive brain part. The mid- and hind-brain are the intuitive, affective part of our brains, the part that handles moods, feelings and attitudes. A raft of research since the 1980s\footnote{Reviewed in Haidt, Op. Cit.} indicates that our affective brains automatically, quickly, and intuitively interpret the world around us. Morals---our deeply held beliefs of what is fundamentally good or bad, what actions are just and which are not---emerge unconsciously from the affective brain, where they are thoughtfully shaped into actions by the cognitive part.\\
 
Evolution selects for traits, including actions, that make an individual more likely to reproduce. But our actions are driven by our moral beliefs, which are ultimately built and shaped by the affective brain. As the brain evolves, so too must our morals. Morality emerges in humans because it increases reproductive success, just as Darwin suggested. Altruism, the moral choice of putting others before self, exists because, it increases the chance that the altruists' genes will be reproduced.\\

\subsubsection{Not the altruist, just their genes!}
Altruism increases reproductive success in three different ways, each less direct than the other. The most direct way is through \emph{kin selection}. Your kin, or family, share many of the same genes. So altruistically sacrificing self for a child, sibling, or parent propagates $1/2$ of one's own genes, if they benefit at least twice as much as you sacrifice. Cousins? Not so much---they share only $1/8$ of your genes, so they'll need a benefit 8 times your cost. Acting altruistically to kin makes evolutionary sense, and does much to explain why all human societies have complicated rules for deciding who's kin, and how close those kin are. But kin selection doesn't explain why anyone would help a non-relative within their group, let alone a complete stranger. \\

Reciprocity---you scratch my back, and I'll scratch yours---is a basic component of the social glue that binds societies together. And not just human societies! Many primate species (including chimpanzees, bonobos, capuchins, tamarins) also share reciprocally with non-related others in their group. The best example is food sharing\footnote{Jaeggi, A. V., \& Gurven, M. (2013). Reciprocity explains food sharing in humans and other primates independent of kin selection and tolerated scrounging: a phylogenetic meta-analysis. Proceedings of the Royal Society B: Biological Sciences, 280(1768), 20131615–20131615. https://doi.org/10.1098/rspb.2013.1615} (Figure \ref{fig:bonobos_food_sharing}), where individuals share food when they have a surplus, on the condition that such sharing is reciprocated in the future. This \emph{reciprocal altruism} increases the fitness and health of all group members, if every individual shares. Free-loaders could``game'' this system by taking, but not giving, food or other resources. But human (and other primate) brains are hard-wired, probably in the affective portion, do detect, remember and punish free-loaders\footnote{Jaeggi, A. V., \& Gurven, M. Op. Cit.}. You are familiar with this if you have an aquantance who is happy to share your pizza, but never seems to offer any of hers. Reciprocal altruism increases the fitness of large groups because it leads to effective punishment of free-loaders, reducing their frequency in a population, and making the entire group more successful.\\
 
The third, and most controversial, of the three evolutionary forces shaping morality is \emph{group selection}. Kin selection and reciprocal altruism don't explain all aspects of human cooperation\footnote{Bowles, S., \& Gintis, H. (2011). A cooperative species: Human reciprocity and its evolution. Princeton University Press, pg 198-9, as quoted in Callicott, J. B. (2014). Thinking like a planet: The land ethic and the earth ethic. Oxford University Press.}. Members of the armed forces, for example, suffer extraordinary sacrifices with little hope of reciprocal benefit from to benefit their fellow citizens, their extended``group''. Shared morals allow individuals to identify their``group,'' and allows that group to prosper in the future. As Haidt noted, ``morality
binds and builds; it constrains individuals and ties them to each other to create groups that are emergent entities with new properties.... Humans attain their extreme group solidarity by forming moral communities within which selfishness is punished and virtue rewarded.'' Examples of this group selection are currently prominent in many areas of the world, where residents are objecting, sometimes violently, to the immigration of``other'' people into their country\footnote{See, for example, \href{http://www.desmoinesregister.com/story/news/politics/2017/03/13/iowa-gop-chair-jeff-kaufmann-condemns-steve-king-our-civilization-tweet/99116748/} {this article} from the Des Moines Register}. In 2017 CE,``other'' has been defined by religion\footnote{\href{http://www.bbc.com/news/world-europe-39287689}{BBC} Dutch election: Wilders defeat celebrated by PM Rutte, accessed 17 March 2017}, nationality\footnote{\href{http://www.foxnews.com/politics/2017/01/25/trump-to-order-construction-us-mexican-border-wall-reportedly-to-suspend-refugee-program.html}{Fox News} Trump to order construction of US-Mexico border wall; expected to suspend refugee program, accessed 17 March 2017}, ethnic background\footnote{\href{http://www.newsweek.com/2016/10/07/why-ethiopian-jews-israel-face-discrimination-racism-police-brutality-502697.html}{US News and World Report}Why Ethiopian Jews Face Increasing Discrimination and Police Brutality in Israel, accessed 17 March 2017}, and socio-economic class.
  
\subsubsection{Reciprocity requires trust that others will, well, reciprocate}
The previous sections demonstrated that altruism is an inherent moral imperative, practiced intimately with one's kin, directly with one's friends and acquaintances, and indirectly in large groups with whom we have a common moral stance. Reputation is the currency of all this altruism. Reputation grants the promise of future reciprocity and benefit; it fuels trust and hence altruism. Humans spend a remarkable amount of time managing their reputations (deleted anything from a social media account recently?), and clearly it has played a role in shaping our views of morality\footnote{Haidt, Op. Cit.} In a large group, individuals with positive reputations can lead the group in sanctioning or punishing free-loaders. Social reputation becomes increasingly important as a substitute for personal reputation as human groups grow. In very large groups, social reputation may be the only currency that matters, as any one who has taken down a social media page applying for a job can attest.

\subsection{Who is your neighbor? What are your responsibilities?}
Humans evolved in small groups, where reputation was personal, kin were close by, and group identity was easy to manage. Our moral machinery evolved in exactly the same circumstances, so our morals have evolved to be personal, direct and reciprocal. Our``standard moral calculus,'' the default ways our affective and cognitive brains``see'' morality, reflects this small group upbringing. Humans think harms and their causes are \emph{individual}, that cause and effect are close in space and time\footnote{Jamieson, D. (1992) Ethics, Public Policy, and Global Warming, Science, Technology, \& Human Values, Vol. 17, No. 2 (Spring, 1992), pg. 148}. If we return to our car and find a new dent in the fender (harm), we immediately investigate to see who (cause) might have done it, because we know the culprit was here since we parked the car (close).\\

\subsubsection{Groups and Globalization}\label{groups_global}
But the idea of close has changed substantially over human history, and particularly in the past 30 years. The``groups'' of which we are all part have grown by a factor of at least a few million in the past ten thousand years of human history (Figure \ref{big_city_pop}), from small clans of perhaps 2 dozen individuals, to cities and nations with tens or hundreds of million cohabitants. This increase in population has been matched by the increase in the range over which humans travel. For most of human history, travel was by foot, at roughly 5 kph (roughly 3 mph), or by rivers and oceans. Using a few aircraft, one can travel around the world in about 24 hours, all be it 10 km (35,000 ft) above ground. Communication, even to the once remotest parts of Earth, is now trivially easy with satellite phones that work \emph{anywhere} on Earth's surface, and cost less per day than many people spend on coffee. \\

What size is a``group'' of humans now? A reasonable model is to imagine yourself surrounded by  hierarchy of groups, from kin to everyone (see Figure \ref{groups}). Now add to this diagram the economic hierarchy centered about you, the circular flow model of Figure \ref{fig:circ_flow}. For most of us, that circular flow model extends to the hinterlands and wastelands that now, in our modern world, extend across every part of Earth, geosphere, biosphere, hydrosphere and atmosphere. And if, as philosopher J. Baird Callicott observed,``ethics exist to maintain the integrity of groups: they evolved to hold human societies together, as Darwin first observed, and as the work of... many other moral psychologists now confirms\footnote{Callicott, J. B. (2014). Thinking like a planet: The land ethic and the earth ethic. Oxford University Press, pg 147},'' where's the boundary between you and the groups to whom you show moral behavior? If your moral``group'' extends as far as your economic reach does, you may well be morally responsible for people on the other side of the planet. For people you may never meet. Even for people who aren't yet born.\\

\subsection{Values and morals belong in discussions about Climate Change}\label{values_morals}
Philosopher Dale Jameison has struggled for 25 years to understand the complicated morals of climate change. In 1992 CE, when global surface temperatures were 0.6 K ($1^\circ F$) cooler than today\footnote{Based on five-year running smooth of \href{https://www.ncdc.noaa.gov/cag/time-series/global/globe/land_ocean/ytd/12/1880-2016.csv}{NOAA} data, accessed 17 March 2017}, he wrote that climate change 
\begin{quotation}
is not a purely scientific problem that can be solved by the accumulation of scientific information. Science has altered us to [the] problem, but the problem also concerns our values. It is about how we ought to live, and how humans should relate to each other and to the rest of nature. These are problems of ethics and politics as well as problems of science.
\end{quotation}
Few aspects of contemporary concern in the United States and Australia are as politically and culturally divisive as climate change. Despite the fact that roughly $97\%$ of practicing climate scientists think humans are causing global warming \footnote{Cook, J., Oreskes, N., Doran, P. T., Anderegg, W. R. L., Verheggen, B., Maibach, E. W., … Rice, K. (2016). Consensus on consensus: a synthesis of consensus estimates on human-caused global warming. Environmental Research Letters, 11(4), 048002. \href{https://doi.org/10.1088/1748-9326/11/4/048002}{link}} only 65\% of adults in the United States agree \footnote{Gallup Inc, G. (n.d.). U.S. Concern About Global Warming at Eight-Year High. Retrieved March 24, 2017, from \href{http://www.gallup.com/poll/190010/concern-global-warming-eight-year-high.aspx}{Link}}, with politically conservative people less likely, and politically liberal people more likely, to agree with the consensus view of scientists. A growing literature suggests that people reject scientific information when it challenges their values. In this situation, their affective brain turns on and essentially``buries'' the threatening information in the cognitive background\footnote{Lewandowsky, S., \& Oberauer, K. (2016). Motivated Rejection of Science. Current Directions in Psychological Science, 25(4), 217–222. \href{https://doi.org/10.1177/0963721416654436}{Link}}. Our values can change the way we hear, process, and react to information. No amount of arguing about the facts changes our perceptions because we fear agreeing to the facts means disagreeing with our values. As Thomas Dietz\footnote{Dietz, T. (2013). Bringing values and deliberation to science communication. Proceedings of the National Academy of Sciences, 110(Supplement 3), 14081–14087. \href{https://doi.org/10.1073/pnas.1212740110}{Link}} notes, humans``tend to argue about facts when values are at stake...our reluctance to [discuss] values may lead us astray.'' Throughout this book we'll deliberate what climate change means to individuals, societies and humanity. The values we bring to this discussion are paramount, and determine how each of will choose to act in face of climate change. For now, this choice is about how to resolve the collective action problem posed by the atmospheric CPR.   

 
\section{Elinor Ostrom and effective governance of the common} \label{Elinor Ostrom's insights}
The difference between successfully and unsuccessfully managed CPRs lies in how they are \emph{governed}. Think of``governance..as the structures and processes by which [individuals] in societies make decisions and share power, creating the conditions for ordered rule and collective action.\footnote{Schultz, L., Folke, C., Österblom, H., \& Olsson, P. (2015). Adaptive governance, ecosystem management, and natural capital. Proceedings of the National Academy of Sciences, 112(24), 7369–7374}.'' The science of CPR governance was pioneered by Professor Elinor Ostrom, one of the winners of the Nobel Memorial Prize for Economics in 2009. Her work, with hundreds of colleagues and students, demonstrated through both field work and careful laboratory studies that successful governance strategies exist, and can be implemented by societies who want them.\\

So how does a collection of free-loaders, would-be cooperators, and the occasional altruists successfully manage a CPR? Observations of well-managed CPRs shows that the Tragedy of the Commons is \emph{not} inevitable.  Individuals in the group must eventually learn from experience that conditional cooperation---acting with the group as long as free-loaders are punished---leads to increased cooperation over time, and the long-term sustainability of the resource. This cooperation is begun by those rare altruists, but is ultimately completed by the remainder of the users, who enforce cooperation by preventing free-loaders from unfair gain. 

\subsection{Observed practices of successfully managed CPR}\label{practices}
These same observations show that a suite of core features are shared by these well-managed CPRs\footnote{Dietz, T., Ostrom, E., \& Stern, P. C. (2003). The struggle to govern the commons. Science, 302(5652), 1907–1912}. Fundamental to successful governance is data on the resource: how much there is, how fast that amount is changing, how healthy the resources is, are crucial observations that allow the users to maintain the resource. Users build and maintain reputations for trustworthiness by frequent face-to-face conversations. For example, lobster fishers in Maine accomplish both these tasks simultaneously, often over a doughnut and coffee in the pre-dawn hours before they head out for the day. This communication also helps users gauge each others behavior, and make sure that everyone follows the rules they've agreed to. Well-run CPRs have efficient and low-cost ways of excluding outsiders from the resource, either through formal rules or informal behaviors. Lobstering is an intensely local activity, and fishers who try to place their gear in another town's waters will quickly find their gear``missing'' if they continue the practice. Finally, and this might be out of control of a group, CPRs are easier to manage when the resource only changes slowly. Rapid change, as see in the cod fisheries of Maine, produces so much anxiety in users that free-loading becomes seductively attractive\footnote{Acheson, J., \& Gardner, R. (2014). Fishing failure and success in the Gulf of Maine: lobster and groundfish management. Maritime Studies, 13(1), 8. \href{https://doi.org/10.1186/2212-9790-13-8}{Link}}. 
   
\subsection{8 Successful Strategies}\label{no_toc}
With those goals in mind, Ostrom and her co-authors found that a few good strategies were enough to bring stability to CPRs (See Table \ref{tab:8 strats}). Each of them faces unique challenges when applied to global CPRs, such as the atmosphere CPR. Below we examine each strategy, grouped by the practices we saw in Section \ref{practices}

\begin{table} 
\caption{8 Strategies for Successful Governance of Common Pool Resources}
\label{tab:8 strats}
\centering
\begin{tabular}{@{}ll@{}} \toprule
Number & Title\\ \midrule 
 0 & No Panaceas\\
 1 & Be Prepared for Change\\
 2 & Dealing with Conflict\\
 3 & Collect Lots of Data\\
 4 & Analytical Discussions\\
 5 & Provide Infrastructure\\
 6 & Inducing Rules Compliance\\
 7 & Nested Governance\\ \bottomrule
\end{tabular}
\end{table}

\paragraph{Change is inevitable (Strategies 0, 5)}
A``panacea'' is a cure-all, something that fixes all ailments. Unfortunately, there are no panaceas when it comes to governing a CPR. Strategies that work in one CPR may not work in another one, and strategies that work now may not work in the future. Because change is inevitable, users should be open to review of the rules and limits governing the CPR. This requires experimentation with what works, often repeatedly. Even in small CPRs this can lead to confusion and anger amongst users. Experimentation is hard for any CPR, but is exceptionally hard for the atmospheric CPR. We have only one planet after all, and experiments take decades to play out. So our``experiments'' have to be done indirectly---through computer simulation, small-scale models, or using Earth's past natural climate changes as natural experiences.\\

%Despite the global nature of the consequences, the causes are local and local areas can explore various ways of dealing w the problem. For example, in the summer of 2016, California decided to close one of its nuclear power generation stations, deciding that the risks of keeping the plant open outweighed the carbon-free energy provided by the plant. New York state did just he opposite, deciding the same week to subsidize the nuclear power to its mix\href{https://www.washingtonpost.com/news/energy-environment/wp/2016/08/01/why-new-york-state-just-delivered-extremely-good-news-to-the-nuclear-industry/?utm_term=.b471e5ad7064}, {Washington Post}.

\paragraph{Build trust (Strategy 2)}
As we've seen repeatedly in this chapter, reputation is the key to initiating and maintaining the conditional cooperation so crucial to running a CPR. Honest, early and frequent communication is a key aspect of building reputation, trust and cooperation with in a group. Conflict within any group is inevitable, (partly because power is usually unevenly distributed between individuals), but having existing relationships helps to both reduce the frequency and severity of the conflicts.\\

Building trust on global CPRs is triply difficult. Everyone, all 7 billion humans, use the atmosphere CPR, so we can only build trust through reputation, which takes time, technology and money. But most of us will never, meet directly. So we need to rely on representative (often political) leaders, many of whom have vague reputations even with their own people. Finally, most CPRs are small enough that users share a common culture, which is definitely not true for global CPRs. The enormous variation in cultural beliefs across the human population makes reputation building difficult.   

\paragraph{Collect, analyze and discuss data (Strategies 1,6)}
Having reliable data on the status of the CPR is invaluable. Knowing how much of the resource is left, how fast the amount of the resource is changing, and where the remaining resources are located are all as valuable as the resource itself. But data alone is not enough: the data has to be presented to users in ways they can understand and act on, but in enough detail to be useful. Finally the users of the resource need to discuss the data amongst themselves and others so they can create a proper course of action for the CPRs management.\\

Applying this to climate change and the atmosphere CPR is exceptionally difficult. Surprisingly few people actually understand the greenhouse effect, and hence don't understand the causal relationship between emission of greenhouse gasses and global warming\footnote{Just 35 words,``Earth's [surface] transforms sunlight’s \emph{visible} light energy into \emph{infrared} light energy, which leaves Earth slowly because it is absorbed by greenhouse gases. When people produce greenhouse gases, energy leaves Earth \emph{even more slowly}---\emph{raising} Earth’s temperature."are enough to  get the point across, even to a lay audience. Ranney, M. A., \& Clark, D. (2016). Climate Change Conceptual Change: Scientific Information Can Transform Attitudes. Topics in Cognitive Science, 8(1), 49–75. \href{https://doi.org/10.1111/tops.12187}{Link}}. No amount of data on the status of the atmosphere can help users manage it successfully if they don't understand how the CPR actually works. Clearly, the goal of this book is to change that!   

\paragraph{Infrastructure is important (Strategy 4)}
Infrastructure--the built facilities and structures of society---conditions what is a CPR. Something as simple as barb-wire provides a great example. Prior to the 1880s, the Great Plains of the United States were largely open lands, through which``cowmen'' drove their grazing herds. Closing the millions of acres of land was impractical, and the grass remained a CPR. With the invention of barb-wire, enclosing vast stretches of land became possible, and the open prairie was rapidly fenced in. Barb-wire killed a CPR\footnote{Cook, S. The Rise of Barbed Wire and Its Transformation of the American Frontier \href{http://xroads.virginia.edu/~class/am485_98/cook/develp2.htm}{Link}accessed 24 March 2017}. Global communications and transport has allowed the flourishing of a global marketplace for resources from many CPRs: lobster from Maine is flown nightly to markets around the world.\\

\paragraph{Governance (Strategies 3 and 7)}
Users in well-run CPRs believe, often through extensive experience and experimentation, that the rules governing use of the resource are beneficial to them in the long-run. Users who have this``buy-in'' are more likely to follow the rules themselves, and are less tolerant of free-loading in others. Developing this ethic is easier when there are tiered penalties for rule-breakers. Thus mild or initial infractions should have mild penalties, but subsequent or serious infractions earn more serious sanctions.\\

Finally, there is how all this governance is structured. Research clearly indicates that CPRs fail if the only governance is from centralized, top-down administration through laws by national governments. Instead, governance is more effective when  it starts at the local level with users, and continues to larger scales. These groups form a``nested'' chain of institutions, all of which share management decisions, focused on maintaining the CPR\footnote{Brewer, J. (2012). Revisiting Maine’s lobster commons: rescaling political subjects. International Journal of the Commons, 6(2). https://doi.org/10.18352/ijc.336}. The lobster fishery in Maine is governed by local fishers at the port level, who effectively and collectively enforce the rules on the number and type of lobster caught. A fisher from each local collection of ports serves on a state-sponsored``Lobster Zone Council,'' which fosters communication and data sharing at the regional level. One person from each of these Councils also serves on the``Lobster Advisory Council,'' which advises Maine's Governor, sponsors lobster research and resolves the inevitable disputes between ports, zones and regions. The``Atlantic States Marine Fishery Council'' has a Lobster Board, which coordinates maintenance of lobster fisheries all along the New England coast. Most controversially, the``National Marine Fisheries Service'' provides a Federal presence in the management.\\

This is a sprawling, apparently cumbersome structure, and some levels were designed specifically to prevent other levels from gaining too much power. But it \emph{works}. Despite 70 years of intensive fishing, climate change, technological growth and two World Wars, lobster have survived as a resource, and lobstering has survived as a way of life.  \\  

\subsection{Governing the Atmospheric CPR is more difficult than governing other CPRs}
The global scale and slow change of the atmospheric CPR means it has different features than the vast majority of other CPRs. These unique features make governing the atmospheric CPR uniquely difficult. As we saw above, Ostrom's research showed that CPRs are well-governed when the users communicate repeatedly\footnote{Dietz,  et al. Op. Cit.} and personally with each other. This builds trust and reputation among the users. This trust allows users to escape the Prisoner's Dilemma, and avoid the worst-of-all-outcomes we saw in section \ref{Prisoners_D}. But the atmospheric CPR is global! Meeting, personally and repeatedly, with fellow emitters of greenhouse gasses from China, Brazil, or Australia is impossible. The only people who can meet repeatedly face to face are political entities, such as national governments, multi-national organizations (The UN), and others private entities. But even these meetings can be overly large, with well over 23,000 participants\footnote{Carbon Brief, quoting offical UN documentation https://www.carbonbrief.org/analysis-which-countries-have-sent-the-most-delegates-to-cop21, accessed 14 March 2017} at the important 2015 Climate talks in Paris. Building trust and reputation is hard when you have to meet 23,000 people in a few weeks.  

Another subtle difference is the slow, inter-generational nature of the atmospheric CPR's collapse. Cod fishing went from profitable to impossible in a generation. But the collapse of the atmospheric CPR has been happening for at least 150 years, and will continue for at least as long. In his perfectly titled essay \textit{The Perfect Moral Storm}, Steven Gardiner\footnote{Gardiner, S. M. (2006). A perfect moral storm: climate change, inter-generational ethics and the problem of moral corruption. Environmental values, 397-413.} argues that those same political entities who met in Paris tend to act with their current (not future) citizens in mind and so ignore the interests of future citizens. While true, this behavior isn't relegated just to politicians. Most people tend to discount the needs of future people in favor of those already here; many individuals even tend to discount their own future needs over their current needs. But are you any less``valuable'' in 20 years than you are now? This practice of discounting the future in favor of the present has always been a difficulty of managing CPRs, but it reaches new levels when applied to global CPRs such as the atmosphere's ability to hold the waste products of the Anthroposphere.  


%Cost of rule enforcement can be significant. TEAs and Sectors are both models of inducing rules compliance. So to can internalizing external costs. Finally there is the question of how and who manages all this. This was one of Gardiner's ``Perfect Moral Storm'' components. Research suggests CPRs are most-likely to be successful when the governing is done at many inter-laced levels of governance. 
 %
%These last two strategies show the difficulty in managing a global CPR. Citizens of the United States have a particularly difficult and complicated relationship with their Federal Government \href{http://www.people-press.org/2015/11/23/2-general-opinions-about-the-federal-government/#size-and-scope-of-government}{Pew Trust}. A small majority (53\%) of adult US citizens feel the federal government should be smaller and provide fewer services than it is now. This opinion is held more strongly by males, whites, the elderly and the wealthier. Adults in the US are evenly split (within uncertainties) on the role of the federal government in solving problems. According to the Pew Trust, ``47\% [of US adults] say the government should do more to solve problems, while 48\% say it is doing too many things better left to businesses and individuals.'' While half of adult citizens think the government shouldn't be fighting global warming, fully 88\% of them think the federal government {\it should} play a major role responding to natural disasters, and a whopping 79\% think the federal government is doing a {\it good} job doing so. It's hard to avoid the conclusion that most US citizen’s don't want the federal government to play a role preventing climate change, just to pay for its cleanup.

\section{A Safe Operating Space for Humanity} \label{Humanities Safe Place}
All complex systems, natural or artificial, have a ``safe operating space,\footnote{Rockstr\"{o}m, J., Steffen, W., Noone, K., Persson, Å., Chapin, F. S. I., Lambin, E., … Foley, J. (2009). Planetary Boundaries: Exploring the Safe Operating Space for Humanity. Ecology and Society, 14(2). https://doi.org/10.5751/ES-03180-140232}'' a realm where the machine runs predictably and with a low risk of sudden failure. A car, for example, can't be driven too fast for too long without damaging the engine; the human heart can't function well if the arteries are blocked by fatty deposits. Earth's climate is an especially complex system, and so we'd expect it too has boundaries. We are particularly interested in the boundaries that keep the climate operating in a way that is safe for humanity---resistant to sudden changes or failures that would endanger the Anthroposphere's continuing existence. Just like running a car too hard, running the Earth through those boundaries risks moving climate out of a safe space for humanity. If virtue is the practice of moral excellence, then the first step to solving the collective action dilemma posed by climate change is to avoid crossing planetary boundaries, and leaving Earth in a safe operating space for humanity.
 
\subsection{Planetary Boundaries}\label{pbs}
Societies and the humans in them need water, food, energy and raw materials to thrive. This requires healthy and well-functioning hinterlands. When human populations were small, so too were their needs. Over the course of the Holocene---the climatic period extending over the past 11,700 years---human population has increased by a factor of around 1400\footnote{Amongst other sources, Zahid, H. J., Robinson, E., \& Kelly, R. L. (2016). Agriculture, population growth, and statistical analysis of the radiocarbon record. Proceedings of the National Academy of Sciences, 113(4), 931–935. \href{https://doi.org/10.1073/pnas.1517650112}{link} note that human population growth rates prior to the modern era $\approx1650$ CE averaged $0.04\pm0.01\%/yr$. Given population at $1650\:CE\:(P_{1650})\approx0.5x10^9$, then $P_{9700\:BCE}=5x10^6$ with a factor of 3 uncertainty. So the ratio $P_{2018}/P_{9700\:BCE}=1.4x10^{3\pm0.5}$}. Each modern person, on average, consumes about 15 times\footnote{Baccini and Brewer, op. cit.} the raw materials as our early Holocene ancestors. Our hinterlands and wastelands are now 20,000 times larger than they were at the beginning of the Holocene Period! Our ancient ancestors couldn't possibly push the boundaries of the planet's ability to provide key services to us. We are.\\

A large international group of scientists proposed the planetary boundary idea\footnote{Rockstr\"{o}m, J., Steffen, W., Noone, K., Persson, Å., Chapin, F. S. I., Lambin, E., … Foley, J. (2009). Planetary Boundaries: Exploring the Safe Operating Space for Humanity. Ecology and Society, 14(2). \href{https://doi.org/10.5751/ES-03180-140232}{link}} in 2009. Their idea was to identify conditions in Earth's various spheres the crossing of which might initiate rapid and even dangerous planet-wide environmental change, challenging human societies' survival. Think of each boundary as the edge of a hill leading to an unknown, but dangerous, chasm. Because changing behaviors is difficult, and systems can have inertia, it is best to draw the boundary before the steep downward sloping portion of the hill. Figure \ref{fig:planet_bound_pot_surf} sketches the idea, with safe operating spaces shown in green, a ``warning area'' in yellow, and a ``danger area'' in red. The planetary boundary itself is the boundary between the yellow and red areas. Rockstr\"{o}m and his colleagues originally defined nine boundaries. Subsequent research and 8 more years of data indicates that only 7 of those original boundaries are essential for our purposes, and one new one is needed. Each boundary has a associated quantitative value attached to it, but only some of these are crucial for our story.\\ 

\subsubsection{Climate Change}
Of course climate change is one of the boundaries! The ``warning'' and ``danger'' boundaries are drawn at $350\: ppm$ and $500\pm50\:ppm\: CO_2$ in the atmosphere respectively. October of 1988 CE was the last time carbon dioxide concentration in the atmosphere was $<350\: ppm$; 30 years later concentrations have increased to 408 ppm. We are deep in the ``warning zone'' on this boundary, and we'll spend the remainder of this book examining the causes and solutions of this.

\subsubsection{Ocean Acidification}
All that $CO_2$ in the atmosphere leads to increased acidity in the world's oceans, as we noted above in section \ref{sl_acid}. Figure \ref{fig:ocean_acidity} shows this increase as measured in one representative area near Bermuda, where acidity is increasing by about 4\% per decade. Increased ocean acidity makes shell-secretion harder for marine organisms, and so the warning boundary is 125\% more difficulty in shell secretion than in 1750CE. We are current near 120\% more difficult, and so are still in the green part of the boundary. At current rates of carbon dioxide emission, we will cross the boundary in about 60 years.

\subsubsection{Net Plant Production}
At the base of all food webs lie plants. Terrestrial, or land-based, plants are particularly important in this regard, because they combine through photosynthesis water, $CO_2$ and light from the Sun to form carbohydrates. ``Net Plant Production,'' or NPP, measures the net mass of new plant material growing each year. Although it may vary locally over the year (See Figure \ref{fig:npp}), the annual total varies by $\leq2\%$ around $54x10^{12}$ kg. To put that number in perspective, the mass of plant growth every year is about 100 times greater than the mass of \emph{all} humans on the planet! Despite this enormous value, there's little variation in it, meaning that Earth's ability to produce plant mass is limited. And we, as a species, already appropriate\footnote{Smith, W. K., Zhao, M., \& Running, S. W. (2012). Global bioenergy capacity as constrained by observed biospheric productivity rates. BioScience, 62(10), 911-922} an astonishing 81\% of all available plant production on the globe. As noted by many authors\footnote{Running, S. W. (2012). A measurable planetary boundary for the biosphere. Science, 337(6101), 1458-1459.}, NPP is ``the source of food,fiber, and fuel for humanity,'' so the exceeding this boundary would be significant. While we are firmly in the ``warning zone,'' the proportion of NPP we use has been constant for the past 30 years, largely because of industrial-scale agricultural. This most fundamental of planetary boundaries will be threatened by all other boundaries, in ways that may be complex and unexpected. 

\subsubsection{Chemical Cycles}
Evidence for life on Earth is found in the oldest rocks geologist can find\footnote{Reviewed beautifully in Knoll, A. H., \& Nowak, M. A. (2017). The timetable of evolution. Science Advances, 3(5), e1603076.}. Excellent fossil evidence demonstrates that microbial live existed as early as $3400\pm 50 Ma$, and somewhat more controversial evidence pushes the age back to $\approx 3700 Ma$. Whatever the precise date of its origin, life has thrived on our planet for nearly 4 billion years. All that life needs energy and chemical building blocks, most importantly the elements C, H, N, O, S and P\footnote{Falkowski, P. G., Fenchel, T., \& Delong, E. F. (2008). The microbial engines that drive Earth’s biogeochemical cycles. Science, 320(5879), 1034–1039.} Life's long history on Earth is possible because those elements are continually cycled from the biosphere, into the geosphere, and then back into the biosphere.

These various cycles are energized by solar energy and the internal heat of the Earth. But small,unicellular microbes are the ``machines'' actually doing the cycling, converting the waste products of one life form (such as $O_2$ from photosynthesis in plants) to reactants for another form (such as $)_2$ for respiration in humans). Over the past 150 years, humans have collected vast quantities of N and P to use as fertilizers. These fertilizers have made it possible to feed billions of people, but at the cost of upsetting the delicate microbial machinery driving chemical cycles. Run-off from agricultural fields delivers N and P to the ocean, where the nutrients create ``dead zones'' in the oceans, potentially affecting the health of
the very microbial life forms necessary to drive the cycles. Warning areas for release of P and N are the natural release rates. Danger areas are 180\% and 130\% of natural release rates respectively. As shown in Figure \ref{fig:planet_bound}, we have already pushed beyond the danger boundary, at roughly 225\% for P and  nearly 250\% for N. This boundary is a particularly troubling one: growing sufficient food for 7.5 billion inhabitants seemingly requires N and P-containing fertilizers, but continued use of them may threaten the ecosystem's ability to provide those services.

\subsection{Ecosystem Services}\label{eco_servs}
``Ecosystem services''describes the various natural cycles and resources Earth processes provide to human societies. Figure \ref{fig:circ_flow} shows these services as the natural resources that underlie all human societies, and the natural recycling that converts human waste products back into resources. Without these services, human societies would lack the inputs we need not only to drive a healthy economy, but also to provide the very food and water we need for survival. The intimate relationship human societies have with ecosystem services suggests that once again we need to expand our concept of ``group'' to include those ecosystem services fundamental to our group's survival. Just as we argued in Section \ref{groups_global} that even distant peoples have ``standing'' as being members of ``our'' group, we now face the realization that we depend upon Earth systems for a functioning society. The idea that land, and even Earth itself, has moral standing, was first formulated by at least if one agrees that a functioning economy is important to society. The idea that land, perhaps even the Earth itself, has moral standing began in 1948, with the publication of \textit{A Sand County Almanac} by the American author Aldo Leopold, and has continued to develop since then\footnote{As examined by Callicott, J. B. (2014). Thinking like a planet: The land ethic and the earth ethic. Oxford University Press.}.\\

This moral standing sprouts from a spectrum of viewpoints\ref{fig:eco_values}. For example, many people may find an \textbf{instrumental value} in the Earth and its ecosystem services: there's monetary value in the resources one extracts in the hinterlands, and in services provided by wastelands. This viewpoint might be characterized as ``Nature \emph{for} me'' Others may find \textbf{relational value}\footnote{Chan, K. M., Balvanera, P., Benessaiah, K., Chapman, M., Diaz, S., Gómez-Baggethun, E., ... \& Luck, G. W. (2016). Opinion: Why protect nature? Rethinking values and the environment. Proceedings of the National Academy of Sciences, 113(6), 1462-1465.} in Earth and its ecosystems: Earth provides personally- and culturally-important meaning to them. Think of this viewpoint as ``Nature \emph{and} me'' Finally, some people think that Earth has \textbf{intrinsic value}: Earth itself has inherent worth. Think of this perspective as ``Nature \emph{for} Nature.'' These various ways of finding value in Earth and its services are not mutually incompatible, and most people probably have a mixture of all three, as shown in Figure \ref{fig:eco_values}.\\
 
\subsection{Cost of breaching Planetary Boundaries}\label{breach}
Regardless of which value (or values!) you place on Earth and the ecosystem services it provides, everyone has them. At the very least, if you eat food or drive a car, you associate instrumental values with Earth. So now our circle of ``ethical behavior'' (Figure \ref{fig:groups}) has to expand again, to include the Earth and the ecosystem services it provides, as shown on Figure \ref{fig:groups_and_circ}. Planetary boundaries (Section \ref{pbs}and Figure \ref{fig:planet_bound}) guard the gates of the a purely instrumental view of ecosystem services provided by Earth. If human civilization breaks through these gates, we are likely---in a few generations---to find ourselves in a radically different environment than the Holocene one in which we have thrived for nearly 12 millennia. 

As noted by Gretchen Daily\footnote{Daily, Gretchen, 1997, Introduction: What are Ecosystem Services. In Daily, Gretchen, (Ed.), 1997. Nature's Services: Societal Dependence On Natural Ecosystems 4th ed. Edition, Island Press, Washington, D.C.} ``assigning a value to'' something as huge, complicated and un-physical as 12,000 years of climate ``may arouse great suspicion, and for good reason. Valuation involves resolving fundamental philosophical issues,'' such as what's more important: the instrumental, relational or intrinsic value of Earth's climate. Climate change is a planetary issue, but its seriousness and solutions are inevitably deeply intertwined in how we---as individuals, societies, nations and a species---value nature\footnote{Hulme, M. (2009). Why we disagree about climate change: Understanding controversy, inaction and opportunity. Cambridge University Press}. What a mess! No wonder no progress has been made on such a ``wicked'' problem\footnote{Jenkins, W. (2016). The Turn to Virtue in Climate Ethics: Wickedness and Goodness in the Anthropocene. Environmental Ethics, 38(1), 77-96.}! Is there a path forward a majority of us can agree to?

\subsection{The virtue of not breaching global boundaries}
\subsubsection{A brief review of where we have been}
In Section \ref{values_morals} we noted the necessity of discussing morals when discussing climate change. To fully understand why, we examined common pool resources (Section \ref{cprs}), collective action problems (Section \ref{collective_action}), how reputation in a ``group'' (Section \ref{moral_science}) leads to altruism by a small minorty of individuals in the group, and a potentially successful solution to the Tragedy of the Common (Section \ref{no_toc}). Then we examined planetary boundaries (Section \ref{pbs}), and found that climate change is but one of many changes altering the services Earth's ecosystem provides us (Section \ref{eco_servs}) regardless of the value(s) we have for those services (Figure \ref{fig:eco_values}). 
   
\subsubsection{And a suggestion on how to move forward}
Virtue is an old concept in both Western and Eastern civilizations, and essentially means an ``excellent trait of character, coupled with a moral wisdom of understanding not only what is good, but how to achieve that good thing effectively\footnote{Hursthouse, Rosalind and Pettigrove, Glen, ``Virtue Ethics'', The Stanford Encyclopedia of Philosophy (Winter 2016 Edition), Edward N. Zalta (ed.), \href{https://plato.stanford.edu/archives/win2016/entries/ethics-virtue/}{Link}.}.'' Most humans would agree that maintaining a ``safe operating space'' for humanity of the planet is a virtuous idea: the outcome itself is a ``good thing,'' and the way to do that effectively is to not blow through the planetary boundaries that allow continued ecosystem services. This idea might be called ``planetary stewardship\footnote{Jenkins, W. (2016). The Turn to Virtue in Climate Ethics: Wickedness and Goodness in the Anthropocene. Environmental Ethics, 38(1), pg 89, and references therein.}.'' As Jenkins notes, ``stewardship shifts attention from protecting what nature \emph{would be} without human interference to maintaining what is required for the long-term flourishing of humanity''. We noted in Section \ref{GCC_diff} that global change is a different sort of problem because each of our small contributions to the problem seem so inconsequential. But the virtue of ``planetary stewardship'' rescues each of us from that depressing fate. Virtue is about doing good in an effective way. And so we each get to choose how to contribute to the virtue of not ``blowing through the boundaries.'' Interestingly, virtuous behavior is an effective way of building reputation, which of course is needed to solve collective action problems. 

\section{Choosing not to choose is choosing} \label{Choosing to choose}
This chapter is one long ethical and scientific argument about what, if any, responsibilities we have for the welfare of other people, even those we may never meet or even those born after we die. Such questions are unusual in any textbook, particularly one on science. But the ethical dilemmas posed by the global collective action problem of climate change are unique. Climate change happens so slowly, and the causes are so diffuse, that the affective moral outrage we normally associate with ethical transgressions are simply missing in our species\footnote{Jamieson, D. (2014). Reason in a dark time: Why the struggle against climate change failed--and what It means for our future. Oxford University Press.}. If you look back over this chapter you'll see that our examination of ethics, morals and collective actions has all been very data-driven: we've examined morals and ethics \emph{scientifically}. Values-the morals buried deep in our affective brains-matter in any discussion of climate change and how these changes will affect us. This is particularly true because it focuses our attention on where disagreements belong: on how we choose to react to climate change, not to the actuality of the change or its largely human-induced causes.

So now the choice is yours. You have to choose what moral obligations you have to your group (whatever that group is), and how you can help them flourish by contributing to planetary stewardship. Neither I, your professor, nor your classmates can tell you what choices to make in this regard. Whether you explicitly choose to act a certain way, or choose not to make a choice at all and just ``muddle through,'' your individual actions matter. To you, to your group, and ultimately our species. 

\section{Figures} \label{Figures}
\newpage

\begin{figure}[p]
\centering
\subfloat{%
  \includegraphics[width=5 in]{china_pollution_a}%
}

\subfloat{%
  \includegraphics[width=5 in]{china_pollution_b}%
}

\caption{Who owns the air? Eastern China routinely has amongst the worst air quality in the world. In the upper photograph (panel a) a relatively clear sky on January 3, 2013 shows a light snow dusting the ground around Beijing and Tianjin. The lower photograph (panel b) taken on January 14, 3013 shows clouds (brightest white) and heavy dense smog---air pollution---obscuring the ground; even the clouds are tinged with gray and brown from the smog. As explained more fully in the text, the air quality in Beijing that day was off-the-scale terrible.  Parts (a) and (b) are courtesy of NASA’s Earth Observatory (\href{http://earthobservatory.nasa.gov/IOTD/view.php?id=80152&src=ve}{link}).}
\label{fig:china_pollution_ab}

\end{figure}

\newpage
\begin{figure}[p]
\centering
\includegraphics[width=5 in]{china_pollution_c}%

\caption{The photograph is a ground level picture in Beijing on January 23, 2013. Note the time (8:30 am) and the weather forecast icon in the upper left corner of the phone-sunny. The gray is not fog, it is smog: air pollution from burning of fossil fuels and from burning wood. From the Atlantic magazine web site, \href{http://www.theatlantic.com/photo/2013/01/chinas-toxic-sky/100449/}{link}. I DO NOT have permission (yet) to use this image.}

\label{fig:china_pollution_c}

\end{figure}

\newpage
\begin{figure}[p]
\centering
\subfloat{%
  \includegraphics[width=5 in]{world_night_a}%
}

\subfloat{%
  \includegraphics[width=5 in]{world_night_b}%
}

\caption{Earth at night, as revealed by night-time imaging from the NASA Suomi satellite. (panel a, top) The world at night. Brightness and density of lights accurately maps wealth: brighter areas are richer. Note the incredible density of people in the Nile River valley (northeast Africa), the  \href{https://people.hofstra.edu/geotrans/eng/ch6en/appl6en/tokaido.html}{Tokaido Corridor} along the eastern coast of Japan, northern India, and central Europe from the Po river valley in Italy, through Germany, Belgium and Luxembourg, and on to London and Manchester in the United Kingdom.  Courtesy \href{http://eoimages.gsfc.nasa.gov/images/imagerecords/79000/79765/dnb_land_ocean_ice.2012.3600x1800.jpg}{NASA}) (panel b, bottom) The remarkable geography of human settlement and industry in  North America is beautifully revealed in this detailed image. Note the continuous city of Bosnywash (BOSton-New York-WASHington) in the northeast coast, the Great Lakes Megaloplis arcing from Toronto, Canada to Minneapolis, Minnesota. The two areas alone account for more than one third of the total US population! Note the sudden decrease in light intensity along the line stretching from Autin, TX to Fargo, ND. West of this line  precipitation rapidly falls from 30'' to 15'' per year. Large metropolitan areas need lots of water, more than 15'' can provide. Images are courtesy \href{https://www.nasa.gov/sites/default/files/images/712129main_8247975848_88635d38a1_o.jpg}{NASA}.}

\label{fig:world_night_ab}
\end{figure}
 
\newpage
\begin{figure}[p]
\centering
\includegraphics[width=5 in]{EPA_pm}%

\caption{Particulate matter (PM) is a particularly dangerous form of pollution, and is a common by-product of burning fossil fuels and other bio-matter. Smaller particles can be inhaled and then become lodged in the lungs, where they nucleate diseases. The finest and most dangerous PM are smaller than \SI{2.5}{\micro\meter}, are generally called ``PM\textsub{2.5}''. Particulate material less than \SI{10}{\micro\meter}, PM\textsub{10} are still quite dangerous. Their size can be hard to visualize. Here, a human hair (about \SI{70}{\micro\meter} across, and fine beach sand (\SI{90}{\micro\meter}) dwarf the PM particles that make air pollution fatal. As many as a million Chinese die prematurely from PM\textsub{2.5} pollution; it and its larger cousin contribute to the gray pallor seen in Figures \ref{fig:china_pollution_ab} and \ref{fig:china_pollution_c}. Courtesy of the US Environmental Protection Agency \href{https://www.epa.gov/sites/production/files/2016-09/pm2.5_scale_graphic-color_2.jpg}{link}.}

\label{fig:EPA_pm}

\end{figure}

\newpage
\begin{figure}[p]
\centering
\includegraphics[width=5 in]{O3_pac_winds}%

\caption{Brisk westerly winds (for historic reasons winds are named after the direction from which they blow) carry natural and anthropogenic material across the oceans. Here a series of three images show the movement of air and dust during a week-long interval in April of 2010. Darker shades of brown indicate exponentially higher concentrations of dust. The green star marks the approximate location of one air ``package'' during the period. Despite the intricate winding and unwinding associated with the air's fine-scale movement, the migration from east Asia to North America in a few days is evident. Courtesy of NASA's Earth Observatory page \href{http://earthobservatory.nasa.gov/IOTD/view.php?id=78742}{link}. The site includes an intriguing animation of the data}

\label{fig:O3_pac_winds}
\end{figure}

\newpage
\begin{figure}[p]
\centering
\includegraphics[width=5.6 in]{slr}%

\caption{Sea level rise, averaged over the globe, for the remainder of the century. Total sea level rise is sensitive to global warming, and hence to the mass of greenhouse gasses we chose to emit. The red line is a model of sea level rise in a world where little is done to constrain release of greenhouse gasses. This ``business as usual'' world would have \SIrange{0.4}{0.7}{\metre} (2.3 feet) of sea level rise by 2100 CE. If human society chooses to limit growth of greenhouse gas emissions so that they peak in 2100 CE, a more modest \SIrange{0.4}{0.2}{\metre} of sea level growth (1.3 feet, green line) results. The black line (in both the main figure and inset) shows historic sea levels to 900 BCE. Historic data are from Kopp \textit{et al.}, 2016 (Temperature-driven global sea-level variability in the Common Era. Proceedings of the National Academy of Sciences, 2016 113 (11) E1434-E1441; published ahead of print February 22, 2016, doi:10.1073/pnas.1517056113). Forecast for RCP 8.5 (red) and 6.0 (green) are from Church \textsl{et al.}, 2013, Sea Level Change. In: Climate Change 2013: The Physical Science Basis. Contribution of Working Group I to the Fifth Assessment Report of the Intergovernmental Panel on Climate Change [Stocker, \textit{et al.} (eds.)]. Cambridge University Press, Cambridge, United Kingdom and New York, NY, USA. }

\label{fig:slr}

\end{figure}

\newpage
\begin{figure}[p]
\centering
\subfloat{%
  \includegraphics[width=4 in]{Qannaaq}%
}

\subfloat{%
  \includegraphics[width=4 in]{Dawson_City}%
}

\caption{Melting permafrost will change the way people live in the polar regions. (Panel a, top) The Inuit village of Qannaaq in northwest Greenland is one of the most northerly (\ang{77.47}N latitude) occupied places in the planet. The village is built on \emph{permafrost}, permanently frozen ground that makes an excellent construction substrate. Until it melts. Photograph by Andy Mahoney, courtesy of the National Snow and Ice Data Center, \textsl{All About Frozen Ground}, accessed 20 January 2017. \href{https://nsidc.org/cryosphere/frozenground}{NSIDC}.  (panel b, bottom) When the permfrost does melt, as in this example of a building in Dawson City, Yukon Territory, Canada, the ground looses structural strength and buildings collapse. This problem is now particularly severe in portions of Alaska and Siberian Russia.Photograph by Andrew Slater, courtesy of the National Snow and Ice Data Center, \textsl{All About Frozen Ground}, accessed 20 January 2017. \href{https://nsidc.org/cryosphere/frozenground}{NSIDC}}

\label{fig:impermafrost}
\end{figure}


\newpage
\begin{figure}[p]
\centering
\subfloat{%
  \includegraphics[width=5.5 in]{rcp85T2090}%
}

\subfloat{%
  \includegraphics[width=5.5 in]{rcp45T2090}%
}

\caption{pablum}

\label{fig:rcp_T}
\end{figure}

\newpage
\begin{figure}[p]
\centering
\subfloat{%
  \includegraphics[width=5.5 in]{rcp85P2090}%
}

\subfloat{%
  \includegraphics[width=5.5 in]{rcp45P2090}%
}

\caption{pablum}

\label{fig:rcp_P}
\end{figure}

\newpage
\begin{figure}[p]
\centering
\includegraphics[width=6. in]{mass_ext}%
\caption{The large scale pattern of extinctions and originations on Earth over the past 540 My.``Originations'' are the opposite of extinction, the evolution of new genera.The gray band indicates the typical “background” range of origination and extinction rates. The``Big Five'' mass depletions are the events plunging below the -13\% extinction rate. Those marked with a circle are true mass extinctions, those marked with the asterisk may be due to low origination. The proportion of species going extinct in these events is staggering: In all of the Big Five $>75\%$ of animal species went extinct, as shown by the red scale on the right hand axis. The blue arrow on the right side of the figure indicates that the current mass extinction has moved about a quarter of the way toward the 75\% extinction level typical of other mass extinctions. Modified from Figures 1 and 2 of Richard K. Bambach, Andrew H. Knoll, and Steve C. Wang, “Origination, extinction, and mass depletions of marine diversity”, Paleobiology, 30(4), 2004, pp. 522–542. Special extinction totals are from Barnosky, A. D., Matzke, N., Tomiya, S., Wogan, G. O., Swartz, B., Quental, T. B., ... \& Mersey, B. (2011). Has the Earth's sixth mass extinction already arrived? Nature, 471(7336), 51-57 .}
\label{fig:mass_ext}
\end{figure}

\newpage
\begin{figure}[p]
\centering
\includegraphics[width=5.5 in]{hill_top}%

\caption{A composite photograph illustrating the relationship between people and their need for extensive hinterlands. The lower part of the figure is of the Hobet-21 mine, in southwestern West Virginia. The yellow line indicates the total mined area. The brownish area on the west was the active mine in 2015; the light green areas to the east are older areas of the mine undergoing regrowth. The  upper portion of the composite shows the island of Manhattan, outlined in white, to the same scale as the mine. In 2007 CE, the coal produced at the Hobet-21 mine would have provided about 50 days worth of energy (\href{https://www.nycedc.com/economic-data/july-2013-economic-snapshot}{link}, accessed 28 February 2017) to Manhattan. Roughly 6 more Hobet-21 mines would be needed just to provide one portion of one city with its energy needs. Because of the density of housing in Manhattan, it is one of the most energy efficient cities in the United States. Both photographs are courtesy of NASA Earth Observatory. The lower, via the Earth Observatory World of Change series, \href{http://earthobservatory.nasa.gov/Features/WorldOfChange/hobet.php}{Mountaintop Mining, West Virginia}, accessed 23 February 2017, and the upper from the Image of the Day archive \href{http://earthobservatory.nasa.gov/IOTD/view.php?id=82417}{Manhattan and Thanksgiving}, accessed 23 February 2017}  

\label{fig:hill_top}
\end{figure}

\newpage
\begin{figure}[p]
\centering
  \includegraphics[width=5.5 in]{circ_model}%
\caption{The simple but powerful``Circular Flow'' model of economic activity. In this model, economic activity centers on households, which can be individuals, nuclear families, multiple families, or even unrelated people, who sharing living costs and space (US Census Bureau, . Households provide labor and human capital to business firms (lower red arrow), in return \href{https://www.census.gov/quickfacts/table/PST045216/00}{link}, accessed 3 March 2017) for which firms pay wages and other income (lower green arrow). Households use these wages to purchase (upper green arrow) goods and services from other firms (upper red arrow), completing the circular flow of the economy. But this model in incomplete: both households and firms require input of raw materials (food, water, metals, energy, and so on) from Earth's natural resources to fuel the inner economic circles (blue arrows). The hinterlands that supply these natural resources are an important source of natural capital to the world's economy. Like human capital, they have economic value, even if that value is hard to precisely quantify. As you know from taking out your households trash, flushing a toilet, or watching smoke rise from a smokestack, all households and firms generate waste (gray arrows). This waste is generally expelled back into the environment, often into the hydrosphere and atmosphere. Natural recycling can move these waste products back in to the biosphere and geosphere (dashed gray line), while anthropic recycling can intercept the movement of waste back in to the environment.} 

\label{fig:circ_flow}
\end{figure}


\newpage
\begin{figure}[p]
\centering
  \includegraphics[width=5.5 in]{pipeline_1}%

\caption{One way of visualizing the raw materials needed to fuel the Anthroposphere. Randall Munroe imagines pipelines carrying all the liquids consumed in the United States in a year. The liquids move at 4 m/s, or about 10 mph. The pipes in the first panel are actual size. The second panel is scaled; the stick figure women provides the scale. Pipes with natural materials (water, petroleum,milk) represent extracted natural capital (the blue arrows in Figure \ref{fig:circ_flow}), while the pipes with manufactured products (ketchup, cement) are goods produced by firms (upper red arrow in Figure \ref{fig:circ_flow}). Note that water is by far the most important raw material to the Anthroposphere, larger than all the other fluids \emph{combined}. Not shown in this diagram is that an equal number of pipes would be necessary to carry wastes back into the Earth. This is XKCD cartoon \href{https://xkcd.com/1649/}{1649}, and is courtesy of Randall Munroe.}

\label{fig:pipelines}
\end{figure}

\newpage
\begin{figure}[p]
\centering
  \includegraphics[width=5.5 in]{pipeline_2}%

\caption{ }
\label{fig:pipelines_2}
\end{figure}


\newpage
\begin{figure}[p]
\centering
  \includegraphics[width=5.5 in]{coal_powerplant}%

\caption{Resource use leads inevitably to  discharge of wastes. A coal-fired power plant, such as this coal-fired electrical generating plant in northern Illinois, requires input of coal, delivered by train or barge. The desired output of burning the coal is electricity (transported by the high-voltage wires on the towers in the background). But combustion also includes the generation of soot, ash, and gasseous by products, some of which are emitted into the atmosphere by the smoke stack in the right background. The wide tower on the far right uses water from Lake Michigan as a coolant; the steam rising from this tower is relatively clean, but does carry significant waste heat lost to the environment. I do not yet have permission to use this image, which is \copyright 2010``codeeightythree
'' at \href{https://www.flickr.com/photos/railpictures/4316794283/in/photostream/}{this link}}
\label{fig:coal_powerplant}
\end{figure}

\newpage
\begin{figure}[p]
\centering
  \includegraphics[width=5.5 in]{ins_and_outs}%
\caption{}
\label{fig:ins_and_outs}
\end{figure}

\newpage
\begin{figure}[p]
\centering
  \includegraphics[width=5.5 in]{trash_barrels}%

\caption{Trash barrels are great examples of the use (and abuse) of common property resources. The resource here is a``wasteland,'' a place to put waste products of some activity. The trash barrel is finite in extent, so its use is a zero-sum game: the more you use, the less any one else can use. (Clearly the barrels shown here were far \emph{too} finite.) The barrels are freely available for all to use, so excluding anyone from depositing trash is difficult, even when the resource is completely full! These two characteristics---finite size and low excludability---are the defining characteristics of common property resources. Unfortunately, the overuse of the resource, as seen here, is all too common. Image courtesy of NOAA,  \href{https://marinedebrisblog.wordpress.com/tag/plastics/}{link}, accessed 08 March 2017.}
\label{fig:trash}
\end{figure}

\newpage
\begin{figure}[p]
\centering
  \includegraphics[width=6.5 in]{lobstah_cod}%


\caption{Fish are an excellent example of common property resources. This graph shows the weight of lobster (solid blue line) and cod (dash-dot red line) landed on docks in Maine. (The weight is the amount caught in a given year, divided by the maximum ever caught, and is shown as a percentage.) Cod landings peaked in the early 1990s, and then declined exponentially and catastrophically to essentially 0. Cod fishing in Maine is effectively extinct. This crash was due to a variety of causes, including poor governance of the resource (which contributed to over-fishing) and climate change. The fate of cod exemplify the \emph{Tragedy of the Commons}, the destruction of a resource from users choosing short-term gain over long-term sustainability. The lobster fishery is still going strong, although there is some evidence that climate change is eroding lobster habitats. The lobster fishery in Maine is an excellent example of a common property resource that has avoided the Tragedy of the Commons through robust governance by fishers, communities, local, state and Federal governments. (Data from \href{http://www.maine.gov/dmr/commercial-fishing/landings/documents/cod.table.pdf}{Maine DMR Cod} and  
\href{http://www.maine.gov/dmr/commercial-fishing/landings/documents/lobster.table.pdf}{Maine DMR lobstah},  accessed 09 March 2017.)}
\label{fig:cod_lobstah}
\end{figure}


\newpage
\begin{figure}[p]
\centering
  \includegraphics[width=5.5 in]{CPR_users}%


\caption{Users of CPRs fall into four behavioral categories, defined by inclined they are to work cooperatively to sustain the resource. Panel A (left) shows the relative frequency of the different types with the black dashed line, with cooperativeness increasing to the right. Altruists initiate cooperation regardless of others' actions, while free-loaders cooperate only with punishment. In the middle are those who cooperate on the provision that others do, and those whose cooperation is conditional on the elimination of free-loaders. Societies solve the collective action dilemma when their rules and behaviors increase the proportion of conditional cooperators over time, as shown by the blue, solid line. In societies where free-loaders are too common, or where rules and behaviors allow free-loaders to increase, tend to spiral into the Tragedy of the Common (dash-dot red line)(Vollan, B., \& Ostrom, E. (2010). Cooperation and  the Commons. Science, 330(6006), 923–924. https://doi.org/10.1126/science.1198347). These fates are schematically shown in Panel B, on the right, with more cooperation leading to long-term success, and less leading to, well, less. Note the similarity of these trajectories to those of lobster and cod in Figure \ref{fig:cod_lobstah}.} 
\label{fig:CPR_users}
\end{figure}


\newpage
\begin{figure}[p]
\centering
  \includegraphics[width=5.5 in]{brain_affec_cog}%

\caption{The human brain consists of two grand portions: the affective (in red) and the cognitive (in green). The affective brain evolved long before the cognitive portions, and is the source of our feelings of affection to kin (family) and friends, anger, sympathy and empathy. This is the center of our moral``intuition,'' where we distinguish between the morally good and bad. These raw thoughts are automatic and emerge quickly and unconsciously, sometimes to the embarrassment of our cognitive brains. That part of our brains is responsible for higher order moral``reasoning,'' of deciding how to do what is ethically right \textit{versus} wrong. You may have experienced this dichotomy of function if you have ever witnessed someone``cheating'' in a social situation---cutting in line, receiving undeserved benefits, or cheating on an exam---and felt an instant sense of anger or indignation. But did you then take a moment and think about what to do? If so, you have felt the affective message emerge and then shaped it by cognitive thought (Buss, D. (2012). Evolutionary psychology: The new science of the mind, 4th Edition Psychology Press, page 276-9. }
\label{fig:brain_affec_cog}
\end{figure}

\newpage
\begin{figure}[p]
\centering
  \includegraphics[width=5.5 in]{bonobo_food_sharing}%

\caption{https://www.flickr.com/photos/estherfotografeert/236212800/in/photostream/ Reciprocal altruism is an essential social activity performed by many primates, including these bonobos (a the sister species of the common chimpanzee). Here a dominant female (on the right) shares food with a subordinate female the left). Primates, with their relatively long life spans, penchant for living in small groups, and well-developed ability to identify and remember other individuals, make primates particularly pre-disposed to altruism (Trivers, R. L. (1971). The Evolution of Reciprocal Altruism. The Quarterly Review of Biology, 46(1), 35–57. https://doi.org/10.1086/406755). Even though food sharing looks like a simple short-term exchange, each cycle is a successful round of the ``prisoner's dillemma'' game. Evidence indicates that food sharing, grooming and other short-term practices add up to significant long-term reputation between the individuals (Jaeggi, A. V., De Groot, E., Stevens, J. M. G., \& Van Schaik, C. P. (2013). Mechanisms of reciprocity in primates: testing for short-term contingency of grooming and food sharing in bonobos and chimpanzees. Evolution and Human Behavior, 34(2), 69–77. https://doi.org/10.1016/j.evolhumbehav.2012.09.005). Small acts matter.}
\label{fig:bonobos_food_sharing}
\end{figure}


\newpage
\begin{figure}[p]
\centering
  \includegraphics[width=5.5 in]{groups}%
\caption{Humans are a social species: we thrive in our interactions with communities other humans. Not all interactions are equal, though. The reputation an individual has with family and close kin is generally more important to the individual than their reputation with more distant (geographically, culturally, or otherwise) communities. Like onions and ogres, human's ethical behavior is layered, with closer layers (communities) receiving more ethical standing. As noted by generations of philosophers, humans ethical and  ``moral duties and obligations are generated by our community memberships...which are multiple and`` layered (Callicott, J. B. (2014). Thinking like a planet: The land ethic and the earth ethic. Oxford University Press.). As our opinion of who is ``close'' changes, so too do our duties and obligations. Over our species's cultural history, ``close'' has tended to expand from an early, small community (in red) to a later, broader one (in green).}
\label{fig:groups}
\end{figure}


\newpage
\begin{figure}[p]
\centering
  \includegraphics[width=5.5 in]{planet_bound_pot_surf}%

\caption{A ball on a surface will naturally come to rest at a low point on the surface, as shown in Panel A (top) by the circle. The green coloring in the center indicates ``safe'' areas where the ball remains stable. Yellow indicates areas where, if change continues, the ball risks rolling away. Red indicates the areas of natural, perhaps sudden and catastrophic, change.  In Panel B (bottom), ball ``A'' has moved to the brink of a shallow hill. If it moves too much more to the left, it will slowly roll downhill, naturally, seeking another stable place. Ball ``B'' has already started to roll down its shorter and steeper hill.  Planetary boundaries are one way of envisioning this affect for Earth's systems. The inner yellow boundaries form the ``warning'' zone of change, while the outer red boundaries indicate the ``danger zone.'' Earth's Holocene climate is, like the ball in Panel A, stable because it is in an equilibrium created by interactions of Earth's spheres. Change the conditions of those spheres, or the interactions between them, and climate can be driven out of its stable area, where it can fall into places not seen since the dawn of human civilization.}
\label{fig:planet_bound_pot_surf}
\end{figure}

\newpage
\begin{figure}[p]
\centering
  \includegraphics[width=5.5 in]{planet_bound}%

\caption{The eight planetary boundaries we will consider, modified from Rockstr\"{o}m et al. (2009). Sectors in green are still in their safe zones, those in yellow are now in the ``warning'' zone, and those in red  are in the ``danger'' zone. Human activities, mostly associated with growing food, have pushed biodiversity and global geochemical cycles well in to the ``danger'' area. Three other systems, including climate change, land use, and ocean acidification, are all in , or close to, their respective ``warning zones.'' Two systems, fresh water and ozone depletion, are still firmly in the safe area. Ozone depletion is actually \emph{decreasing}, thanks to the remarkable success of the Montreal Protocols, an international environmental agreement signed by the United States and 196 other countries in 1987.}
\label{fig:planet_bound}
\end{figure}

\newpage
\begin{figure}[p]
\centering
  \includegraphics[width=5.5 in]{oceanacidity}%


\caption{Oceans absorb $CO_2$ from the atmosphere, so as the concentration of $CO_2$ increases in the latter, it increases in the former as well. Through a series of chemical reactions, this leads to an increase in the acidity of the oceans. This acidity is measured on the pH scale, with lower numbers indicating more acidic conditions. The graph shows the pH of sea water sampled near Bermuda (gray line) over the past 30 years. (Note the pH scale is reversed, so more acidic waters are toward the top of the diagram.)  Although a strong seasonal pattern is evident, over time the pH is increasing by about $0.016 pH/decade$, which amounts to almost 4\% increase in acidity every 10 years. The data are from the Environmental Protection Agency (\href{https://www.epa.gov/climate-indicators/climate-change-indicators-ocean-acidity\#ref5}{link}), who quote Bates,  N.R. 2016 update to data originally published in: Bates, N.R., M.H. Best, K. Neely, R. Garley, A.G. Dickson, and R.J. Johnson. 2012. Indicators of anthropogenic carbon dioxide uptake and ocean acidification in the North Atlantic Ocean. Biogeosciences 9:2509–2522.}
\label{fig:ocean_acidity}
\end{figure}

\newpage
\begin{figure}[p]
\centering
\subfloat{%
  \includegraphics[width=5.5 in]{npp_jul.jpg}%
	}

\subfloat{%
  \includegraphics[width=5.5 in]{npp_jan.jpg}%
}

\caption{Net Planet Production varies widely across Earth and season, but annual production varies by only $\pm 2\%/year$. NPP in July and January reflect the availability of light and water in controlling NPP. During northern hemisphere summer (upper panel), robust plant growth, shown by the deep green areas, includes the northern boreal forests (near $60\circ$ N) and the tropical rain forests (near the equator). The dry deserts ((near $30\circ$ N)) and polar ice caps lack water, soil, or light. In Northern hemisphere winter (lower panel), when light is scarce and temperatures cold, no growth occurs over the vast stretches of the northern landmasses. The southern continents bloom, but the relatively small areas of these continents means their productivity is small. This pattern will become clear when we examine the annual cycles in $CO_2$ concentrations in the atmosphere.  }

\label{fig:npp}
\end{figure}

\newpage
\begin{figure}[p]
\centering
  \includegraphics[width=5.5 in]{eco_values}%


\caption{Most people have a mixture of relationships with the natural world, visualized in the diagram above. Individuals who see nature as providing goods and services to humans see the natural world as having instrumental value: Nature for humans. Those who see the natural world as providing personally or culturally-important meaning to them respond to a relational values: Nature and humans. Finally, those who think the natural world has inherent worth, independent of human needs or wants. This intrinsic value system n=might be labeled: Nature for Nature. Where does your combination of values plot on this diagram? Each vertex represents a single value, where you'd be on the diagram if you thought the natural world had only, say, instrumental value. But if you enjoy the beauty of a sunset, the wash of waves at a beach, or even the smell of freshly-mowed grass, then you also find relational value in the natural world. }
\label{fig:eco_values}
\end{figure}



\newpage
\begin{figure}[p]
\centering
  \includegraphics[width=5.5 in]{groups_and_circ}%

\caption{Pablum}
\label{fig:groups_and_circ}
\end{figure}



\end{document}
